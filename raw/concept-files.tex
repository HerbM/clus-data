

This section describes the \clisp\ interface to file systems.

The model used by this interface assumes 
     that \newtermidx{files}{file} are named by \newtermidx{filenames}{filename},
     that a //[[CL:Glossary:filename]]// can be represented by a //[[CL:Glossary:pathname]]// //[[CL:Glossary:object]]//, 
 and that given a //[[CL:Glossary:pathname]]// a //[[CL:Glossary:stream]]// can be constructed 
      that connects to a //[[CL:Glossary:file]]// whose //[[CL:Glossary:filename]]// it represents.

For information about opening and closing //[[CL:Glossary:files]]//, and manipulating their contents, \seechapter\Streams.

\Thenextfigure\ lists some //[[CL:Glossary:operators]]//  that are applicable to //[[CL:Glossary:files]]// and directories.

\displaythree{File and Directory Operations}{ compile-file&file-length&open\cr delete-file&file-position&probe-file\cr directory&file-write-date&rename-file\cr file-author&load&with-open-file\cr }

\beginsubsection{Coercion of Streams to Pathnames} \DefineSection{StreamsToPathnames}

A //[[CL:Glossary:stream associated with a file]]// is either a //[[CL:Glossary:file stream]]// or a //[[CL:Glossary:synonym stream]]// whose target is a //[[CL:Glossary:stream associated with a file]]//. Such streams can be used as //[[CL:Glossary:pathname designators]]//.

Normally, when a //[[CL:Glossary:stream associated with a file]]// is used as a //[[CL:Glossary:pathname designator]]//, it denotes the //[[CL:Glossary:pathname]]// used to  open the //[[CL:Glossary:file]]//; this may be, but is not required to be, the actual name of the //[[CL:Glossary:file]]//.

Some functions, such as **[[CL:Functions:truename]]** and **[[CL:Functions:delete-file]]**, coerce //[[CL:Glossary:streams]]// to //[[CL:Glossary:pathnames]]// in a different way that  involves referring to the actual //[[CL:Glossary:file]]// that is open, which might or might not be the file whose name was opened originally.  Such special situations are always notated specifically and are not the default.

\endsubsection%{Coercion of Streams to Pathnames}

\beginsubsection{File Operations on Open and Closed Streams} \DefineSection{OpenAndClosedStreams}

\issue{CLOSED-STREAM-FUNCTIONS:ALLOW-INQUIRY}

Many //[[CL:Glossary:functions]]// that perform //[[CL:Glossary:file]]// operations accept either //[[CL:Glossary:open]]// or //[[CL:Glossary:closed]]// //[[CL:Glossary:streams]]// as //[[CL:Glossary:arguments]]//; \seesection\StreamArgsToStandardizedFns.

Of these, the //[[CL:Glossary:functions]]// in \thenextfigure\ treat //[[CL:Glossary:open]]// and  //[[CL:Glossary:closed]]// //[[CL:Glossary:streams]]// differently.

\displaythree{File Functions that Treat Open and Closed Streams Differently}{ delete-file&file-author&probe-file\cr directory&file-write-date&truename\cr }

Since treatment of //[[CL:Glossary:open]]// //[[CL:Glossary:streams]]// by the //[[CL:Glossary:file system]]//  may vary considerably between //[[CL:Glossary:implementations]]//, however,  a //[[CL:Glossary:closed]]// //[[CL:Glossary:stream]]// might be the most reliable kind of //[[CL:Glossary:argument]]// for some of these functions---in particular, those in \thenextfigure.  For example, in some //[[CL:Glossary:file systems]]//,  //[[CL:Glossary:open]]// //[[CL:Glossary:files]]// are written under temporary names  and not renamed until //[[CL:Glossary:closed]]// and/or are held invisible until //[[CL:Glossary:closed]]//. In general, any code that is intended to be portable should use such //[[CL:Glossary:functions]]// carefully.

\displaythree{File Functions where Closed Streams Might Work Best}{ directory&probe-file&truename\cr }

\endsubsection%{File Operations on Open and Closed Streams}

\beginsubsection{Truenames} \DefineSection{Truenames}

Many //[[CL:Glossary:file systems]]// permit more than one //[[CL:Glossary:filename]]// to designate  a particular //[[CL:Glossary:file]]//.

Even where multiple names are possible, most //[[CL:Glossary:file systems]]// have a convention for generating a canonical //[[CL:Glossary:filename]]// in such situations.  Such a canonical //[[CL:Glossary:filename]]// (or the //[[CL:Glossary:pathname]]// representing such a //[[CL:Glossary:filename]]//) is called a //[[CL:Glossary:truename]]//.  

The //[[CL:Glossary:truename]]// of a //[[CL:Glossary:file]]// may differ from other //[[CL:Glossary:filenames]]// for the file because of
     symbolic links,
     version numbers,

     logical device translations in the //[[CL:Glossary:file system]]//,

     //[[CL:Glossary:logical pathname]]// translations within \clisp,
  or other artifacts of the //[[CL:Glossary:file system]]//.

The //[[CL:Glossary:truename]]// for a //[[CL:Glossary:file]]// is often, but not necessarily, unique for each //[[CL:Glossary:file]]//.  For instance, a Unix //[[CL:Glossary:file]]// with multiple hard links  could have several //[[CL:Glossary:truenames]]//.

\beginsubsubsection{Examples of Truenames}

For example, a DEC TOPS-20 system with //[[CL:Glossary:files]]// \f{PS:<JOE>FOO.TXT.1}  and \f{PS:<JOE>FOO.TXT.2} might permit the second //[[CL:Glossary:file]]// to be referred to as \f{PS:<JOE>FOO.TXT.0}, since the ``\f{.0}'' notation denotes ``newest'' version of several //[[CL:Glossary:files]]//. In the same //[[CL:Glossary:file system]]//, a ``logical device'' ``\f{JOE:}'' might be  taken to refer to \f{PS:<JOE>}'' and so the names \f{JOE:FOO.TXT.2} or \f{JOE:FOO.TXT.0} might refer to \f{PS:<JOE>FOO.TXT.2}. In all of these cases, the //[[CL:Glossary:truename]]// of the file would probably be \f{PS:<JOE>FOO.TXT.2}.

If a //[[CL:Glossary:file]]// is a symbolic link to another //[[CL:Glossary:file]]// (in a //[[CL:Glossary:file system]]// permitting such a thing), it is conventional for the //[[CL:Glossary:truename]]// to be the canonical name of the //[[CL:Glossary:file]]// after any symbolic links have been followed; that is, it is the canonical name of the //[[CL:Glossary:file]]// whose contents would become available if an //[[CL:Glossary:input]]// //[[CL:Glossary:stream]]// to that //[[CL:Glossary:file]]// were  opened.

In the case of a //[[CL:Glossary:file]]// still being created (that is, of an //[[CL:Glossary:output]]// //[[CL:Glossary:stream]]// open to such a //[[CL:Glossary:file]]//), the exact //[[CL:Glossary:truename]]// of the file might not be known until the //[[CL:Glossary:stream]]// is closed.  In this case,  \thefunction{truename} might return different values for such a //[[CL:Glossary:stream]]// before and after it was closed.  In fact, before it is closed, the name returned might not even be a valid name in the //[[CL:Glossary:file system]]//---for example, while a file is being written, it might have version \kwdref{newest} and might only take on  a specific numeric value later when the file is closed even in a //[[CL:Glossary:file system]]// where all files have numeric versions.

\endsubsubsection%{Examples of Truenames}

\endsubsection%{Truenames}
