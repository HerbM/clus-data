


This standard presents the syntax and semantics to be implemented by a
//[[CL:Glossary:conforming implementation]]// (and its accompanying documentation).
In addition, it imposes requirements on //[[CL:Glossary:conforming programs]]//.

\beginsubSection{Conforming Implementations}
 
A //[[CL:Glossary:conforming implementation]]// shall adhere to the requirements outlined
in this section.

\beginsubsubsection{Required Language Features}
\DefineSection{ReqLangFeatures}

A //[[CL:Glossary:conforming implementation]]// shall accept all features 
(including deprecated features)
of the language specified in this standard,
with the meanings defined in this standard.

A //[[CL:Glossary:conforming implementation]]// shall not require the inclusion of substitute
or additional language elements in code in order to accomplish a feature of
the language that is specified in this standard.

\endsubsubsection%{Required Language Features}

\beginsubsubsection{Documentation of Implementation-Dependent Features}

A //[[CL:Glossary:conforming implementation]]// shall be accompanied by a document
that provides a definition of all //[[CL:Glossary:implementation-defined]]//
aspects of the language defined by this specification.

In addition, a //[[CL:Glossary:conforming implementation]]// is encouraged (but not required) 
to document items in this standard that are identified as
//[[CL:Glossary:implementation-dependent]]//, although in some cases
such documentation might simply identify the item as ``undefined.''

\endsubsubsection%{Documentation of Implementation-Dependent Features}

\beginsubsubsection{Documentation of Extensions}

A //[[CL:Glossary:conforming implementation]]// shall be accompanied by a
document that separately describes any features accepted by the
//[[CL:Glossary:implementation]]// that are not specified in this standard, but that do not
cause any ambiguity or contradiction when added to the language
standard.  Such extensions shall be described as being ``extensions to
\clisp\ as specified by ANSI \metavar{standard number}.''
 
\endsubsubsection%{Documentation of Extensions}

\beginsubsubsection{Treatment of Exceptional Situations}

A //[[CL:Glossary:conforming implementation]]// shall treat exceptional situations 
in a manner consistent with this specification.
 
\beginsubsubsubsection{Resolution of Apparent Conflicts in Exceptional Situations}

If more than one passage in this specification appears to apply to the
same situation but in conflicting ways, the passage that appears
to describe the situation in the most specific way (not necessarily the
passage that provides the most constrained kind of error detection) 
takes precedence.

\beginsubsubsubsubsection{Examples of Resolution of Apparent Conflicts 
			  in Exceptional Situations}

Suppose that function \f{foo} is a member of a set $S$ of //[[CL:Glossary:functions]]// that
operate on numbers.  Suppose that one passage states that an error must be
signaled if any //[[CL:Glossary:function]]// in $S$ is ever given an argument of \f{17}.
Suppose that an apparently conflicting passage states that the consequences 
are undefined if \f{foo} receives an argument of \f{17}.  Then the second passage
(the one specifically about \f{foo}) would dominate because the description of
the situational context is the most specific, and it would not be required that
\f{foo} signal an error on an argument of \f{17} even though other functions in 
the set $S$ would be required to do so.

\endsubsubsubsubsection%{Examples of Resolution of Apparent Conflicts


\endsubsubsubsection%{Resolution of Apparent Conflicts in Exceptional Situations}

\endsubsubsection%{Treatment of Exceptional Situations}

\beginsubsubsection{Conformance Statement}

A //[[CL:Glossary:conforming implementation]]// shall produce a conformance statement 
as a consequence of using the implementation, or that statement
shall be included in the accompanying documentation.  If the implementation
conforms in all respects with this standard, the conformance statement
shall be
 
\beginlist
\item{} ``\metavar{Implementation} conforms with the requirements 
	  of ANSI \metavar{standard number}''
\endlist
 
If the //[[CL:Glossary:implementation]]// conforms with some but not all of the requirements of this
standard, then the conformance statement shall be
 
\beginlist
\item{} ``\metavar{Implementation} conforms with the requirements of
	  ANSI \metavar{standard number} with the following exceptions: 
	  \metavar{reference to or complete list of the requirements of
		   the standard with which the implementation does not conform}.''
\endlist 
 
\endsubsubsection%{Conformance Statement}

\endsubSection%{Conforming Implementations}

\beginsubSection{Conforming Programs}
\idxterm{conforming program}\idxterm{conforming code}

Code conforming with the requirements of this standard shall adhere to the
following:

\beginlist
 \itemitem{1.} //[[CL:Glossary:Conforming code]]// shall use only those features of the
               language syntax and semantics that are 
	       either specified in this standard
		   or defined using the extension mechanisms 
		      specified in the standard.







 





\itemitem{2.} //[[CL:Glossary:Conforming code]]// may use
	      //[[CL:Glossary:implementation-dependent]]// features and values, 
	      but shall not rely upon
	      any particular interpretation of these features and values 
	      other than those that are discovered by the execution of //[[CL:Glossary:code]]//.

\itemitem{3.} //[[CL:Glossary:Conforming code]]// shall not depend on the consequences
	      of undefined or unspecified situations.

\itemitem{4.} //[[CL:Glossary:Conforming code]]// does not use any constructions 
	      that are prohibited by the standard.

\itemitem{5.} //[[CL:Glossary:Conforming code]]// does not depend on extensions 
	      included in an implementation.
\endlist 









































\beginsubsubsection{Use of Implementation-Defined Language Features}

Note that //[[CL:Glossary:conforming code]]// may rely on particular
//[[CL:Glossary:implementation-defined]]// values or features. Also note that the
requirements for //[[CL:Glossary:conforming code]]// and //[[CL:Glossary:conforming implementations]]// do not
require that the results produced by conforming code always be the
same when processed by a //[[CL:Glossary:conforming implementation]]//. The results may be the
same, or they may differ.





















 
Conforming code may run in all conforming implementations, but might
have allowable //[[CL:Glossary:implementation-defined]]// behavior that makes it
non-portable code.

For example, the following are examples of //[[CL:Glossary:forms]]// that are conforming, but
that might return different //[[CL:Glossary:values]]// in different implementations:

\code
 (evenp most-positive-fixnum) \EV //[[CL:Glossary:implementation-dependent]]//
 (random) \EV //[[CL:Glossary:implementation-dependent]]//
 (> lambda-parameters-limit 93) \EV //[[CL:Glossary:implementation-dependent]]//
 (char-name #\\A) \EV //[[CL:Glossary:implementation-dependent]]//
\endcode 

\beginsubsubsubsection{Use of Read-Time Conditionals}
\DefineSection{ReadTimeConditionals}





Use of \f{\#+} and \f{\#-} does not automatically disqualify a program
from being conforming.  A program which uses \f{\#+} and \f{\#-} is 
considered conforming if there is no set of //[[CL:Glossary:features]]// in which the
program would not be conforming.  Of course, //[[CL:Glossary:conforming programs]]// are
not necessarily working programs.  The following program is conforming:

\code
(defun foo ()
  \#+ACME (acme:initialize-something)
  (print 'hello-there))
\endcode

However, this program might or might not work, depending on whether the
presence of the feature \f{ACME} really implies that a function named
\f{acme:initialize-something} is present in the environment.  In effect,
using \f{\#+} or \f{\#-} in a //[[CL:Glossary:conforming program]]// means that the variable
\varref{*features*}\idxref{*features*}
becomes just one more piece of input data to that 
program.  Like any other data coming into a program, the programmer
is responsible for assuring that the program does not make unwarranted
assumptions on the basis of input data.

\endsubsubsubsection%{Use of Read-Time Conditionals}

\endsubsubsection%{Use of Implementation-Defined Language Features}

\beginsubsubsection{Character Set for Portable Code}

//[[CL:Glossary:Portable code]]// is written using only //[[CL:Glossary:standard characters]]//.

\endsubsubsection%{Character Set for Portable Code}

\endsubSection%{Conforming Programs}
