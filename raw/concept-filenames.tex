

There are many kinds of //[[CL:Glossary:file systems]]//, varying widely both in their superficial syntactic details, 		and in their underlying power and structure. The facilities provided by \clisp\ for referring to and manipulating //[[CL:Glossary:files]]// has been chosen to be compatible with many kinds of //[[CL:Glossary:file systems]]//, while at the same time minimizing the program-visible differences  between kinds of //[[CL:Glossary:file systems]]//.

Since //[[CL:Glossary:file systems]]// vary in their conventions for naming //[[CL:Glossary:files]]//, there are two distinct ways to represent //[[CL:Glossary:filenames]]//: as //[[CL:Glossary:namestrings]]// and as //[[CL:Glossary:pathnames]]//.

\beginSubsection{Namestrings as Filenames}

\issue{PATHNAME-HOST-PARSING:RECOGNIZE-LOGICAL-HOST-NAMES} A //[[CL:Glossary:namestring]]// is a //[[CL:Glossary:string]]// that represents a //[[CL:Glossary:filename]]//.

In general, the syntax of //[[CL:Glossary:namestrings]]// involves the use of  //[[CL:Glossary:implementation-defined]]// conventions,  usually those customary for the //[[CL:Glossary:file system]]// in which the named //[[CL:Glossary:file]]// resides. The only exception is the syntax of a //[[CL:Glossary:logical pathname]]// //[[CL:Glossary:namestring]]//, which is defined in this specification; \seesection\LogPathNamestrings.

A //[[CL:Glossary:conforming program]]// must never unconditionally use a //[[CL:Glossary:literal]]// //[[CL:Glossary:namestring]]// other than a //[[CL:Glossary:logical pathname]]// //[[CL:Glossary:namestring]]// because \clisp\ does not define any //[[CL:Glossary:namestring]]// syntax  other than that for //[[CL:Glossary:logical pathnames]]// that would be guaranteed to be portable. However, a //[[CL:Glossary:conforming program]]// can, if it is careful,  successfully manipulate user-supplied data  which contains or refers to non-portable //[[CL:Glossary:namestrings]]//.

A //[[CL:Glossary:namestring]]// can be //[[CL:Glossary:coerced]]// to a //[[CL:Glossary:pathname]]// by \thefunctions{pathname} or **[[CL:Functions:parse-namestring]]**.

\endSubsection%{Namestrings as Filenames}

\beginSubsection{Pathnames as Filenames} \DefineSection{PathnamesAsFilenames}

\newtermidx{Pathnames}{pathname} are structured //[[CL:Glossary:objects]]// that can represent, in an //[[CL:Glossary:implementation-independent]]// way, the //[[CL:Glossary:filenames]]// that are used natively by an underlying //[[CL:Glossary:file system]]//.

In addition, //[[CL:Glossary:pathnames]]// can also represent certain partially composed  //[[CL:Glossary:filenames]]// for which an underlying //[[CL:Glossary:file system]]//  might not have a specific //[[CL:Glossary:namestring]]// representation.

A //[[CL:Glossary:pathname]]// need not correspond to any file that actually exists,  and more than one //[[CL:Glossary:pathname]]// can refer to the same file. For example, the //[[CL:Glossary:pathname]]// with a version of **'':newest''**  might refer to the same file as a //[[CL:Glossary:pathname]]//  with the same components except a certain number as the version. Indeed, a //[[CL:Glossary:pathname]]// with version **'':newest''** might refer to different files as time passes, because the meaning of such a //[[CL:Glossary:pathname]]// depends on the state of the file system.  

Some //[[CL:Glossary:file systems]]// naturally use a structural model for their //[[CL:Glossary:filenames]]//, while others do not.  Within the \clisp\ //[[CL:Glossary:pathname]]// model,  all //[[CL:Glossary:filenames]]// are seen as having a particular structure, even if that structure is not reflected in the underlying //[[CL:Glossary:file system]]//. The nature of the mapping between structure imposed by //[[CL:Glossary:pathnames]]// and the structure, if any, that is used by the underlying //[[CL:Glossary:file system]]// is //[[CL:Glossary:implementation-defined]]//.

Every //[[CL:Glossary:pathname]]// has six components:
     a host,
     a device,
     a directory,
     a name,
     a type,
 and a version. By naming //[[CL:Glossary:files]]// with //[[CL:Glossary:pathnames]]//,  \clisp\ programs can work in essentially the same way even in //[[CL:Glossary:file systems]]// that seem superficially quite different. For a detailed description of these components, \seesection\PathnameComponents.

\issue{FILE-OPEN-ERROR:SIGNAL-FILE-ERROR} \issue{PATHNAME-SYNTAX-ERROR-TIME:EXPLICITLY-VAGUE}%

The mapping of the //[[CL:Glossary:pathname]]// components into the concepts peculiar to each //[[CL:Glossary:file system]]// is //[[CL:Glossary:implementation-defined]]//. There exist conceivable //[[CL:Glossary:pathnames]]// for which there is no mapping to a syntactically valid //[[CL:Glossary:filename]]// in a particular //[[CL:Glossary:implementation]]//. An //[[CL:Glossary:implementation]]// may use various strategies in an attempt to find a mapping; for example,  an //[[CL:Glossary:implementation]]// may quietly truncate //[[CL:Glossary:filenames]]// that exceed length limitations imposed by the underlying //[[CL:Glossary:file system]]//, or ignore certain //[[CL:Glossary:pathname]]// components for which the //[[CL:Glossary:file system]]// provides no support. If such a mapping cannot be found, an error \oftype{file-error} is signaled.

The time at which this mapping and associated error signaling  occurs is //[[CL:Glossary:implementation-dependent]]//. Specifically, it may occur 
    at the time the //[[CL:Glossary:pathname]]// is constructed,
    when coercing a //[[CL:Glossary:pathname]]// to a //[[CL:Glossary:namestring]]//,
 or when an attempt is made to //[[CL:Glossary:open]]// or otherwise access the //[[CL:Glossary:file]]// 
     designated by the //[[CL:Glossary:pathname]]//.

\Thenextfigure\ lists some //[[CL:Glossary:defined names]]// that are applicable to //[[CL:Glossary:pathnames]]//.

\displaythree{Pathname Operations}{ *default-pathname-defaults*&namestring&pathname-name\cr directory-namestring&open&pathname-type\cr enough-namestring&parse-namestring&pathname-version\cr file-namestring&pathname&pathnamep\cr file-string-length&pathname-device&translate-pathname\cr host-namestring&pathname-directory&truename\cr make-pathname&pathname-host&user-homedir-pathname\cr merge-pathnames&pathname-match-p&wild-pathname-p\cr } \endSubsection%{Pathnames as Filenames}

\beginSubsection{Parsing Namestrings Into Pathnames}

Parsing is the operation used to convert a //[[CL:Glossary:namestring]]// into a //[[CL:Glossary:pathname]]//. \issue{PATHNAME-HOST-PARSING:RECOGNIZE-LOGICAL-HOST-NAMES} Except in the case of parsing //[[CL:Glossary:logical pathname]]// //[[CL:Glossary:namestrings]]//,

this operation is //[[CL:Glossary:implementation-dependent]]//, because the format of //[[CL:Glossary:namestrings]]// is //[[CL:Glossary:implementation-dependent]]//.

A //[[CL:Glossary:conforming implementation]]// is free to accommodate other //[[CL:Glossary:file system]]// features in its //[[CL:Glossary:pathname]]// representation and provides a parser that can process  such specifications in //[[CL:Glossary:namestrings]]//.   //[[CL:Glossary:Conforming programs]]// must not depend on any such features,  since those features will not be portable.

\endSubsection%{Parsing Namestrings Into Pathnames}
