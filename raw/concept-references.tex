


\beginlist

\item{\bull} {\AnatomyOfLisp},
        John Allen, McGraw-Hill, Inc., 1978.


\item{\bull} {\KnuthVolThree},
             Donald E. Knuth, Addison-Wesley Company (Reading, MA), 1973.

\item{\bull} {\MetaObjectProtocol},
	Kiczales et al., MIT Press (Cambridge, MA), 1991.

\item{\bull} ``\CLOSPaper,''
        D. Bobrow, L. DiMichiel, R.P. Gabriel, S. Keene, G. Kiczales, D. Moon,
        \i{SIGPLAN Notices} V23, September, 1988.

\item{\bull} {\CLtL},
        Guy L. Steele Jr., Digital Press (Burlington, MA), 1984.

\item{\bull} {\CLtLTwo},
        Guy L. Steele Jr., Digital Press (Bedford, MA), 1990.

\item{\bull} {\CondSysPaper},
	Kent M. Pitman,
	{\it Proceedings of the First European Conference
	     on the Practical Application of LISP\/}
        (EUROPAL '90),
	Churchill College, Cambridge, England,
	March 27-29, 1990.

\item{\bull} {\FlavorsPaper},
	Howard I. Cannon, 1982.

\item{\bull} {\IEEEFloatingPoint},
        ANSI/IEEE Std 754-1985,
        Institute of Electrical and Electronics Engineers, Inc. (New York), 1985.

\item{\bull} {\IEEEScheme},
	IEEE Std 1178-1990,
	Institute of Electrical and Electronic Engineers, Inc. (New York), 1991.

\item{\bull} {\InterlispManual}, Third Revision,
	Teitelman, Warren, et al,
	Xerox Palo Alto Research Center (Palo Alto, CA), 1978.

\item{\bull} \ISOChars,
	\i{Information processing---Coded character sets 
           for text communication---Part 2: Latin alphabetic and non-alphabetic
           graphic characters}, 
	ISO, 1983.

\item{\bull} {\LispOnePointFive},
	John McCarthy, MIT Press (Cambridge, MA), August, 1962.

\item{\bull} {\Chinual},
	D.L. Weinreb and D.A. Moon,
	Artificial Intelligence Laboratory, MIT (Cambridge, MA), July, 1981.

\item{\bull} {\Moonual},
	David A. Moon, Project MAC (Laboratory for Computer Science),
        MIT (Cambridge, MA), March, 1974.

\item{\bull} ``{\NILReport},'' 
        JonL White, \i{Macsyma User's Conference}, 1979.

\item{\bull} {\GabrielBenchmarks},
	Richard P. Gabriel, MIT Press (Cambridge, MA), 1985.

\item{\bull} ``{\PrincipalValues},'' 
        Paul Penfield Jr., \i{APL 81 Conference Proceedings},
        ACM SIGAPL (San Francisco, September 1981), 248-256.
        Proceedings published as \i{APL Quote Quad 12}, 1 (September 1981).

\item{\bull} {\Pitmanual},
	Kent M. Pitman, 
	Technical Report 295,
	Laboratory for Computer Science, MIT (Cambridge, MA), May 1983.

\item{\bull} ``{\RevisedCubedScheme},''
        Jonathan Rees and William Clinger (editors), 
        \i{SIGPLAN Notices} V21, \#12, December, 1986.

\item{\bull} ``\SOneCLPaper,''
	R.A. Brooks, R.P. Gabriel, and G.L. Steele,
	\i{Conference Record of the 1982 ACM Symposium on Lisp and Functional Programming},
	108-113, 1982.

\item{\bull} \SmalltalkBook,
        A. Goldberg and D. Robson, Addison-Wesley, 1983.

\item{\bull} ``{\StandardLispReport},''
        J.B. Marti, A.C. Hearn, M.L. Griss, and C. Griss,
        \i{SIGPLAN Notices} V14, \#10, October, 1979.

\item{\bull} {\WebstersDictionary},
	Merriam Webster (Springfield, MA), 1986.

\item{\bull} \XPPaper,
        R.C. Waters,
	Memo 1102a,
	Artificial Intelligence Laboratory, MIT (Cambridge, MA), September 1989.

\endlist
