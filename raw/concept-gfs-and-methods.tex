



\beginSubsection{Introduction to Generic Functions}
\DefineSection{IntroToGFs}

A //[[CL:Glossary:generic function]]// is a function whose behavior depends on
the //[[CL:Glossary:classes]]// or identities of the //[[CL:Glossary:arguments]]// supplied to it.
A //[[CL:Glossary:generic function]]// //[[CL:Glossary:object]]// 

is associated with 
     a set of //[[CL:Glossary:methods]]//,
     a //[[CL:Glossary:lambda list]]//,
     a //[[CL:Glossary:method combination]]//\meaning{2}, 
 and other information.

Like an //[[CL:Glossary:ordinary function]]//, a //[[CL:Glossary:generic function]]// takes //[[CL:Glossary:arguments]]//,
performs a series of operations, and perhaps returns useful //[[CL:Glossary:values]]//.
An //[[CL:Glossary:ordinary function]]// has a single body of //[[CL:Glossary:code]]// that is always //[[CL:Glossary:executed]]//
when the //[[CL:Glossary:function]]// is called.  A //[[CL:Glossary:generic function]]// has a set of bodies
of //[[CL:Glossary:code]]// of which a subset is selected for //[[CL:Glossary:execution]]//. The selected
bodies of //[[CL:Glossary:code]]// and the manner of their combination are determined by
the //[[CL:Glossary:classes]]// or identities of one or more of the //[[CL:Glossary:arguments]]// to the
//[[CL:Glossary:generic function]]// and by its //[[CL:Glossary:method combination]]//.

//[[CL:Glossary:Ordinary functions]]// and //[[CL:Glossary:generic functions]]// are called with identical syntax.
 
//[[CL:Glossary:Generic functions]]// are true //[[CL:Glossary:functions]]// that can be passed as //[[CL:Glossary:arguments]]//
and used as the first //[[CL:Glossary:argument]]// to **[[CL:Functions:funcall]]** and **[[CL:Functions:apply]]**.

A //[[CL:Glossary:binding]]// of a //[[CL:Glossary:function name]]// to a //[[CL:Glossary:generic function]]//
can be //[[CL:Glossary:established]]// in one of several ways.  It can be
//[[CL:Glossary:established]]// in the //[[CL:Glossary:global environment]]// by 
 **[[CL:Functions:ensure-generic-function]]**,
 \macref{defmethod} (implicitly, due to **[[CL:Functions:ensure-generic-function]]**)
or
 \macref{defgeneric} (also implicitly, due to **[[CL:Functions:ensure-generic-function]]**).
\issue{GENERIC-FLET-POORLY-DESIGNED:DELETE}
No //[[CL:Glossary:standardized]]// mechanism is provided for //[[CL:Glossary:establishing]]// a
//[[CL:Glossary:binding]]// of a //[[CL:Glossary:function name]]// to a //[[CL:Glossary:generic function]]//
in the //[[CL:Glossary:lexical environment]]//.

\issue{WITH-ADDED-METHODS:DELETE}























\issue{WITH-ADDED-METHODS:DELETE}










\issue{GENERIC-FLET-POORLY-DESIGNED:DELETE}





When a \macref{defgeneric} form is evaluated, one of three actions
is taken (due to **[[CL:Functions:ensure-generic-function]]**):




\beginlist

\itemitem{\bull} If a generic function of the given name already exists,
the existing generic function object is modified.  Methods specified
by the current \macref{defgeneric} form are added, and any methods in the
existing generic function that were defined by a previous \macref{defgeneric}
form are removed.  Methods added by the current \macref{defgeneric} 
form might replace methods defined by \macref{defmethod}, 
\macref{defclass}, \macref{define-condition}, or \macref{defstruct}.  
No other methods in the generic function are affected
or replaced.

\itemitem{\bull} If the given name names 
    an //[[CL:Glossary:ordinary function]]//, 
    a  //[[CL:Glossary:macro]]//,
 or a //[[CL:Glossary:special operator]]//, 
an error is signaled.

\itemitem{\bull} Otherwise a generic function is created with the
methods specified by the method definitions in the \macref{defgeneric}
form.

\endlist

Some //[[CL:Glossary:operators]]// permit specification of the options of a
//[[CL:Glossary:generic function]]//, such as 
the //[[CL:Glossary:type]]// of //[[CL:Glossary:method combination]]// it uses 
or its //[[CL:Glossary:argument precedence order]]//.
These //[[CL:Glossary:operators]]// will be referred to as
``operators that specify generic function options.''
\issue{GENERIC-FLET-POORLY-DESIGNED:DELETE}



The only //[[CL:Glossary:standardized]]// //[[CL:Glossary:operator]]// in this category is \macref{defgeneric}.

\issue{WITH-ADDED-METHODS:DELETE}







Some //[[CL:Glossary:operators]]// define //[[CL:Glossary:methods]]// for a //[[CL:Glossary:generic function]]//.
These //[[CL:Glossary:operators]]// will be referred to as
\newtermidx{method-defining operators}{method-defining operator};
their associated //[[CL:Glossary:forms]]// are called //[[CL:Glossary:method-defining forms]]//.
The //[[CL:Glossary:standardized]]// //[[CL:Glossary:method-defining operators]]// are listed in \thenextfigure.















\DefineFigure{StdMethDefOps}
\issue{GENERIC-FLET-POORLY-DESIGNED:DELETE}
\issue{WITH-ADDED-METHODS:DELETE}


\displaythree{Standardized Method-Defining Operators}{
defgeneric&defmethod&defclass\cr
define-condition&defstruct&\cr
}


Note that of the //[[CL:Glossary:standardized]]// //[[CL:Glossary:method-defining operators]]//






only \macref{defgeneric}
can specify //[[CL:Glossary:generic function]]// options.
\macref{defgeneric} and any //[[CL:Glossary:implementation-defined]]// //[[CL:Glossary:operators]]//
that can specify //[[CL:Glossary:generic function]]// options
are also referred to as ``operators that specify generic function options.''

\endSubsection%{Introduction to Generic Functions}

\beginSubsection{Introduction to Methods}
\DefineSection{IntroToMethods}

//[[CL:Glossary:Methods]]// define the class-specific or identity-specific behavior
and operations of a //[[CL:Glossary:generic function]]//. 



A //[[CL:Glossary:method]]// //[[CL:Glossary:object]]// 

is associated with 

     //[[CL:Glossary:code]]// that implements the method's behavior,
     a sequence of //[[CL:Glossary:parameter specializers]]// 
       that specify when the given //[[CL:Glossary:method]]// is applicable,
     a //[[CL:Glossary:lambda list]]//,
 and a sequence of //[[CL:Glossary:qualifiers]]// that are used by the method combination
       facility to distinguish among //[[CL:Glossary:methods]]//.

A method object is not a function and cannot be invoked as a function. 
Various mechanisms in the \OS\ take a method object and invoke its method
function, as is the case when a generic function is invoked.  When this
occurs it is said that the method is invoked or called.

A method-defining form contains the //[[CL:Glossary:code]]// that is to be run when the
arguments to the generic function cause the method that it defines to
be invoked.  When a method-defining form is evaluated, a method object
is created and one of four actions is taken:

\beginlist

\itemitem{\bull} If a //[[CL:Glossary:generic function]]// of the given name already exists
and if a //[[CL:Glossary:method object]]// already exists that agrees with the new one on
//[[CL:Glossary:parameter specializers]]// and //[[CL:Glossary:qualifiers]]//, the new //[[CL:Glossary:method object]]// replaces
the old one.  For a definition of one method agreeing with another on
//[[CL:Glossary:parameter specializers]]// and //[[CL:Glossary:qualifiers]]//, 
\seesection\SpecializerQualifierAgreement.

\itemitem{\bull} If a //[[CL:Glossary:generic function]]// of the given name already exists
and if there is no //[[CL:Glossary:method object]]// that agrees with the new one on
//[[CL:Glossary:parameter specializers]]// and //[[CL:Glossary:qualifiers]]//, the existing //[[CL:Glossary:generic function]]//
//[[CL:Glossary:object]]// is modified to contain the new //[[CL:Glossary:method]]// //[[CL:Glossary:object]]//.

\itemitem{\bull} If the given //[[CL:Glossary:name]]// names an //[[CL:Glossary:ordinary function]]//, a //[[CL:Glossary:macro]]//,
or a //[[CL:Glossary:special operator]]//, an error is signaled.

\itemitem{\bull} Otherwise a //[[CL:Glossary:generic function]]// is created with the //[[CL:Glossary:method]]//
specified by the //[[CL:Glossary:method-defining form]]//.

\endlist

If the //[[CL:Glossary:lambda list]]// of a new //[[CL:Glossary:method]]// is not
//[[CL:Glossary:congruent]]// with the //[[CL:Glossary:lambda list]]// of the //[[CL:Glossary:generic function]]//,
an error is signaled.  If a //[[CL:Glossary:method-defining operator]]// that cannot specify
//[[CL:Glossary:generic function]]// options creates a new //[[CL:Glossary:generic function]]//, 
a //[[CL:Glossary:lambda list]]// for that //[[CL:Glossary:generic function]]// is derived from the
//[[CL:Glossary:lambda list]]// of the //[[CL:Glossary:method]]// in the //[[CL:Glossary:method-defining form]]// in such a way
as to be //[[CL:Glossary:congruent]]// with it.  For a discussion of //[[CL:Glossary:congruence]]//,
\seesection\GFMethodLambdaListCongruency.

Each method has a //[[CL:Glossary:specialized lambda list]]//, which determines
when that method can be applied.  A //[[CL:Glossary:specialized lambda list]]// is like
an //[[CL:Glossary:ordinary lambda list]]// except that a specialized parameter
may occur instead of the name of a required parameter.  A specialized parameter
is a list \f{(\i{variable-name} \i{parameter-specializer-name})},
where \i{parameter-specializer-name} is one of the following:

\beginlist

\itemitem{a //[[CL:Glossary:symbol]]//}

denotes a //[[CL:Glossary:parameter specializer]]// which is the //[[CL:Glossary:class]]// 
named by that //[[CL:Glossary:symbol]]//.

\issue{CLASS-OBJECT-SPECIALIZER:AFFIRM}


\itemitem{a //[[CL:Glossary:class]]//}

denotes a //[[CL:Glossary:parameter specializer]]// which is the //[[CL:Glossary:class]]// itself.


\itemitem{\f{(eql \i{form})}}

denotes a //[[CL:Glossary:parameter specializer]]// which satisfies the //[[CL:Glossary:type specifier]]//
\f{(eql \i{object})}, where \i{object} is the 
result of evaluating \i{form}.  The form \i{form} is evaluated in 
the lexical environment in which the method-defining form is evaluated.
Note that \i{form} is evaluated only once, at the time the method is
defined, not each time the generic function is called.
\endlist

//[[CL:Glossary:Parameter specializer names]]// are used in macros intended as the
user-level interface (\macref{defmethod}), while //[[CL:Glossary:parameter specializers]]//
are used in the functional interface.

Only required parameters may be specialized, and there must be a
//[[CL:Glossary:parameter specializer]]// for each required parameter.  For notational
simplicity, if some required parameter in a //[[CL:Glossary:specialized lambda list]]// in
a method-defining form is simply a variable name, its 
//[[CL:Glossary:parameter specializer]]// defaults to \theclass{t}.

Given a generic function and a set of arguments, an applicable
method is a method for that generic function whose parameter
specializers are satisfied by their corresponding arguments.  The
following definition specifies what it means for a method to be
applicable and for an argument to satisfy a //[[CL:Glossary:parameter specializer]]//.






Let $\langle A\sub 1, \ldots, A\sub n\rangle$ be the required
arguments to a generic function in order. Let $\langle P\sub 1,
\ldots, P\sub n\rangle$ be the //[[CL:Glossary:parameter specializers]]// corresponding to
the required parameters of the method $M$ in order.  The method $M$ is








applicable when each $A\sub i$ is of the //[[CL:Glossary:type]]// specified by 
the //[[CL:Glossary:type specifier]]// $P\sub i$.























Because every valid //[[CL:Glossary:parameter specializer]]// is 
also a valid //[[CL:Glossary:type specifier]]//, \thefunction{typep} can be used during method
selection to determine whether an argument satisfies a //[[CL:Glossary:parameter specializer]]//.  

A method all of whose //[[CL:Glossary:parameter specializers]]// are 
\theclass{t} is called a //[[CL:Glossary:default method]]//; it is always applicable but
may be shadowed by a more specific method.

Methods can have //[[CL:Glossary:qualifiers]]//, which give the method combination
procedure a way to distinguish among methods.  A method that has one
or more //[[CL:Glossary:qualifiers]]// is called a //[[CL:Glossary:qualified method]]//.
A method with no //[[CL:Glossary:qualifiers]]// is called an //[[CL:Glossary:unqualified method]]//. 
A //[[CL:Glossary:qualifier]]// is any //[[CL:Glossary:non-list]]//.



The //[[CL:Glossary:qualifiers]]// defined by the //[[CL:Glossary:standardized]]// method combination types 
are //[[CL:Glossary:symbols]]//.

In this specification, the terms ``//[[CL:Glossary:primary method]]//'' and 
``//[[CL:Glossary:auxiliary method]]//'' are used to partition //[[CL:Glossary:methods]]//
within a method combination type according to their intended use.  
In standard method combination, //[[CL:Glossary:primary methods]]// are 
//[[CL:Glossary:unqualified methods]]// 
and //[[CL:Glossary:auxiliary methods]]// are methods with a single //[[CL:Glossary:qualifier]]// 
that is one of **'':around''**, **'':before''**, or **'':after''**.
//[[CL:Glossary:Methods]]// with these //[[CL:Glossary:qualifiers]]// are called //[[CL:Glossary:around methods]]//,
//[[CL:Glossary:before methods]]//, and //[[CL:Glossary:after methods]]//, respectively.
When a method combination type is defined using the short form of
\macref{define-method-combination}, //[[CL:Glossary:primary methods]]// are 
methods qualified with the name of the type of method combination, 
and auxiliary methods have the //[[CL:Glossary:qualifier]]// **'':around''**.
Thus the terms ``//[[CL:Glossary:primary method]]//'' and ``//[[CL:Glossary:auxiliary method]]//''
have only a relative definition within a given method combination type.

\endSubsection%{Introduction to Methods}

\beginSubsection{Agreement on Parameter Specializers and Qualifiers}
\DefineSection{SpecializerQualifierAgreement}

Two //[[CL:Glossary:methods]]// are said to agree with each other on //[[CL:Glossary:parameter specializers]]//
and //[[CL:Glossary:qualifiers]]// if the following conditions hold:

\beginlist

\itemitem{1.} Both methods have the same number of required parameters.
Suppose the //[[CL:Glossary:parameter specializers]]// of the two methods are
$P\sub{1,1}\ldots P\sub{1,n}$ and $P\sub{2,1}\ldots P\sub{2,n}$.

\itemitem{2.} For each $1\leq i\leq n$, $P\sub{1,i}$ agrees with $P\sub{2,i}$.
The //[[CL:Glossary:parameter specializer]]// $P\sub{1,i}$ agrees with $P\sub{2,i}$ if
$P\sub{1,i}$ and $P\sub{2,i}$ are the same class or if 
$P\sub{1,i}=\hbox{{\tt(\b{eql} $\hbox{\i{object}}\sub 1$)}}$,
$P\sub{2,i}=\hbox{{\tt(\b{eql} $\hbox{\i{object}}\sub 2$)}}$, and
{\tt (\b{eql} $\hbox{\i{object}}\sub 1$ $\hbox{\i{object}}\sub 2$)}.
Otherwise $P\sub{1,i}$ and $P\sub{2,i}$ do not agree.


\itemitem{3.} The two //[[CL:Glossary:lists]]// of //[[CL:Glossary:qualifiers]]// are the //[[CL:Glossary:same]]// 
under **[[CL:Functions:equal]]**.




\endlist

\endSubsection%{Agreement on Parameter Specializers and Qualifiers}

\beginSubsection{Congruent Lambda-lists for all Methods of a Generic Function}
\DefineSection{GFMethodLambdaListCongruency}

These rules define the congruence of a set of //[[CL:Glossary:lambda lists]]//, including the
//[[CL:Glossary:lambda list]]// of each method for a given generic function and the
//[[CL:Glossary:lambda list]]// specified for the generic function itself, if given.

\beginlist

\itemitem{1.} Each //[[CL:Glossary:lambda list]]// must have the same number of required
parameters.

\itemitem{2.} Each //[[CL:Glossary:lambda list]]// must have the same number of optional
parameters.  Each method can supply its own default for an optional
parameter.
                                       
\itemitem{3.} If any //[[CL:Glossary:lambda list]]// mentions ''&rest'' or ''&key'', each
//[[CL:Glossary:lambda list]]// must mention one or both of them.
                                                        
\itemitem{4.} If the //[[CL:Glossary:generic function]]// //[[CL:Glossary:lambda list]]//
mentions ''&key'', each
method must accept all of the keyword names mentioned after ''&key'',
either by accepting them explicitly, by specifying ''&allow-other-keys'',
or by specifying ''&rest'' but not ''&key''.
Each method can accept additional keyword arguments of its own.  The
checking of the validity of keyword names is done in the generic
function, not in each method.

A method is invoked as if the keyword
argument pair whose name is **'':allow-other-keys''** and whose value
is //[[CL:Glossary:true]]// were supplied, though no such argument pair will be passed.

                      
\itemitem{5.} The use of ''&allow-other-keys'' need not be consistent
across //[[CL:Glossary:lambda lists]]//.  If ''&allow-other-keys'' is mentioned in 
the //[[CL:Glossary:lambda list]]// of any applicable //[[CL:Glossary:method]]// or of the //[[CL:Glossary:generic function]]//,
any keyword arguments may be mentioned in the call to the //[[CL:Glossary:generic function]]//.
                      
\itemitem{6.} The use of ''&aux'' need not be consistent across methods.

If a //[[CL:Glossary:method-defining operator]]// that cannot specify //[[CL:Glossary:generic function]]// options
creates a //[[CL:Glossary:generic function]]//, and if the //[[CL:Glossary:lambda list]]// for the method
mentions keyword arguments, the //[[CL:Glossary:lambda list]]// of the generic function
will mention ''&key'' (but no keyword arguments).

\endlist

\endSubsection%{Congruent Lambda-lists for All Methods of a Generic Function}



\beginSubsection{Keyword Arguments in Generic Functions and Methods}
\DefineSection{KwdArgsInGFsAndMeths}
                                                        
When a generic function or any of its methods mentions 
''&key'' in a //[[CL:Glossary:lambda list]]//, the specific set of keyword
arguments accepted by the generic function varies according to the
applicable methods.  The set of keyword arguments accepted by the
generic function for a particular call is the union of the keyword
arguments accepted by all applicable methods and the keyword arguments
mentioned after ''&key'' in the generic function definition,
if any.  A method that has ''&rest'' but not ''&key'' does not affect the
set of acceptable keyword arguments.  If
the //[[CL:Glossary:lambda list]]// of any applicable method or of the generic
function definition contains ''&allow-other-keys'', all
keyword arguments are accepted by the generic function.

The //[[CL:Glossary:lambda list]]// congruence rules require that each method
accept all of the keyword arguments mentioned after ''&key'' in the
generic function definition, by accepting them explicitly, by
specifying ''&allow-other-keys'', or by specifying ''&rest'' but
not ''&key''.  Each method can accept additional keyword arguments
of its own, in addition to the keyword arguments mentioned in the
generic function definition.

If a //[[CL:Glossary:generic function]]// is passed a keyword argument that no applicable
method accepts, an error should be signaled; \seesection\FuncallErrorChecking.

\beginsubsubsection{Examples of Keyword Arguments in Generic Functions and Methods}

For example, suppose there are two methods defined for {\tt width}
as follows:

\code
 (defmethod width ((c character-class) &key font) ...)
 
 (defmethod width ((p picture-class) &key pixel-size) ...)
\endcode

\noindent Assume that there are no other methods and no generic
function definition for {\tt width}. The evaluation of the
following form should signal an error because 
the keyword argument **'':pixel-size''** is not accepted by the applicable method.

\code
 (width (make-instance `character-class :char #\\Q) 
        :font 'baskerville :pixel-size 10)
\endcode

The evaluation of the following form should signal an error.

\code
 (width (make-instance `picture-class :glyph (glyph #\\Q)) 
        :font 'baskerville :pixel-size 10)
\endcode

The evaluation of the following form will not signal an error
if the class named {\tt character-picture-class} is a subclass of
both {\tt picture-class} and {\tt character-class}.

\code
 (width (make-instance `character-picture-class :char #\\Q)
        :font 'baskerville :pixel-size 10)
\endcode

\endsubsubsection%{Examples of Keyword Arguments in Generic Functions and Methods}

\endSubsection%{Keyword Arguments in Generic Functions and Methods}

\beginSubsection{Method Selection and Combination}
\DefineSection{MethodSelectionAndCombination}

When a //[[CL:Glossary:generic function]]// is called with particular arguments, it must
determine the code to execute.  This code is called the 
//[[CL:Glossary:effective method]]// for those //[[CL:Glossary:arguments]]//.
The //[[CL:Glossary:effective method]]// is a 
combination of the //[[CL:Glossary:applicable methods]]// in the //[[CL:Glossary:generic function]]//
that //[[CL:Glossary:calls]]// some or all of the //[[CL:Glossary:methods]]//.

\issue{CLOS-ERROR-CHECKING-ORDER:NO-APPLICABLE-METHOD-FIRST}



If a //[[CL:Glossary:generic function]]// is called and no //[[CL:Glossary:methods]]// are 
//[[CL:Glossary:applicable]]//, the //[[CL:Glossary:generic function]]// **[[CL:Functions:no-applicable-method]]**
is invoked, with the //[[CL:Glossary:results]]// from that call being used as the
//[[CL:Glossary:results]]// of the call to the original //[[CL:Glossary:generic function]]//.  Calling
**[[CL:Functions:no-applicable-method]]** takes precedence over checking for acceptable
keyword arguments; \seesection\KwdArgsInGFsAndMeths.


When the //[[CL:Glossary:effective method]]// has been determined,
it is invoked with the same //[[CL:Glossary:arguments]]// as were passed to the //[[CL:Glossary:generic function]]//.  
Whatever //[[CL:Glossary:values]]// it returns are returned as the //[[CL:Glossary:values]]//
of the //[[CL:Glossary:generic function]]//.

\beginsubsubsection{Determining the Effective Method}
\DefineSection{DeterminingtheEffectiveMethod}

The effective method is determined by the following three-step procedure:

\beginlist

\itemitem{1.}{Select the applicable methods.}

\itemitem{2.}{Sort the applicable methods by precedence order, putting
the most specific method first.}

\itemitem{3.}{Apply method combination to the sorted list of
applicable methods, producing the effective method.}

\endlist

\beginsubsubsubsection{Selecting the Applicable Methods}
\DefineSection{SelApplMeth}

This step is described in \secref\IntroToMethods.

\endsubsubsubsection%{Selecting the Applicable Methods}

\beginsubsubsubsection{Sorting the Applicable Methods by Precedence Order}

To compare the precedence of two methods, their //[[CL:Glossary:parameter specializers]]//
are examined in order.  The default examination order is from left to
right, but an alternative order may be specified by the 
**'':argument-precedence-order''** option to \macref{defgeneric} or to any of
the other operators that specify generic function options.

The corresponding //[[CL:Glossary:parameter specializers]]// from each method are
compared.  When a pair of //[[CL:Glossary:parameter specializers]]// agree, the next
pair are compared for agreement.  If all corresponding parameter
specializers agree, the two methods must have different
//[[CL:Glossary:qualifiers]]//; in this case, either method can be selected to precede the
other.  For information about agreement, \seesection\SpecializerQualifierAgreement.

If some corresponding //[[CL:Glossary:parameter specializers]]// do not agree, the first
pair of //[[CL:Glossary:parameter specializers]]// that do not agree determines the
precedence.  If both //[[CL:Glossary:parameter specializers]]// are classes, the more
specific of the two methods is the method whose //[[CL:Glossary:parameter specializer]]//
appears earlier in the //[[CL:Glossary:class precedence list]]// of the corresponding
argument.  Because of the way in which the set of applicable methods
is chosen, the //[[CL:Glossary:parameter specializers]]// are guaranteed to be present in
the class precedence list of the class of the argument.

If just one of a pair of corresponding //[[CL:Glossary:parameter specializers]]// is {\tt (eql \i{object})},
the //[[CL:Glossary:method]]// with that //[[CL:Glossary:parameter specializer]]// precedes the
other //[[CL:Glossary:method]]//.  If both //[[CL:Glossary:parameter specializers]]// are **[[CL:Functions:eql]]**
//[[CL:Glossary:expressions]]//, the
specializers must agree (otherwise the two //[[CL:Glossary:methods]]// would
not both have been applicable to this argument).

The resulting list of //[[CL:Glossary:applicable methods]]// has the most specific
//[[CL:Glossary:method]]// first and the least specific //[[CL:Glossary:method]]// last.    

\endsubsubsubsection%{Sorting the Applicable Methods by Precedence Order}

\beginsubsubsubsection{Applying method combination to the sorted list of applicable methods}
\DefineSection{ApplyMethCombToSortedMethods}

In the simple case---if standard method combination is used and all
applicable methods are primary methods---the 

effective method is the most specific method.
That method can call the next most specific
method by using \thefunction{call-next-method}.  The method that
**[[CL:Functions:call-next-method]]** will call is referred to as the 
//[[CL:Glossary:next method]]//.  The predicate **[[CL:Functions:next-method-p]]** tests whether a next
method exists.  If **[[CL:Functions:call-next-method]]** is called and there is no
next most specific method, the generic function **[[CL:Functions:no-next-method]]**
is invoked.

In general, the effective method is some combination of the applicable
methods.  It is described by a //[[CL:Glossary:form]]// that contains calls to some or
all of the applicable methods, returns the value or values that will
be returned as the value or values of the generic function, and
optionally makes some of the methods accessible by means of 
**[[CL:Functions:call-next-method]]**.





The role of each method in the effective method is determined by its

//[[CL:Glossary:qualifiers]]// and the specificity of the method.  A //[[CL:Glossary:qualifier]]//
serves to mark a method, and the meaning of a //[[CL:Glossary:qualifier]]// is
determined by the way that these marks are used by this step
of the procedure.  If an applicable method has an unrecognized
//[[CL:Glossary:qualifier]]//, this step signals an error and does not include that method
in the effective method.

When standard method combination is used together with qualified methods, 
the effective method is produced as described in \secref\StdMethComb.
                                                                  
Another type of method combination can be specified by using the
**'':method-combination''** option of \macref{defgeneric} or
of any of the other operators that specify generic function options.  In
this way this step of the procedure can be customized.

New types of method combination can be defined by using 
\themacro{define-method-combination}. 

\endsubsubsubsection%{Applying method combination to the sorted list of applicable methods}

\endsubsubsection%{Determining the Effective Method}

\beginsubsubsection{Standard Method Combination}
\DefineSection{StdMethComb}
\idxref{standard}
                                                       

Standard method combination is supported by \theclass{standard-generic-function}.
It is used if no other type of method
combination is specified or if the built-in method combination type
\misc{standard} is specified. 

Primary methods define the main action of the effective method,  
while auxiliary methods modify that action in one of three ways.
A primary method has no method //[[CL:Glossary:qualifiers]]//.
                                                           
An auxiliary method is a method whose 

//[[CL:Glossary:qualifier]]// is **'':before''**, **'':after''**, or **'':around''**.
Standard method combination
allows no more than one //[[CL:Glossary:qualifier]]// per method; if a method definition
specifies more than one //[[CL:Glossary:qualifier]]// per method, an error is signaled.

\beginlist

\itemitem{\bull}
A //[[CL:Glossary:before method]]// has the keyword **'':before''** as its only //[[CL:Glossary:qualifier]]//.
A //[[CL:Glossary:before method]]// specifies //[[CL:Glossary:code]]// that is to be run before any 
//[[CL:Glossary:primary methods]]//.

\itemitem{\bull}
An //[[CL:Glossary:after method]]// has the keyword **'':after''** as its only //[[CL:Glossary:qualifier]]//.
An //[[CL:Glossary:after method]]// specifies //[[CL:Glossary:code]]// that is to be run after
//[[CL:Glossary:primary methods]]//.

\itemitem{\bull}
An //[[CL:Glossary:around method]]// has the keyword **'':around''** as its only //[[CL:Glossary:qualifier]]//.
An //[[CL:Glossary:around method]]// specifies //[[CL:Glossary:code]]// that is to be run instead of other
//[[CL:Glossary:applicable methods]]//,


but which might contain explicit //[[CL:Glossary:code]]//
which calls some of those //[[CL:Glossary:shadowed]]// //[[CL:Glossary:methods]]//
(via **[[CL:Functions:call-next-method]]**).

\endlist

The semantics of standard method combination is as follows:

\beginlist
                               
\itemitem{\bull} If there are any //[[CL:Glossary:around methods]]//, the most specific
//[[CL:Glossary:around method]]// is called.  It supplies the value or values of the
generic function.

\itemitem{\bull} Inside the body of an //[[CL:Glossary:around method]]//, 
**[[CL:Functions:call-next-method]]** can be used to call the //[[CL:Glossary:next method]]//.  When the next
method returns, the //[[CL:Glossary:around method]]// can execute more code,
perhaps based on the returned value or values.



\TheGF{no-next-method} is invoked if **[[CL:Functions:call-next-method]]** is used and
there is no //[[CL:Glossary:applicable method]]// to call.  \Thefunction{next-method-p}
may be used to determine whether a //[[CL:Glossary:next method]]// exists.

\itemitem{\bull} 
If an //[[CL:Glossary:around method]]// invokes **[[CL:Functions:call-next-method]]**,
the next most specific //[[CL:Glossary:around method]]//
is called, if one is applicable.  If there are no //[[CL:Glossary:around methods]]// 
or if **[[CL:Functions:call-next-method]]** is called by the least
specific //[[CL:Glossary:around method]]//, the other methods are called as
follows:
\beginlist
\itemitem{--} All the //[[CL:Glossary:before methods]]// are called, in
most-specific-first order.  Their values are ignored.
An error is signaled if **[[CL:Functions:call-next-method]]** is used in a
//[[CL:Glossary:before method]]//.

\itemitem{--} The most specific primary method is called.  Inside the
body of a primary method, **[[CL:Functions:call-next-method]]** may be used to call
the next most specific primary method.  When that method returns, the
previous primary method can execute more code, perhaps based on the
returned value or values.  The generic function **[[CL:Functions:no-next-method]]**
is invoked if **[[CL:Functions:call-next-method]]** is used and there are no more
applicable primary methods.  \Thefunction{next-method-p} may be
used to determine whether a //[[CL:Glossary:next method]]// exists.  If **[[CL:Functions:call-next-method]]**
is not used, only the most specific //[[CL:Glossary:primary method]]// is called.

\itemitem{--} All the //[[CL:Glossary:after methods]]// are called in
most-specific-last order.  Their values are ignored.
An error is signaled if **[[CL:Functions:call-next-method]]** is used in an
//[[CL:Glossary:after method]]//.
\endlist
\itemitem{\bull} If no //[[CL:Glossary:around methods]]// were invoked, the most
specific primary method supplies the value or values returned by the
generic function.  The value or values returned by the invocation of
**[[CL:Functions:call-next-method]]** in the least specific //[[CL:Glossary:around method]]// are
those returned by the most specific primary method.

\endlist

In standard method combination, if there is an applicable method
but no applicable primary method, an error is signaled.
     
The //[[CL:Glossary:before methods]]// are run in most-specific-first order while
the //[[CL:Glossary:after methods]]// are run in least-specific-first order.  The
design rationale for this difference can be illustrated with an
example.  Suppose class $C\sub 1$ modifies the behavior of its
superclass, $C\sub 2$, by adding //[[CL:Glossary:before methods]]// and //[[CL:Glossary:after methods]]//.
Whether the behavior of the class $C\sub 2$ is defined
directly by methods on $C\sub 2$ or is inherited from its superclasses
does not affect the relative order of invocation of methods on
instances of the class $C\sub 1$.  Class $C\sub 1$'s 
//[[CL:Glossary:before method]]// runs before all of class $C\sub 2$'s methods.  
Class $C\sub 1$'s //[[CL:Glossary:after method]]// runs after all of class $C\sub 2$'s methods.

By contrast, all //[[CL:Glossary:around methods]]// run before any other methods
run.  Thus a less specific //[[CL:Glossary:around method]]// runs before a more
specific primary method.

If only primary methods are used and if **[[CL:Functions:call-next-method]]** is not
used, only the most specific method is invoked; that is, more specific
methods shadow more general ones. 

\endsubsubsection%{Standard Method Combination}

\beginsubsubsection{Declarative Method Combination}
           
The macro \macref{define-method-combination} defines new forms of method
combination.  It provides a mechanism for customizing the production
of the effective method. The default procedure for producing an
effective method is described in \secref\DeterminingtheEffectiveMethod.
There are two forms of
\macref{define-method-combination}.  The short form is a simple facility while
the long form is more powerful and more verbose.  The long form
resembles \macref{defmacro} in that the body is an expression that
computes a Lisp form; it provides mechanisms for implementing
arbitrary control structures within method combination and for
arbitrary processing of method //[[CL:Glossary:qualifiers]]//.  

\endsubsubsection%{Declarative Method Combination}

\beginsubsubsection{Built-in Method Combination Types}
\DefineSection{BuiltInMethCombTypes}

The \CLOS\ provides a set of built-in method combination types.  To
specify that a generic function is to use one of these method
combination types, the name of the method combination type is given as
the argument to the **'':method-combination''** option to 
\macref{defgeneric} or to the **'':method-combination''** option to any of the
other operators that specify generic function options.

The names of the built-in  method combination types are listed in \thenextfigure.
\idxref{+}%
\idxref{and}%
\idxref{append}%
\idxref{list}%
\idxref{max}%
\idxref{min}%
\idxref{nconc}%
\idxref{or}%
\idxref{progn}%
\idxref{standard}

\displayfive{Built-in Method Combination Types}{
+&append&max&nconc&progn\cr
and&list&min&or&standard\cr
}

The semantics of the \misc{standard} built-in method combination type is
described in \secref\StdMethComb.  The other
built-in method combination types are called simple built-in method
combination types.

The simple built-in method combination types act as though they were
defined by the short form of \macref{define-method-combination}.  
They recognize two roles for //[[CL:Glossary:methods]]//:

\beginlist
                                                                  
\itemitem{\bull} An //[[CL:Glossary:around method]]// has the keyword symbol 
**'':around''** as its sole //[[CL:Glossary:qualifier]]//.  The meaning of 
**'':around''** //[[CL:Glossary:methods]]// is the same as in standard method combination.
Use of the functions **[[CL:Functions:call-next-method]]** and **[[CL:Functions:next-method-p]]**
is supported in //[[CL:Glossary:around methods]]//.

\itemitem{\bull} A primary method has the name of the method combination
type as its sole //[[CL:Glossary:qualifier]]//.  For example, the built-in method
combination type {\tt and} recognizes methods whose sole //[[CL:Glossary:qualifier]]// is
{\tt and}; these are primary methods. Use of the functions 
**[[CL:Functions:call-next-method]]** and **[[CL:Functions:next-method-p]]** is not supported 
in //[[CL:Glossary:primary methods]]//.

\endlist

The semantics of the simple built-in method combination types is as
follows:

\beginlist
\itemitem{\bull}                                                    
If there are any //[[CL:Glossary:around methods]]//, the most specific //[[CL:Glossary:around method]]//
is called.   It supplies the value or values of the //[[CL:Glossary:generic function]]//. 
                                    
\itemitem{\bull} Inside the body of an //[[CL:Glossary:around method]]//, the function
**[[CL:Functions:call-next-method]]** can be used to call the //[[CL:Glossary:next method]]//.



\TheGF{no-next-method} is invoked if 
**[[CL:Functions:call-next-method]]** is used and there is no applicable method to call.
\Thefunction{next-method-p} may be used to determine whether a
//[[CL:Glossary:next method]]// exists. When the //[[CL:Glossary:next method]]// returns, 
the //[[CL:Glossary:around method]]// can execute more code,
perhaps based on the returned value or values.
                    
\itemitem{\bull} If an //[[CL:Glossary:around method]]// invokes **[[CL:Functions:call-next-method]]**,
the next most specific //[[CL:Glossary:around method]]// is
called, if one is applicable.  If there are no //[[CL:Glossary:around methods]]//
or if **[[CL:Functions:call-next-method]]** is called by the least specific
//[[CL:Glossary:around method]]//, a Lisp form derived from the name of the built-in
method combination type and from the list of applicable primary
methods is evaluated to produce the value of the generic function.
Suppose the name of the method combination type is \i{operator}
and the call to the generic function is of the form

$$(\hbox{\i{generic-function}}\ a\sub 1\ldots a\sub n)$$

\itemitem{} Let $M\sub 1,\ldots,M\sub k$ be the applicable primary methods
in order; then the derived Lisp form is

$$(\hbox{\i{operator}}\ \langle M\sub  1%
\ a\sub 1\ldots a\sub n\rangle\ldots\langle%
M\sub k\ a\sub 1\ldots a\sub n\rangle)$$

\itemitem{} If the expression $\langle M\sub i \ a\sub 1\ldots a\sub
n\rangle$ is
evaluated, the method $M\sub i$ will be applied to the arguments
$a\sub 1\ldots a\sub n$.  
For example,
if \i{operator} is {\tt or},
the expression $\langle M\sub i \ a\sub 1\ldots a\sub n\rangle$ is
evaluated only if $\langle M\sub j \ a\sub 1\ldots a\sub n\rangle$,
$1\leq j<i$, returned {\tt nil}.
                                                      
\itemitem{} The default order for the primary methods is 
**'':most-specific-first''**.  However, the order can be reversed by supplying
**'':most-specific-last''** as the second argument to the **'':method-combination''** option.
\endlist

The simple built-in method combination types require exactly one
//[[CL:Glossary:qualifier]]// per method.  An error is signaled if there are applicable
methods with no //[[CL:Glossary:qualifiers]]// or with //[[CL:Glossary:qualifiers]]// that are not supported
by the method combination type. An error is signaled if there are
applicable //[[CL:Glossary:around methods]]// and no applicable primary
methods.

\endsubsubsection%{Built-in Method Combination Types}

\endSubsection%{Method Selection and Combination}

\beginSubsection{Inheritance of Methods}
\DefineSection{MethodInheritance}

A subclass inherits methods in the sense that any method applicable to
all instances of a class is also applicable to all instances of any
subclass of that class.

The inheritance of methods acts the same way regardless of 
which of the //[[CL:Glossary:method-defining operators]]// created the methods.





The inheritance of methods is described in detail in 
\secref\MethodSelectionAndCombination.

\endSubsection%{Inheritance of Methods}
