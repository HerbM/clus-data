

\beginsubSection{Introduction to Characters} \DefineSection{IntroToChars}

A //[[CL:Glossary:character]]// is an //[[CL:Glossary:object]]// that represents a unitary token  (\eg a letter, a special symbol, or a ``control character'') in an aggregate quantity of text (\eg a //[[CL:Glossary:string]]// or a text //[[CL:Glossary:stream]]//).

\issue{CHARACTER-PROPOSAL:2-6-1} \clisp\ allows an implementation to provide support  for international language //[[CL:Glossary:characters]]// as well as //[[CL:Glossary:characters]]// used in specialized arenas (\eg mathematics).

The following figures contain lists of //[[CL:Glossary:defined names]]// applicable to  //[[CL:Glossary:characters]]//.

\Thenextfigure\ lists some //[[CL:Glossary:defined names]]// relating to  //[[CL:Glossary:character]]// //[[CL:Glossary:attributes]]// and //[[CL:Glossary:character]]// //[[CL:Glossary:predicates]]//.

\displaythree{Character defined names -- 1}{ alpha-char-p&char-not-equal&char>\cr alphanumericp&char-not-greaterp&char>=\cr both-case-p&char-not-lessp&digit-char-p\cr char-code-limit&char/=&graphic-char-p\cr char-equal&char<&lower-case-p\cr char-greaterp&char<=&standard-char-p\cr char-lessp&char=&upper-case-p\cr }
                            \Thenextfigure\ lists some //[[CL:Glossary:character]]// construction and conversion //[[CL:Glossary:defined names]]//.

\displaythree{Character defined names -- 2}{ char-code&char-name&code-char\cr char-downcase&char-upcase&digit-char\cr char-int&character&name-char\cr }

\endsubSection%{Introduction to Characters}

\beginsubSection{Introduction to Scripts and Repertoires}

\beginsubsubsection{Character Scripts} \DefineSection{CharScripts}

A //[[CL:Glossary:script]]// is one of possibly several sets that form an //[[CL:Glossary:exhaustive partition]]// of the type \typeref{character}.

The number of such sets and boundaries between them is //[[CL:Glossary:implementation-defined]]//. \clisp\ does not require these sets to be //[[CL:Glossary:types]]//, but an //[[CL:Glossary:implementation]]// is permitted to define such //[[CL:Glossary:types]]// as an extension.  Since no //[[CL:Glossary:character]]// from one //[[CL:Glossary:script]]// can ever be a member of another //[[CL:Glossary:script]]//, it is generally more useful to speak about //[[CL:Glossary:character]]// //[[CL:Glossary:repertoires]]//.

\issue{LINDEN-COMMENTS-ON-CHARACTERS:X3J13-APR-92}

Although the term ``//[[CL:Glossary:script]]//'' is chosen for 

definitional  compatibility with ISO terminology, no //[[CL:Glossary:conforming implementation]]//  is required to use any particular //[[CL:Glossary:scripts]]// standardized by ISO or by any other standards organization.

\issue{CHARACTER-PROPOSAL:2-4-1} Whether and how the //[[CL:Glossary:script]]// or //[[CL:Glossary:scripts]]// used by any given //[[CL:Glossary:implementation]]// are named is //[[CL:Glossary:implementation-dependent]]//.

\endsubsubsection%{Character Scripts}

\beginsubsubsection{Character Repertoires} \DefineSection{CharRepertoires}

\issue{CHARACTER-PROPOSAL:2-4-3} A //[[CL:Glossary:repertoire]]// is a //[[CL:Glossary:type specifier]]// for a \subtypeof{character}.

This term is generally used when describing a collection of //[[CL:Glossary:characters]]// independent of their coding. //[[CL:Glossary:Characters]]// in //[[CL:Glossary:repertoires]]// are only identified
    by name,
    by //[[CL:Glossary:glyph]]//, or
    by character description.

A //[[CL:Glossary:repertoire]]// can contain //[[CL:Glossary:characters]]// from several //[[CL:Glossary:scripts]]//, and a //[[CL:Glossary:character]]// can appear in more than one //[[CL:Glossary:repertoire]]//.

\issue{LINDEN-COMMENTS-ON-CHARACTERS:X3J13-APR-92} For some examples of //[[CL:Glossary:repertoires]]//, see the coded character standards ISO 8859/1, ISO 8859/2, and ISO 6937/2. Note, however, that although

the term ``//[[CL:Glossary:repertoire]]//'' is chosen for 

definitional  compatibility with ISO terminology, no //[[CL:Glossary:conforming implementation]]//  is required to use //[[CL:Glossary:repertoires]]// standardized by ISO or any other  standards organization.

\endsubsubsection%{Character Repertoires}

\endsubSection%{Introduction to Repertoires and Scripts}

\beginsubSection{Character Attributes} \DefineSection{CharacterAttributes}

\issue{CHARACTER-PROPOSAL:2-1-1} //[[CL:Glossary:Characters]]// have only one //[[CL:Glossary:standardized]]// //[[CL:Glossary:attribute]]//: a //[[CL:Glossary:code]]//.  A //[[CL:Glossary:character]]//'s //[[CL:Glossary:code]]// is a non-negative //[[CL:Glossary:integer]]//. This //[[CL:Glossary:code]]// is composed from a character //[[CL:Glossary:script]]// and a character label in an //[[CL:Glossary:implementation-dependent]]// way.  \Seefuns{char-code} and **[[CL:Functions:code-char]]**.

\issue{CHARACTER-PROPOSAL:2-1-1}

Additional, //[[CL:Glossary:implementation-defined]]// //[[CL:Glossary:attributes]]// of //[[CL:Glossary:characters]]// are also permitted so that, for example,  two //[[CL:Glossary:characters]]// with the same //[[CL:Glossary:code]]// may differ  in some other, //[[CL:Glossary:implementation-defined]]// way.

For any //[[CL:Glossary:implementation-defined]]// //[[CL:Glossary:attribute]]// there is a distinguished value called the //[[CL:Glossary:null]]// value for that //[[CL:Glossary:attribute]]//.  A //[[CL:Glossary:character]]// for which each //[[CL:Glossary:implementation-defined]]// //[[CL:Glossary:attribute]]// has the null value for that //[[CL:Glossary:attribute]]// is called a //[[CL:Glossary:simple]]// //[[CL:Glossary:character]]//. If the //[[CL:Glossary:implementation]]// has no //[[CL:Glossary:implementation-defined]]// //[[CL:Glossary:attributes]]//, then all //[[CL:Glossary:characters]]// are //[[CL:Glossary:simple]]// //[[CL:Glossary:characters]]//.

\endSubsection%{Character Attributes}

\beginSubsection{Character Categories}

There are several (overlapping) categories of //[[CL:Glossary:characters]]// that have no formally associated //[[CL:Glossary:type]]// but that are nevertheless useful to name.  They include
     //[[CL:Glossary:graphic]]// //[[CL:Glossary:characters]]//, 
     //[[CL:Glossary:alphabetic]]//\meaning{1} //[[CL:Glossary:characters]]//,
     //[[CL:Glossary:characters]]// with //[[CL:Glossary:case]]// 
       (//[[CL:Glossary:uppercase]]// and //[[CL:Glossary:lowercase]]// //[[CL:Glossary:characters]]//),
     //[[CL:Glossary:numeric]]// //[[CL:Glossary:characters]]//,
     //[[CL:Glossary:alphanumeric]]// //[[CL:Glossary:characters]]//,
 and //[[CL:Glossary:digits]]// (in a given //[[CL:Glossary:radix]]//).

For each //[[CL:Glossary:implementation-defined]]// //[[CL:Glossary:attribute]]// of a //[[CL:Glossary:character]]//, the documentation for that //[[CL:Glossary:implementation]]// must specify whether  //[[CL:Glossary:characters]]// that differ only in that //[[CL:Glossary:attribute]]// are permitted to differ in whether are not they are members of one of the aforementioned categories.

Note that these terms are defined independently of any special syntax  which might have been enabled in the //[[CL:Glossary:current readtable]]//.

\beginsubsubsection{Graphic Characters} \DefineSection{GraphicChars}

\issue{CHARACTER-PROPOSAL:2-1-1} //[[CL:Glossary:Characters]]// that are classified as //[[CL:Glossary:graphic]]//, or displayable, are each associated with a glyph, a visual representation of the //[[CL:Glossary:character]]//.

A //[[CL:Glossary:graphic]]// //[[CL:Glossary:character]]// is one that has a standard textual  representation as a single //[[CL:Glossary:glyph]]//, such as \f{A} or \f{*} or \f{=}. //[[CL:Glossary:Space]]//, which effectively has a blank //[[CL:Glossary:glyph]]//, is defined to be a //[[CL:Glossary:graphic]]//.

Of the //[[CL:Glossary:standard characters]]//,
     //[[CL:Glossary:newline]]// is //[[CL:Glossary:non-graphic]]// 
 and all others are //[[CL:Glossary:graphic]]//; \seesection\StandardChars.

\issue{CHARACTER-PROPOSAL:2-1-1} //[[CL:Glossary:Characters]]// that are not //[[CL:Glossary:graphic]]// are called //[[CL:Glossary:non-graphic]]//.

//[[CL:Glossary:Non-graphic]]// //[[CL:Glossary:characters]]// are sometimes informally called
    ``formatting characters'' 
 or ``control characters.''

\f{\#\\Backspace}, \f{\#\\Tab}, \f{\#\\Rubout}, \f{\#\\Linefeed},  \f{\#\\Return}, and \f{\#\\Page}, if they are supported by the //[[CL:Glossary:implementation]]//, are //[[CL:Glossary:non-graphic]]//.

\endsubsubsection%{Graphic Characters}

\beginsubsubsection{Alphabetic Characters}

The //[[CL:Glossary:alphabetic]]//\meaning{1} //[[CL:Glossary:characters]]// are a subset of the //[[CL:Glossary:graphic]]// //[[CL:Glossary:characters]]//. Of the //[[CL:Glossary:standard characters]]//, only these are the //[[CL:Glossary:alphabetic]]//\meaning{1} //[[CL:Glossary:characters]]//:

\f{A B C D E F G H I J K L M N O P Q R S T U V W X Y Z}

\f{a b c d e f g h i j k l m n o p q r s t u v w x y z}

Any //[[CL:Glossary:implementation-defined]]// //[[CL:Glossary:character]]// that has //[[CL:Glossary:case]]//  must be //[[CL:Glossary:alphabetic]]//\meaning{1}. For each //[[CL:Glossary:implementation-defined]]// //[[CL:Glossary:graphic]]// //[[CL:Glossary:character]]//  that has no //[[CL:Glossary:case]]//, it is //[[CL:Glossary:implementation-defined]]// whether  that //[[CL:Glossary:character]]// is //[[CL:Glossary:alphabetic]]//\meaning{1}.

\endsubsubsection%{Alphabetic Characters}

\beginsubsubsection{Characters With Case} \DefineSection{CharactersWithCase}

The //[[CL:Glossary:characters]]// with //[[CL:Glossary:case]]// are  a subset of the //[[CL:Glossary:alphabetic]]//\meaning{1} //[[CL:Glossary:characters]]//. A //[[CL:Glossary:character]]// with //[[CL:Glossary:case]]// has the property of being either //[[CL:Glossary:uppercase]]// or //[[CL:Glossary:lowercase]]//. Every //[[CL:Glossary:character]]// with //[[CL:Glossary:case]]// is in one-to-one correspondence with some other //[[CL:Glossary:character]]// with the opposite //[[CL:Glossary:case]]//.

\beginsubsubsubsection{Uppercase Characters}

An uppercase //[[CL:Glossary:character]]// is one that has a corresponding //[[CL:Glossary:lowercase]]// //[[CL:Glossary:character]]// that is //[[CL:Glossary:different]]//  (and can be obtained using **[[CL:Functions:char-downcase]]**).

Of the //[[CL:Glossary:standard characters]]//, only these are //[[CL:Glossary:uppercase]]// //[[CL:Glossary:characters]]//:

\f{A B C D E F G H I J K L M N O P Q R S T U V W X Y Z}

\endsubsubsubsection%{Uppercase Characters}

\beginsubsubsubsection{Lowercase Characters}

A lowercase //[[CL:Glossary:character]]// is one that has a corresponding  //[[CL:Glossary:uppercase]]// //[[CL:Glossary:character]]// that is //[[CL:Glossary:different]]//  (and can be obtained using **[[CL:Functions:char-upcase]]**).

Of the //[[CL:Glossary:standard characters]]//, only these are //[[CL:Glossary:lowercase]]// //[[CL:Glossary:characters]]//:

\f{a b c d e f g h i j k l m n o p q r s t u v w x y z}

\endsubsubsubsection%{Lowercase Characters}

\beginsubsubsubsection{Corresponding Characters in the Other Case}

The //[[CL:Glossary:uppercase]]// //[[CL:Glossary:standard characters]]// \f{A} through \f{Z} mentioned above respectively correspond to the //[[CL:Glossary:lowercase]]// //[[CL:Glossary:standard characters]]// \f{a} through \f{z} mentioned above. For example, the //[[CL:Glossary:uppercase]]// //[[CL:Glossary:character]]// \f{E}  corresponds to the //[[CL:Glossary:lowercase]]// //[[CL:Glossary:character]]// \f{e}, and vice versa.

\endsubsubsubsection%{Corresponding Characters in the Other Case}

\beginsubsubsubsection{Case of Implementation-Defined Characters}

An //[[CL:Glossary:implementation]]// may define that other //[[CL:Glossary:implementation-defined]]// //[[CL:Glossary:graphic]]// //[[CL:Glossary:characters]]// have //[[CL:Glossary:case]]//.  Such definitions must always be done in pairs---one //[[CL:Glossary:uppercase]]// //[[CL:Glossary:character]]// in one-to-one  //[[CL:Glossary:correspondence]]// with one //[[CL:Glossary:lowercase]]// //[[CL:Glossary:character]]//.

\endsubsubsubsection%{Case of Implementation-Defined Characters}

\endsubsubsection%{Characters With Case}

\beginsubsubsection{Numeric Characters}

The //[[CL:Glossary:numeric]]// //[[CL:Glossary:characters]]// are a subset of the //[[CL:Glossary:graphic]]// //[[CL:Glossary:characters]]//. Of the //[[CL:Glossary:standard characters]]//, only these are //[[CL:Glossary:numeric]]// //[[CL:Glossary:characters]]//:

\f{0 1 2 3 4 5 6 7 8 9}

For each //[[CL:Glossary:implementation-defined]]// //[[CL:Glossary:graphic]]// //[[CL:Glossary:character]]//  that has no //[[CL:Glossary:case]]//, the //[[CL:Glossary:implementation]]// must define whether or not it is a //[[CL:Glossary:numeric]]// //[[CL:Glossary:character]]//.

\endsubsubsection%{Numeric Characters}

\beginsubsubsection{Alphanumeric Characters}

The set of //[[CL:Glossary:alphanumeric]]// //[[CL:Glossary:characters]]// is the union of 
    the set of //[[CL:Glossary:alphabetic]]//\meaning{1} //[[CL:Glossary:characters]]//  and the set of //[[CL:Glossary:numeric]]// //[[CL:Glossary:characters]]//.

\endsubsubsection%{Alphanumeric Characters}

\beginsubsubsection{Digits in a Radix} \DefineSection{Digits}

What qualifies as a //[[CL:Glossary:digit]]// depends on the //[[CL:Glossary:radix]]//  (an //[[CL:Glossary:integer]]// between \f{2} and \f{36}, inclusive). The potential //[[CL:Glossary:digits]]// are:

\f{0 1 2 3 4 5 6 7 8 9 A B C D E F G H I J K L M N O P Q R S T U V W X Y Z}

Their respective weights are \f{0}, \f{1}, \f{2}, $\ldots$ \f{35}. In any given radix $n$, only the first $n$ potential //[[CL:Glossary:digits]]//  are considered to be //[[CL:Glossary:digits]]//. For example, the digits in radix \f{2}  are \f{0} and \f{1},  the digits in radix \f{10} are \f{0} through \f{9}, and the digits in radix \f{16} are \f{0} through \f{F}.

//[[CL:Glossary:Case]]// is not significant in //[[CL:Glossary:digits]]//;  for example, in radix \f{16}, both \f{F} and \f{f}  are //[[CL:Glossary:digits]]// with weight \f{15}.

\endsubsubsection%{Digits in a Radix}

\endsubSection%{Character Categories}

\beginsubSection{Identity of Characters}

Two //[[CL:Glossary:characters]]// that are **[[CL:Functions:eql]]**, **[[CL:Functions:char=]]**, or **[[CL:Functions:char-equal]]**  are not necessarily **[[CL:Functions:eq]]**.

\endsubSection%{Identity of Characters}

\beginsubSection{Ordering of Characters}

The total ordering on //[[CL:Glossary:characters]]// is guaranteed to have  the following properties: 

\beginlist

\issue{CHARACTER-PROPOSAL:2-1-1}

\itemitem{\bull}  If two //[[CL:Glossary:characters]]// have the same //[[CL:Glossary:implementation-defined]]// //[[CL:Glossary:attributes]]//, then their ordering by **[[CL:Functions:char<]]** is consistent with the numerical ordering by the predicate **[[CL:Functions:<]]** on their code //[[CL:Glossary:attributes]]//.

\itemitem{\bull} If two //[[CL:Glossary:characters]]// differ in any //[[CL:Glossary:attribute]]//, then they

are not **[[CL:Functions:char=]]**.

\reviewer{Barmar: I wonder if we should say that the ordering may be dependent on the 	          //[[CL:Glossary:implementation-defined]]// //[[CL:Glossary:attributes]]//.}

\itemitem{\bull}
  The total ordering is not necessarily the same as the total ordering
  on the //[[CL:Glossary:integers]]// produced by applying **[[CL:Functions:char-int]]** to the
  //[[CL:Glossary:characters]]//.

\issue{LINDEN-COMMENTS-ON-CHARACTERS:X3J13-APR-92} \itemitem{\bull} 
  While //[[CL:Glossary:alphabetic]]//\meaning{1} //[[CL:Glossary:standard characters]]// of a given //[[CL:Glossary:case]]//
  must    

  obey a partial ordering,
  they need not be contiguous; it is permissible for 
  //[[CL:Glossary:uppercase]]// and //[[CL:Glossary:lowercase]]// //[[CL:Glossary:characters]]// to be interleaved. 
  Thus \f{(char<= \#\\a x \#\\z)} 
  is not a valid way of determining whether or not \f{x} is a
  //[[CL:Glossary:lowercase]]// //[[CL:Glossary:character]]//.  

\issue{CHARACTER-PROPOSAL:2-1-1}

\endlist

\issue{LINDEN-COMMENTS-ON-CHARACTERS:X3J13-APR-92}

Of the //[[CL:Glossary:standard characters]]//,  those which are //[[CL:Glossary:alphanumeric]]// obey the following partial ordering:

\code
 A<B<C<D<E<F<G<H<I<J<K<L<M<N<O<P<Q<R<S<T<U<V<W<X<Y<Z
 a<b<c<d<e<f<g<h<i<j<k<l<m<n<o<p<q<r<s<t<u<v<w<x<y<z
 0<1<2<3<4<5<6<7<8<9
 either 9<A or Z<0
 either 9<a or z<0                                                       \endcode \issue{LINDEN-COMMENTS-ON-CHARACTERS:X3J13-APR-92} This implies that, for //[[CL:Glossary:standard characters]]//, //[[CL:Glossary:alphabetic]]//\meaning{1}  ordering holds within each //[[CL:Glossary:case]]// (//[[CL:Glossary:uppercase]]// and //[[CL:Glossary:lowercase]]//),  and that the //[[CL:Glossary:numeric]]// //[[CL:Glossary:characters]]// as a group are not interleaved with //[[CL:Glossary:alphabetic]]// //[[CL:Glossary:characters]]//. However, the ordering or possible interleaving of //[[CL:Glossary:uppercase]]// //[[CL:Glossary:characters]]// and //[[CL:Glossary:lowercase]]// //[[CL:Glossary:characters]]// is //[[CL:Glossary:implementation-defined]]//.

\endsubSection%{Ordering of Characters}

\beginsubsection{Character Names} \DefineSection{CharacterNames}

The following //[[CL:Glossary:character]]// //[[CL:Glossary:names]]// must be present in all  //[[CL:Glossary:conforming implementations]]//:

\beginlist \itemitem{\f{Newline}}

The character that represents the division between lines. An implementation must translate between \f{\#\\Newline},  a single-character representation, and whatever external representation(s) may be used.

\itemitem{\f{Space}}

The space or blank character. \endlist

The following names are //[[CL:Glossary:semi-standard]]//;  if an //[[CL:Glossary:implementation]]// supports them, they should be used for the described //[[CL:Glossary:characters]]// and no others.

\beginlist          \itemitem{\f{Rubout}}

The rubout or delete character.

\itemitem{\f{Page}}

The form-feed or page-separator character.

\itemitem{\f{Tab}}

The tabulate character.

\itemitem{\f{Backspace}}

The backspace character.

\itemitem{\f{Return}}

The carriage return character.

\itemitem{\f{Linefeed}}

The line-feed character. \endlist

In some //[[CL:Glossary:implementations]]//, one or more of these //[[CL:Glossary:character]]// //[[CL:Glossary:names]]//  might denote a //[[CL:Glossary:standard character]]//;  for example, \f{\#\\Linefeed} and \f{\#\\Newline} might be the //[[CL:Glossary:same]]// //[[CL:Glossary:character]]// in some //[[CL:Glossary:implementations]]//.

\endsubsection%{Character Names}

\beginsubSection{Treatment of Newline during Input and Output} \DefineSection{TreatmentOfNewline}

When the character \f{\#\\Newline} is written to an output file, the implementation must take the appropriate action to produce a line division.  This might involve writing out a record or translating \f{\#\\Newline} to a CR/LF sequence. When reading, a corresponding reverse transformation must take place.

\endsubSection%{Treatment of Newline during Input and Output}

\beginsubSection{Character Encodings} \DefineSection{CharEncodings}

\issue{CHARACTER-PROPOSAL:2-1-1}

A //[[CL:Glossary:character]]// is sometimes represented merely by its //[[CL:Glossary:code]]//, and sometimes by another //[[CL:Glossary:integer]]// value which is composed from the //[[CL:Glossary:code]]// and all  //[[CL:Glossary:implementation-defined]]// //[[CL:Glossary:attributes]]// (in an //[[CL:Glossary:implementation-defined]]// way that might vary between //[[CL:Glossary:Lisp images]]// even in the same //[[CL:Glossary:implementation]]//). This //[[CL:Glossary:integer]]//, returned by the function **[[CL:Functions:char-int]]**, is called the character's ``encoding.'' There is no corresponding function from a character's encoding back to the //[[CL:Glossary:character]]//,  since its primary intended uses include things like hashing where an inverse operation is not really called for.

\endsubSection%{Character Encodings}

\issue{CHARACTER-PROPOSAL:2-1-1}

\beginsubsection{Documentation of Implementation-Defined Scripts} \DefineSection{ImplementationDefinedScripts}

\issue{CHARACTER-PROPOSAL:2-4-2} 
   An //[[CL:Glossary:implementation]]// must document the //[[CL:Glossary:character]]// //[[CL:Glossary:scripts]]// 
   it supports. For each //[[CL:Glossary:character]]// //[[CL:Glossary:script]]// supported,
   the documentation must describe at least the following: \beginlist \itemitem{\bull}
   Character labels, glyphs, and descriptions.
   Character labels must be uniquely named using only Latin capital letters A--Z,
   hyphen (-), and digits 0--9. \itemitem{\bull}
   Reader canonicalization.
   Any mechanisms by which **[[CL:Functions:read]]** treats
   //[[CL:Glossary:different]]// characters as equivalent must be documented. \itemitem{\bull}
   The impact on **[[CL:Functions:char-upcase]]**, 		 **[[CL:Functions:char-downcase]]**, 	     and the case-sensitive //[[CL:Glossary:format directives]]//.
   In particular, for each //[[CL:Glossary:character]]// with //[[CL:Glossary:case]]//,
   whether it is //[[CL:Glossary:uppercase]]// or //[[CL:Glossary:lowercase]]//,
   and which //[[CL:Glossary:character]]// is its equivalent in the opposite case. \itemitem{\bull}
   The behavior of the case-insensitive //[[CL:Glossary:functions]]//
     **[[CL:Functions:char-equal]]**, **[[CL:Functions:char-not-equal]]**,
     **[[CL:Functions:char-lessp]]**, **[[CL:Functions:char-greaterp]]**, 
     **[[CL:Functions:char-not-greaterp]]**, and **[[CL:Functions:char-not-lessp]]**. \itemitem{\bull}
   The behavior of any //[[CL:Glossary:character]]// //[[CL:Glossary:predicates]]//;
   in particular, the effects of
   **[[CL:Functions:alpha-char-p]]**,
   **[[CL:Functions:lower-case-p]]**,
   **[[CL:Functions:upper-case-p]]**,
   **[[CL:Functions:both-case-p]]**,
   **[[CL:Functions:graphic-char-p]]**, 
   and
   **[[CL:Functions:alphanumericp]]**. \itemitem{\bull}
   The interaction with file I/O, in particular,
   the supported coded character sets (for example, ISO8859/1-1987)
   and external encoding schemes supported are documented. \endlist

\endsubsection%{Documentation of Implementation-Defined Scripts}
