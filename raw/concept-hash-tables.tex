

\beginsubsection{Hash-Table Operations}

\Thenextfigure\ lists some //[[CL:Glossary:defined names]]// that are applicable  to //[[CL:Glossary:hash tables]]//.  The following rules apply to //[[CL:Glossary:hash tables]]//.

\beginlist \itemitem{--} A //[[CL:Glossary:hash table]]// can only associate one value with a given key. If an attempt is made to add a second value for a given key, the second value will replace the first. Thus, adding a value to a //[[CL:Glossary:hash table]]// is a destructive operation; the //[[CL:Glossary:hash table]]// is modified.  

\itemitem{--} There are four kinds of //[[CL:Glossary:hash tables]]//:
  those whose keys are compared with **[[CL:Functions:eq]]**,
  those whose keys are compared with **[[CL:Functions:eql]]**,
  those whose keys are compared with **[[CL:Functions:equal]]**, and \issue{HASH-TABLE-TESTS:ADD-EQUALP}
  those whose keys are compared with **[[CL:Functions:equalp]]**.  

\itemitem{--} //[[CL:Glossary:Hash tables]]// are created by **[[CL:Functions:make-hash-table]]**.  **[[CL:Functions:gethash]]** is used to look up a key and find the associated value. New entries are added to //[[CL:Glossary:hash tables]]// using \macref{setf} with **[[CL:Functions:gethash]]**. **[[CL:Functions:remhash]]** is used to remove an entry. For example:

\code
 (setq a (make-hash-table)) \EV #<HASH-TABLE EQL 0/120 32536573>
 (setf (gethash 'color a) 'brown) \EV BROWN
 (setf (gethash 'name a) 'fred) \EV FRED
 (gethash 'color a) \EV BROWN, //[[CL:Glossary:true]]//
 (gethash 'name a) \EV FRED, //[[CL:Glossary:true]]//
 (gethash 'pointy a) \EV NIL, //[[CL:Glossary:false]]// \endcode

In this example, the symbols \f{color} and \f{name} are being used as keys, and the symbols \f{brown} and \f{fred} are being used as the associated values.  The //[[CL:Glossary:hash table]]//  has two items in it, one of which                               associates from \f{color} to \f{brown}, and the other of which associates from \f{name} to \f{fred}.

\itemitem{--} A key or a value may be any //[[CL:Glossary:object]]//.

\issue{HASH-TABLE-SIZE:INTENDED-ENTRIES}

\itemitem{--}

The existence of an entry in the //[[CL:Glossary:hash table]]// can be determined from the //[[CL:Glossary:secondary value]]// returned by **[[CL:Functions:gethash]]**. \endlist               

\displaythree{Hash-table defined names}{ clrhash&hash-table-p&remhash\cr gethash&make-hash-table&sxhash\cr hash-table-count&maphash&\cr }

\endsubsection%{Hash-Table Operations}

\beginsubsection{Modifying Hash Table Keys} \DefineSection{ModifyingHashKeys}

\issue{HASH-TABLE-KEY-MODIFICATION:SPECIFY}

The function supplied as the **'':test''** argument to **[[CL:Functions:make-hash-table]]** specifies the `equivalence test' for the //[[CL:Glossary:hash table]]// it creates.
  An //[[CL:Glossary:object]]// is `visibly modified' with regard to an equivalence test if there exists some set of //[[CL:Glossary:objects]]// (or potential //[[CL:Glossary:objects]]//) which are equivalent to the //[[CL:Glossary:object]]// before the modification but are no longer equivalent afterwards.

If an //[[CL:Glossary:object]]// $O\sub 1$ is used as a key in a //[[CL:Glossary:hash table]]// $H$ and is then visibly modified with regard to the equivalence test of $H$, then the consequences are unspecified if $O\sub 1$, or any //[[CL:Glossary:object]]// $O\sub 2$ equivalent to $O\sub 1$ under the equivalence test (either before or after the modification), is used as a key in further operations on $H$. The consequences of using $O\sub 1$ as a key are unspecified  even if $O\sub 1$ is visibly modified  and then later modified again in such a way as  to undo the visible modification.
  Following are specifications of the modifications which are visible to the equivalence tests which must be supported by //[[CL:Glossary:hash tables]]//.  The modifications are described in terms of modification of components, and are defined recursively.  Visible modifications of components of the //[[CL:Glossary:object]]// are  visible modifications of the //[[CL:Glossary:object]]//.

\beginsubsubsection{Visible Modification of Objects with respect to EQ and EQL} \DefineSection{VisModEQ} \DefineSection{VisModEQL}

No //[[CL:Glossary:standardized]]// //[[CL:Glossary:function]]// is provided that is capable of visibly modifying an //[[CL:Glossary:object]]// with regard to **[[CL:Functions:eq]]** or **[[CL:Functions:eql]]**.

\endsubsubsection%{Visible Modification of Objects with respect to EQ and EQL}

\beginsubsubsection{Visible Modification of Objects with respect to EQUAL} \DefineSection{VisModEQUAL}

As a consequence of the behavior for **[[CL:Functions:equal]]**, the rules for visible modification of //[[CL:Glossary:objects]]// not explicitly mentioned in this section are inherited from those in \secref\VisModEQL.
  \beginsubsubsubsection{Visible Modification of Conses with respect to EQUAL}

Any visible change to the //[[CL:Glossary:car]]// or the //[[CL:Glossary:cdr]]// of a //[[CL:Glossary:cons]]// is considered a visible modification with regard to **[[CL:Functions:equal]]**.
  \endsubsubsubsection%{Visible Modification of Conses with respect to EQUAL}

\beginsubsubsubsection{Visible Modification of Bit Vectors and Strings with respect to EQUAL}

For a //[[CL:Glossary:vector]]// \oftype{bit-vector} or \oftype{string}, any visible change
     to an //[[CL:Glossary:active]]// //[[CL:Glossary:element]]// of the //[[CL:Glossary:vector]]//,
  or to the //[[CL:Glossary:length]]// of the //[[CL:Glossary:vector]]// (if it is //[[CL:Glossary:actually adjustable]]//  					           or has a //[[CL:Glossary:fill pointer]]//) is considered a visible modification with regard to **[[CL:Functions:equal]]**.
  \endsubsubsubsection%{Visible Modification of Bit Vectors and Strings with respect to EQUAL}

\endsubsubsection%{Visible Modification of Objects with respect to EQUAL}

\beginsubsubsection{Visible Modification of Objects with respect to EQUALP}

As a consequence of the behavior for **[[CL:Functions:equalp]]**, the rules for visible modification of //[[CL:Glossary:objects]]// not explicitly mentioned in this section are inherited from those in \secref\VisModEQUAL.

\beginsubsubsubsection{Visible Modification of Structures with respect to EQUALP}

Any visible change to a //[[CL:Glossary:slot]]// of a //[[CL:Glossary:structure]]// is considered a visible modification with regard to **[[CL:Functions:equalp]]**.
  \endsubsubsubsection%{Visible Modification of Structures with respect to EQUALP}
  \beginsubsubsubsection{Visible Modification of Arrays with respect to EQUALP}

In an //[[CL:Glossary:array]]//, any visible change
     to an //[[CL:Glossary:active]]// //[[CL:Glossary:element]]//,
     to the //[[CL:Glossary:fill pointer]]// (if the //[[CL:Glossary:array]]// can and does have one),
  or to the //[[CL:Glossary:dimensions]]// (if the //[[CL:Glossary:array]]// is //[[CL:Glossary:actually adjustable]]//) is considered a visible modification with regard to **[[CL:Functions:equalp]]**.

\endsubsubsubsection%{Visible Modification of Arrays with respect to EQUALP}

\beginsubsubsubsection{Visible Modification of Hash Tables with respect to EQUALP}
  In a //[[CL:Glossary:hash table]]//, any visible change
     to the count of entries in the //[[CL:Glossary:hash table]]//,
     to the keys,
  or to the values associated with the keys is considered a visible modification with regard to **[[CL:Functions:equalp]]**.

Note that the visibility of modifications to the keys depends on the equivalence test of the //[[CL:Glossary:hash table]]//, not on the specification of **[[CL:Functions:equalp]]**.
  \endsubsubsubsection%{Visible Modification of Hash Tables with respect to EQUALP}

\endsubsubsection%{Visible Modification of Objects with respect to EQUALP}

\beginsubsubsection{Visible Modifications by Language Extensions}

//[[CL:Glossary:Implementations]]// that extend the language by providing additional mutator functions (or additional behavior for existing mutator functions) must document how the use of these extensions interacts with equivalence tests and //[[CL:Glossary:hash table]]// searches.
  //[[CL:Glossary:Implementations]]// that extend the language by defining additional acceptable equivalence tests for //[[CL:Glossary:hash tables]]// (allowing additional values for the **'':test''** argument to **[[CL:Functions:make-hash-table]]**) must document the visible components of these tests.

\endsubsubsection%{Visible Modifications by Language Extensions}

\endsubsection%{Modifying Hash Table Keys}
