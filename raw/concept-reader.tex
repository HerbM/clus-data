

\beginsubsection{Dynamic Control of the Lisp Reader}

Various aspects of the //[[CL:Glossary:Lisp reader]]// can be controlled dynamically. \Seesection\Readtables\ and \secref\ReaderVars.

\endsubsection%{Dynamic Control of the Lisp Reader}
  \beginsubsection{Effect of Readtable Case on the Lisp Reader} \DefineSection{ReadtableCaseReadEffect}

The //[[CL:Glossary:readtable case]]// of the //[[CL:Glossary:current readtable]]// affects the //[[CL:Glossary:Lisp reader]]// in the following ways:

\beginlist \itemitem{**'':upcase''**}

 When the //[[CL:Glossary:readtable case]]// is **'':upcase''**,
 unescaped constituent //[[CL:Glossary:characters]]// are converted to //[[CL:Glossary:uppercase]]//,
 as specified in \secref\ReaderAlgorithm.

\itemitem{**'':downcase''**}

 When the //[[CL:Glossary:readtable case]]// is **'':downcase''**,
 unescaped constituent //[[CL:Glossary:characters]]// are converted to //[[CL:Glossary:lowercase]]//.

\itemitem{**'':preserve''**}

When the //[[CL:Glossary:readtable case]]// is **'':preserve''**,
 the case of all //[[CL:Glossary:characters]]// remains unchanged.

\itemitem{**'':invert''**}

When the //[[CL:Glossary:readtable case]]// is **'':invert''**,
 then if all of the unescaped letters in the extended token are of the same //[[CL:Glossary:case]]//, 
 those (unescaped) letters are converted to the opposite //[[CL:Glossary:case]]//.

\endlist

\beginsubsubsection{Examples of Effect of Readtable Case on the Lisp Reader} \DefineSection{ReadtableCaseReadExamples}

\code
 (defun test-readtable-case-reading ()
   (let ((*readtable* (copy-readtable nil)))
     (format t "READTABLE-CASE  Input   Symbol-name~
              ~%-----------------------------------~
              ~%")
     (dolist (readtable-case '(:upcase :downcase :preserve :invert))
       (setf (readtable-case *readtable*) readtable-case)
       (dolist (input '("ZEBRA" "Zebra" "zebra"))
         (format t "~&:~A~16T~A~24T~A"
                 (string-upcase readtable-case)
                 input
                 (symbol-name (read-from-string input))))))) \endcode
  The output from \f{(test-readtable-case-reading)} should be as follows:

\code
 READTABLE-CASE     Input Symbol-name
 -------------------------------------
    :UPCASE         ZEBRA   ZEBRA
    :UPCASE         Zebra   ZEBRA
    :UPCASE         zebra   ZEBRA
    :DOWNCASE       ZEBRA   zebra
    :DOWNCASE       Zebra   zebra
    :DOWNCASE       zebra   zebra
    :PRESERVE       ZEBRA   ZEBRA
    :PRESERVE       Zebra   Zebra
    :PRESERVE       zebra   zebra
    :INVERT         ZEBRA   zebra
    :INVERT         Zebra   Zebra
    :INVERT         zebra   ZEBRA \endcode

\endsubsubsection%{Examples of Effect of Readtable Case on the Lisp Reader}

\endsubsection%{Effect of Readtable Case on the Lisp Reader}

\beginsubsection{Argument Conventions of Some Reader Functions}

\beginsubsubsection{The EOF-ERROR-P argument}

//Eof-error-p// in input function calls controls what happens if input is from a file (or any other input source that has a definite end) and the end of the file is reached. If //eof-error-p// is //[[CL:Glossary:true]]// (the default),  an error \oftype{end-of-file} is signaled at end of file.  If it is //[[CL:Glossary:false]]//, then no error is signaled, and instead the function returns //eof-value//.

Functions such as **[[CL:Functions:read]]** that read the representation of an //[[CL:Glossary:object]]// rather than a single character always signals an error, regardless of //eof-error-p//, if the file ends in the middle of an object representation. For example, if a file does not contain enough right parentheses to balance the left parentheses in it, **[[CL:Functions:read]]** signals an error.  If a file ends in a  //[[CL:Glossary:symbol]]// or a //[[CL:Glossary:number]]// immediately followed by end-of-file, **[[CL:Functions:read]]** reads the  //[[CL:Glossary:symbol]]// or //[[CL:Glossary:number]]//  successfully and when called again will

act according to //eof-error-p//. Similarly, \thefunction{read-line} successfully reads the last line of a file even if that line is terminated by end-of-file rather than the newline character. Ignorable text, such as lines containing only //[[CL:Glossary:whitespace]]//\meaning{2} or comments, are not considered to begin an //[[CL:Glossary:object]]//;  if **[[CL:Functions:read]]** begins to read an //[[CL:Glossary:expression]]// but sees only such ignorable text, it does not consider the file to end in the middle of an //[[CL:Glossary:object]]//. Thus an //eof-error-p// argument controls what happens when the file ends between //[[CL:Glossary:objects]]//.

\endsubsubsection%{The EOF-ERROR-P argument}

\beginsubsubsection{The RECURSIVE-P argument}

If //recursive-p// is supplied and not \nil, it specifies that this function call is not an outermost call to **[[CL:Functions:read]]** but an  embedded call, typically from a //[[CL:Glossary:reader macro function]]//. It is important to distinguish such recursive calls for three reasons.

\beginlist \itemitem{1.} An outermost call establishes the context within which the \f{\#//n//=} and \f{\#//n//\#} syntax is scoped.  Consider, for example, the expression

\code
 (cons '#3=(p q r) '(x y . #3#)) \endcode If the //[[CL:Glossary:single-quote]]// //[[CL:Glossary:reader macro]]// were defined in this way:

\code
 (set-macro-character #\\'       ;incorrect
    #'(lambda (stream char)
         (declare (ignore char))
         (list 'quote (read stream)))) \endcode

then each call to the //[[CL:Glossary:single-quote]]// //[[CL:Glossary:reader macro function]]// would establish independent contexts for the scope of **[[CL:Functions:read]]** information, including the scope of identifications between markers like ``\f{\#3=}'' and ``\f{\#3\#}''.  However, for this expression, the scope was clearly intended to be determined by the outer set  of parentheses, so such a definition would be incorrect. The correct way to define the //[[CL:Glossary:single-quote]]// //[[CL:Glossary:reader macro]]// uses //recursive-p//: 

\code
 (set-macro-character #\\'       ;correct
    #'(lambda (stream char)
         (declare (ignore char))
         (list 'quote (read stream t nil t)))) \endcode

\itemitem{2.} A recursive call does not alter whether the reading process is to preserve //[[CL:Glossary:whitespace]]//\meaning{2} or not (as determined by whether the outermost call was to **[[CL:Functions:read]]** or **[[CL:Functions:read-preserving-whitespace]]**). Suppose again that //[[CL:Glossary:single-quote]]// 

were to be defined as shown above in the incorrect definition. Then a call to **[[CL:Functions:read-preserving-whitespace]]** that read the expression \f{'foo\SpaceChar} would fail to preserve the space character following the symbol \f{foo} because the //[[CL:Glossary:single-quote]]// //[[CL:Glossary:reader macro function]]// calls **[[CL:Functions:read]]**,  not **[[CL:Functions:read-preserving-whitespace]]**, to read the following expression (in this case \f{foo}). The correct definition, which passes the value //[[CL:Glossary:true]]// for //recursive-p// to **[[CL:Functions:read]]**, allows the outermost call to determine whether //[[CL:Glossary:whitespace]]//\meaning{2} is preserved.

\itemitem{3.} When end-of-file is encountered and the //eof-error-p// argument is not \nil, the kind of error that is signaled may depend on the value of //recursive-p//.  If //recursive-p//  is //[[CL:Glossary:true]]//, then the end-of-file is deemed to have occurred within the middle of a printed representation; if //recursive-p// is //[[CL:Glossary:false]]//, then the end-of-file may be deemed to have occurred between //[[CL:Glossary:objects]]// rather than within the middle of one.

\endlist

\endsubsubsection%{The EOF-ERROR-P argument}

\endsubsection%{Argument Conventions of Some Reader Functions}
