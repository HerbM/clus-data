






The //[[CL:Glossary:Lisp reader]]// takes //[[CL:Glossary:characters]]// from a //[[CL:Glossary:stream]]//, 
interprets them as a printed representation of an //[[CL:Glossary:object]]//,
constructs that //[[CL:Glossary:object]]//, and returns it.

\DefineSection{TheStandardSyntax}
The syntax described by this chapter is called the //[[CL:Glossary:standard syntax]]//.
Operations are provided by \clisp\ so that
various aspects of the syntax information represented by a //[[CL:Glossary:readtable]]// 
can be modified under program control; \seechapter\Reader.
Except as explicitly stated otherwise, 
the syntax used throughout this document is //[[CL:Glossary:standard syntax]]//.

\beginSubsection{Readtables}
\DefineSection{Readtables}

Syntax information for use by the //[[CL:Glossary:Lisp reader]]// is embodied in an
//[[CL:Glossary:object]]// called a //[[CL:Glossary:readtable]]//.  Among other things, 
the //[[CL:Glossary:readtable]]// contains the association between //[[CL:Glossary:characters]]// 
and //[[CL:Glossary:syntax types]]//.

\Thenextfigure\ lists some //[[CL:Glossary:defined names]]// that are applicable to
//[[CL:Glossary:readtables]]//.

\displaythree{Readtable defined names}{
*readtable*&readtable-case\cr
copy-readtable&readtablep\cr
get-dispatch-macro-character&set-dispatch-macro-character\cr
get-macro-character&set-macro-character\cr
make-dispatch-macro-character&set-syntax-from-char\cr
}


\beginsubsubsection{The Current Readtable}
\DefineSection{CurrentReadtable}

Several //[[CL:Glossary:readtables]]// describing different syntaxes can exist,
but at any given time only one, called the //[[CL:Glossary:current readtable]]//, 
affects the way in which //[[CL:Glossary:expressions]]//\meaning{2} are parsed 
into //[[CL:Glossary:objects]]// by the //[[CL:Glossary:Lisp reader]]//.
The //[[CL:Glossary:current readtable]]// in a given //[[CL:Glossary:dynamic environment]]//
is \thevalueof{*readtable*} in that //[[CL:Glossary:environment]]//.
To make a different //[[CL:Glossary:readtable]]// become the //[[CL:Glossary:current readtable]]//,
\varref{*readtable*} can be //[[CL:Glossary:assigned]]// or //[[CL:Glossary:bound]]//.

\endsubsubsection%{The Current Readtable}

\beginsubsubsection{The Standard Readtable}

The //[[CL:Glossary:standard readtable]]// conforms to //[[CL:Glossary:standard syntax]]//.
The consequences are undefined if an attempt is made
to modify the //[[CL:Glossary:standard readtable]]//.

To achieve the effect of altering or extending //[[CL:Glossary:standard syntax]]//,
a copy of the //[[CL:Glossary:standard readtable]]// can be created; \seefun{copy-readtable}.

The //[[CL:Glossary:readtable case]]// of the //[[CL:Glossary:standard readtable]]// is **'':upcase''**.

\endsubsubsection%{The Standard Readtable}

\beginsubsubsection{The Initial Readtable}

The //[[CL:Glossary:initial readtable]]// is
the //[[CL:Glossary:readtable]]// that is the //[[CL:Glossary:current readtable]]//
at the time when the //[[CL:Glossary:Lisp image]]// starts.
At that time, it conforms to //[[CL:Glossary:standard syntax]]//.
The //[[CL:Glossary:initial readtable]]// is //[[CL:Glossary:distinct]]// 
from the //[[CL:Glossary:standard readtable]]//.
It is permissible for a //[[CL:Glossary:conforming program]]// 
to modify the //[[CL:Glossary:initial readtable]]//.

\endsubsubsection%{The Initial Readtable}

\endSubsection%{Readtables}

\beginsubsection{Variables that affect the Lisp Reader}
\DefineSection{ReaderVars}

The //[[CL:Glossary:Lisp reader]]// is influenced not only by the //[[CL:Glossary:current readtable]]//,
but also by various //[[CL:Glossary:dynamic variables]]//.  \Thenextfigure\ lists
the //[[CL:Glossary:variables]]// that influence the behavior of the //[[CL:Glossary:Lisp reader]]//.

\displaythree{Variables that influence the Lisp reader.}{
*package*&*read-default-float-format*&*readtable*\cr
*read-base*&*read-suppress*&\cr
}

\endsubsection%{Variables that affect the Lisp Reader}

\beginSubsection{Standard Characters}
\DefineSection{StandardChars}




\issue{CHARACTER-PROPOSAL:2-2-1}
All //[[CL:Glossary:implementations]]// must support a //[[CL:Glossary:character]]// //[[CL:Glossary:repertoire]]//
called \typeref{standard-char}; //[[CL:Glossary:characters]]// that are members of that
//[[CL:Glossary:repertoire]]// are called \newtermidx{standard characters}{standard character}.

The \typeref{standard-char} //[[CL:Glossary:repertoire]]// consists of
the //[[CL:Glossary:non-graphic]]// //[[CL:Glossary:character]]// //[[CL:Glossary:newline]]//,
the //[[CL:Glossary:graphic]]// //[[CL:Glossary:character]]// //[[CL:Glossary:space]]//,
and the following additional
ninety-four //[[CL:Glossary:graphic]]// //[[CL:Glossary:characters]]// or their equivalents:
 
\tablefigsix{Standard Character Subrepertoire (Part 1 of 3: Latin Characters)}%
{Graphic ID}{Glyph}{Description}%
{Graphic ID}{Glyph}{Description}{
  LA01  &  \f{a}  &  small a    &
  LN01  &  \f{n}  &  small n    \cr
  LA02  &  \f{A}  &  capital A  &
  LN02  &  \f{N}  &  capital N  \cr
  LB01  &  \f{b}  &  small b    &
  LO01  &  \f{o}  &  small o    \cr
  LB02  &  \f{B}  &  capital B  &
  LO02  &  \f{O}  &  capital O  \cr
  LC01  &  \f{c}  &  small c    &
  LP01  &  \f{p}  &  small p    \cr
  LC02  &  \f{C}  &  capital C  &
  LP02  &  \f{P}  &  capital P  \cr
  LD01  &  \f{d}  &  small d    &
  LQ01  &  \f{q}  &  small q    \cr
  LD02  &  \f{D}  &  capital D  &
  LQ02  &  \f{Q}  &  capital Q  \cr
  LE01  &  \f{e}  &  small e    &
  LR01  &  \f{r}  &  small r    \cr
  LE02  &  \f{E}  &  capital E  &
  LR02  &  \f{R}  &  capital R  \cr
  LF01  &  \f{f}  &  small f    &
  LS01  &  \f{s}  &  small s    \cr
  LF02  &  \f{F}  &  capital F  &
  LS02  &  \f{S}  &  capital S  \cr
  LG01  &  \f{g}  &  small g    &
  LT01  &  \f{t}  &  small t    \cr
  LG02  &  \f{G}  &  capital G  &
  LT02  &  \f{T}  &  capital T  \cr
  LH01  &  \f{h}  &  small h    &
  LU01  &  \f{u}  &  small u    \cr
  LH02  &  \f{H}  &  capital H  &
  LU02  &  \f{U}  &  capital U  \cr
  LI01  &  \f{i}  &  small i    &
  LV01  &  \f{v}  &  small v    \cr
  LI02  &  \f{I}  &  capital I  &
  LV02  &  \f{V}  &  capital V  \cr
  LJ01  &  \f{j}  &  small j    &
  LW01  &  \f{w}  &  small w    \cr
  LJ02  &  \f{J}  &  capital J  &
  LW02  &  \f{W}  &  capital W  \cr
  LK01  &  \f{k}  &  small k    &
  LX01  &  \f{x}  &  small x    \cr
  LK02  &  \f{K}  &  capital K  &
  LX02  &  \f{X}  &  capital X  \cr
  LL01  &  \f{l}  &  small l    &
  LY01  &  \f{y}  &  small y    \cr
  LL02  &  \f{L}  &  capital L  &
  LY02  &  \f{Y}  &  capital Y  \cr
  LM01  &  \f{m}  &  small m    &
  LZ01  &  \f{z}  &  small z    \cr
  LM02  &  \f{M}  &  capital M  &
  LZ02  &  \f{Z}  &  capital Z  \cr
}

\tablefigsix{Standard Character Subrepertoire (Part 2 of 3: Numeric Characters)}%
{Graphic ID}{Glyph}{Description}%
{Graphic ID}{Glyph}{Description}{
  ND01  &  \f{1}  &  digit 1 &
  ND06  &  \f{6}  &  digit 6 \cr
  ND02  &  \f{2}  &  digit 2 &
  ND07  &  \f{7}  &  digit 7 \cr
  ND03  &  \f{3}  &  digit 3 &
  ND08  &  \f{8}  &  digit 8 \cr
  ND04  &  \f{4}  &  digit 4 &
  ND09  &  \f{9}  &  digit 9 \cr
  ND05  &  \f{5}  &  digit 5 &
  ND10  &  \f{0}  &  digit 0 \cr
}

\DefineFigure{StdCharsThree}
\tablefigthree{Standard Character Subrepertoire (Part 3 of 3: Special Characters)}%
{Graphic ID}{Glyph}{Description}{
  SP02  &  \f{!}        &  exclamation mark                        \cr
  SC03  &  \f{\$}       &  dollar sign                             \cr
  SP04  &  \f{"}        &  quotation mark, or double quote         \cr
  SP05  &  \f{'}        &  apostrophe, or \brac{single} quote      \cr
  SP06  &  \f{(}        &  left parenthesis, or open parenthesis   \cr
  SP07  &  \f{)}        &  right parenthesis, or close parenthesis \cr
  SP08  &  \f{,}        &  comma                                   \cr
  SP09  &  \f{_}        &  low line, or underscore                 \cr
  SP10  &  \f{-}        &  hyphen, or minus \brac{sign}            \cr
  SP11  &  \f{.}        &  full stop, period, or dot               \cr
  SP12  &  \f{/}        &  solidus, or slash                       \cr
  SP13  &  \f{:}        &  colon                                   \cr
  SP14  &  \f{;}        &  semicolon                               \cr
  SP15  &  \f{?}        &  question mark                           \cr
  SA01  &  \f{+}        &  plus \brac{sign}                        \cr
  SA03  &  \f{<}        &  less-than \brac{sign}                   \cr
  SA04  &  \f{=}        &  equals \brac{sign}                      \cr
  SA05  &  \f{>}        &  greater-than \brac{sign}                \cr
  SM01  &  \f{\#}       &  number sign, or sharp\brac{sign}        \cr
  SM02  &  \f{\%}       &  percent \brac{sign}                     \cr
  SM03  &  \f{\&}       &  ampersand			           \cr
  SM04  &  \f{*}        &  asterisk, or star                       \cr
  SM05  &  \f{@}        &  commercial at, or at-sign               \cr
  SM06  &  \f{[}        &  left \brac{square} bracket              \cr
  SM07  &  \f{\\}       &  reverse solidus, or backslash           \cr
  SM08  &  \f{]}        &  right \brac{square} bracket             \cr
  SM11  &  \f{\{}       &  left curly bracket, or left brace       \cr
  SM13  &  \f{|}        &  vertical bar                            \cr
  SM14  &  \f{\}}       &  right curly bracket, or right brace     \cr
  SD13  &  \f{`}        &  grave accent, or backquote              \cr
  SD15  &  \f{\hat}     &  circumflex accent                       \cr
  SD19  &  \f{~}        &  tilde                                   \cr
}

The graphic IDs are not used within \clisp,
but are provided for cross reference purposes with {\ISOChars}.
Note that the first letter of the graphic ID 
categorizes the character as follows:
L---Latin, N---Numeric, S---Special.

\endSubsection%{Standard Characters}



\beginSubsection{Character Syntax Types}
\DefineSection{CharacterSyntaxTypes}



The //[[CL:Glossary:Lisp reader]]// constructs an //[[CL:Glossary:object]]// 
from the input text by interpreting each //[[CL:Glossary:character]]// 
according to its //[[CL:Glossary:syntax type]]//.
The //[[CL:Glossary:Lisp reader]]// cannot accept as input 
everything that the //[[CL:Glossary:Lisp printer]]// produces,
and the //[[CL:Glossary:Lisp reader]]// has features that are not used by the //[[CL:Glossary:Lisp printer]]//.
The //[[CL:Glossary:Lisp reader]]// can be used as a lexical analyzer 
for a more general user-written parser.



When the //[[CL:Glossary:Lisp reader]]// is invoked, it reads a single character from 
the //[[CL:Glossary:input]]// //[[CL:Glossary:stream]]// and dispatches according to the
//[[CL:Glossary:syntax type]]// of that //[[CL:Glossary:character]]//.
Every //[[CL:Glossary:character]]// that can appear in the //[[CL:Glossary:input]]// //[[CL:Glossary:stream]]//
is of one of the //[[CL:Glossary:syntax types]]// shown in \figref\PossibleSyntaxTypes.

\DefineFigure{PossibleSyntaxTypes}
\showthree{Possible Character Syntax Types}{
//[[CL:Glossary:constituent]]//&//[[CL:Glossary:macro character]]//&//[[CL:Glossary:single escape]]//\cr
//[[CL:Glossary:invalid]]//&//[[CL:Glossary:multiple escape]]//&//[[CL:Glossary:whitespace]]//\meaning{2}\cr
}

The //[[CL:Glossary:syntax type]]// of a //[[CL:Glossary:character]]// in a //[[CL:Glossary:readtable]]//
determines how that character is interpreted by the //[[CL:Glossary:Lisp reader]]//
while that //[[CL:Glossary:readtable]]// is the //[[CL:Glossary:current readtable]]//.
At any given time, every character has exactly one //[[CL:Glossary:syntax type]]//.



\Figref\CharSyntaxTypesInStdSyntax\ 
lists the //[[CL:Glossary:syntax type]]// of each //[[CL:Glossary:character]]// in //[[CL:Glossary:standard syntax]]//.

\DefineFigure{CharSyntaxTypesInStdSyntax}

{\def\w{//[[CL:Glossary:whitespace]]//\meaning{2}}
\def\n{//[[CL:Glossary:non-terminating]]// //[[CL:Glossary:macro char]]//} %cheat on "char" => "character" to make fit
\def\t{//[[CL:Glossary:terminating]]// //[[CL:Glossary:macro char]]//}     %ditto
\def\c{//[[CL:Glossary:constituent]]//}
\def\C{//[[CL:Glossary:constituent]]//*}
\def\SE{//[[CL:Glossary:single escape]]//}
\def\ME{//[[CL:Glossary:multiple escape]]//}
\tablefigfour{Character Syntax Types in Standard Syntax}{character}{syntax type}{character}{syntax type}{
Backspace&\c&0--9&\c\cr
Tab&\w&:&\c\cr
Newline&\w&;&\t\cr
Linefeed&\w&{\tt<}&\c\cr
Page&\w&=&\c\cr
Return&\w&{\tt>}&\c\cr
Space&\w&?&\C\cr
!&\C&{\tt @}&\c\cr
{\tt"}&\t&A--Z&\c\cr
\#&\n&\f{[}&\C\cr
\$&\c&\f{\\}&\SE\cr
\%&\c&\f{]}&\C\cr
\&&\c&\hat&\c\cr
'&\t&\f{\_}&\c\cr
(&\t&`&\t\cr
)&\t&a--z&\c\cr
{\tt*}&\c&\f{\{}&\C\cr
+&\c&\f{|}&\ME\cr
,&\t&\f{\}}&\C\cr
-&\c&\f{~}&\c\cr
.&\c&Rubout&\c\cr
/&\c\cr
}}


The characters marked with an asterisk (*) are initially //[[CL:Glossary:constituents]]//,








but they are not used in any standard \clisp\ notations.




These characters are explicitly reserved to the //[[CL:Glossary:programmer]]//.
\f{~} is not used in \clisp, and reserved to implementors.
\f{\$} and \f{\%} are //[[CL:Glossary:alphabetic]]//\meaning{2} //[[CL:Glossary:characters]]//,
but are not used in the names of any standard \clisp\ //[[CL:Glossary:defined names]]//.

//[[CL:Glossary:Whitespace]]//\meaning{2} characters serve as separators but are otherwise
ignored.  //[[CL:Glossary:Constituent]]// and //[[CL:Glossary:escape]]// //[[CL:Glossary:characters]]// are accumulated
to make a //[[CL:Glossary:token]]//, which is then interpreted as a //[[CL:Glossary:number]]// or //[[CL:Glossary:symbol]]//.
//[[CL:Glossary:Macro characters]]// trigger the invocation of //[[CL:Glossary:functions]]// (possibly
user-supplied) that can perform arbitrary parsing actions.
//[[CL:Glossary:Macro characters]]// are divided into two kinds,
//[[CL:Glossary:terminating]]// and //[[CL:Glossary:non-terminating]]//,
depending on whether or not they terminate a //[[CL:Glossary:token]]//.
The following are descriptions of each kind of //[[CL:Glossary:syntax type]]//.

\beginsubsubsection{Constituent Characters}
\DefineSection{ConstituentChars}

//[[CL:Glossary:Constituent]]// //[[CL:Glossary:characters]]// are used in //[[CL:Glossary:tokens]]//.


A //[[CL:Glossary:token]]// is a representation of a //[[CL:Glossary:number]]// or a //[[CL:Glossary:symbol]]//.  
Examples of //[[CL:Glossary:constituent]]// //[[CL:Glossary:characters]]// are letters and digits.


Letters in symbol names are sometimes converted to 
letters in the opposite //[[CL:Glossary:case]]// when the name is read;
\seesection\ReadtableCaseReadEffect.
//[[CL:Glossary:Case]]// conversion can be suppressed by the use 
of //[[CL:Glossary:single escape]]// or //[[CL:Glossary:multiple escape]]// characters.





\beginsubsubsection{Constituent Traits}
\DefineSection{ConstituentTraits}

Every //[[CL:Glossary:character]]// has one or more //[[CL:Glossary:constituent traits]]//
that define how the //[[CL:Glossary:character]]// is to be interpreted by the //[[CL:Glossary:Lisp reader]]//
when the //[[CL:Glossary:character]]// is a //[[CL:Glossary:constituent]]// //[[CL:Glossary:character]]//.
These //[[CL:Glossary:constituent traits]]// are 
     //[[CL:Glossary:alphabetic]]//\meaning{2},                  
     digit,
     //[[CL:Glossary:package marker]]//,
     plus sign,
     minus sign, 
     dot,
     decimal point,
     //[[CL:Glossary:ratio marker]]//,
     //[[CL:Glossary:exponent marker]]//,
 and //[[CL:Glossary:invalid]]//.
\Figref\ConstituentTraitsOfStdChars\ shows the //[[CL:Glossary:constituent traits]]//
of the //[[CL:Glossary:standard characters]]//
and of certain //[[CL:Glossary:semi-standard]]// //[[CL:Glossary:characters]]//;


no mechanism is provided for changing the //[[CL:Glossary:constituent trait]]// of a //[[CL:Glossary:character]]//.
Any //[[CL:Glossary:character]]// with the alphadigit //[[CL:Glossary:constituent trait]]//
in that figure is a digit if the //[[CL:Glossary:current input base]]// is greater
than that character's digit value,
otherwise the //[[CL:Glossary:character]]// is //[[CL:Glossary:alphabetic]]//\meaning{2}.  


Any //[[CL:Glossary:character]]// quoted by a //[[CL:Glossary:single escape]]// 
is treated as an //[[CL:Glossary:alphabetic]]//\meaning{2} constituent, regardless of its normal syntax.

\DefineFigure{ConstituentTraitsOfStdChars}
\boxfig
{\dimen0=.75pc
\def\a{//[[CL:Glossary:alphabetic]]//\meaning{2}}
\def\ad{alphadigit}
\def\i{//[[CL:Glossary:invalid]]//}
\def\pm{//[[CL:Glossary:package marker]]//}
\tabskip \dimen0
\halign to \hsize {#\hfil\tabskip \dimen0&#\hfil\tabskip 0pt plus 1fil
&#\hfil\tabskip \dimen0&#\hfil\cr
\noalign{\vskip -9pt}
\bf constituent&\bf traits&\bf constituent&\bf traits\cr
\bf character&&\bf character\cr
\noalign{\vskip 2pt\hrule\vskip 2pt}
Backspace&\i&\f{\{}&\a\cr
Tab&\i*&\f{\}}&\a\cr
Newline&\i*&+&\a, plus sign\cr
Linefeed&\i*&-&\a, minus sign\cr
Page&\i*&.&\a, dot, decimal point\cr
Return&\i*&/&\a, //[[CL:Glossary:ratio marker]]//\cr
Space&\i*&A, a&\ad\cr
! &\a&B, b&\ad\cr
{\tt "}&\a*&C, c&\ad\cr
\#&\a*&D, d&\ad, double-float //[[CL:Glossary:exponent marker]]//\cr
\$&\a&E, e&\ad, float //[[CL:Glossary:exponent marker]]//\cr
\%&\a&F, f&\ad, single-float //[[CL:Glossary:exponent marker]]//\cr
\&&\a&G, g&\ad\cr
'&\a*&H, h&\ad\cr
(&\a*&I, i&\ad\cr
)&\a*&J, j&\ad\cr
{\tt *}&\a&K, k&\ad\cr
,&\a*&L, l&\ad, long-float //[[CL:Glossary:exponent marker]]//\cr
0-9&\ad&M, m&\ad\cr
:&\pm&N, n&\ad\cr
;&\a*&O, o&\ad\cr
{\tt<}&\a&P, p&\ad\cr
=&\a&Q, q&\ad\cr
{\tt>}&\a&R, r&\ad\cr
?&\a&S, s&\ad, short-float //[[CL:Glossary:exponent marker]]//\cr
\f{@}&\a&T, t&\ad\cr
\f{[}&\a&U, u&\ad\cr
\f{\\}&\a*&V, v&\ad\cr
\f{]}&\a&W, w&\ad\cr
\hat&\a&X, x&\ad\cr
\f{\_}&\a&Y, y&\ad\cr
`&\a*&Z, z&\ad\cr
\f{|}&\a*&Rubout&\i\cr
\f{~}&\a\cr
\noalign{\vskip -9pt}
}}
\caption{Constituent Traits of Standard Characters and Semi-Standard Characters}
\endfig
                   
The interpretations in this table apply only to //[[CL:Glossary:characters]]//
whose //[[CL:Glossary:syntax type]]// is //[[CL:Glossary:constituent]]//.
Entries marked with an asterisk (*) are normally //[[CL:Glossary:shadowed]]//\meaning{2} 
because the indicated //[[CL:Glossary:characters]]// are of //[[CL:Glossary:syntax type]]//
//[[CL:Glossary:whitespace]]//\meaning{2},
//[[CL:Glossary:macro character]]//,
//[[CL:Glossary:single escape]]//,
or //[[CL:Glossary:multiple escape]]//;
these //[[CL:Glossary:constituent traits]]// apply to them only if their //[[CL:Glossary:syntax types]]// 
are changed to //[[CL:Glossary:constituent]]//.

\endsubsubsection%{Constituent Traits}

\endsubsubsection%{Constituent Characters}
 
\beginsubsubsection{Invalid Characters}
\DefineSection{InvalidChars}

//[[CL:Glossary:Characters]]// with the //[[CL:Glossary:constituent trait]]// //[[CL:Glossary:invalid]]// 
cannot ever appear in a //[[CL:Glossary:token]]// 
except under the control of a //[[CL:Glossary:single escape]]// //[[CL:Glossary:character]]//.
If an //[[CL:Glossary:invalid]]// //[[CL:Glossary:character]]// is encountered while an //[[CL:Glossary:object]]// is
being read, an error \oftype{reader-error} is signaled.
If an //[[CL:Glossary:invalid]]// //[[CL:Glossary:character]]// is preceded by a //[[CL:Glossary:single escape]]// //[[CL:Glossary:character]]//,
it is treated as an //[[CL:Glossary:alphabetic]]//\meaning{2} //[[CL:Glossary:constituent]]// instead.

\endsubsubsection%{Invalid Characters}

\beginsubsubsection{Macro Characters}
\DefineSection{MacroChars}

When the //[[CL:Glossary:Lisp reader]]// encounters a //[[CL:Glossary:macro character]]// 
on an //[[CL:Glossary:input]]// //[[CL:Glossary:stream]]//,
special parsing of subsequent //[[CL:Glossary:characters]]// 
on the //[[CL:Glossary:input]]// //[[CL:Glossary:stream]]// 
is performed.

A //[[CL:Glossary:macro character]]// has an associated //[[CL:Glossary:function]]//
called a //[[CL:Glossary:reader macro function]]// that implements its specialized parsing behavior.
An association of this kind can be established or modified under control of
a //[[CL:Glossary:conforming program]]// by using 
\thefunctions{set-macro-character} and **[[CL:Functions:set-dispatch-macro-character]]**.

Upon encountering a //[[CL:Glossary:macro character]]//, the //[[CL:Glossary:Lisp reader]]// calls its
//[[CL:Glossary:reader macro function]]//, which parses one specially formatted object 
from the //[[CL:Glossary:input]]// //[[CL:Glossary:stream]]//.
The //[[CL:Glossary:function]]// either returns the parsed //[[CL:Glossary:object]]//,
or else it returns no //[[CL:Glossary:values]]// 
    to indicate that the characters scanned by the //[[CL:Glossary:function]]//
    are being ignored (\eg in the case of a comment).
Examples of //[[CL:Glossary:macro characters]]//
are //[[CL:Glossary:backquote]]//, //[[CL:Glossary:single-quote]]//, //[[CL:Glossary:left-parenthesis]]//, and 
//[[CL:Glossary:right-parenthesis]]//.

A //[[CL:Glossary:macro character]]// is either //[[CL:Glossary:terminating]]// or //[[CL:Glossary:non-terminating]]//.
The difference between //[[CL:Glossary:terminating]]// and //[[CL:Glossary:non-terminating]]// //[[CL:Glossary:macro characters]]// 
lies in what happens when such characters occur in the middle of a //[[CL:Glossary:token]]//.  
If a //[[CL:Glossary:non-terminating]]// //[[CL:Glossary:macro character]]// occurs in the middle of a //[[CL:Glossary:token]]//,
the //[[CL:Glossary:function]]// associated 
with the //[[CL:Glossary:non-terminating]]// //[[CL:Glossary:macro character]]// is not called,
and the
//[[CL:Glossary:non-terminating]]// //[[CL:Glossary:macro character]]// does not terminate the //[[CL:Glossary:token]]//'s name; it
becomes part of the name as if the //[[CL:Glossary:macro character]]// were really a constituent
character.  A //[[CL:Glossary:terminating]]// //[[CL:Glossary:macro character]]// terminates any //[[CL:Glossary:token]]//,
and its associated //[[CL:Glossary:reader macro function]]//
is called no matter where the //[[CL:Glossary:character]]// appears.
The only //[[CL:Glossary:non-terminating]]// //[[CL:Glossary:macro character]]// in //[[CL:Glossary:standard syntax]]// 
is //[[CL:Glossary:sharpsign]]//.

If a //[[CL:Glossary:character]]// is a //[[CL:Glossary:dispatching macro character]]// $C\sub 1$,
its //[[CL:Glossary:reader macro function]]// is a //[[CL:Glossary:function]]// supplied by the //[[CL:Glossary:implementation]]//.
This //[[CL:Glossary:function]]// reads decimal //[[CL:Glossary:digit]]// //[[CL:Glossary:characters]]// until a non-//[[CL:Glossary:digit]]//
$C\sub 2$ is read.
If any //[[CL:Glossary:digits]]// were read,
they are converted into a corresponding //[[CL:Glossary:integer]]// infix parameter $P$;
otherwise, the infix parameter $P$ is \nil.  
The terminating non-//[[CL:Glossary:digit]]// $C\sub 2$ is a //[[CL:Glossary:character]]// 
(sometimes called a ``sub-character'' to emphasize its subordinate role in the dispatching)
that is looked up in the dispatch table associated with
the //[[CL:Glossary:dispatching macro character]]// $C\sub 1$.
The //[[CL:Glossary:reader macro function]]// associated with the sub-character $C\sub 2$ 
is invoked with three arguments:
     the //[[CL:Glossary:stream]]//,
     the sub-character $C\sub 2$,
 and the infix parameter $P$.
For more information about dispatch characters,
\seefun{set-dispatch-macro-character}.

For information about the //[[CL:Glossary:macro characters]]// 
that are available in //[[CL:Glossary:standard syntax]]//,
\seesection\StandardMacroChars.

\endsubsubsection%{Macro Characters}

\beginsubsubsection{Multiple Escape Characters}
\DefineSection{MultipleEscapeChar}

A pair of //[[CL:Glossary:multiple escape]]// //[[CL:Glossary:characters]]//
is used to indicate that an enclosed sequence of characters,
including possible //[[CL:Glossary:macro characters]]// and //[[CL:Glossary:whitespace]]//\meaning{2} //[[CL:Glossary:characters]]//,
are to be treated as //[[CL:Glossary:alphabetic]]//\meaning{2} //[[CL:Glossary:characters]]// 
with //[[CL:Glossary:case]]// preserved.
Any //[[CL:Glossary:single escape]]// and //[[CL:Glossary:multiple escape]]// //[[CL:Glossary:characters]]// 
that are to appear in the sequence must be preceded by a //[[CL:Glossary:single escape]]// 
//[[CL:Glossary:character]]//.  

//[[CL:Glossary:Vertical-bar]]// is a //[[CL:Glossary:multiple escape]]// //[[CL:Glossary:character]]//
in //[[CL:Glossary:standard syntax]]//.

\beginsubsubsubsection{Examples of Multiple Escape Characters}

\code
 ;; The following examples assume the readtable case of *readtable* 
 ;; and *print-case* are both :upcase.
 (eq 'abc 'ABC) \EV //[[CL:Glossary:true]]//
 (eq 'abc '|ABC|) \EV //[[CL:Glossary:true]]//
 (eq 'abc 'a|B|c) \EV //[[CL:Glossary:true]]//
 (eq 'abc '|abc|) \EV //[[CL:Glossary:false]]//
\endcode

\endsubsubsubsection%{Examples of Multiple Escape Characters}

\endsubsubsection%{Multiple Escape Characters}

\beginsubsubsection{Single Escape Character}
\DefineSection{SingleEscapeChar}

A //[[CL:Glossary:single escape]]// is used to indicate that 
the next //[[CL:Glossary:character]]// is to be treated as 
an //[[CL:Glossary:alphabetic]]//\meaning{2} //[[CL:Glossary:character]]//
with its //[[CL:Glossary:case]]// preserved,
no matter what the //[[CL:Glossary:character]]// is 
or which //[[CL:Glossary:constituent traits]]// it has.  




//[[CL:Glossary:Backslash]]// is a //[[CL:Glossary:single escape]]// //[[CL:Glossary:character]]// in //[[CL:Glossary:standard syntax]]//.

\beginsubsubsubsection{Examples of Single Escape Characters}

\code
 ;; The following examples assume the readtable case of *readtable* 
 ;; and *print-case* are both :upcase.
 (eq 'abc '\\A\\B\\C) \EV //[[CL:Glossary:true]]//
 (eq 'abc 'a\\Bc) \EV //[[CL:Glossary:true]]//
 (eq 'abc '\\ABC) \EV //[[CL:Glossary:true]]//
 (eq 'abc '\\abc) \EV //[[CL:Glossary:false]]//
\endcode

\endsubsubsubsection%{Examples of Single Escape Characters}

\endsubsubsection%{Single Escape Character}

\beginsubsubsection{Whitespace Characters}
\DefineSection{WhitespaceChars}

//[[CL:Glossary:Whitespace]]//\meaning{2} //[[CL:Glossary:characters]]// are used to separate //[[CL:Glossary:tokens]]//.

//[[CL:Glossary:Space]]// and //[[CL:Glossary:newline]]// are //[[CL:Glossary:whitespace]]//\meaning{2} //[[CL:Glossary:characters]]//
in //[[CL:Glossary:standard syntax]]//.

\beginsubsubsubsection{Examples of Whitespace Characters}

\code
 (length '(this-that)) \EV 1
 (length '(this - that)) \EV 3
 (length '(a
           b)) \EV 2
 (+ 34) \EV 34
 (+ 3 4) \EV 7
\endcode

\endsubsubsubsection%{Examples of Whitespace Characters}

\endsubsubsection%{Whitespace Characters}

\endSubsection%{Character Syntax Types}
