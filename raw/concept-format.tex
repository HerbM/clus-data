


\issue{PRETTY-PRINT-INTERFACE}
\editornote{KMP: This is transplanted from FORMAT and will need a bit of work before
 it looks good standing alone. Bear with me.}


**[[CL:Functions:format]]** is useful for producing nicely formatted text, producing
good-looking messages, and so on.  **[[CL:Functions:format]]** can generate and return
a //[[CL:Glossary:string]]// or output to //destination//.

The //control-string// argument to **[[CL:Functions:format]]** is actually a //[[CL:Glossary:format control]]//.



That is, it can be either a //[[CL:Glossary:format string]]// or a //[[CL:Glossary:function]]//,
for example a //[[CL:Glossary:function]]// returned
by the \macref{formatter} //[[CL:Glossary:macro]]//.

If it is a //[[CL:Glossary:function]]//, the //[[CL:Glossary:function]]// is called with the appropriate
output stream as its first argument and the data arguments to **[[CL:Functions:format]]**
as its remaining arguments.  The function should perform whatever output is 
necessary and return the unused tail of the arguments (if any).

The compilation process performed by \macref{formatter} produces a //[[CL:Glossary:function]]//
that would do with its //[[CL:Glossary:arguments]]// as the **[[CL:Functions:format]]** interpreter
would do with those //[[CL:Glossary:arguments]]//.

The remainder of this section describes what happens if the //control-string//
is a //[[CL:Glossary:format string]]//.  











//Control-string// is composed of simple text (//[[CL:Glossary:characters]]//) 
and embedded directives.


**[[CL:Functions:format]]** writes the simple text as is;
each embedded directive specifies further text output 
that is to appear at the corresponding point within the simple text.  
Most directives use one or more elements of //args// to
create their output.


A directive consists of a //[[CL:Glossary:tilde]]//,
optional prefix parameters
separated by commas, optional //[[CL:Glossary:colon]]// and //[[CL:Glossary:at-sign]]// modifiers,
and a single character indicating what kind of directive this is.
\issue{FORMAT-ATSIGN-COLON}
There is no required ordering between the //[[CL:Glossary:at-sign]]// and //[[CL:Glossary:colon]]// modifier.

The //[[CL:Glossary:case]]// of the directive character is ignored.
Prefix parameters are notated as signed (sign is optional) decimal numbers,
or as a //[[CL:Glossary:single-quote]]// followed by a character.
For example, \f{~5,'0d} can be used
to print an //[[CL:Glossary:integer]]// 
in decimal radix in five columns with leading zeros,
or \f{~5,'*d} to get leading asterisks.


In place of a prefix parameter to a directive, \f{V} (or \f{v}) can be used. 
In this case, **[[CL:Functions:format]]** takes an argument from //args// as a parameter to
the directive.  The argument should be an //[[CL:Glossary:integer]]// or //[[CL:Glossary:character]]//.
If the //arg// used by a \f{V} parameter is \nil,
the effect is as if the parameter had been omitted.
\f{\#} can be used in place of a prefix parameter; it
represents the number of //args// remaining to be processed.
When used within a recursive format, in the context of \f{~?} or \f{~\{},
the \f{\#} prefix parameter represents the number of //[[CL:Glossary:format arguments]]//
remaining within the recursive call.

Examples of //[[CL:Glossary:format strings]]//:

\showtwo{Examples of format control strings}{
\f{"~S"}       & ;This is an S directive with no parameters or modifiers.       \cr
\f{"~3,-4:@s"} & ;This is an S directive with two parameters, \f{3} and \f{-4}, \cr
 	       & ; and both the //[[CL:Glossary:colon]]// and //[[CL:Glossary:at-sign]]// flags.          \cr
\f{"~,+4S"}    & ;Here the first prefix parameter is omitted and takes          \cr
 	       & ; on its default value, while the second parameter is \f{4}.   \cr
}

**[[CL:Functions:format]]** sends the output to //destination//.
If //destination// is \nil,
**[[CL:Functions:format]]** creates and returns a //[[CL:Glossary:string]]// 
containing the output from //control-string//.
If //destination// is //[[CL:Glossary:non-nil]]//, 
it must be a //[[CL:Glossary:string]]// with a //[[CL:Glossary:fill pointer]]//, 
a //[[CL:Glossary:stream]]//, or the symbol \t.
If //destination// is a //[[CL:Glossary:string]]// with a //[[CL:Glossary:fill pointer]]//, 
the output is added to the end of the //[[CL:Glossary:string]]//. 
If //destination// is a //[[CL:Glossary:stream]]//, 
the output is sent to that //[[CL:Glossary:stream]]//. 
If //destination// is \t,
the output is sent to //[[CL:Glossary:standard output]]//.






In the description of the directives that follows,
the term \j{arg} in general
refers to the next item of the set of //args// to be processed.
The word or phrase at the beginning of each description is a mnemonic
for the directive.
\issue{FORMAT-PRETTY-PRINT:YES}
**[[CL:Functions:format]]** directives do not bind any of the printer control
variables (\varref{*print-...*}) except as specified in the following
descriptions.
   Implementations may specify the binding of new, implementation-specific 
printer control variables for each **[[CL:Functions:format]]** directive, but they
    may neither bind any standard printer control variables not
    specified in description of a **[[CL:Functions:format]]** 
directive nor fail to bind
    any standard printer control variables as specified in the
    description.









\beginsubsection{FORMAT Basic Output}

\beginsubsubsection{Tilde C: Character}
\idxtext{C (format directive)}\idxtext{Tilde C (format directive)}



               
The next \j{arg} should be a //[[CL:Glossary:character]]//; 
it is printed
according to the modifier flags.



\issue{FORMAT-OP-C}
\f{~C} prints the //[[CL:Glossary:character]]// 
as if by using **[[CL:Functions:write-char]]** if it is a //[[CL:Glossary:simple character]]//.
//[[CL:Glossary:Characters]]// that are not //[[CL:Glossary:simple]]//
are not necessarily printed as if by **[[CL:Functions:write-char]]**,
but are displayed in an //[[CL:Glossary:implementation-defined]]//, abbreviated format.
For example,

\code
 (format nil "~C" #\\A) \EV "A"
 (format nil "~C" #\\Space) \EV " "
\endcode



\f{~:C} is the same as \f{~C} for //[[CL:Glossary:printing]]// //[[CL:Glossary:characters]]//,
but other //[[CL:Glossary:characters]]// are ``spelled out.''  The intent is that this
is a ``pretty'' format for printing characters.
For //[[CL:Glossary:simple]]// //[[CL:Glossary:characters]]// that are not //[[CL:Glossary:printing]]//,
what is spelled out is the //[[CL:Glossary:name]]// of the //[[CL:Glossary:character]]// (see **[[CL:Functions:char-name]]**).
For //[[CL:Glossary:characters]]// that are not //[[CL:Glossary:simple]]// and not //[[CL:Glossary:printing]]//,
what is spelled out is //[[CL:Glossary:implementation-defined]]//.
For example, 

\code
 (format nil "~:C" #\\A) \EV "A"
 (format nil "~:C" #\\Space) \EV "Space"
;; This next example assumes an implementation-defined "Control" attribute.
 (format nil "~:C" #\\Control-Space)
\EV "Control-Space"
\OV "c-Space"
\endcode


\f{~:@C} prints what \f{~:C} would, and then
if the //[[CL:Glossary:character]]// requires unusual shift keys on the keyboard to type it,
this fact is mentioned.  For example,

\begingroup
\def\Partial{$\partial$}
\code
 (format nil "~:@C" #\\Control-Partial) \EV "Control-{\Partial} (Top-F)"  
\endcode
\endgroup

This is the format used for telling the user about a key he is expected to type,
in prompts, for instance.  The precise output may depend not only
on the implementation, but on the particular I/O devices in use.
                           

\f{~@C} 
prints the //[[CL:Glossary:character]]// in a way that the //[[CL:Glossary:Lisp reader]]// can understand,
using \f{\#\\} syntax.

\issue{FORMAT-PRETTY-PRINT:YES}
\f{~@C}  binds \varref{*print-escape*} to \t.


\endsubsubsection%{Tilde C: Character}

\beginsubsubsection{Tilde Percent: Newline}
\idxtext{Percent (format directive)}\idxtext{Tilde Percent (format directive)}



This outputs a \f{\#\\Newline} character, thereby terminating the current
output line and beginning a new one.
\f{~\j{n}\%} outputs \j{n} newlines.
No \j{arg} is used.  

\endsubsubsection%{Tilde Percent: Newline}

\beginsubsubsection{Tilde Ampersand: Fresh-Line}
\idxtext{Ampersand (format directive)}\idxtext{Tilde Ampersand (format directive)}



Unless it can be determined that the output stream
is already at the beginning of a line,
this outputs a newline.
\f{~\j{n}\&} calls **[[CL:Functions:fresh-line]]**
and then outputs \j{n}\minussign 1 newlines.
\f{~0\&} does nothing.

\endsubsubsection%{Tilde Ampersand: Fresh-Line}

\beginsubsubsection{Tilde Vertical-Bar: Page}
\idxtext{Vertical-Bar (format directive)}\idxtext{Tilde Vertical-Bar (format directive)}



This outputs a page separator character, if possible.
\f{~\j{n}|} does this \j{n} times.

\endsubsubsection%{Tilde Vertical-Bar: Page}



\beginsubsubsection{Tilde Tilde: Tilde}
\idxtext{Tilde (format directive)}\idxtext{Tilde Tilde (format directive)}

This outputs a //[[CL:Glossary:tilde]]//.  \f{~\j{n}~} outputs \j{n} tildes.

\endsubsubsection%{Tilde Tilde: Tilde}

\endsubsection%{FORMAT Basic Output}

\beginsubsection{FORMAT Radix Control}

\beginsubsubsection{Tilde R: Radix}
\idxtext{R (format directive)}\idxtext{Tilde R (format directive)}



\f{~\j{n}R} prints \j{arg} in radix \j{n}.
The modifier flags and any remaining parameters are used as for
the \f{~D} directive.
\f{~D} is the same as \f{~10R}.  
The full form is 
\f{~\j{radix},\j{mincol},\j{padchar},\j{commachar},\j{comma-interval}R}.


If no prefix parameters are given to \f{~R}, then a different
interpretation is given.  The argument should be an //[[CL:Glossary:integer]]//.
For example, if \j{arg} is 4:


\beginlist
\itemitem{\bull}
\f{~R} prints \j{arg} as a cardinal English number: \f{four}.


\itemitem{\bull}
\f{~:R} prints \j{arg} as an ordinal English number: \f{fourth}.


\itemitem{\bull}   
\f{~@R} prints \j{arg} as a Roman numeral: \f{IV}.


\itemitem{\bull}      
\f{~:@R} prints \j{arg} as an old Roman numeral: \f{IIII}.
\endlist

\issue{FORMAT-COMMA-INTERVAL}
For example:

\code
 (format nil "~,,' ,4:B" 13) \EV "1101"
 (format nil "~,,' ,4:B" 17) \EV "1 0001"
 (format nil "~19,0,' ,4:B" 3333) \EV "0000 1101 0000 0101"
 (format nil "~3,,,' ,2:R" 17) \EV "1 22"
 (format nil "~,,'|,2:D" #xFFFF) \EV  "6|55|35"
\endcode


\issue{FORMAT-PRETTY-PRINT:YES}
If and only if the first parameter, \j{n}, is supplied,
\f{~R} binds
     \varref{*print-escape*} to //[[CL:Glossary:false]]//,
     \varref{*print-radix*} to //[[CL:Glossary:false]]//, 
     \varref{*print-base*} to \j{n},
\issue{PRINC-READABLY:X3J13-DEC-91}
 and \varref{*print-readably*} to //[[CL:Glossary:false]]//.


If and only if no parameters are supplied,
\f{~R} binds \varref{*print-base*} to \f{10}.


\endsubsubsection%{Tilde R: Radix}

\beginsubsubsection{Tilde D: Decimal}
\idxtext{D (format directive)}\idxtext{Tilde D (format directive)}



An \j{arg}, which should be an //[[CL:Glossary:integer]]//, 
is printed in decimal radix.
\f{~D} will never put a decimal point after the number.


\f{~\j{mincol}D} uses 
a column width of \j{mincol}; spaces are inserted on
the left if the number requires fewer than \j{mincol} columns for its digits
and sign.  If the number doesn't fit in \j{mincol} columns, additional columns
are used as needed.


\f{~\j{mincol},\j{padchar}D} uses \j{padchar} as the pad character
instead of space.


If \j{arg} is not an //[[CL:Glossary:integer]]//, it is printed in \f{~A} format and decimal base.


The \f{@} modifier causes the number's sign to be printed always; the default
is to print it only if the number is negative.
\issue{FORMAT-COMMA-INTERVAL}
The \f{:} modifier causes commas to be printed between groups of digits;
\j{commachar} may be used to change the character used as the comma.
\j{comma-interval} 
must be an //[[CL:Glossary:integer]]// and defaults to 3.  When the \f{:} 
modifier is given to any of
these directives, the \j{commachar} 
is printed between groups of \j{comma-interval}
digits.


Thus the most general form of \f{~D} is
\f{~\j{mincol},\j{padchar},\j{commachar},\j{comma-interval}D}.

\issue{FORMAT-PRETTY-PRINT:YES}
\f{~D} binds
     \varref{*print-escape*} to //[[CL:Glossary:false]]//,
     \varref{*print-radix*} to //[[CL:Glossary:false]]//,
     \varref{*print-base*} to \f{10},
\issue{PRINC-READABLY:X3J13-DEC-91}
 and \varref{*print-readably*} to //[[CL:Glossary:false]]//.



\endsubsubsection%{Tilde D: Decimal}

\beginsubsubsection{Tilde B: Binary}
\idxtext{B (format directive)}\idxtext{Tilde B (format directive)}



This is just like \f{~D} but prints in binary radix (radix 2)
instead of decimal.  The full form is therefore
\f{~\j{mincol},\j{padchar},\j{commachar},\j{comma-interval}B}.

\issue{FORMAT-PRETTY-PRINT:YES}
\f{~B} binds
     \varref{*print-escape*} to //[[CL:Glossary:false]]//,
     \varref{*print-radix*} to //[[CL:Glossary:false]]//,
     \varref{*print-base*} to \f{2},
\issue{PRINC-READABLY:X3J13-DEC-91}
 and \varref{*print-readably*} to //[[CL:Glossary:false]]//.



\endsubsubsection%{Tilde B: Binary}

\beginsubsubsection{Tilde O: Octal}
\idxtext{O (format directive)}\idxtext{Tilde O (format directive)}



This is just like \f{~D} but prints in octal radix (radix 8)
instead of decimal.  The full form is therefore
\f{~\j{mincol},\j{padchar},\j{commachar},\j{comma-interval}O}.

\issue{FORMAT-PRETTY-PRINT:YES}
\f{~O} binds
     \varref{*print-escape*} to //[[CL:Glossary:false]]//,
     \varref{*print-radix*} to //[[CL:Glossary:false]]//,
     \varref{*print-base*} to \f{8},
\issue{PRINC-READABLY:X3J13-DEC-91}
 and \varref{*print-readably*} to //[[CL:Glossary:false]]//.



\endsubsubsection%{Tilde O: Octal}

\beginsubsubsection{Tilde X: Hexadecimal}
\idxtext{X (format directive)}\idxtext{Tilde X (format directive)}



This is just like \f{~D} but prints in hexadecimal radix
(radix 16) instead of decimal.  The full form is therefore
\f{~\j{mincol},\j{padchar},\j{commachar},\j{comma-interval}X}.

\issue{FORMAT-PRETTY-PRINT:YES}
\f{~X} binds
     \varref{*print-escape*} to //[[CL:Glossary:false]]//,
     \varref{*print-radix*} to //[[CL:Glossary:false]]//,
     \varref{*print-base*} to \f{16},
\issue{PRINC-READABLY:X3J13-DEC-91}
 and \varref{*print-readably*} to //[[CL:Glossary:false]]//.



\endsubsubsection%{Tilde X: Hexadecimal}

\endsubsection%{FORMAT Radix Control}

\beginsubsection{FORMAT Floating-Point Printers}

\beginsubsubsection{Tilde F: Fixed-Format Floating-Point}
\idxtext{F (format directive)}\idxtext{Tilde F (format directive)}



The next \j{arg} is printed as a //[[CL:Glossary:float]]//.


The full form is \f{~\j{w},\j{d},\j{k},\j{overflowchar},\j{padchar}F}.
The parameter \j{w}
is the width of the field to be printed; \j{d} is the number
of digits to print after the decimal point; \j{k} is a scale factor
that defaults to zero.


Exactly \j{w} characters will
be output.  First, leading copies of the character \j{padchar}
(which defaults to a space) are printed, if necessary, to pad the
field on the left.
If the \j{arg} is negative, then a minus sign is printed;
if the \j{arg} is not negative, then a plus sign is printed
if and only if the \f{@}
modifier was supplied.  Then a sequence
of digits, containing a single embedded decimal point, is printed;
this represents the magnitude of the value of \j{arg} times $10^\j{k}$,
rounded to \j{d} fractional digits.                         
When rounding up and rounding down would produce printed values
equidistant from the scaled value of \j{arg}, then the implementation
is free to use either one.  For example, printing the argument
\f{6.375} using the format \f{~4,2F} may correctly produce
either \f{6.37} or \f{6.38}.
Leading zeros are not permitted, except that a single
zero digit is output before the decimal point if the printed value
is less than one, and this single zero digit is not output
at all if \j{w}=\j{d}+1.


If it is impossible to print the value in the required format in a field
of width \j{w}, then one of two actions is taken.  If the
parameter \j{overflowchar} is supplied, then \j{w} copies of that
parameter are printed instead of the scaled value of \j{arg}.
If the \j{overflowchar} parameter is omitted, then the scaled value
is printed using more than \j{w} characters, as many more as may be
needed.


If the \j{w} parameter is omitted, then the field is of variable width.
In effect, a value is chosen
for \j{w} in such a way that no leading pad characters need to be printed
and exactly \j{d} characters will follow the decimal point.
For example, the directive \f{~,2F} will print exactly
two digits after the decimal point and as many as necessary before the
decimal point.


If the parameter \j{d} is omitted, then there is no constraint
on the number of digits to appear after the decimal point.
A value is chosen for \j{d} in such a way that as many digits
as possible may be printed subject to the width constraint
imposed by the parameter \j{w} and the constraint that no trailing
zero digits may appear in the fraction, except that if the
fraction to be printed is zero, then a single zero digit should
appear after the decimal point if permitted by the width constraint.


If both \j{w} and \j{d} are omitted, then the effect is to print
the value using ordinary free-format output; **[[CL:Functions:prin1]]** uses this format
for any number whose magnitude is either zero or between
$10^{-3}$ (inclusive) and $10^7$ (exclusive).


If \j{w} is omitted, then if the magnitude of \j{arg} is so large (or, if
\j{d} is also omitted, so small) that more than 100 digits would have to
be printed, then an implementation is free, at its discretion, to print
the number using exponential notation instead, as if by the directive
\f{~E} (with all parameters to \f{~E} defaulted, not
taking their values from the \f{~F} directive).


If \j{arg} is a //[[CL:Glossary:rational]]// 
number, then it is coerced to be a //[[CL:Glossary:single float]]//
and then printed.  Alternatively, an implementation is permitted to
process a //[[CL:Glossary:rational]]// 
number by any other method that has essentially the
same behavior but avoids loss of precision or overflow
because of the coercion.  If \j{w} and \j{d} are
not supplied and the number has no exact decimal representation,
for example \f{1/3}, some precision cutoff must be chosen
by the implementation since only a finite number of digits may be printed.


If \j{arg} is a //[[CL:Glossary:complex]]// number or some non-numeric
//[[CL:Glossary:object]]//, 
then it is printed using the format directive \f{~\j{w}D},
thereby printing it in decimal radix and a minimum field width of \j{w}.

\issue{FORMAT-PRETTY-PRINT:YES}
\f{~F} binds
    \varref{*print-escape*} to //[[CL:Glossary:false]]//
\issue{PRINC-READABLY:X3J13-DEC-91}
 and \varref{*print-readably*} to //[[CL:Glossary:false]]//.



\endsubsubsection%{Tilde F: Fixed-Format Floating-Point}

\beginsubsubsection{Tilde E: Exponential Floating-Point}
\idxtext{E (format directive)}\idxtext{Tilde E (format directive)}



The next \j{arg} is printed as a //[[CL:Glossary:float]]// in exponential notation.


The full form is
\f{~\j{w},\j{d},\j{e},\j{k},\j{overflowchar},\j{padchar},\j{exponentchar}E}.
The parameter \j{w}
is the width of the field to be printed; \j{d} is the number
of digits to print after the decimal point; \j{e} is the number
of digits to use when printing the exponent;
\j{k} is a scale factor that defaults to one (not zero).


Exactly \j{w} characters will
be output.  First, leading copies of the character \j{padchar}
(which defaults to a space) are printed, if necessary, to pad the
field on the left.
If the \j{arg} is negative, then a minus sign is printed;
if the \j{arg} is not negative, then a plus sign is printed
if and only if the \f{@}
modifier was supplied.  Then a sequence
of digits containing a single embedded decimal point is printed.
The form of this sequence of digits depends on the scale factor \j{k}.
If \j{k} is zero, then \j{d} digits are printed after the decimal
point, and a single zero digit appears before the decimal point if
the total field width will permit it.  If \j{k} is positive,
then it must be strictly less than \j{d}+2;  \j{k} significant digits
are printed before the decimal point, and \j{d}\minussign \j{k}+1
digits are printed after the decimal point.  If \j{k} is negative,
then it must be strictly greater than \minussign \j{d};
a single zero digit appears before the decimal point if
the total field width will permit it, and after the decimal point
are printed first
\minussign \j{k} zeros and then \j{d}+\j{k} significant digits.
The printed fraction must be properly rounded.         
When rounding up and rounding down would produce printed values
equidistant from the scaled value of \j{arg}, then the implementation
is free to use either one.  For example, printing the argument
\f{637.5} using the format \f{~8,2E} may correctly produce
either \f{6.37E+2} or \f{6.38E+2}.


Following the digit sequence, the exponent is printed.
First the character parameter \j{exponentchar} is printed; if this
parameter is omitted, then the //[[CL:Glossary:exponent marker]]// that
**[[CL:Functions:prin1]]** would use is printed, as determined from the
type of the //[[CL:Glossary:float]]// and the current value of
\varref{*read-default-float-format*}.
Next, either a plus sign or a minus sign
is printed, followed by \j{e} digits representing the power of
ten by which the printed fraction must be multiplied
to properly represent the rounded value of \j{arg}.


If it is impossible to print the value in the required format in a field
of width \j{w}, possibly because \j{k} is too large or too small
or because the exponent cannot be printed in \j{e} character positions,
then one of two actions is taken.  If the
parameter \j{overflowchar} is supplied, then \j{w} copies of that
parameter are printed instead of the scaled value of \j{arg}.
If the \j{overflowchar} parameter is omitted, then the scaled value
is printed using more than \j{w} characters, as many more as may be
needed; if the problem is that \j{d} is too small for the supplied \j{k}
or that \j{e} is too small, then a larger value is used for \j{d} or \j{e}
as may be needed.


If the \j{w} parameter is omitted, then the field is of variable width.
In effect a value is chosen
for \j{w} in such a way that no leading pad characters need to be printed.


If the parameter \j{d} is omitted, then there is no constraint
on the number of digits to appear.
A value is chosen for \j{d} in such a way that as many digits
as possible may be printed subject to the width constraint
imposed by the parameter \j{w}, the constraint of the scale factor \j{k},
and the constraint that no trailing
zero digits may appear in the fraction, except that if the
fraction to be printed is zero then a single zero digit should
appear after the decimal point.


If the parameter \j{e} is omitted, then the exponent is printed
using the smallest number of digits necessary to represent its value.


If all of \j{w}, \j{d}, and \j{e} are omitted, then the effect is to print
the value using ordinary free-format exponential-notation output;
**[[CL:Functions:prin1]]** uses 
\issue{FORMAT-E-EXPONENT-SIGN:FORCE-SIGN}
a similar

format for any non-zero number whose magnitude
is less than $10^{-3}$ or greater than or equal to $10^7$.
\issue{FORMAT-E-EXPONENT-SIGN:FORCE-SIGN}
The only difference is that the \f{~E} 
directive always prints a plus or minus sign in front of the
    exponent, while **[[CL:Functions:prin1]]** omits the plus sign if the exponent is
    non-negative.



If \j{arg} is a //[[CL:Glossary:rational]]// 
number, then it is coerced to be a //[[CL:Glossary:single float]]//
and then printed.  Alternatively, an implementation is permitted to
process a //[[CL:Glossary:rational]]// 
number by any other method that has essentially the
same behavior but avoids loss of precision or overflow
because of the coercion.  If \j{w} and \j{d} are
unsupplied and the number has no exact decimal representation,
for example \f{1/3}, some precision cutoff must be chosen
by the implementation since only a finite number of digits may be printed.


If \j{arg} is a //[[CL:Glossary:complex]]// number or some non-numeric
//[[CL:Glossary:object]]//, 
then it is printed using the format directive \f{~\j{w}D},
thereby printing it in decimal radix and a minimum field width of \j{w}.

\issue{FORMAT-PRETTY-PRINT:YES}
\f{~E} binds
     \varref{*print-escape*} to //[[CL:Glossary:false]]//
\issue{PRINC-READABLY:X3J13-DEC-91}
 and \varref{*print-readably*} to //[[CL:Glossary:false]]//.



\endsubsubsection%{Tilde E: Exponential Floating-Point}

\beginsubsubsection{Tilde G: General Floating-Point}
\idxtext{G (format directive)}\idxtext{Tilde G (format directive)}



The next \j{arg} is printed as a //[[CL:Glossary:float]]// 
in either fixed-format or exponential notation as appropriate.


The full form is \f{~\j{w},\j{d},\j{e},\j{k},\j{overflowchar},\j{padchar},\j{exponentchar}G}.
The format in which to print \j{arg} depends on the magnitude (absolute
value) of the \j{arg}.  Let \j{n} be an integer such that
$10^{\j{n}-1}$ $\le$ |\j{arg}| < $10^\j{n}$.
Let \j{ee} equal \j{e}+2, or 4 if \j{e} is omitted.
Let \j{ww} equal \j{w}\minussign \j{ee},
or \nil\ if \j{w} is omitted.  If \j{d} is omitted, first let \j{q}
be the number of digits needed to print \j{arg} with no loss
of information and without leading or trailing zeros;
then let \j{d} equal \f{(max \j{q} (min \j{n} 7))}.
Let \j{dd} equal \j{d}\minussign \j{n}.


If 0 $\le$ \j{dd} $\le$ \j{d}, then \j{arg} is printed
as if by the format directives
                                                                         

\f{~\j{ww},\j{dd},,\j{overflowchar},\j{padchar}F~\j{ee}@T}

Note that the scale factor \j{k} is not passed to the \f{~F}
directive.  For all other values of \j{dd}, \j{arg} is printed as if
by the format directive

\f{~\j{w},\j{d},\j{e},\j{k},\j{overflowchar},\j{padchar},\j{exponentchar}E}


In either case, an \f{@}
modifier is supplied to the \f{~F}
or \f{~E} directive if and only if one was supplied to the
\f{~G} directive.

\issue{FORMAT-PRETTY-PRINT:YES}
\f{~G} binds
     \varref{*print-escape*} to //[[CL:Glossary:false]]//
\issue{PRINC-READABLY:X3J13-DEC-91}
 and \varref{*print-readably*} to //[[CL:Glossary:false]]//.



\endsubsubsection%{Tilde G: General Floating-Point}

\beginsubsubsection{Tilde Dollarsign: Monetary Floating-Point}
\idxtext{Dollarsign (format directive)}\idxtext{Tilde Dollarsign (format directive)}



The next \j{arg} is printed as a //[[CL:Glossary:float]]// in fixed-format notation.  


The full form is \f{~\j{d},\j{n},\j{w},\j{padchar}\$}.
The parameter \j{d} is the number
of digits to print after the decimal point (default value 2);
\j{n} is the minimum number of digits to print before the decimal
point (default value 1);
\j{w} is the minimum total width of the field to be printed (default
value 0).


First padding and the sign are output.
If the \j{arg} is negative, then a minus sign is printed;
if the \j{arg} is not negative, then a plus sign is printed
if and only if the \f{@} modifier was supplied.  
If the \f{:} modifier is used, the sign appears before any padding,
and otherwise after the padding.
If \j{w} is supplied and the number of other characters to be output
is less than \j{w}, then copies of \j{padchar} (which defaults
to a space) are output to
make the total field width equal \j{w}.
Then \j{n} digits are printed for the integer part of \j{arg},
with leading zeros if necessary; then a decimal point;
then \j{d} digits of fraction, properly rounded.


If the magnitude of \j{arg} is so large that more than \j{m} digits would
have to be printed, where \j{m} is the larger of \j{w} and 100, then an
implementation is free, at its discretion, to print the number using
exponential notation instead, as if by the directive
\f{~\j{w},\j{q},,,,\j{padchar}E}, where \j{w} and \j{padchar} are
present or omitted according to whether they were present or omitted in
the \f{~\$} directive, and where \j{q}=\j{d}+\j{n}\minussign 1,
where \j{d} and \j{n} are the (possibly default) values given to the
\f{~\$} directive.


If \j{arg} is a //[[CL:Glossary:rational]]// 
number, then it is coerced to be a //[[CL:Glossary:single float]]//
and then printed.  Alternatively, an implementation is permitted to
process a //[[CL:Glossary:rational]]// number by any 
other method that has essentially the
same behavior but avoids loss of precision or overflow
because of the coercion.


If \j{arg} is a //[[CL:Glossary:complex]]// number or some non-numeric
//[[CL:Glossary:object]]//,
then it is printed using the format directive \f{~\j{w}D},
thereby printing it in decimal radix and a minimum field width of \j{w}.

\issue{FORMAT-PRETTY-PRINT:YES}
\f{~\$} binds \varref{*print-escape*} to //[[CL:Glossary:false]]//
\issue{PRINC-READABLY:X3J13-DEC-91}
 and \varref{*print-readably*} to //[[CL:Glossary:false]]//.




\endsubsubsection%{Tilde Dollarsign: Monetary Floating-Point}

\endsubsection%{FORMAT Floating-Point Printers}

\beginsubsection{FORMAT Printer Operations}
\DefineSection{FORMATPrinterOps}

\beginsubsubsection{Tilde A: Aesthetic}
\idxtext{A (format directive)}\idxtext{Tilde A (format directive)}






An \j{arg}, any //[[CL:Glossary:object]]//, 
is printed without escape characters
(as by **[[CL:Functions:princ]]**).  If \j{arg} is a //[[CL:Glossary:string]]//, 
its //[[CL:Glossary:characters]]//
will be output verbatim.
If \j{arg} is \nil\ it will be printed as \nil;
the //[[CL:Glossary:colon]]// modifier (\f{~:A}) will cause an \j{arg} of \nil\ to be printed as \empty,
but if \j{arg} is a composite structure, such as a //[[CL:Glossary:list]]// or //[[CL:Glossary:vector]]//,
any contained occurrences of \nil\ will still be printed as \nil.


\f{~\j{mincol}A} inserts spaces on the right, if necessary, to make the
width at least \j{mincol} columns.  The \f{@}
modifier causes the spaces
to be inserted on the left rather than the right.


\f{~\j{mincol},\j{colinc},\j{minpad},\j{padchar}A} 
is the full form of \f{~A},
which allows control of the padding.
The //[[CL:Glossary:string]]// is padded on the right (or on the left if the
\f{@} modifier is used) with at least \j{minpad} copies
of \j{padchar}; padding characters are then inserted \j{colinc} characters
at a time until the total width is at least \j{mincol}.
The defaults are \f{0} for \j{mincol} and \j{minpad}, \f{1} for \j{colinc},
and the space character for \j{padchar}.

\issue{FORMAT-PRETTY-PRINT:YES}
\f{~A} binds \varref{*print-escape*} to //[[CL:Glossary:false]]//,
\issue{PRINC-READABLY:X3J13-DEC-91}
and \varref{*print-readably*} to //[[CL:Glossary:false]]//.



\endsubsubsection%{Tilde A: Aesthetic}

\beginsubsubsection{Tilde S: Standard}
\idxtext{S (format directive)}\idxtext{Tilde S (format directive)}







This is just like \f{~A}, but \j{arg} is printed with escape
characters (as by **[[CL:Functions:prin1]]** rather than \f{princ}).  The output is
therefore suitable for input to **[[CL:Functions:read]]**.  \f{~S} accepts
all the arguments and modifiers that \f{~A} does.

\issue{FORMAT-PRETTY-PRINT:YES}
\f{~S} binds \varref{*print-escape*} to \t.


\endsubsubsection%{Tilde S: Standard}

\issue{PRETTY-PRINT-INTERFACE}
\beginsubsubsection{Tilde W: Write}
\idxtext{W (format directive)}\idxtext{Tilde W (format directive)}

An argument, any //[[CL:Glossary:object]]//, is printed obeying every printer control
variable (as by **[[CL:Functions:write]]**).  In addition, \f{~W} interacts correctly with depth
abbreviation, by not resetting the depth counter to zero.  \f{~W} does not
accept parameters.  If given the //[[CL:Glossary:colon]]// modifier, \f{~W} binds \varref{*print-pretty*}
to //[[CL:Glossary:true]]//.  If given the //[[CL:Glossary:at-sign]]// modifier, \f{~W} binds \varref{*print-level*}
and \varref{*print-length*} to \nil.
 
\f{~W} provides automatic support for the detection of circularity and
sharing.  If \thevalueof{*print-circle*} is not \nil\ and \f{~W} is applied
to an argument that is a circular (or shared) reference, an appropriate 
\f{\#//n//\#} marker is inserted in the output instead of printing the argument.
 
\endsubsubsection%{Tilde W: Write}


\endsubsection%{FORMAT Printer Operations}

\beginsubsection{FORMAT Pretty Printer Operations}

\issue{PRETTY-PRINT-INTERFACE}

The following constructs provide access to the //[[CL:Glossary:pretty printer]]//:

\beginsubsubsection{Tilde Underscore: Conditional Newline}
\DefineSection{TildeUnderscore}
\idxtext{Underscore (format directive)}\idxtext{Tilde Underscore (format directive)}

Without any modifiers, \f{~_} is the same as \f{(pprint-newline :linear)}.
\f{~@_}  is the same as \f{(pprint-newline :miser)}.
\f{~:_}  is the same as \f{(pprint-newline :fill)}.
\f{~:@_} is the same as \f{(pprint-newline :mandatory)}.

\endsubsubsection%{Tilde Underscore: Conditional Newline}

\beginsubsubsection{Tilde Less-Than-Sign: Logical Block}
\DefineSection{TildeLessThanLogicalBlock}
\idxtext{Less-Than-Sign (format directive)}\idxtext{Tilde Less-Than-Sign (format directive)}

\f{~<...~:>}

If \f{~:>} is used to terminate a \f{~<...~>},
the directive is equivalent to a call to \macref{pprint-logical-block}.
The argument corresponding to the \f{~<...~:>} directive is treated in
the same way as the //[[CL:Glossary:list]]// argument to **[[CL:Functions:pprint-logical-block]]**,
thereby providing automatic support for non-//[[CL:Glossary:list]]// arguments and
the detection of circularity, sharing, and depth abbreviation.  
The portion of the //control-string// nested within the \f{~<...~:>}
specifies the **'':prefix''** (or **'':per-line-prefix''**), **'':suffix''**,
and body of the \macref{pprint-logical-block}.
 
The //control-string// portion enclosed by \f{~<...~:>} can be divided
into segments \f{~<//prefix//~;//body//~;//suffix//~:>}
by \f{~;} directives.  If the first section is terminated by \f{~@;}, 
it specifies a per-line prefix rather than a simple prefix.  
The //prefix// and //suffix// cannot contain format directives.  
An error is signaled if either the prefix or suffix fails to be a
constant string or if the enclosed portion is divided into more than three segments.
 
If the enclosed portion is divided into only two segments, the //suffix//
defaults to the null string.  If the enclosed portion consists of only
a single segment, both the //prefix// and the //suffix// default to 
the null string.  If the //[[CL:Glossary:colon]]// modifier is used (\ie \f{~:<...~:>}),
the //prefix// and //suffix// default to \f{"("} and \f{")"}
(respectively) instead of the null string.
 
The body segment can be any arbitrary //[[CL:Glossary:format string]]//.
This //[[CL:Glossary:format string]]// is applied to the elements of the list
corresponding to the \f{~<...~:>} directive as a whole.
Elements are extracted from this list using \macref{pprint-pop},
thereby providing automatic support for malformed lists, and the detection
of circularity, sharing, and length abbreviation.
Within the body segment, \f{~{\hat}} acts like \macref{pprint-exit-if-list-exhausted}.
 
\f{~<...~:>} supports a feature not supported by \macref{pprint-logical-block}.
If \f{~:@>} is used to terminate the directive (\ie \f{~<...~:@>}), 
then a fill-style conditional newline is automatically inserted after each
group of blanks immediately contained in the body (except for blanks
after a ~\NewlineChar\ directive).  This makes it easy to achieve the
equivalent of paragraph filling.
 
If the //[[CL:Glossary:at-sign]]// modifier is used with \f{~<...~:>}, the entire remaining argument
list is passed to the directive as its argument.  All of the remaining
arguments are always consumed by \f{~@<...~:>}, even if they are not all used
by the //[[CL:Glossary:format string]]// nested in the directive.  Other than the difference in
its argument, \f{~@<...~:>} is exactly the same as \f{~<...~:>} except that
circularity detection is not applied if \f{~@<...~:>} is encountered at top
level in a //[[CL:Glossary:format string]]//.  This ensures that circularity detection is
applied only to data lists, not to //[[CL:Glossary:format argument]]// //[[CL:Glossary:lists]]//.

\f{" . \#//n//\#"} is printed if circularity or sharing has to be indicated
for its argument as a whole.
 
To a considerable extent, the basic form of the directive \f{~<...~>} is
incompatible with the dynamic control of the arrangement of output by
\f{~W}, \f{~_}, \f{~<...~:>}, \f{~I}, and \f{~:T}.  As a result, an error 
is signaled if any of these directives is nested within \f{~<...~>}.  
Beyond this, an error is also signaled if the \f{~<...~:;...~>} form of
\f{~<...~>} is used in the same //[[CL:Glossary:format string]]// with 
\f{~W}, \f{~_}, \f{~<...~:>}, \f{~I}, or \f{~:T}.
 
See also \secref\TildeLessThanJustification.

\endsubsubsection%{Tilde Less-Than-Sign: Logical Block}

\beginsubsubsection{Tilde I: Indent}
\DefineSection{TildeI}
\idxtext{I (format directive)}\idxtext{Tilde I (format directive)}
 
\f{~//n//I}  is the same as \f{(pprint-indent :block n)}.

\f{~//n//:I} is the same as \f{(pprint-indent :current n)}.
In both cases, //n// defaults to zero, if it is omitted.
 
\endsubsubsection%{Tilde I: Indent}

\beginsubsubsection{Tilde Slash: Call Function}
\idxtext{Slash (format directive)}\idxtext{Tilde Slash (format directive)}

\f{~///name///}

User defined functions can be called from within a format
string by using the directive \f{~///name///}.
The //[[CL:Glossary:colon]]// modifier, the //[[CL:Glossary:at-sign]]// modifier, and arbitrarily many parameters 
can be specified with the \f{~///name///} directive.
//name// can be any arbitrary string that does not contain a "/".
All of the characters in //name// are treated as if they were upper case.
If //name// contains a single //[[CL:Glossary:colon]]// (\f{:}) or double //[[CL:Glossary:colon]]// (\f{::}),
then everything up to but not including the first \f{":"} or \f{"::"}
is taken to be a //[[CL:Glossary:string]]// that names a //[[CL:Glossary:package]]//.
Everything after the first \f{":"} or \f{"::"} (if any) is taken to be a 
//[[CL:Glossary:string]]// that names a \f{symbol}.  The function corresponding to a 
\f{~/name/} directive is obtained by looking up the //[[CL:Glossary:symbol]]//
that has the indicated name in the indicated //[[CL:Glossary:package]]//.
If //name// does not contain a \f{":"} or \f{"::"},

then the whole //name// string is looked up in \thepackage{common-lisp-user}. 

When a \f{~/name/} directive is encountered,
the indicated function is called with four or more arguments.
The first four arguments are:
     the output stream,
     the //[[CL:Glossary:format argument]]// corresponding to the directive,
     a //[[CL:Glossary:generalized boolean]]// that is //[[CL:Glossary:true]]// if the //[[CL:Glossary:colon]]// modifier was used,
 and a //[[CL:Glossary:generalized boolean]]// that is //[[CL:Glossary:true]]// if the //[[CL:Glossary:at-sign]]// modifier was used.
The remaining arguments consist of any parameters specified with the directive.
The function should print the argument appropriately.
Any values returned by the function are ignored.
 
The three //[[CL:Glossary:functions]]// 
      **[[CL:Functions:pprint-linear]]**,
      **[[CL:Functions:pprint-fill]]**,
  and **[[CL:Functions:pprint-tabular]]**
are specifically designed so that they can be called by \f{~/.../}
(\ie \f{~/pprint-linear/}, \f{~/pprint-fill/}, and \f{~/pprint-tabular/}).
In particular they take //[[CL:Glossary:colon]]// and //[[CL:Glossary:at-sign]]// arguments.



\endsubsubsection%{Tilde Slash: Call Function}

\endsubsection%{FORMAT Pretty Printer Operations}

\beginsubsection{FORMAT Layout Control}

\beginsubsubsection{Tilde T: Tabulate}
\idxtext{T (format directive)}\idxtext{Tilde T (format directive)}



This spaces over to a given column.
\f{~\j{colnum},\j{colinc}T} will output
sufficient spaces to move the cursor to column \j{colnum}.  If the cursor
is already at or beyond column \j{colnum}, it will output spaces to move it to
column \j{colnum}+\j{k}*\j{colinc} for the smallest positive integer
\j{k} possible, unless \j{colinc} is zero, in which case no spaces
are output if the cursor is already at or beyond column \j{colnum}.
\j{colnum} and \j{colinc} default to \f{1}.


If for some reason the current absolute column position cannot be determined
by direct inquiry,
**[[CL:Functions:format]]** 
may be able to deduce the current column position by noting         
that certain directives (such as \f{~\%}, or \f{~\&},
or \f{~A} 
with the argument being a string containing a newline) cause
the column position to be reset to zero, and counting the number of characters
emitted since that point.  If that fails, **[[CL:Functions:format]]** 
may attempt a
similar deduction on the riskier assumption that the destination was
at column zero when **[[CL:Functions:format]]** 
was invoked.  If even this heuristic fails
or is implementationally inconvenient, at worst
the \f{~T} operation will simply output two spaces.


\f{~@T} performs relative tabulation.
\f{~\j{colrel},\j{colinc}@T} outputs \j{colrel} spaces
and then outputs the smallest non-negative
number of additional spaces necessary to move the cursor
to a column that is a multiple                                       
of \j{colinc}.  For example, the directive 
\f{~3,8@T} outputs
three spaces and then moves the cursor to a ``standard multiple-of-eight
tab stop'' if not at one already.
If the current output column cannot be determined, however,
then \j{colinc} is ignored, and exactly \j{colrel} spaces are output.

If the //[[CL:Glossary:colon]]// modifier is used with the \f{~T} directive,
the tabbing computation is done relative to the horizontal position where the
section immediately containing the directive begins, rather than with
respect to a horizontal position of zero.  The numerical parameters are
both interpreted as being in units of //[[CL:Glossary:ems]]// and both default to \f{1}.
\f{~//n//,//m//:T} is the same as 
  \f{(pprint-tab :section //n// //m//)}.
\f{~//n//,//m//:@T} is the same as
  \f{(pprint-tab :section-relative //n// //m//)}.

\endsubsubsection%{Tilde T: Tab}

\beginsubsubsection{Tilde Less-Than-Sign: Justification}
\DefineSection{TildeLessThanJustification}
\idxtext{Less-Than-Sign (format directive)}\idxtext{Tilde Less-Than-Sign (format directive)}


\f{~\j{mincol},\j{colinc},\j{minpad},\j{padchar}<\j{str}~>}

This justifies the text produced by processing \j{str}
within a field at least \j{mincol} columns wide.  \j{str}
may be divided up into segments with \f{~;}, in which case the
spacing is evenly divided between the text segments.


With no modifiers, the leftmost text segment is left justified in the
field, and the rightmost text segment is right justified.  If there is
only one text element, as a special case, it is right justified.
The \f{:} modifier causes                                            
spacing to be introduced before the first text segment;  the 
\f{@} modifier causes spacing to be added after the last.
The \j{minpad} parameter (default \f{0}) is the minimum number of
padding characters to be output between each segment.
The padding character is supplied by \j{padchar},
which defaults to the space character.
If the total width needed to satisfy these constraints is greater
than \j{mincol}, then the width used is \j{mincol}+\j{k}*\j{colinc}
for the smallest possible non-negative integer value \j{k}.
\j{colinc} defaults to \f{1}, and \j{mincol} defaults to \f{0}.


Note that \j{str} may include **[[CL:Functions:format]]** directives.
All the clauses in \j{str} are processed in order;
it is the resulting pieces of text that are justified.


The \f{~\hat } directive may be used to terminate processing of the
clauses prematurely, in which case only the completely processed clauses
are justified.


If the first clause of a \f{~<} 
is terminated with \f{~:;} instead of
\f{~;}, then it is used in a special way.  All of the clauses are
processed (subject to \f{~\hat }, of course), but the 
first one is not used
in performing the spacing and padding.  When the padded result has been
determined, then if it will fit on the current line of output, it is
output, and the text for the first clause is discarded.  If, however, the
padded text will not fit on the current line, then the text segment for
the first clause is output before the padded text.  The first clause
ought to contain a newline (such as a \f{~\%} directive).  The first
clause is always processed, and so any arguments it refers to will be
used; the decision is whether to use the resulting segment of text, not
whether to process the first clause.  If the \f{~:;} has a prefix
parameter \j{n}, then the padded text must fit on the current line with
\j{n} character positions to spare to avoid outputting the first clause's
text.  For example, the control string

\code
 "~%;; ~\lbr\ ~<~%;; ~1:; ~S~>~\hat\ ,~\rbr\ .~%"
\endcode

can be used to print a list of items separated by commas without
breaking items over line boundaries, beginning each line with
\f{;; }.  The prefix parameter 
\f{1} in \f{~1:;} accounts for the width of the
comma that will follow the justified item if it is not the last
element in the list, or the period 
if it is.  If \f{~:;} has a second
prefix parameter, then it is used as the width of the line,
thus overriding the natural line width of the output stream.  To make
the preceding example use a line width of 50, one would write
     
\code
 "~%;; ~\lbr\ ~<~%;; ~1,50:; ~S~>~\hat\ ,~\rbr \ .~%"
\endcode

If the second argument is not supplied, then **[[CL:Functions:format]]** uses the
line width of the //destination// output stream.
If this cannot be determined (for example, when producing a 
//[[CL:Glossary:string]]// result), then **[[CL:Functions:format]]** uses \f{72} as the line length.

See also \secref\TildeLessThanLogicalBlock.

\endsubsubsection%{Tilde Less-Than-Sign: Justification}

\beginsubsubsection{Tilde Greater-Than-Sign: End of Justification}
\idxtext{Greater-Than-Sign (format directive)}\idxtext{Tilde Greater-Than-Sign (format directive)}



\f{~>} terminates a \f{~<}.
The consequences of using it elsewhere are undefined.

\endsubsubsection%{Tilde Greater-Than-Sign: End of Justification}

\endsubsection%{FORMAT Layout Control}

\beginsubsection{FORMAT Control-Flow Operations}

\beginsubsubsection{Tilde Asterisk: Go-To}
\idxtext{Asterisk (format directive)}\idxtext{Tilde Asterisk (format directive)}


                                                
The next \j{arg} is ignored.
\f{~\j{n}*} ignores the next \j{n} arguments.


\f{~:*} backs up in the list of
arguments so that the argument last processed will be processed again.
\f{~\j{n}:*} backs up \j{n} arguments.


When within a \f{~\{} construct
(see below), the ignoring (in either direction) is relative to the list
of arguments being processed by the iteration.


\f{~\j{n}@*} 
goes to the \j{n}th \j{arg}, where 0 means the first one;
\j{n} defaults to 0, so \f{~@*} goes back to the first \j{arg}.      
Directives after a \f{~\j{n}@*}
will take arguments in sequence beginning with the one gone to.
When within a \f{~\{} construct, the ``goto''
is relative to the list of arguments being processed by the iteration.

\endsubsubsection%{Tilde Asterisk: Go-To}

\beginsubsubsection{Tilde Left-Bracket: Conditional Expression}
\idxtext{Left-Bracket (format directive)}\idxtext{Tilde Left-Bracket (format directive)}


\f{~[\j{str0}~;\j{str1}~;\j{...}~;\j{strn}~]}

This is a set of control strings, called \j{clauses}, one of which is
chosen and used.  The clauses are separated by \f{~;}
and the construct is terminated by \f{~]}.  For example,

\f{"~[Siamese~;Manx~;Persian~] Cat"}
                              
The \j{arg}th
clause is selected, where the first clause is number 0.
If a prefix parameter is given (as \f{~\j{n}[}),
then the parameter is used instead of an argument.    
If \j{arg} is out of range then no clause is selected
and no error is signaled.
After the selected alternative has been processed, the control string
continues after the \f{~]}.


\f{~[\j{str0}~;\j{str1}~;\j{...}~;\j{strn}~:;\j{default}~]}
has a default case.
If the \j{last} \f{~;} used to separate clauses
is \f{~:;} instead, then the last clause is an else clause
that is performed if no other clause is selected.
For example:

\f{"~[Siamese~;Manx~;Persian~:;Alley~] Cat"}



\f{~:[//alternative//~;//consequent//~]} 
selects the //alternative// control string if \j{arg} is //[[CL:Glossary:false]]//,
and selects the //consequent// control string otherwise.
                                                             

\f{~@[//consequent//~]} 
tests the argument.  If it is //[[CL:Glossary:true]]//,           
then the argument is not used up by the \f{~[} command
but remains as the next one to be processed,
and the one clause //consequent// is processed.
If the \j{arg} is //[[CL:Glossary:false]]//, then the argument is used up,
and the clause is not processed.
The clause therefore should normally use exactly one argument,
and may expect it to be //[[CL:Glossary:non-nil]]//.
For example:

\code
 (setq *print-level* nil *print-length* 5)
 (format nil
        "~@[ print level = ~D~]~@[ print length = ~D~]"
        *print-level* *print-length*)
\EV  " print length = 5"
\endcode

Note also that

\code
 (format //stream// "...~@[//str//~]..." ...)
\EQ (format //stream// "...~:[~;~:*//str//~]..." ...)
\endcode


The combination of \f{~[} and \f{\#} is useful, for
example, for dealing with English conventions for printing lists:
                         
\code
 (setq foo "Items:~#[ none~; ~S~; ~S and ~S~
           ~:;~@\{~#[~; and~] ~S~\hat\ ,~\}~].")
 (format nil foo) \EV  "Items: none."
 (format nil foo 'foo) \EV  "Items: FOO."
 (format nil foo 'foo 'bar) \EV  "Items: FOO and BAR."
 (format nil foo 'foo 'bar 'baz) \EV  "Items: FOO, BAR, and BAZ."
 (format nil foo 'foo 'bar 'baz 'quux) \EV  "Items: FOO, BAR, BAZ, and QUUX."
\endcode

\endsubsubsection%{Tilde Left-Bracket: Conditional Expression}

\beginsubsubsection{Tilde Right-Bracket: End of Conditional Expression}
\idxtext{Right-Bracket (format directive)}\idxtext{Tilde Right-Bracket (format directive)}



\f{~]} terminates a \f{~[}.
The consequences of using it elsewhere are undefined.

\endsubsubsection%{Tilde Right-Bracket: End of Conditional Expression}

\beginsubsubsection{Tilde Left-Brace: Iteration}
\idxtext{Left-Brace (format directive)}\idxtext{Tilde Left-Brace (format directive)}


\f{~\{\j{str}~\}}

This is an iteration construct.  The argument should be a //[[CL:Glossary:list]]//,
which is used as a set of arguments 
as if for a recursive call to **[[CL:Functions:format]]**.
The //[[CL:Glossary:string]]// \j{str} is used repeatedly as the control string.
Each iteration can absorb as many elements of the //[[CL:Glossary:list]]// as it likes
as arguments;
if \j{str} uses up two arguments by itself, then two elements of the
//[[CL:Glossary:list]]// will get used up each time around the loop.
If before any iteration step the //[[CL:Glossary:list]]// 
is empty, then the iteration is terminated.
Also, if a prefix parameter \j{n} is given, then there will be at most \j{n}
repetitions of processing of \j{str}.  
Finally, the \f{~\hat } directive can be
used to terminate the iteration prematurely.


For example:
                                                                       
\code
 (format nil "The winners are:~\{ ~S~\}." 
         '(fred harry jill)) 
\EV "The winners are: FRED HARRY JILL."                           
 (format nil "Pairs:~\{ <~S,~S>~\}." 
         '(a 1 b 2 c 3))
\EV "Pairs: <A,1> <B,2> <C,3>."
\endcode


\f{~:\lbr \j{str}~\rbr  } is similar, 
but the argument should be a //[[CL:Glossary:list]]// of sublists.
At each repetition step, one sublist 
is used as the set of arguments for
processing \j{str}; on the next repetition, a new sublist 
is used, whether
or not all of the last sublist had been processed.  
For example:

                                                                               

\code
 (format nil "Pairs:~:\lbr <~S,~S>~\rbr\ ." 
                 '((a 1) (b 2) (c 3)))
\EV "Pairs: <A,1> <B,2> <C,3>."
\endcode


\f{~@\lbr \j{str}~\rbr }
is similar to \f{~\lbr \j{str}~\rbr  }, but instead of
using one argument that is a list, all the remaining arguments
are used as the list of arguments for the iteration.
Example:
                                    
\code
 (format nil "Pairs:~@\lbr <~S,~S>~\rbr\ ." 'a 1 'b 2 'c 3)
\EV "Pairs: <A,1> <B,2> <C,3>."
\endcode
If the iteration is terminated before all the remaining arguments are
consumed, then any arguments not processed by the iteration remain to be
processed by any directives following the iteration construct.


\f{~:@\lbr \j{str}~\rbr  } 
combines the features                              
of \f{~:\lbr \j{str}~\rbr  }
and \f{~@\lbr \j{str}~\rbr  }.
All the remaining arguments
are used, and each one must be a //[[CL:Glossary:list]]//.
On each iteration, the next argument is 
used as a //[[CL:Glossary:list]]// of arguments to \j{str}.
Example:
                                     
\code
 (format nil "Pairs:~:@\lbr <~S,~S>~\rbr\ ." 
              '(a 1) '(b 2) '(c 3)) 
\EV "Pairs: <A,1> <B,2> <C,3>."
\endcode

Terminating the repetition construct with \f{~:\rbr } 
instead of \f{~\rbr  }
forces \j{str} to be processed at least once, even if the initial
list of arguments is null. However, this will not override an explicit
prefix parameter of zero.


If \j{str} is empty, then an argument is used as \j{str}.  
It must be a //[[CL:Glossary:format control]]//
and precede any arguments processed by the iteration.  As an example,
the following are equivalent:

\code
    (apply #'format stream string arguments)
 \EQ (format stream "~1\{~:\}" string arguments)
\endcode

This will use \f{string} as a formatting string.  
The \f{~1\lbr } says it will                 
be processed at most once, and the \f{~:\rbr } 
says it will be processed at least once.
Therefore it is processed exactly once, using \f{arguments} as the arguments.
This case may be handled more clearly by the \f{~?} directive,
but this general feature of \f{~\lbr  }
is more powerful than \f{~?}.

\endsubsubsection%{Tilde Left-Brace: Iteration}

\beginsubsubsection{Tilde Right-Brace: End of Iteration}
\idxtext{Right-Brace (format directive)}\idxtext{Tilde Right-Brace (format directive)}


                               
\f{~\}} terminates a \f{~\{}.
The consequences of using it elsewhere are undefined.

\endsubsubsection%{Tilde Right-Brace: End of Iteration}

\beginsubsubsection{Tilde Question-Mark: Recursive Processing}
\idxtext{Question-Mark (format directive)}\idxtext{Tilde Question-Mark (format directive)}





The next \j{arg} must be a //[[CL:Glossary:format control]]//, and the one after it a //[[CL:Glossary:list]]//;
both are consumed by the \f{~?} directive.
The two are processed as a //control-string//, with the elements of the //[[CL:Glossary:list]]// 
as the arguments.  Once the recursive processing
has been finished, the processing of the control
string containing the \f{~?} directive is resumed.
Example:

\code
 (format nil "~? ~D" "<~A ~D>" '("Foo" 5) 7) \EV "<Foo 5> 7"
 (format nil "~? ~D" "<~A ~D>" '("Foo" 5 14) 7) \EV "<Foo 5> 7"
\endcode
Note that in the second example three arguments are supplied
to the //[[CL:Glossary:format string]]// \f{"<~A ~D>"}, but only two are processed
and the third is therefore ignored.


With the \f{@}
modifier, only one \j{arg} is directly consumed.
The \j{arg} must be a //[[CL:Glossary:string]]//; 
it is processed as part of the control
string as if it had appeared in place of the \f{~@?} construct,
and any directives in the recursively processed control string may      
consume arguments of the control string containing the \f{~@?}
directive.
Example:

\code
 (format nil "~@? ~D" "<~A ~D>" "Foo" 5 7) \EV "<Foo 5> 7"
 (format nil "~@? ~D" "<~A ~D>" "Foo" 5 14 7) \EV "<Foo 5> 14"
\endcode

\endsubsubsection%{Tilde Question-Mark: Recursive Processing}

\endsubsection%{FORMAT Control-Flow Operations}

\beginsubsection{FORMAT Miscellaneous Operations}

\beginsubsubsection{Tilde Left-Paren: Case Conversion}
\idxtext{Left-Paren (format directive)}\idxtext{Tilde Left-Paren (format directive)}


\f{~(\j{str}~)}

The contained control string \j{str} is processed, and what it produces
is subject to case conversion.


With no flags, every //[[CL:Glossary:uppercase]]// //[[CL:Glossary:character]]//
is converted to the corresponding //[[CL:Glossary:lowercase]]// //[[CL:Glossary:character]]//.


\f{~:(} capitalizes all words, as if by **[[CL:Functions:string-capitalize]]**.
                        

\f{~@(} 
capitalizes just the first word and forces the rest to lower
case.


\f{~:@(} converts every lowercase character
to the corresponding uppercase character.


In this example \f{~@(} is used to cause the first word
produced by \f{~@R} to be capitalized:

\code
 (format nil "~@R ~(~@R~)" 14 14) 
\EV "XIV xiv"
 (defun f (n) (format nil "~@(~R~) error~:P detected." n)) \EV F
 (f 0) \EV "Zero errors detected."
 (f 1) \EV "One error detected."
 (f 23) \EV "Twenty-three errors detected."
\endcode


When case conversions appear nested, the outer conversion dominates,
as illustrated in the following example:

\code
 (format nil "~@(how is ~:(BOB SMITH~)?~)")
 \EV "How is bob smith?"
 \NV "How is Bob Smith?"
\endcode

\endsubsubsection%{Tilde Left-Paren: Case Conversion}

\beginsubsubsection{Tilde Right-Paren: End of Case Conversion}
\idxtext{Right-Paren (format directive)}\idxtext{Tilde Right-Paren (format directive)}



\f{~)} terminates a \f{~(}.
The consequences of using it elsewhere are undefined.

\endsubsubsection%{Tilde Right-Paren: End of Case Conversion}

\beginsubsubsection{Tilde P: Plural}
\idxtext{P (format directive)}\idxtext{Tilde P (format directive)}



If \j{arg} is not **[[CL:Functions:eql]]** 
to the integer \f{1}, a lowercase \f{s} is
printed; if \j{arg} is **[[CL:Functions:eql]]** to \f{1}, nothing is printed.  
If \j{arg} is a floating-point \f{1.0}, the \f{s} is
printed.


\f{~:P} does the same thing, 
after doing a \f{~:*} to back up one argument;
that is, it prints a lowercase \f{s} if the previous argument was not
\f{1}.  


\f{~@P} 
prints \f{y} if the argument is \f{1}, or \f{ies} if it is
not.  \f{~:@P} does the same thing, but backs up first.

\code
 (format nil "~D tr~:@P/~D win~:P" 7 1) \EV "7 tries/1 win"
 (format nil "~D tr~:@P/~D win~:P" 1 0) \EV "1 try/0 wins"
 (format nil "~D tr~:@P/~D win~:P" 1 3) \EV "1 try/3 wins"
\endcode

\endsubsubsection%{Tilde P: Plural}

\endsubsection%{FORMAT Miscellaneous Operations}

\beginsubsection{FORMAT Miscellaneous Pseudo-Operations}

\beginsubsubsection{Tilde Semicolon: Clause Separator}
\idxtext{Semicolon (format directive)}\idxtext{Tilde Semicolon (format directive)}



This separates clauses in \f{~[} and \f{~<} constructs.
The consequences of using it elsewhere are undefined.

\endsubsubsection%{Tilde Semicolon: Clause Separator}

\beginsubsubsection{Tilde Circumflex: Escape Upward}
\idxtext{Circumflex (format directive)}\idxtext{Tilde Circumflex (format directive)}


{\f{~\hat }}

This is an escape construct.  If there are no more arguments remaining to
be processed, then the immediately           
enclosing \f{~\lbr  } or \f{~<} construct
is terminated.  If there is no such enclosing construct, then the entire
formatting operation is terminated.  
In the \f{~<} case, the formatting
is performed, but no more segments are processed before doing the
justification.     
\f{~\hat } may appear anywhere in a \f{~\lbr  }
construct.

\code
 (setq donestr "Done.~{\hat} ~D warning~:P.~{\hat} ~D error~:P.")
\EV "Done.~{\hat} ~D warning~:P.~{\hat} ~D error~:P."
 (format nil donestr) \EV "Done."
 (format nil donestr 3) \EV "Done. 3 warnings."
 (format nil donestr 1 5) \EV "Done. 1 warning. 5 errors."
\endcode
                                

If a prefix parameter is given, then termination occurs if the parameter
is zero.  (Hence \f{~{\hat}} is equivalent to 
\f{~\#{\hat}}.)  If two
parameters are given, termination occurs if they are equal.
\reviewer{Barmar: Which equality predicate?}  If three
parameters are given, termination occurs if the first is less than or
equal to the second and the second is less than or equal to the third.
Of course, this is useless if all the prefix parameters are constants; at
least one of them should be a \f{\#} or a \f{V} parameter.


If \f{~{\hat}} is used within a \f{~:\lbr  } 
construct, then it terminates
the current iteration step because in the standard case it tests for
remaining arguments of the current step only; the next iteration step
commences immediately.  \f{~:{\hat}} is used to terminate
the iteration process.
\issue{FORMAT-COLON-UPARROW-SCOPE}
\f{~:{\hat}} 
may be used only if the command it would terminate is 
\f{~:\lbr  } or \f{~:@\lbr  }.
The entire iteration process is terminated if and only if the sublist that is
supplying the arguments for the current iteration step is the last sublist in
the case of \f{~:\lbr }, 
or the last **[[CL:Functions:format]]**        
argument in the case of \f{~:@\lbr  }. 
\f{~:{\hat}} is not
equivalent to \f{~\#:{\hat}}; 
the latter terminates the entire iteration if and only if no
arguments remain for the current iteration step.
For example:

\code
 (format nil "~:\lbr\ ~@?~:\hat\ ...~\rbr\ " '(("a") ("b"))) \EV "a...b"
\endcode



If \f{~{\hat}} appears within a control string being processed
under the control of a \f{~?} directive, but not within
any \f{~\lbr  } or \f{~<} construct within that string,
then the string being
processed will be terminated, thereby ending processing
of the \f{~?} directive.  Processing then
continues within the string   
containing the \f{~?} directive at the point following that directive.


If \f{~{\hat}}                                          
appears within a \f{~[} or \f{~(} construct,
then all the commands up to the \f{~{\hat}} are properly selected
or case-converted,                   
the \f{~[} or \f{~(} processing is terminated,
and the outward search continues         
for a \f{~\lbr  } or \f{~<} construct
to be terminated.  For example:
                            
\code
 (setq tellstr "~@(~@[~R~]~{\hat} ~A!~)")
\EV "~@(~@[~R~]~{\hat} ~A!~)"
 (format nil tellstr 23) \EV "Twenty-three!"
 (format nil tellstr nil "losers") \EV " Losers!"
 (format nil tellstr 23 "losers") \EV "Twenty-three losers!"
\endcode


Following are examples of the use of \f{~{\hat}} 
within a \f{~<} construct.

\code
 (format nil "~15<~S~;~{\hat}~S~;~{\hat}~S~>" 'foo)
\EV  "            FOO"
 (format nil "~15<~S~;~{\hat}~S~;~{\hat}~S~>" 'foo 'bar)
\EV  "FOO         BAR"
 (format nil "~15<~S~;~{\hat}~S~;~{\hat}~S~>" 'foo 'bar 'baz)
\EV  "FOO   BAR   BAZ"
\endcode

\endsubsubsection%{Tilde Circumflex: Escape Upward}

\beginsubsubsection{Tilde Newline: Ignored Newline}
\idxtext{Newline (format directive)}\idxtext{Tilde Newline (format directive)}



//[[CL:Glossary:Tilde]]// immediately followed by a //[[CL:Glossary:newline]]// ignores the //[[CL:Glossary:newline]]//
and any following non-newline //[[CL:Glossary:whitespace]]//\meaning{1} characters.
With a \f{:},
 the //[[CL:Glossary:newline]]// is ignored, 
 but any following //[[CL:Glossary:whitespace]]//\meaning{1} is left in place.
With an \f{@},
 the //[[CL:Glossary:newline]]// is left in place,
 but any following //[[CL:Glossary:whitespace]]//\meaning{1} is ignored.
For example:


\code
 (defun type-clash-error (fn nargs argnum right-type wrong-type)
   (format *error-output*
           "~&~S requires its ~:[~:R~;~*~]~ 
           argument to be of type ~S,~%but it was called ~
           with an argument of type ~S.~%"
           fn (eql nargs 1) argnum right-type wrong-type))
 (type-clash-error 'aref nil 2 'integer 'vector)  prints:
AREF requires its second argument to be of type INTEGER,
but it was called with an argument of type VECTOR.
NIL
 (type-clash-error 'car 1 1 'list 'short-float)  prints:
CAR requires its argument to be of type LIST,
but it was called with an argument of type SHORT-FLOAT.
NIL
\endcode
Note that in this example newlines appear in the output only as specified
by the \f{~\&} and \f{~\%} directives; the 
actual newline characters
in the control string are suppressed because each is preceded by a tilde.
                                                                     
\endsubsubsection%{Tilde Newline: Ignored Newline}

\endsubsection%{FORMAT Miscellaneous Pseudo-Operations}

\beginsubsection{Additional Information about FORMAT Operations}

\beginsubsubsection{Nesting of FORMAT Operations}




The case-conversion, conditional, iteration, and justification
constructs can contain other formatting constructs by bracketing them.
These constructs must nest properly with respect to each other.
For example, it is not legitimate to put the start of a case-conversion
construct in each arm of a conditional and the
end of the case-conversion construct outside the conditional:

\code
 (format nil "~:[abc~:@(def~;ghi~
:@(jkl~]mno~)" x) ;Invalid!
\endcode
This notation is invalid because the \f{~[...~;...~]}
and \f{~(...~)} constructs are not properly nested.
                                   

The processing indirection caused by the \f{~?} directive
is also a kind of nesting for the purposes of this rule of proper nesting.
It is not permitted to
start a bracketing construct within a string processed
under control of a \f{~?}                                      
directive and end the construct at some point after the \f{~?} construct
in the string containing that construct, or vice versa.
For example, this situation is invalid:
                                                                               
\code
 (format nil "~@?ghi~)" "abc~@(def") ;Invalid!
\endcode
This notation
is invalid because the \f{~?}
and \f{~(...~)} constructs are not properly nested.

\endsubsubsection%{Nesting of FORMAT Operations}

\beginsubsubsection{Missing and Additional FORMAT Arguments}


The consequences are undefined if no //arg// remains for a directive 
requiring an argument.  However, it is permissible for one or more //args// 
to remain unprocessed by a directive; such //args// are ignored.
 
\endsubsubsection%{Missing and Additional FORMAT Arguments}

\beginsubsubsection{Additional FORMAT Parameters}


The consequences are undefined if a format directive is given more parameters 
than it is described here as accepting.

\endsubsubsection%{Additional FORMAT Parameters}

\beginsubsubsection{Undefined FORMAT Modifier Combinations}


The consequences are undefined if //[[CL:Glossary:colon]]// or //[[CL:Glossary:at-sign]]// modifiers
are given to a directive in a combination not specifically described
here as being meaningful.

\endsubsubsection%{Undefined FORMAT Modifier Combinations}

\endsubsection%{Additional Information about FORMAT Operations}

\beginsubsection{Examples of FORMAT}




\code
 (format nil "foo") \EV "foo"
 (setq x 5) \EV 5
 (format nil "The answer is ~D." x) \EV "The answer is 5."
 (format nil "The answer is ~3D." x) \EV "The answer is   5."
 (format nil "The answer is ~3,'0D." x) \EV "The answer is 005."
 (format nil "The answer is ~:D." (expt 47 x))
\EV "The answer is 229,345,007."
 (setq y "elephant") \EV "elephant"
 (format nil "Look at the ~A!" y) \EV "Look at the elephant!"
 (setq n 3) \EV 3
 (format nil "~D item~:P found." n) \EV "3 items found."
 (format nil "~R dog~:[s are~; is~] here." n (= n 1))
\EV "three dogs are here."
 (format nil "~R dog~:*~[s are~; is~:;s are~] here." n)
\EV "three dogs are here."
 (format nil "Here ~[are~;is~:;are~] ~:*~R pupp~:@P." n)
\EV "Here are three puppies."
\endcode

\code
 (defun foo (x)
   (format nil "~6,2F|~6,2,1,'*F|~6,2,,'?F|~6F|~,2F|~F"
           x x x x x x)) \EV FOO
 (foo 3.14159)  \EV "  3.14| 31.42|  3.14|3.1416|3.14|3.14159"
 (foo -3.14159) \EV " -3.14|-31.42| -3.14|-3.142|-3.14|-3.14159"
 (foo 100.0)    \EV "100.00|******|100.00| 100.0|100.00|100.0"
 (foo 1234.0)   \EV "1234.00|******|??????|1234.0|1234.00|1234.0"
 (foo 0.006)    \EV "  0.01|  0.06|  0.01| 0.006|0.01|0.006"
\endcode

\code
 (defun foo (x)  
    (format nil
           "~9,2,1,,'*E|~10,3,2,2,'?,,'\$E|~
            ~9,3,2,-2,'%@E|~9,2E"
           x x x x))
 (foo 3.14159)  \EV "  3.14E+0| 31.42\$-01|+.003E+03|  3.14E+0"
 (foo -3.14159) \EV " -3.14E+0|-31.42\$-01|-.003E+03| -3.14E+0"
 (foo 1100.0)   \EV "  1.10E+3| 11.00\$+02|+.001E+06|  1.10E+3"
 (foo 1100.0L0) \EV "  1.10L+3| 11.00\$+02|+.001L+06|  1.10L+3"
 (foo 1.1E13)   \EV "*********| 11.00\$+12|+.001E+16| 1.10E+13"
 (foo 1.1L120)  \EV "*********|??????????|%%%%%%%%%|1.10L+120"
 (foo 1.1L1200) \EV "*********|??????????|%%%%%%%%%|1.10L+1200"
\endcode
As an example of the effects of varying the scale factor, the code

\code
 (dotimes (k 13)
   (format t "~%Scale factor ~2D: |~13,6,2,VE|"
           (- k 5) (- k 5) 3.14159))
\endcode
produces the following output:

\code
Scale factor -5: | 0.000003E+06|
Scale factor -4: | 0.000031E+05|
Scale factor -3: | 0.000314E+04|
Scale factor -2: | 0.003142E+03|
Scale factor -1: | 0.031416E+02|
Scale factor  0: | 0.314159E+01|
Scale factor  1: | 3.141590E+00|
Scale factor  2: | 31.41590E-01|
Scale factor  3: | 314.1590E-02|
Scale factor  4: | 3141.590E-03|
Scale factor  5: | 31415.90E-04|
Scale factor  6: | 314159.0E-05|
Scale factor  7: | 3141590.E-06|
\endcode


\code
 (defun foo (x)
   (format nil "~9,2,1,,'*G|~9,3,2,3,'?,,'\$G|~9,3,2,0,'%G|~9,2G"
          x x x x))                                     
 (foo 0.0314159) \EV "  3.14E-2|314.2\$-04|0.314E-01|  3.14E-2"
 (foo 0.314159)  \EV "  0.31   |0.314    |0.314    | 0.31    "
 (foo 3.14159)   \EV "   3.1   | 3.14    | 3.14    |  3.1    "
 (foo 31.4159)   \EV "   31.   | 31.4    | 31.4    |  31.    "
 (foo 314.159)   \EV "  3.14E+2| 314.    | 314.    |  3.14E+2"
 (foo 3141.59)   \EV "  3.14E+3|314.2\$+01|0.314E+04|  3.14E+3"
 (foo 3141.59L0) \EV "  3.14L+3|314.2\$+01|0.314L+04|  3.14L+3"
 (foo 3.14E12)   \EV "*********|314.0\$+10|0.314E+13| 3.14E+12"
 (foo 3.14L120)  \EV "*********|?????????|%%%%%%%%%|3.14L+120"
 (foo 3.14L1200) \EV "*********|?????????|%%%%%%%%%|3.14L+1200"
\endcode


\code
 (format nil "~10<foo~;bar~>")   \EV "foo    bar"
 (format nil "~10:<foo~;bar~>")  \EV "  foo  bar"
 (format nil "~10<foobar~>")     \EV "    foobar"
 (format nil "~10:<foobar~>")    \EV "    foobar"
 (format nil "~10:@<foo~;bar~>") \EV "  foo bar "
 (format nil "~10@<foobar~>")    \EV "foobar    "
 (format nil "~10:@<foobar~>")   \EV "  foobar  "
\endcode

\issue{PATHNAME-PRINT-READ:SHARPSIGN-P}
\code
  (FORMAT NIL "Written to ~A." #P"foo.bin")
  \EV "Written to foo.bin."
\endcode


\endsubsection%{Examples of FORMAT}

\beginsubsection{Notes about FORMAT}


\issue{FORMAT-STRING-ARGUMENTS:SPECIFY}
Formatted output is performed not only by **[[CL:Functions:format]]**,
but by certain other functions that accept a //[[CL:Glossary:format control]]//
the way **[[CL:Functions:format]]** does.  For example, error-signaling functions
such as **[[CL:Functions:cerror]]** accept //[[CL:Glossary:format controls]]//.


Note that the meaning of \nil\ and \t\ as destinations to **[[CL:Functions:format]]** 
are different than those of \nil\ and \t\ as //[[CL:Glossary:stream designators]]//.

The \f{~{\hat}} should appear only at the beginning of a \f{~<} clause,
because it aborts the entire clause in which it appears (as well as
all following clauses).                                     

\endsubsection%{Notes about FORMAT}
