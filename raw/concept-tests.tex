

\beginsubsection{Satisfying a Two-Argument Test} \DefineSection{SatisfyingTheTwoArgTest}

When an //[[CL:Glossary:object]]// $O$ is being considered iteratively  against each //[[CL:Glossary:element]]// $E\sub i$ of a //[[CL:Glossary:sequence]]// $S$ by an //[[CL:Glossary:operator]]// $F$ listed in \thenextfigure, it is sometimes useful to control the way in which the presence of $O$  is tested in $S$ is tested by $F$. This control is offered on the basis of a //[[CL:Glossary:function]]// designated with  either a **'':test''** or **'':test-not''** //[[CL:Glossary:argument]]//.

\displaythree{Operators that have Two-Argument Tests to be Satisfied}{ adjoin&nset-exclusive-or&search\cr assoc&nsublis&set-difference\cr count&nsubst&set-exclusive-or\cr delete&nsubstitute&sublis\cr find&nunion&subsetp\cr intersection&position&subst\cr member&pushnew&substitute\cr mismatch&rassoc&tree-equal\cr nintersection&remove&union\cr nset-difference&remove-duplicates&\cr }

The object $O$ might not be compared directly to $E\sub i$. If a **'':key''** //[[CL:Glossary:argument]]// is provided, it is a //[[CL:Glossary:designator]]// for a //[[CL:Glossary:function]]// of one //[[CL:Glossary:argument]]//  to be called with each $E\sub i$ as an //[[CL:Glossary:argument]]//,  and //[[CL:Glossary:yielding]]// an //[[CL:Glossary:object]]// $Z\sub i$ to be used for comparison. (If there is no **'':key''** //[[CL:Glossary:argument]]//, $Z\sub i$ is $E\sub i$.)

The //[[CL:Glossary:function]]// designated by \thekeyarg{key} is never called on $O$ itself. However, if the function operates on multiple sequences (\eg as happens in **[[CL:Functions:set-difference]]**), $O$ will be the result of calling the **'':key''** function on an //[[CL:Glossary:element]]// of the other sequence.  

A **'':test''** //[[CL:Glossary:argument]]//, if supplied to $F$, is a //[[CL:Glossary:designator]]// for a  //[[CL:Glossary:function]]// of two //[[CL:Glossary:arguments]]//, $O$ and $Z\sub i$. An $E\sub i$ is said (or, sometimes, an $O$ and an $E\sub i$ are said) to //[[CL:Glossary:satisfy the test]]//  if this **'':test''** //[[CL:Glossary:function]]// returns a //[[CL:Glossary:generalized boolean]]// representing  //[[CL:Glossary:true]]//.

A **'':test-not''** //[[CL:Glossary:argument]]//, if supplied to $F$,  is //[[CL:Glossary:designator]]// for a //[[CL:Glossary:function]]//  of two //[[CL:Glossary:arguments]]//, $O$ and $Z\sub i$. An $E\sub i$ is said (or, sometimes, an $O$ and an $E\sub i$ are said) to //[[CL:Glossary:satisfy the test]]//  if this **'':test-not''** //[[CL:Glossary:function]]// returns a //[[CL:Glossary:generalized boolean]]// representing //[[CL:Glossary:false]]//.

If neither a **'':test''** nor a **'':test-not''** //[[CL:Glossary:argument]]// is supplied,  it is as if a **'':test''** argument of \f{\#'eql} was supplied.

The consequences are unspecified if both a **'':test''** and a **'':test-not''** //[[CL:Glossary:argument]]// are supplied in the same //[[CL:Glossary:call]]// to $F$.

\beginsubsubsection{Examples of Satisfying a Two-Argument Test}

\code
 (remove "FOO" '(foo bar "FOO" "BAR" "foo" "bar") :test #'equal) \EV (foo bar "BAR" "foo" "bar")
 (remove "FOO" '(foo bar "FOO" "BAR" "foo" "bar") :test #'equalp) \EV (foo bar "BAR" "bar")
 (remove "FOO" '(foo bar "FOO" "BAR" "foo" "bar") :test #'string-equal) \EV (bar "BAR" "bar")
 (remove "FOO" '(foo bar "FOO" "BAR" "foo" "bar") :test #'string=) \EV (BAR "BAR" "foo" "bar")

 (remove 1 '(1 1.0 #C(1.0 0.0) 2 2.0 #C(2.0 0.0)) :test-not #'eql) \EV (1)
 (remove 1 '(1 1.0 #C(1.0 0.0) 2 2.0 #C(2.0 0.0)) :test-not #'=) \EV (1 1.0 #C(1.0 0.0))
 (remove 1 '(1 1.0 #C(1.0 0.0) 2 2.0 #C(2.0 0.0)) :test (complement #'=)) \EV (1 1.0 #C(1.0 0.0))

 (count 1 '((one 1) (uno 1) (two 2) (dos 2)) :key #'cadr) \EV 2

 (count 2.0 '(1 2 3) :test #'eql :key #'float) \EV 1

 (count "FOO" (list (make-pathname :name "FOO" :type "X")  
                    (make-pathname :name "FOO" :type "Y"))
        :key #'pathname-name
        :test #'equal) \EV 2 \endcode

\endsubsubsection%{Examples of Satisfying a Two-Argument Test}

\endsubsection%{Satisfying a Two-Argument Test}

\beginsubsection{Satisfying a One-Argument Test} \DefineSection{SatisfyingTheOneArgTest}

When using one of the //[[CL:Glossary:functions]]// in \thenextfigure, the elements $E$ of a //[[CL:Glossary:sequence]]// $S$ are filtered not on the basis of the presence or absence of an object $O$  under a two //[[CL:Glossary:argument]]// //[[CL:Glossary:predicate]]//, as with the //[[CL:Glossary:functions]]// described in \secref\SatisfyingTheTwoArgTest, but rather on the basis of a one //[[CL:Glossary:argument]]// //[[CL:Glossary:predicate]]//.

\displaythree{Operators that have One-Argument Tests to be Satisfied}{ assoc-if&member-if&rassoc-if\cr assoc-if-not&member-if-not&rassoc-if-not\cr count-if&nsubst-if&remove-if\cr count-if-not&nsubst-if-not&remove-if-not\cr delete-if&nsubstitute-if&subst-if\cr delete-if-not&nsubstitute-if-not&subst-if-not\cr find-if&position-if&substitute-if\cr find-if-not&position-if-not&substitute-if-not\cr }

The element $E\sub i$ might not be considered directly. If a **'':key''** //[[CL:Glossary:argument]]// is provided, it is a //[[CL:Glossary:designator]]// for a //[[CL:Glossary:function]]// of one //[[CL:Glossary:argument]]//  to be called with each $E\sub i$ as an //[[CL:Glossary:argument]]//,  and //[[CL:Glossary:yielding]]// an //[[CL:Glossary:object]]// $Z\sub i$ to be used for comparison. (If there is no **'':key''** //[[CL:Glossary:argument]]//, $Z\sub i$ is $E\sub i$.)

//[[CL:Glossary:Functions]]// defined in this specification and having a name that ends in ``\f{-if}'' accept a first //[[CL:Glossary:argument]]// that is a //[[CL:Glossary:designator]]// for a  //[[CL:Glossary:function]]// of one //[[CL:Glossary:argument]]//, $Z\sub i$. An $E\sub i$ is said to //[[CL:Glossary:satisfy the test]]// if this **'':test''** //[[CL:Glossary:function]]// returns a //[[CL:Glossary:generalized boolean]]// representing //[[CL:Glossary:true]]//.

//[[CL:Glossary:Functions]]// defined in this specification and having a name that ends in ``\f{-if-not}'' accept a first //[[CL:Glossary:argument]]// that is a //[[CL:Glossary:designator]]// for a  //[[CL:Glossary:function]]// of one //[[CL:Glossary:argument]]//, $Z\sub i$. An $E\sub i$ is said to //[[CL:Glossary:satisfy the test]]// if this **'':test''** //[[CL:Glossary:function]]// returns a //[[CL:Glossary:generalized boolean]]// representing //[[CL:Glossary:false]]//.

\beginsubsubsection{Examples of Satisfying a One-Argument Test}

\code
 (count-if #'zerop '(1 #C(0.0 0.0) 0 0.0d0 0.0s0 3)) \EV 4

 (remove-if-not #'symbolp '(0 1 2 3 4 5 6 7 8 9 A B C D E F)) \EV (A B C D E F)
 (remove-if (complement #'symbolp) '(0 1 2 3 4 5 6 7 8 9 A B C D E F)) \EV (A B C D E F)

 (count-if #'zerop '("foo" "" "bar" "" "" "baz" "quux") :key #'length) \EV 3 \endcode

\endsubsubsection%{Examples of Satisfying a One-Argument Test}

\endsubsection%{Satisfying a One-Argument Test}
