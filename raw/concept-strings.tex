

\beginsubsection{Implications of Strings Being Arrays} \DefineSection{StringsAreArrays}

Since all //[[CL:Glossary:strings]]// are //[[CL:Glossary:arrays]]//, all rules which apply generally to //[[CL:Glossary:arrays]]// also apply to //[[CL:Glossary:strings]]//. \Seesection\ArrayConcepts.

For example,
     //[[CL:Glossary:strings]]// can have //[[CL:Glossary:fill pointers]]//,
 and //[[CL:Glossary:strings]]// are also subject to the rules of //[[CL:Glossary:element type]]// //[[CL:Glossary:upgrading]]//
        that apply to //[[CL:Glossary:arrays]]//.

\endsubsection%{Implications of Strings Being Arrays}

\beginsubsection{Subtypes of STRING} \issue{CHARACTER-PROPOSAL:2}

All functions that operate on //[[CL:Glossary:strings]]//  will operate on //[[CL:Glossary:subtypes]]// of //[[CL:Glossary:string]]// as well.

However, the consequences are undefined if a //[[CL:Glossary:character]]// is inserted into a //[[CL:Glossary:string]]// for which the //[[CL:Glossary:element type]]// of the //[[CL:Glossary:string]]// does not include that //[[CL:Glossary:character]]//.

\endsubsection%{Subtypes of STRING}
