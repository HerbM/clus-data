


                      
The //[[CL:Glossary:generic function]]// **[[CL:Functions:make-instance]]** creates and returns a new
//[[CL:Glossary:instance]]// of a //[[CL:Glossary:class]]//.  The first argument is a //[[CL:Glossary:class]]// or
the //[[CL:Glossary:name]]// of a //[[CL:Glossary:class]]//, and the remaining arguments form an 
//[[CL:Glossary:initialization argument list]]//.

The initialization of a new //[[CL:Glossary:instance]]// consists of several distinct
steps, including the following: combining the explicitly supplied initialization
arguments with default values for the unsupplied initialization arguments, 
checking the validity of the initialization arguments, allocating storage 
for the //[[CL:Glossary:instance]]//, filling //[[CL:Glossary:slots]]// with
values, and executing user-supplied //[[CL:Glossary:methods]]// that perform additional
initialization.  Each step of **[[CL:Functions:make-instance]]** is implemented by a
//[[CL:Glossary:generic function]]// to provide a mechanism for customizing that step.  
In addition, **[[CL:Functions:make-instance]]** is itself a //[[CL:Glossary:generic function]]// 
and thus also can be customized.

The \OS\ specifies system-supplied primary //[[CL:Glossary:methods]]// for each step 
and thus specifies a well-defined standard behavior for the entire
initialization process.  The standard behavior provides four simple
mechanisms for controlling initialization:

\beginlist

\itemitem{\bull} Declaring a //[[CL:Glossary:symbol]]// to be an initialization argument 
for a //[[CL:Glossary:slot]]//.  An initialization argument is declared by using the
**'':initarg''** slot option to \macref{defclass}.  This provides a mechanism
for supplying a value for a //[[CL:Glossary:slot]]// in a call to **[[CL:Functions:make-instance]]**.

\itemitem{\bull} Supplying a default value form for an initialization argument.
Default value forms for initialization arguments are defined by using the
**'':default-initargs''** class option to \macref{defclass}.  If an 
initialization argument is not explicitly provided
as an argument to **[[CL:Functions:make-instance]]**, the default value form is
evaluated in the lexical environment of the \macref{defclass} form that
defined it, and the resulting value is used as the value of the
initialization argument.

\itemitem{\bull} Supplying a default initial value form for a //[[CL:Glossary:slot]]//.  
A default initial value form for a //[[CL:Glossary:slot]]// is defined by using the 
**'':initform''** slot option to \macref{defclass}.  If no initialization
argument associated with that //[[CL:Glossary:slot]]// is given as an argument to 
**[[CL:Functions:make-instance]]** or is defaulted by **'':default-initargs''**, this
default initial value form is evaluated in the lexical environment of
the \macref{defclass} form that defined it, and the resulting value is
stored in the //[[CL:Glossary:slot]]//.  The **'':initform''** form for a
//[[CL:Glossary:local slot]]// may be used when creating an //[[CL:Glossary:instance]]//, when 
updating an //[[CL:Glossary:instance]]// to conform to a redefined //[[CL:Glossary:class]]//, 
or when updating an //[[CL:Glossary:instance]]// to conform to the definition of a
different //[[CL:Glossary:class]]//. The **'':initform''** form for a
//[[CL:Glossary:shared slot]]// may be used when defining or re-defining the //[[CL:Glossary:class]]//.
                                                                       
\itemitem{\bull} 
Defining //[[CL:Glossary:methods]]// for **[[CL:Functions:initialize-instance]]** and
**[[CL:Functions:shared-initialize]]**.  The slot-filling behavior described above is
implemented by a system-supplied primary //[[CL:Glossary:method]]// for
**[[CL:Functions:initialize-instance]]** which invokes **[[CL:Functions:shared-initialize]]**. The
//[[CL:Glossary:generic function]]// **[[CL:Functions:shared-initialize]]** implements the parts of
initialization shared by these four situations: when making an //[[CL:Glossary:instance]]//, 
when re-initializing an //[[CL:Glossary:instance]]//, when updating an //[[CL:Glossary:instance]]//
to conform to a redefined //[[CL:Glossary:class]]//, and when updating an //[[CL:Glossary:instance]]// 
to conform to the definition of a different //[[CL:Glossary:class]]//. The system-supplied
primary //[[CL:Glossary:method]]// for **[[CL:Functions:shared-initialize]]** directly implements the
slot-filling behavior described above, and **[[CL:Functions:initialize-instance]]**
simply invokes **[[CL:Functions:shared-initialize]]**.

\endlist

\beginsubsection{Initialization Arguments}

An initialization argument controls //[[CL:Glossary:object]]// creation and
initialization.  It is often convenient to use keyword //[[CL:Glossary:symbols]]//
to name initialization arguments, but the //[[CL:Glossary:name]]// of an
initialization argument can be any //[[CL:Glossary:symbol]]//, including \nil.  An
initialization argument can be used in two ways: to fill a //[[CL:Glossary:slot]]//
with a value or to provide an argument for an initialization
//[[CL:Glossary:method]]//.  A single initialization argument can be used for both
purposes.

\issue{PLIST-DUPLICATES:ALLOW}
An //[[CL:Glossary:initialization argument list]]// is a

//[[CL:Glossary:property list]]// of
initialization argument names and values.
Its structure is identical
to a //[[CL:Glossary:property list]]// and also 
to the portion of an argument list
processed for ''&key'' parameters.
As in those lists,
if an initialization
argument name appears more than once in an initialization argument list,
the leftmost occurrence supplies the value and the remaining occurrences
are ignored.  The arguments to **[[CL:Functions:make-instance]]** (after the first
argument) form an //[[CL:Glossary:initialization argument list]]//.
\issue{INITIALIZATION-FUNCTION-KEYWORD-CHECKING}







An initialization argument can be associated with a //[[CL:Glossary:slot]]//.  If
the initialization argument has a value in the \term{initialization
argument list}, the value is stored into the //[[CL:Glossary:slot]]// of the newly
created //[[CL:Glossary:object]]//, overriding any **'':initform''** form associated
with the //[[CL:Glossary:slot]]//.  A single initialization argument can initialize
more than one //[[CL:Glossary:slot]]//.  An initialization argument that initializes
a //[[CL:Glossary:shared slot]]// stores its value into the //[[CL:Glossary:shared slot]]//,
replacing any previous value.

An initialization argument can be associated with a //[[CL:Glossary:method]]//.  When
an //[[CL:Glossary:object]]// is created and a particular initialization argument is
supplied, the //[[CL:Glossary:generic functions]]// **[[CL:Functions:initialize-instance]]**,
**[[CL:Functions:shared-initialize]]**, and **[[CL:Functions:allocate-instance]]** are called
with that initialization argument's name and value as a keyword argument
pair.  If a value for the initialization argument is not supplied in the
//[[CL:Glossary:initialization argument list]]//, the //[[CL:Glossary:method]]//'s 
//[[CL:Glossary:lambda list]]// supplies a default value.

Initialization arguments are used in four situations: when making an
//[[CL:Glossary:instance]]//, when re-initializing an //[[CL:Glossary:instance]]//, when updating
an //[[CL:Glossary:instance]]// to conform to a redefined //[[CL:Glossary:class]]//, and when
updating an //[[CL:Glossary:instance]]// to conform to the definition of a different
//[[CL:Glossary:class]]//.

Because initialization arguments are used to control the creation and
initialization of an //[[CL:Glossary:instance]]// of some particular //[[CL:Glossary:class]]//,
we say that an initialization argument is
``an initialization argument for'' that //[[CL:Glossary:class]]//.

\endsubsection%{Initialization Arguments}

\beginsubsection{Declaring the Validity of Initialization Arguments}
\DefineSection{DeclaringInitargValidity}

Initialization arguments are checked for validity in each of the four
situations that use them.  An initialization argument may be valid in
one situation and not another. For example, the system-supplied     
primary //[[CL:Glossary:method]]// for **[[CL:Functions:make-instance]]** defined for 
\theclass{standard-class} checks the validity of its initialization arguments
and signals an error if an initialization argument is supplied that is
not declared as valid in that situation.


There are two means for declaring initialization arguments valid.

\beginlist

\itemitem{\bull}
Initialization arguments that fill //[[CL:Glossary:slots]]// are declared as valid
by the **'':initarg''** slot option to \macref{defclass}.  The
**'':initarg''** slot option is inherited from //[[CL:Glossary:superclasses]]//.  Thus
the set of valid initialization arguments that fill //[[CL:Glossary:slots]]// for a
//[[CL:Glossary:class]]// is the union of the initialization arguments that fill
//[[CL:Glossary:slots]]// declared as valid by that //[[CL:Glossary:class]]// and its
//[[CL:Glossary:superclasses]]//.  Initialization arguments that fill //[[CL:Glossary:slots]]//
are valid in all four contexts.

\itemitem{\bull}
Initialization arguments that supply arguments to //[[CL:Glossary:methods]]// are
declared as valid by defining those //[[CL:Glossary:methods]]//.  The keyword name of
each keyword parameter specified in the //[[CL:Glossary:method]]//'s 
//[[CL:Glossary:lambda list]]// becomes an initialization argument for all //[[CL:Glossary:classes]]// 
for which the //[[CL:Glossary:method]]// is applicable.
\issue{INITIALIZATION-FUNCTION-KEYWORD-CHECKING}
The presence of {\allowotherkeys} in the
//[[CL:Glossary:lambda list]]// of an applicable method disables validity checking of 
initialization arguments.

Thus //[[CL:Glossary:method]]// inheritance
controls the set of valid initialization arguments that supply arguments
to //[[CL:Glossary:methods]]//.  The //[[CL:Glossary:generic functions]]// for which //[[CL:Glossary:method]]//
definitions serve to declare initialization arguments valid are as
follows:
\beginlist                                              
\itemitem{--}
Making an //[[CL:Glossary:instance]]// of a //[[CL:Glossary:class]]//:
**[[CL:Functions:allocate-instance]]**, **[[CL:Functions:initialize-instance]]**, and
**[[CL:Functions:shared-initialize]]**.  Initialization arguments declared as valid
by these //[[CL:Glossary:methods]]// are valid when making 
an //[[CL:Glossary:instance]]// of a //[[CL:Glossary:class]]//.

\itemitem{--} Re-initializing an //[[CL:Glossary:instance]]//:
**[[CL:Functions:reinitialize-instance]]** and **[[CL:Functions:shared-initialize]]**.
Initialization arguments declared as valid by these //[[CL:Glossary:methods]]// are
valid when re-initializing an //[[CL:Glossary:instance]]//.

\itemitem{--}  Updating an //[[CL:Glossary:instance]]// to conform to a redefined //[[CL:Glossary:class]]//:
**[[CL:Functions:update-instance-for-redefined-class]]** and **[[CL:Functions:shared-initialize]]**.
Initialization arguments declared as valid by these //[[CL:Glossary:methods]]// are
valid when updating an //[[CL:Glossary:instance]]// to conform to a redefined //[[CL:Glossary:class]]//.

\itemitem{--} Updating an //[[CL:Glossary:instance]]// to conform to the definition of a
different //[[CL:Glossary:class]]//:
**[[CL:Functions:update-instance-for-different-class]]** and **[[CL:Functions:shared-initialize]]**.
Initialization arguments declared as valid by these //[[CL:Glossary:methods]]// are
valid when updating an //[[CL:Glossary:instance]]// to conform to the definition
of a different //[[CL:Glossary:class]]//.

\endlist
\endlist

The set of valid initialization arguments for a //[[CL:Glossary:class]]// is the set of
valid initialization arguments that either fill //[[CL:Glossary:slots]]// or supply
arguments to //[[CL:Glossary:methods]]//, along with the predefined initialization
argument **'':allow-other-keys''**.  The default value for 
**'':allow-other-keys''** is \nil.
\issue{INITIALIZATION-FUNCTION-KEYWORD-CHECKING}



Validity checking of initialization arguments is disabled if the value of
the initialization argument **'':allow-other-keys''** is //[[CL:Glossary:true]]//.


\endsubsection%{Declaring the Validity of Initialization Arguments}

\beginsubsection{Defaulting of Initialization Arguments}

A default value //[[CL:Glossary:form]]// can be supplied for an initialization
argument by using the **'':default-initargs''** //[[CL:Glossary:class]]// option.  If an
initialization argument is declared valid by some particular //[[CL:Glossary:class]]//,
its default  value form might be specified by a different //[[CL:Glossary:class]]//. 
In this case **'':default-initargs''** is used to supply a default value
for an inherited initialization argument.

The **'':default-initargs''** option is used only to provide default
values for initialization arguments; it does not declare a //[[CL:Glossary:symbol]]// 
as a valid initialization argument name.  Furthermore, 
the **'':default-initargs''** option is used only to provide default values for
initialization arguments when making an //[[CL:Glossary:instance]]//.
                     
The argument to the **'':default-initargs''** class 
option is a list of
alternating initialization argument names and //[[CL:Glossary:forms]]//.  
Each //[[CL:Glossary:form]]// is the
default  value form for the corresponding initialization
argument.  The default  value //[[CL:Glossary:form]]// of an initialization
argument is used and evaluated only if that initialization argument
does not appear in the arguments to **[[CL:Functions:make-instance]]** and is not
defaulted by a more specific //[[CL:Glossary:class]]//.  The default  value //[[CL:Glossary:form]]// is
evaluated in the lexical environment of the \macref{defclass} form that
supplied it; the resulting value is used as the initialization
argument's value.
                                          
The initialization arguments supplied to **[[CL:Functions:make-instance]]** are combined
with defaulted initialization arguments to produce a 
//[[CL:Glossary:defaulted initialization argument list]]//. A 
//[[CL:Glossary:defaulted initialization argument list]]//
is a list of alternating initialization argument names and
values in which unsupplied initialization arguments are defaulted and in
which the explicitly supplied initialization arguments appear earlier in
the list than the defaulted initialization arguments.  Defaulted
initialization arguments are ordered according to the order in the 
//[[CL:Glossary:class precedence list]]// of the //[[CL:Glossary:classes]]// that supplied the default values.
                                                    
There is a distinction between the purposes of the 
**'':default-initargs''** and the **'':initform''** options with respect to the
initialization of //[[CL:Glossary:slots]]//.  The **'':default-initargs''** 
class option
provides a mechanism for the user to give a default  value //[[CL:Glossary:form]]//
for an initialization argument without knowing whether the
initialization argument initializes a //[[CL:Glossary:slot]]// 
or is passed to a //[[CL:Glossary:method]]//.
If that initialization argument is not explicitly supplied in a call
to **[[CL:Functions:make-instance]]**, the default  value //[[CL:Glossary:form]]// is used, just
as if it had been supplied in the call.  In contrast, the 
**'':initform''** slot option provides a mechanism for the user to give a
default initial value form for a //[[CL:Glossary:slot]]//.  An **'':initform''** form is
used to initialize a //[[CL:Glossary:slot]]// only if no initialization argument
associated with that //[[CL:Glossary:slot]]// is given as an argument to 
**[[CL:Functions:make-instance]]** or is defaulted by **'':default-initargs''**.

\idxtext{order of evaluation}\idxtext{evaluation order}
The order of evaluation of default value //[[CL:Glossary:forms]]// for initialization
arguments and the order of evaluation of **'':initform''** forms are
undefined.  If the order of evaluation is important, 
**[[CL:Functions:initialize-instance]]** or **[[CL:Functions:shared-initialize]]** //[[CL:Glossary:methods]]// 
should be used
instead.

\endsubsection%{Defaulting of Initialization Arguments}

\beginsubsection{Rules for Initialization Arguments}
\DefineSection{InitargRules}
     
The **'':initarg''** slot option may be specified more than
once for a given //[[CL:Glossary:slot]]//.

The following rules specify when initialization arguments may be
multiply defined:

\beginlist

\itemitem{\bull} A given initialization argument can be used to
initialize more than one //[[CL:Glossary:slot]]// if the same initialization argument name
appears in more than one **'':initarg''** slot option.

\itemitem{\bull} A given initialization argument name can appear 
in the //[[CL:Glossary:lambda list]]// of more than one initialization //[[CL:Glossary:method]]//.

\itemitem{\bull} A given initialization argument name can
appear both in an **'':initarg''** slot option and 
in the //[[CL:Glossary:lambda list]]//
of an initialization //[[CL:Glossary:method]]//.

\endlist

\reviewer{The next three paragraphs could be replaced by ``If two or more
initialization arguments that initialize the same slot appear in the
//[[CL:Glossary:defaulted initialization argument list]]//, the leftmost of these supplies
the value, even if they have different names.''  And the rest would follow
from the rules above.}

If two or more initialization arguments that initialize the same
//[[CL:Glossary:slot]]// are given in the arguments to **[[CL:Functions:make-instance]]**, the
leftmost of these initialization arguments in the \term{initialization
argument list} supplies the value, even if the initialization arguments
have different names.

If two or more different initialization arguments that initialize the
same //[[CL:Glossary:slot]]// have default values and none is given explicitly in the
arguments to **[[CL:Functions:make-instance]]**, the initialization argument that
appears in a **'':default-initargs''** class option in the most specific
of the //[[CL:Glossary:classes]]// supplies the value. If a single
**'':default-initargs''** class option specifies two or more initialization
arguments that initialize the same //[[CL:Glossary:slot]]// and none is given
explicitly in the arguments to **[[CL:Functions:make-instance]]**, the leftmost in
the **'':default-initargs''** class option supplies the value, and the
values of the remaining default value //[[CL:Glossary:forms]]// are ignored.

Initialization arguments given explicitly in the arguments to
**[[CL:Functions:make-instance]]** appear to the left of defaulted initialization
arguments. Suppose that the classes $C\sub 1$ and $C\sub 2$ supply the
values of defaulted initialization arguments for different //[[CL:Glossary:slots]]//,
and suppose that $C\sub 1$ is more specific than $C\sub 2$; then the
defaulted initialization argument whose value is supplied by $C\sub 1$
is to the left of the defaulted initialization argument whose value is
supplied by $C\sub 2$ in the \term{defaulted initialization argument
list}.  If a single **'':default-initargs''** class option supplies the
values of initialization arguments for two different //[[CL:Glossary:slots]]//, the
initialization argument whose value is specified farther to the left in
the **'':default-initargs''** class option appears farther to the left in
the //[[CL:Glossary:defaulted initialization argument list]]//.

\reviewer{Barmar: End of claim made three paragraphs back.}
                                                        
If a //[[CL:Glossary:slot]]// has both an **'':initform''** form and an 
**'':initarg''** slot option, and the initialization argument is defaulted
using **'':default-initargs''** or is supplied to **[[CL:Functions:make-instance]]**,
the captured **'':initform''** form is neither used nor evaluated.

The following is an example of the above rules:

\code
 (defclass q () ((x :initarg a)))
 (defclass r (q) ((x :initarg b))
   (:default-initargs a 1 b 2))
\endcode

$$\vbox{\halign{\strut#\hfil&\quad\hfil#\hfil&\quad\hfil#\hfil\cr
{}&\bf Defaulted&{}\cr
\bf Form&\bf Initialization Argument List&\bf Contents of Slot X\cr
\noalign{\hrule}
{\tt (make-instance 'r)}&{\tt (a 1 b 2)}&{\tt 1}\cr
{\tt (make-instance 'r 'a 3)}&{\tt (a 3 b 2)}&{\tt 3}\cr
{\tt (make-instance 'r 'b 4)}&{\tt (b 4 a 1)}&{\tt 4}\cr
{\tt (make-instance 'r 'a 1 'a 2)}&{\tt (a 1 a 2 b 2)}&{\tt 1}\cr}}$$

\endsubsection%{Rules for Initialization arguments}

\beginsubsection{Shared-Initialize}
\DefineSection{SharedInitialize}
                      
The //[[CL:Glossary:generic function]]// **[[CL:Functions:shared-initialize]]** is used to fill the 
//[[CL:Glossary:slots]]//
of an //[[CL:Glossary:instance]]// 
using initialization arguments and **'':initform''**
forms when an //[[CL:Glossary:instance]]// is created, when an 
//[[CL:Glossary:instance]]// is re-initialized,
when an //[[CL:Glossary:instance]]// 
is updated to conform to a redefined //[[CL:Glossary:class]]//, and when
an //[[CL:Glossary:instance]]// is updated to conform to a different //[[CL:Glossary:class]]//.
It uses
standard //[[CL:Glossary:method]]// combination. It takes the following arguments: the
//[[CL:Glossary:instance]]// to be initialized, a 
specification of a set of //[[CL:Glossary:names]]// of //[[CL:Glossary:slots]]//
//[[CL:Glossary:accessible]]// in that //[[CL:Glossary:instance]]//, and any number of initialization
arguments.  The arguments after the first two must form an
//[[CL:Glossary:initialization argument list]]//.
                        
The second argument to **[[CL:Functions:shared-initialize]]** may be one of the following:

\beginlist

\itemitem{\bull} It can be a (possibly empty) //[[CL:Glossary:list]]// of //[[CL:Glossary:slot]]// names,
which specifies the set of those //[[CL:Glossary:slot]]// names. 








\itemitem{\bull} It can be the symbol \t, which specifies the set of all of the //[[CL:Glossary:slots]]//.

\endlist
                                               
There is a system-supplied primary //[[CL:Glossary:method]]// for **[[CL:Functions:shared-initialize]]**
whose first //[[CL:Glossary:parameter specializer]]// is \theclass{standard-object}.
This //[[CL:Glossary:method]]// behaves as follows on each //[[CL:Glossary:slot]]//, 
whether shared or local:

\beginlist

\itemitem{\bull} If an initialization argument in the 
//[[CL:Glossary:initialization argument list]]// specifies a value for that //[[CL:Glossary:slot]]//, 
that value is stored
into the //[[CL:Glossary:slot]]//, even if a value has already been stored in the //[[CL:Glossary:slot]]//
before the //[[CL:Glossary:method]]// is run.  
The affected //[[CL:Glossary:slots]]// are independent of which
//[[CL:Glossary:slots]]// are indicated by the second argument to **[[CL:Functions:shared-initialize]]**.

\itemitem{\bull} Any //[[CL:Glossary:slots]]// 
indicated by the second argument that are still
unbound at this point are initialized according to their 
**'':initform''** forms.  For any such //[[CL:Glossary:slot]]// 
that has an **'':initform''** form,
that //[[CL:Glossary:form]]// is evaluated in the 
lexical environment of its defining 
\macref{defclass} form and the result is stored into the //[[CL:Glossary:slot]]//.  
For example,
if a //[[CL:Glossary:before method]]// stores a value in the 
//[[CL:Glossary:slot]]//, the **'':initform''** form will not be used to supply a value 
for the //[[CL:Glossary:slot]]//.  If
the second argument specifies a //[[CL:Glossary:name]]// that does not correspond to any
//[[CL:Glossary:slots]]// //[[CL:Glossary:accessible]]// 
in the //[[CL:Glossary:instance]]//, the results are unspecified.

\itemitem{\bull} The rules mentioned in {\secref\InitargRules} are obeyed.

\endlist
                      
The generic function **[[CL:Functions:shared-initialize]]** is called by the
system-supplied primary //[[CL:Glossary:methods]]// 
for **[[CL:Functions:reinitialize-instance]]**,
**[[CL:Functions:update-instance-for-different-class]]**, 
**[[CL:Functions:update-instance-for-redefined-class]]**, and 
**[[CL:Functions:initialize-instance]]**.  Thus, //[[CL:Glossary:methods]]// can be written for 
**[[CL:Functions:shared-initialize]]** to specify actions that should be taken in all of
these contexts.

\endsubsection%{Shared-Initialize}

\beginsubsection{Initialize-Instance}
                      
The //[[CL:Glossary:generic function]]// **[[CL:Functions:initialize-instance]]** is called by 
**[[CL:Functions:make-instance]]** to initialize a newly created //[[CL:Glossary:instance]]//.
It uses //[[CL:Glossary:standard method combination]]//.  //[[CL:Glossary:Methods]]// for 
**[[CL:Functions:initialize-instance]]** can be defined in order to perform any
initialization that cannot be achieved 


simply by supplying initial values for //[[CL:Glossary:slots]]//.

                        
During initialization, **[[CL:Functions:initialize-instance]]** is invoked
after the following actions have been taken:

\beginlist 

\itemitem{\bull} The //[[CL:Glossary:defaulted initialization argument list]]// 
has been computed by combining the supplied //[[CL:Glossary:initialization argument list]]// 
with any default initialization arguments for the //[[CL:Glossary:class]]//.

\itemitem{\bull} The validity of the //[[CL:Glossary:defaulted initialization argument list]]//
has been checked.  If any of the initialization arguments has not
been declared as valid, an error is signaled. 

\itemitem{\bull} A new //[[CL:Glossary:instance]]// whose //[[CL:Glossary:slots]]// 
are unbound has been created.

\endlist
                      
The generic function **[[CL:Functions:initialize-instance]]** is called with the
new //[[CL:Glossary:instance]]// and the defaulted initialization arguments.  There is
a system-supplied primary //[[CL:Glossary:method]]// for **[[CL:Functions:initialize-instance]]**
whose //[[CL:Glossary:parameter specializer]]// is \theclass{standard-object}.  This
//[[CL:Glossary:method]]// calls the generic function 
**[[CL:Functions:shared-initialize]]** to fill in
the //[[CL:Glossary:slots]]// according to the initialization arguments and the 
**'':initform''** forms for the //[[CL:Glossary:slots]]//; the generic function 
**[[CL:Functions:shared-initialize]]** is called with the following arguments: the //[[CL:Glossary:instance]]//,
\t, and the defaulted initialization arguments.
           
Note that **[[CL:Functions:initialize-instance]]** provides the 
//[[CL:Glossary:defaulted initialization argument list]]// in its call to **[[CL:Functions:shared-initialize]]**,
so the first step performed by the system-supplied primary //[[CL:Glossary:method]]// for
**[[CL:Functions:shared-initialize]]** takes into account both the initialization
arguments provided in the call to **[[CL:Functions:make-instance]]** and the
//[[CL:Glossary:defaulted initialization argument list]]//.

//[[CL:Glossary:Methods]]// for **[[CL:Functions:initialize-instance]]** can be defined to specify
actions to be taken when an //[[CL:Glossary:instance]]// is initialized.  
If only //[[CL:Glossary:after methods]]// for **[[CL:Functions:initialize-instance]]** are defined, they will be
run after the system-supplied primary //[[CL:Glossary:method]]// for initialization and
therefore will not interfere with the default behavior of 
**[[CL:Functions:initialize-instance]]**.

The \OS\ provides two //[[CL:Glossary:functions]]// that are useful in the bodies of 
**[[CL:Functions:initialize-instance]]** methods.  \Thefunction{slot-boundp}
returns a //[[CL:Glossary:generic boolean]]// value that indicates whether a specified //[[CL:Glossary:slot]]// has a
value; this provides a mechanism for writing //[[CL:Glossary:after methods]]// for
**[[CL:Functions:initialize-instance]]** that initialize //[[CL:Glossary:slots]]// only if they have
not already been initialized.  \Thefunction{slot-makunbound}
causes the //[[CL:Glossary:slot]]// to have no value.

\endsubsection%{INITIALIZE-INSTANCE}

\beginsubsection{Definitions of Make-Instance and Initialize-Instance}
                      
The generic function **[[CL:Functions:make-instance]]** behaves as if it were defined as
follows, except that certain optimizations are permitted:

\code
 (defmethod make-instance ((class standard-class) &rest initargs)
   ...
   (let ((instance (apply #'allocate-instance class initargs)))
     (apply #'initialize-instance instance initargs)
     instance))

 (defmethod make-instance ((class-name symbol) &rest initargs)
   (apply #'make-instance (find-class class-name) initargs))
\endcode























                                      
The elided code in the definition of **[[CL:Functions:make-instance]]** 

augments the \f{initargs} with any //[[CL:Glossary:defaulted initialization arguments]]// and
checks the


resulting
initialization arguments to determine whether an initialization
argument was supplied that neither filled a //[[CL:Glossary:slot]]// nor supplied an argument
to an applicable //[[CL:Glossary:method]]//. 





                      
The generic function **[[CL:Functions:initialize-instance]]** behaves as if it were
defined as follows, except that certain optimizations are permitted:

\code
 (defmethod initialize-instance ((instance standard-object) &rest initargs)
   (apply #'shared-initialize instance t initargs)))
\endcode







These procedures can be customized.


                                                                  
Customizing at the Programmer Interface level includes using the 
**'':initform''**, **'':initarg''**, and **'':default-initargs''** options to
\macref{defclass}, as well as defining //[[CL:Glossary:methods]]//
for **[[CL:Functions:make-instance]]**, 

**[[CL:Functions:allocate-instance]]**,
and **[[CL:Functions:initialize-instance]]**.  It is also possible to define
//[[CL:Glossary:methods]]// for **[[CL:Functions:shared-initialize]]**, which would be invoked by the
generic functions **[[CL:Functions:reinitialize-instance]]**, 
**[[CL:Functions:update-instance-for-redefined-class]]**, 
**[[CL:Functions:update-instance-for-different-class]]**, and 
**[[CL:Functions:initialize-instance]]**.  
The meta-object level supports additional
customization.




                                                                
Implementations are permitted to make certain optimizations to 
**[[CL:Functions:initialize-instance]]** and **[[CL:Functions:shared-initialize]]**.  
The description of **[[CL:Functions:shared-initialize]]** in Chapter~7 mentions the
possible optimizations.














\endsubsection%{Definitions of MAKE-INSTANCE and Initialize-Instance}

