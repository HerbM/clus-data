







\beginsubSection{Data Type Definition}




















Information about //[[CL:Glossary:type]]// usage is located in 
the sections specified in \figref\TypeInfoXrefs. 
\figref\ObjectSystemClasses\ lists some //[[CL:Glossary:classes]]// 
that are particularly relevant to the \CLOS.
\figref\StandardizedConditionTypes\ lists the defined //[[CL:Glossary:condition]]// //[[CL:Glossary:types]]//.

\DefineFigure{TypeInfoXrefs}
\showtwo{Cross-References to Data Type Information}{
\hfil\b{Section} & Data Type \cr
\noalign{\vskip 2pt\hrule\vskip 2pt}
\secref\Classes                 & Object System types                \cr
\secref\Slots                   & Object System types                \cr
\secref\Objects                 & Object System types                \cr
\secref\GFsAndMethods           & Object System types                \cr
\secref\ConditionSystemConcepts & Condition System types             \cr
\chapref\TypesAndClasses        & Miscellaneous types		     \cr
\chapref\Syntax                 & All types---read and print syntax  \cr


\secref\TheLispPrinter          & All types---print syntax           \cr
\secref\Compilation             & All types---compilation issues     \cr
}




\endsubSection%{Data Type Definition}

\beginsubSection{Type Relationships}
\DefineSection{TypeRelationships}

\beginlist

\issue{DATA-TYPES-HIERARCHY-UNDERSPECIFIED}
\itemitem{\bull}
\Thetypes{cons}, \typeref{symbol}, \typeref{array}, \typeref{number},
\typeref{character}, \typeref{hash-table}, 
\issue{FUNCTION-TYPE:X3J13-MARCH-88}
\typeref{function},

\typeref{readtable}, \typeref{package}, \typeref{pathname}, \typeref{stream}, 
\typeref{random-state}, \typeref{condition}, \typeref{restart},
and any single other //[[CL:Glossary:type]]// created by \macref{defstruct},

\issue{TYPE-OF-AND-PREDEFINED-CLASSES:UNIFY-AND-EXTEND}
\issue{CLOS-CONDITIONS:INTEGRATE}
\macref{define-condition},


or \macref{defclass} are //[[CL:Glossary:pairwise]]// //[[CL:Glossary:disjoint]]//, 
except for type relations explicitly established by specifying 
//[[CL:Glossary:superclasses]]// in \macref{defclass} 
\issue{TYPE-OF-AND-PREDEFINED-CLASSES:UNIFY-AND-EXTEND}
\issue{CLOS-CONDITIONS:INTEGRATE}
or \macref{define-condition}


or the **'':include''** option of \macref{destruct}.








\issue{DATA-TYPES-HIERARCHY-UNDERSPECIFIED}









\itemitem{\bull} Any two //[[CL:Glossary:types]]// created by \macref{defstruct} are 
//[[CL:Glossary:disjoint]]// unless
one is a //[[CL:Glossary:supertype]]// of the other by virtue of
the \macref{defstruct} **'':include''** option.

\editornote{KMP: The comments in the source say gray suggested some change
from ``common superclass'' to ``common subclass'' in the following, but the
result looks suspicious to me.}







\itemitem{\bull}
Any two //[[CL:Glossary:distinct]]// //[[CL:Glossary:classes]]// created by \macref{defclass} 

or \macref{define-condition}
are //[[CL:Glossary:disjoint]]// unless they have a common //[[CL:Glossary:subclass]]// or
one //[[CL:Glossary:class]]// is a //[[CL:Glossary:subclass]]// of the other.















\issue{COMMON-TYPE:REMOVE}

















\itemitem{\bull} 
An implementation may be extended to add other //[[CL:Glossary:subtype]]//
relationships between the specified //[[CL:Glossary:types]]//, as long as they do
not violate the type relationships and disjointness requirements
specified here.  An implementation may define additional //[[CL:Glossary:types]]//
that are //[[CL:Glossary:subtypes]]// or //[[CL:Glossary:supertypes]]// of any
specified //[[CL:Glossary:types]]//, as long as each additional //[[CL:Glossary:type]]// is
a \subtypeof{t} and a \supertypeof{nil} and the disjointness requirements
are not violated.
 
\issue{TYPE-OF-AND-PREDEFINED-CLASSES:UNIFY-AND-EXTEND}
At the discretion of the implementation, either \typeref{standard-object}
or \typeref{structure-object} might appear in any class precedence list
for a //[[CL:Glossary:system class]]// that does not already specify either 
\typeref{standard-object} or \typeref{structure-object}.  If it does,
it must precede \theclass{t} and follow all other //[[CL:Glossary:standardized]]// //[[CL:Glossary:classes]]//.


\endlist                                     

\endsubSection%{Type relationships}


\beginsubSection{Type Specifiers}
\DefineSection{TypeSpecifiers}

\issue{ARRAY-TYPE-ELEMENT-TYPE-SEMANTICS:UNIFY-UPGRADING}





//[[CL:Glossary:Type specifiers]]// can be //[[CL:Glossary:symbols]]//, //[[CL:Glossary:classes]]//, or //[[CL:Glossary:lists]]//.
\figref\StandardizedAtomicTypeSpecs\ lists //[[CL:Glossary:symbols]]// that are
  //[[CL:Glossary:standardized]]// //[[CL:Glossary:atomic type specifiers]]//, and
\figref\StandardizedCompoundTypeSpecNames\ lists
 //[[CL:Glossary:standardized]]// //[[CL:Glossary:compound type specifier]]// //[[CL:Glossary:names]]//.
For syntax information, see the dictionary entry for the corresponding //[[CL:Glossary:type specifier]]//.
It is possible to define new //[[CL:Glossary:type specifiers]]// using
 \macref{defclass},
 \macref{define-condition},
 \macref{defstruct}, 
or
 \macref{deftype}.

\issue{CHARACTER-VS-CHAR:LESS-INCONSISTENT-SHORT}
\issue{STREAM-ACCESS:ADD-TYPES-ACCESSORS}

\DefineFigure{StandardizedAtomicTypeSpecs}

\displaythree{Standardized Atomic Type Specifiers}{
arithmetic-error&function&simple-condition\cr
array&generic-function&simple-error\cr
atom&hash-table&simple-string\cr
base-char&integer&simple-type-error\cr
base-string&keyword&simple-vector\cr
bignum&list&simple-warning\cr
bit&logical-pathname&single-float\cr
bit-vector&long-float&standard-char\cr
broadcast-stream&method&standard-class\cr
built-in-class&method-combination&standard-generic-function\cr
cell-error&nil&standard-method\cr
character&null&standard-object\cr
class&number&storage-condition\cr
compiled-function&package&stream\cr
complex&package-error&stream-error\cr
concatenated-stream&parse-error&string\cr
condition&pathname&string-stream\cr
cons&print-not-readable&structure-class\cr
control-error&program-error&structure-object\cr
division-by-zero&random-state&style-warning\cr
double-float&ratio&symbol\cr
echo-stream&rational&synonym-stream\cr
end-of-file&reader-error&t\cr
error&readtable&two-way-stream\cr
extended-char&real&type-error\cr
file-error&restart&unbound-slot\cr
file-stream&sequence&unbound-variable\cr
fixnum&serious-condition&undefined-function\cr
float&short-float&unsigned-byte\cr
floating-point-inexact&signed-byte&vector\cr
floating-point-invalid-operation&simple-array&warning\cr
floating-point-overflow&simple-base-string&\cr
floating-point-underflow&simple-bit-vector&\cr
}




\indent               

If a //[[CL:Glossary:type specifier]]// is a //[[CL:Glossary:list]]//, the //[[CL:Glossary:car]]// of the //[[CL:Glossary:list]]// 
is a //[[CL:Glossary:symbol]]//, and the rest of the //[[CL:Glossary:list]]// is subsidiary
//[[CL:Glossary:type]]// information.  Such a //[[CL:Glossary:type specifier]]// is called 
a //[[CL:Glossary:compound type specifier]]//.
Except as explicitly stated otherwise,
the subsidiary items can be unspecified.
The unspecified subsidiary items are indicated
by writing \f{*}.  For example, to completely specify
a //[[CL:Glossary:vector]]//, the //[[CL:Glossary:type]]// of the elements
and the length of the //[[CL:Glossary:vector]]// must be present.

\code
 (vector double-float 100)
\endcode
The following leaves the length unspecified:

\code
 (vector double-float *)
\endcode
The following leaves the element type unspecified:

\code
 (vector * 100)                                      
\endcode
Suppose that two //[[CL:Glossary:type specifiers]]// are the same except that the first
has a \f{*} where the second has a more explicit specification.
Then the second denotes a //[[CL:Glossary:subtype]]// 
of the //[[CL:Glossary:type]]// denoted by the first.


If a //[[CL:Glossary:list]]// has one or more unspecified items at the end, 
those items can be dropped.
If dropping all occurrences of \f{*} results in a //[[CL:Glossary:singleton]]// //[[CL:Glossary:list]]//,
then the parentheses can be dropped as well (the list can be replaced
by the //[[CL:Glossary:symbol]]// in its //[[CL:Glossary:car]]//).  
For example,                       
{\tt (vector double-float *)}                    
can be abbreviated to {\tt (vector double-float)},               
and {\tt (vector * *)} can be abbreviated to {\tt (vector)} 
and then to 
{\tt vector}.

\issue{REAL-NUMBER-TYPE:X3J13-MAR-89}


\DefineFigure{StandardizedCompoundTypeSpecNames}
\displaythree{Standardized Compound Type Specifier Names}{
and&long-float&simple-base-string\cr
array&member&simple-bit-vector\cr
base-string&mod&simple-string\cr
bit-vector&not&simple-vector\cr
complex&or&single-float\cr
cons&rational&string\cr
double-float&real&unsigned-byte\cr
eql&satisfies&values\cr
float&short-float&vector\cr
function&signed-byte&\cr
integer&simple-array&\cr
}


\Thenextfigure\ show the //[[CL:Glossary:defined names]]// that can be used as 
//[[CL:Glossary:compound type specifier]]// //[[CL:Glossary:names]]//
but that cannot be used as //[[CL:Glossary:atomic type specifiers]]//.

\displaythree{Standardized Compound-Only Type Specifier Names}{
and&mod&satisfies\cr
eql&not&values\cr
member&or&\cr
}



New //[[CL:Glossary:type specifiers]]// can come into existence in two ways.
\beginlist
\itemitem{\bull} 
 Defining a structure by using \macref{defstruct} without using
 the **'':type''** specifier or defining a //[[CL:Glossary:class]]// by using 
 \macref{defclass} 

 or \macref{define-condition}
 automatically causes the name of the structure 
 or class to be a new //[[CL:Glossary:type specifier]]// //[[CL:Glossary:symbol]]//.
\itemitem{\bull} 
 \macref{deftype} can be used to define \newtermidx{derived type specifiers}{derived type specifier},
 which act as `abbreviations' for other //[[CL:Glossary:type specifiers]]//.
\endlist

A //[[CL:Glossary:class]]// //[[CL:Glossary:object]]// can be used as a //[[CL:Glossary:type specifier]]//. 
When used this way, it denotes the set of all members of that //[[CL:Glossary:class]]//.

\Thenextfigure\ shows some //[[CL:Glossary:defined names]]// relating to 
//[[CL:Glossary:types]]// and //[[CL:Glossary:declarations]]//.


\DefineFigure{TypesAndDeclsNames}
\displaythree{Defined names relating to types and declarations.}{
coerce&defstruct&subtypep\cr
declaim&deftype&the\cr
declare&ftype&type\cr
defclass&locally&type-of\cr
define-condition&proclaim&typep\cr
}

\Thenextfigure\ shows all //[[CL:Glossary:defined names]]// that are //[[CL:Glossary:type specifier]]// //[[CL:Glossary:names]]//,
whether for //[[CL:Glossary:atomic type specifiers]]// or //[[CL:Glossary:compound type specifiers]]//;
this list is the union of the lists in \figref\StandardizedAtomicTypeSpecs\ 
and \figref\StandardizedCompoundTypeSpecNames.

\DefineFigure{StandardizedTypeSpecifierNames}
\displaythree{Standardized Type Specifier Names}{
and&function&simple-array\cr
arithmetic-error&generic-function&simple-base-string\cr
array&hash-table&simple-bit-vector\cr
atom&integer&simple-condition\cr
base-char&keyword&simple-error\cr
base-string&list&simple-string\cr
bignum&logical-pathname&simple-type-error\cr
bit&long-float&simple-vector\cr
bit-vector&member&simple-warning\cr
broadcast-stream&method&single-float\cr
built-in-class&method-combination&standard-char\cr
cell-error&mod&standard-class\cr
character&nil&standard-generic-function\cr
class&not&standard-method\cr
compiled-function&null&standard-object\cr
complex&number&storage-condition\cr
concatenated-stream&or&stream\cr
condition&package&stream-error\cr
cons&package-error&string\cr
control-error&parse-error&string-stream\cr
division-by-zero&pathname&structure-class\cr
double-float&print-not-readable&structure-object\cr
echo-stream&program-error&style-warning\cr
end-of-file&random-state&symbol\cr
eql&ratio&synonym-stream\cr
error&rational&t\cr
extended-char&reader-error&two-way-stream\cr
file-error&readtable&type-error\cr
file-stream&real&unbound-slot\cr
fixnum&restart&unbound-variable\cr
float&satisfies&undefined-function\cr
floating-point-inexact&sequence&unsigned-byte\cr
floating-point-invalid-operation&serious-condition&values\cr
floating-point-overflow&short-float&vector\cr
floating-point-underflow&signed-byte&warning\cr
}

\endsubSection%{Type Specifiers}

