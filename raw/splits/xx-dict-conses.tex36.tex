====== Accessor REST ======

====Syntax====

**rest** //list// → //tail// (**setf** (**rest** //list//) //new-tail//)

====Arguments and Values====

//list// - a //[[CL:Glossary:list]]//,

which might be a //[[CL:Glossary:dotted list]]// or a //[[CL:Glossary:circular list]]//.

//tail// - an //[[CL:Glossary:object]]//.

====Description====

**[[CL:Functions:rest]]** performs the same operation as **[[CL:Functions:cdr]]**, but mnemonically complements **[[CL:Functions:first]]**. Specifically,

<blockquote> (rest //list//) ≡ (cdr //list//) ([[CL:Macros:setf]] (rest //list//) //new-tail//) ≡ ([[CL:Macros:setf]] (cdr //list//) //new-tail//) </blockquote>

====Examples====

<blockquote> (rest '(1 2)) → (2) (rest '(1 . 2)) → 2 (rest '(1)) → NIL ([[CL:Macros:defparameter]] *cons* '(1 . 2)) → (1 . 2) ([[CL:Macros:setf]] (rest *cons*) "two") → "two" *cons* → (1 . "two") </blockquote>

====Side Effects====

None.

====Affected By====

None.

====Exceptional Situations====

None.

====See Also====

**[[CL:Functions:cdr]]**, **[[CL:Functions:nthcdr]]**

====Notes====

**[[CL:Functions:rest]]** is often preferred stylistically over **[[CL:Functions:cdr]]** when the argument is to being subjectively viewed as a //[[CL:Glossary:list]]// rather than as a //[[CL:Glossary:cons]]//.

\issue{DOTTED-LIST-ARGUMENTS:CLARIFY}
