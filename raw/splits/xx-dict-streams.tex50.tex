====== Function MAKE-STRING-INPUT-STREAM ======

====Syntax====

**make-string-input-stream {string** //\opt} start end// → //string-stream//

====Arguments and Values====

//string// - a //[[CL:Glossary:string]]//.

//start//, //end// - //[[CL:Glossary:bounding index designators]]// of //string//. \Defaults{//start// and //end//}{''0'' and **[[CL:Constant Variables:nil]]**}

//string-stream// - an //[[CL:Glossary:input]]// //[[CL:Glossary:string stream]]//.

====Description====

Returns an //[[CL:Glossary:input]]// //[[CL:Glossary:string stream]]//. This //[[CL:Glossary:stream]]// will supply, in order, the //[[CL:Glossary:characters]]// in the substring of //string// //[[CL:Glossary:bounded]]// by //start// and //end//. After the last //[[CL:Glossary:character]]// has been supplied, the //[[CL:Glossary:string stream]]// will then be at //[[CL:Glossary:end of file]]//.

====Examples====

<blockquote> (let ((string-stream (make-string-input-stream "1 one "))) (list (read string-stream nil nil) (read string-stream nil nil) (read string-stream nil nil))) → (1 ONE NIL)

(read (make-string-input-stream "prefixtargetsuffix" 6 12)) → TARGET </blockquote>

====Side Effects====

None.

====Affected By====

None.

====Exceptional Situations====

None.

====See Also====

**[[CL:Macros:with-input-from-string]]**

====Notes====

None.

\issue{RANGE-OF-START-AND-END-PARAMETERS:INTEGER-AND-INTEGER-NIL} \issue{STREAM-ACCESS:ADD-TYPES-ACCESSORS}
