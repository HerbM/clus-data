====== Type SHORT-FLOAT, SINGLE-FLOAT, DOUBLE-FLOAT, LONG-FLOAT ======

====Supertypes====

**[[CL:Types:short-float]]**: **[[CL:Types:short-float]]**, **[[CL:Types:float]]**,

**[[CL:Types:real]]**, **[[CL:Types:number]]**, **[[CL:Types:t]]**

**[[CL:Types:single-float]]**: **[[CL:Types:single-float]]**, **[[CL:Types:float]]**,

**[[CL:Types:real]]**, **[[CL:Types:number]]**, **[[CL:Types:t]]**

**[[CL:Types:double-float]]**: **[[CL:Types:double-float]]**, **[[CL:Types:float]]**,

**[[CL:Types:real]]**, **[[CL:Types:number]]**, **[[CL:Types:t]]**

**[[CL:Types:long-float]]**: **[[CL:Types:long-float]]**, **[[CL:Types:float]]**,

**[[CL:Types:real]]**, **[[CL:Types:number]]**, **[[CL:Types:t]]**

====Description====

For the four defined subtypes of **[[CL:Types:float]]**, it is true that

intermediate between the type **[[CL:Types:short-float]]** and the type **[[CL:Types:long-float]]** are the type **[[CL:Types:single-float]]** and the type **[[CL:Types:double-float]]**. The precise definition of these categories is

//[[CL:Glossary:implementation-defined]]//. The precision (measured in "bits", computed as ''p\log<sub>2</sub>b'') and the exponent size (also measured in "bits," computed as ''\log<sub>2</sub>(n+1)'', where ''n'' is the maximum exponent value) is recommended to be at least as great as the values in \thenextfigure. Each of the defined subtypes of **[[CL:Types:float]]** might or might not have a minus zero.

\showthree{Recommended Minimum Floating-Point Precision and Exponent Size}{ \hfil\b{Format}& \b{Minimum Precision}& \b{Minimum Exponent Size}\cr \noalign{\vskip 2pt\hrule\vskip 2pt} Short & 13 bits & 5 bits\cr Single & 24 bits & 8 bits\cr Double & 50 bits & 8 bits\cr Long & 50 bits & 8 bits\cr }

There can be fewer than four internal representations for //[[CL:Glossary:floats]]//. If there are fewer distinct representations, the following rules apply: \beginlist \itemitem{--} If there is only one, it is

the type **[[CL:Types:single-float]]**. In this representation, an //[[CL:Glossary:object]]// is simultaneously of //[[CL:Glossary:types]]// **[[CL:Types:single-float]]**, **[[CL:Types:double-float]]**, **[[CL:Types:short-float]]**, and **[[CL:Types:long-float]]**. \itemitem{--} Two internal representations can be arranged in either of the following ways: \beginlist \itemitem{\bull} Two //[[CL:Glossary:types]]// are provided: **[[CL:Types:single-float]]** and **[[CL:Types:short-float]]**. An //[[CL:Glossary:object]]// is simultaneously of //[[CL:Glossary:types]]// **[[CL:Types:single-float]]**, **[[CL:Types:double-float]]**, and **[[CL:Types:long-float]]**. \itemitem{\bull} Two //[[CL:Glossary:types]]// are provided: **[[CL:Types:single-float]]** and **[[CL:Types:double-float]]**. An //[[CL:Glossary:object]]// is simultaneously of //[[CL:Glossary:types]]// **[[CL:Types:single-float]]** and **[[CL:Types:short-float]]**, or **[[CL:Types:double-float]]** and **[[CL:Types:long-float]]**. \endlist \itemitem{--} Three internal representations can be arranged in either of the following ways: \beginlist \itemitem{\bull} Three //[[CL:Glossary:types]]// are provided: **[[CL:Types:short-float]]**, **[[CL:Types:single-float]]**, and **[[CL:Types:double-float]]**. An //[[CL:Glossary:object]]// can simultaneously be of //[[CL:Glossary:type]]// **[[CL:Types:double-float]]** and **[[CL:Types:long-float]]**. \itemitem{\bull} Three //[[CL:Glossary:types]]// are provided: **[[CL:Types:single-float]]**, **[[CL:Types:double-float]]**,and **[[CL:Types:long-float]]**. An //[[CL:Glossary:object]]// can simultaneously be of //[[CL:Glossary:types]]// **[[CL:Types:single-float]]** and **[[CL:Types:short-float]]**. \endlist \endlist

====Compound Type Specifier Kind====

Abbreviating.

====Compound Type Specifier Syntax====

//**short-float** //[short-lower-limit [short-upper-limit]]//// //**single-float** //[single-lower-limit [single-upper-limit]]//// //**double-float** //[double-lower-limit [double-upper-limit]]//// //**long-float** //[long-lower-limit [long-upper-limit]]////

====Compound Type Specifier Arguments====

//short-lower-limit//, //short-upper-limit// - //[[CL:Glossary:interval designators]]// for //[[CL:Glossary:type]]// **[[CL:Types:short-float]]**. \DefaultEach{//lower-limit// and //upper-limit//}{the //[[CL:Glossary:symbol]]// **[[CL:Types:wildcard|*]]**}

//single-lower-limit//, //single-upper-limit// - //[[CL:Glossary:interval designators]]// for //[[CL:Glossary:type]]// **[[CL:Types:single-float]]**. \DefaultEach{//lower-limit// and //upper-limit//}{the //[[CL:Glossary:symbol]]// **[[CL:Types:wildcard|*]]**}

//double-lower-limit//, //double-upper-limit// - //[[CL:Glossary:interval designators]]// for //[[CL:Glossary:type]]// **[[CL:Types:double-float]]**. \DefaultEach{//lower-limit// and //upper-limit//}{the //[[CL:Glossary:symbol]]// **[[CL:Types:wildcard|*]]**}

//long-lower-limit//, //long-upper-limit// - //[[CL:Glossary:interval designators]]// for //[[CL:Glossary:type]]// **[[CL:Types:long-float]]**. \DefaultEach{//lower-limit// and //upper-limit//}{the //[[CL:Glossary:symbol]]// **[[CL:Types:wildcard|*]]**}

====Compound Type Specifier Description====

Each of these denotes the set of //[[CL:Glossary:floats]]// of the indicated //[[CL:Glossary:type]]// that are on the interval specified by the //[[CL:Glossary:interval designators]]//.

\issue{REAL-NUMBER-TYPE:X3J13-MAR-89} \issue{REAL-NUMBER-TYPE:X3J13-MAR-89} \issue{REAL-NUMBER-TYPE:X3J13-MAR-89} \issue{REAL-NUMBER-TYPE:X3J13-MAR-89}
