====== Function ASH ======

====Syntax====

**ash** //integer count// → //shifted-integer//

====Arguments and Values====

//integer// - an //[[CL:Glossary:integer]]//.

//count// - an //[[CL:Glossary:integer]]//.

//shifted-integer// - an //[[CL:Glossary:integer]]//.

====Description====

**[[CL:Functions:ash]]** performs the arithmetic shift operation on the binary representation of //integer//, which is treated as if it were binary.

**[[CL:Functions:ash]]** shifts //integer// arithmetically left by //count// bit positions if //count// is positive, or right //count// bit positions if //count// is negative. The shifted value of the same sign as //integer// is returned.

Mathematically speaking, **[[CL:Functions:ash]]** performs the computation ''floor''(//integer//\centerdot ''2^//count//'').

Logically, **[[CL:Functions:ash]]** moves all of the bits in //integer// to the left, adding zero-bits at the right, or moves them to the right, discarding bits.

**[[CL:Functions:ash]]** is defined to behave as if //integer// were represented in two's complement form, regardless of how //[[CL:Glossary:integers]]// are represented internally. ====Examples==== <blockquote> (ash 16 1) → 32 (ash 16 0) → 16 (ash 16 -1) → 8 (ash -100000000000000000000000000000000 -100) → -79 </blockquote>

====Affected By====

None.

====Exceptional Situations====

Should signal an error of type type-error if //integer// is not an //[[CL:Glossary:integer]]//. Should signal an error of type type-error if //count// is not an //[[CL:Glossary:integer]]//. Might signal **[[CL:Types:arithmetic-error]]**.

====See Also====

None.

====Notes====

<blockquote> (logbitp //j// (ash //n// //k//)) ≡ (and (>= //j// //k//) (logbitp (- //j// //k//) //n//)) </blockquote>

