====== Function MISMATCH ======

====Syntax====

\DefunWithValuesNewline mismatch {sequence-1 sequence-2 ''&key'' from-end test test-not key start1 start2 end1 end2} {position}

====Arguments and Values====

//Sequence-1// - a //[[CL:Glossary:sequence]]//.

//Sequence-2// - a //[[CL:Glossary:sequence]]//.

//from-end// - a //[[CL:Glossary:generalized boolean]]//. The default is //[[CL:Glossary:false]]//.

//test// - a //[[CL:Glossary:designator]]// for a //[[CL:Glossary:function]]// of two //[[CL:Glossary:arguments]]// that returns a //[[CL:Glossary:generalized boolean]]//.

//test-not// - a //[[CL:Glossary:designator]]// for a //[[CL:Glossary:function]]// of two //[[CL:Glossary:arguments]]// that returns a //[[CL:Glossary:generalized boolean]]//.

//start1//, //end1// - //[[CL:Glossary:bounding index designators]]// of //sequence-1//. \Defaults{//start1// and //end1//}{''0'' and **[[CL:Constant Variables:nil]]**}

//start2//, //end2// - //[[CL:Glossary:bounding index designators]]// of //sequence-2//. \Defaults{//start2// and //end2//}{''0'' and **[[CL:Constant Variables:nil]]**}

//key// - a //[[CL:Glossary:designator]]// for a //[[CL:Glossary:function]]// of one argument, or **[[CL:Constant Variables:nil]]**.

//position// - a //[[CL:Glossary:bounding index]]// of //sequence-1//, or **[[CL:Constant Variables:nil]]**.

====Description====

The specified subsequences of //sequence-1// and //sequence-2// are compared element-wise.

The //key// argument is used for both the //sequence-1// and the //sequence-2//.

If //sequence-1// and //sequence-2// are of equal length and match in every element, the result is //[[CL:Glossary:false]]//. Otherwise, the result is a non-negative //[[CL:Glossary:integer]]//, the index within //sequence-1// of the leftmost or rightmost position, depending on //from-end//, at which the two subsequences fail to match. If one subsequence is shorter than and a matching prefix of the other, the result is the index relative to //sequence-1// beyond the last position tested.

If //from-end// is //[[CL:Glossary:true]]//, then one plus the index of the rightmost position in which the //sequences// differ is returned. In effect, the subsequences are aligned at their right-hand ends; then, the last elements are compared, the penultimate elements, and so on. The index returned is an index relative to //sequence-1//.

====Examples==== <blockquote> (mismatch "abcd" "ABCDE" :test #'char-equal) → 4 (mismatch '(3 2 1 1 2 3) '(1 2 3) :from-end t) → 3 (mismatch '(1 2 3) '(2 3 4) :test-not #'eq :key #'oddp) → NIL (mismatch '(1 2 3 4 5 6) '(3 4 5 6 7) :start1 2 :end2 4) → NIL </blockquote>

====Side Effects====

None.

====Affected By====

None.

====Exceptional Situations====

None.

====See Also====

{\secref\TraversalRules}

====Notes====


The **'':test-not''** //[[CL:Glossary:argument]]// is deprecated.

\issue{SUBSEQ-OUT-OF-BOUNDS} \issue{RANGE-OF-START-AND-END-PARAMETERS:INTEGER-AND-INTEGER-NIL} \issue{MAPPING-DESTRUCTIVE-INTERACTION:EXPLICITLY-VAGUE} \issue{TEST-NOT-IF-NOT:FLUSH-ALL}
