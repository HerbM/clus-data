====== Function EVAL ======

====Syntax====

\DefunWithValues eval {form} {\starparam{result}}

====Arguments and Values====

//form// - a //[[CL:Glossary:form]]//.

//results// - the //[[CL:Glossary:values]]// //[[CL:Glossary:yielded]]// by the //[[CL:Glossary:evaluation]]// of //form//.

====Description====

Evaluates //form// in the current //[[CL:Glossary:dynamic environment]]// and the //[[CL:Glossary:null lexical environment]]//.

**[[CL:Functions:eval]]** is a user interface to the evaluator.

The evaluator expands macro calls as if through the use of **[[CL:Functions:macroexpand-1]]**.

Constants appearing in code processed by **[[CL:Functions:eval]]** are not copied nor coalesced. The code resulting from the execution of **[[CL:Functions:eval]]** references //[[CL:Glossary:object|objects]]// that are **[[CL:Functions:eql]]** to the corresponding //[[CL:Glossary:object|objects]]// in the source code.

====Examples====

<blockquote> ([[CL:Macros:defparameter]] form '(1+ a) a 999) → 999 (eval form) → 1000 (eval 'form) → (1+ A) (let ((a '(this would break if eval used local value))) (eval form)) → 1000 </blockquote>

====Affected By====

None.

====Exceptional Situations====

None.

====See Also====

**[[CL:Functions:macroexpand-1]]**, {\secref\EvaluationModel}

====Notes====

To obtain the current dynamic value of a //[[CL:Glossary:symbol]]//, use of **[[CL:Functions:symbol-value]]** is equivalent (and usually preferable) to use of **[[CL:Functions:eval]]**.

Note that an **[[CL:Functions:eval]]** //[[CL:Glossary:form]]// involves two levels of //[[CL:Glossary:evaluation]]// for its //[[CL:Glossary:argument]]//. First, //form// is //[[CL:Glossary:evaluated]]// by the normal argument evaluation mechanism as would occur with any //[[CL:Glossary:call]]//. The //[[CL:Glossary:object]]// that results from this normal //[[CL:Glossary:argument]]// //[[CL:Glossary:evaluation]]// becomes the //[[CL:Glossary:value]]// of the //form// //[[CL:Glossary:parameter]]//, and is then //[[CL:Glossary:evaluated]]// as part of the **[[CL:Functions:eval]]** //[[CL:Glossary:form]]//. For example:

<blockquote> (eval (list 'cdr (car '((quote (a . b)) c)))) → b </blockquote> The //[[CL:Glossary:argument]]// //[[CL:Glossary:form]]// ''(list 'cdr (car '((quote (a . b)) c)))'' is evaluated in the usual way to produce the //[[CL:Glossary:argument]]// ''(cdr (quote (a . b)))'';

**[[CL:Functions:eval]]** then evaluates its //[[CL:Glossary:argument]]//, ''(cdr (quote (a . b)))'', to produce ''b''. Since a single //[[CL:Glossary:evaluation]]// already occurs for any //[[CL:Glossary:argument]]// //[[CL:Glossary:form]]// in any //[[CL:Glossary:function form]]//, **[[CL:Functions:eval]]** is sometimes said to perform "an extra level of evaluation."

\issue{QUOTE-SEMANTICS:NO-COPYING} \issue{EVALHOOK-STEP-CONFUSION:X3J13-NOV-89} \issue{EVALHOOK-STEP-CONFUSION:FIX}
