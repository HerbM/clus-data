====== System Class FLOAT ======

====Class Precedence List==== **[[CL:Types:float]]**, **[[CL:Types:real]]**, **[[CL:Types:number]]**, **[[CL:Types:t]]**

====Description====

A //[[CL:Glossary:float]]// is a mathematical rational (but ''not'' a \clisp\ //[[CL:Glossary:rational]]//) of the form ''s\cdot f\cdot b^{e-p}'', where ''s'' is ''+1'' or ''-1'', the ''sign''; ''b'' is an //[[CL:Glossary:integer]]// greater than~1, the ''base'' or ''radix'' of the representation; ''p'' is a positive //[[CL:Glossary:integer]]//, the ''precision'' (in base-''b'' digits) of the //[[CL:Glossary:float]]//; ''f'' is a positive //[[CL:Glossary:integer]]// between ''b^{p-1}'' and ''b^p-1'' (inclusive), the significand; and ''e'' is an //[[CL:Glossary:integer]]//, the exponent. The value of ''p'' and the range of~''e'' depends on the implementation and on the type of //[[CL:Glossary:float]]// within that implementation. In addition, there is a floating-point zero; depending on the implementation, there can also be a "minus zero". If there is no minus zero, then ''0.0'' and~''-0.0'' are both interpreted as simply a floating-point zero. **[[CL:Functions:(= 0.0 -0.0)]]** is always true. If there is a minus zero, **[[CL:Functions:(eql -0.0 0.0)]]** is //[[CL:Glossary:false]]//, otherwise it is //[[CL:Glossary:true]]//.

\reviewer{Barmar: What about IEEE NaNs and infinities?}

\reviewer{RWK: In the following, what is the "ordering"? precision? range? Can there be additional subtypes of float or does "others" in the list of four?}

The //[[CL:Glossary:types]]// **[[CL:Types:short-float]]**, **[[CL:Types:single-float]]**, **[[CL:Types:double-float]]**, and **[[CL:Types:long-float]]** are subtypes of **[[CL:Types:float]]**. Any two of them must be either //[[CL:Glossary:disjoint]]// //[[CL:Glossary:types]]// or the //[[CL:Glossary:same]]// //[[CL:Glossary:type]]//; if the //[[CL:Glossary:same]]// //[[CL:Glossary:type]]//, then any other //[[CL:Glossary:types]]// between them in the above ordering must also be the //[[CL:Glossary:same]]// //[[CL:Glossary:type]]//. For example, if the type **[[CL:Types:single-float]]** and the type **[[CL:Types:long-float]]** are the //[[CL:Glossary:same]]// //[[CL:Glossary:type]]//, then the type **[[CL:Types:double-float]]** must be the //[[CL:Glossary:same]]// //[[CL:Glossary:type]]// also.

\realtypespec{float}

====See Also====

{\figref\SyntaxForNumericTokens}, {\secref\NumsFromTokens}, {\secref\PrintingFloats}

====Notes====

Note that all mathematical integers are representable not only as \clisp\ //[[CL:Glossary:reals]]//, but also as //[[CL:Glossary:complex floats]]//. For example, possible representations of the mathematical number ''1'' include the //[[CL:Glossary:integer]]// ''1'', the //[[CL:Glossary:float]]// ''1.0'', or the //[[CL:Glossary:complex]]// ''#C(1.0 0.0)''.

\issue{REAL-NUMBER-TYPE:X3J13-MAR-89}
