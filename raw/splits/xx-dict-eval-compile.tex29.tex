====== Special Operator THE ======

====Syntax====

\DefspecWithValues the {value-type form} {\starparam{result}}

====Arguments and Values====

//value-type// - a //[[CL:Glossary:type specifier]]//; \noeval.

//form// - a //[[CL:Glossary:form]]//; \eval.

//results// - the //[[CL:Glossary:values]]// resulting from the //[[CL:Glossary:evaluation]]// of //form//. These //[[CL:Glossary:values]]// must conform to the //[[CL:Glossary:type]]// supplied by //value-type//; see below.

====Description====

\specref{the} specifies that the //[[CL:Glossary:values]]// returned by //form// are of the //[[CL:Glossary:types]]// specified by //value-type//. The consequences are undefined if any //result// is not of the declared type.

It is permissible for //form// to //[[CL:Glossary:yield]]// a different number of //[[CL:Glossary:values]]// than are specified by //value-type//, provided that the values for which //types// are declared are indeed of those //[[CL:Glossary:types]]//. Missing values are treated as **[[CL:Constant Variables:nil]]** for the purposes of checking their //[[CL:Glossary:types]]//.

Regardless of number of //[[CL:Glossary:values]]// declared by //value-type//, the number of //[[CL:Glossary:values]]// returned by \thespecform{the} is the same as the number of //[[CL:Glossary:values]]// returned by //form//.

====Examples====

<blockquote> (the symbol (car (list (gensym)))) → #:G9876 (the fixnum (+ 5 7)) → 12 (the (values) (truncate 3.2 2)) → 1, 1.2 (the integer (truncate 3.2 2)) → 1, 1.2 (the (values integer) (truncate 3.2 2)) → 1, 1.2 (the (values integer float) (truncate 3.2 2)) → 1, 1.2 (the (values integer float symbol) (truncate 3.2 2)) → 1, 1.2 (the (values integer float symbol t null list) (truncate 3.2 2)) → 1, 1.2 (let ((i 100)) (declare (fixnum i)) (the fixnum (1+ i))) → 101 (let* ((x (list 'a 'b 'c)) (y 5)) ([[CL:Macros:setf]] (the fixnum (car x)) y) x) → (5 B C) </blockquote>

====Affected By====

None.

====Exceptional Situations====

The consequences are undefined if the //[[CL:Glossary:values]]// //[[CL:Glossary:yielded]]// by the //form// are not of the //[[CL:Glossary:type]]// specified by //value-type//.

====See Also====

\declref{values}

====Notes====

The \declref{values} //[[CL:Glossary:type specifier]]// can be used to indicate the types of //[[CL:Glossary:multiple values]]//:

<blockquote> (the (values integer integer) (floor x y)) (the (values string t) (gethash the-key the-string-table)) </blockquote>

**[[CL:Macros:setf]]** can be used with \specref{the} type declarations. In this case the declaration is transferred to the form that specifies the new value. The resulting **[[CL:Macros:setf]]** //[[CL:Glossary:form]]// is then analyzed.

\issue{THE-VALUES:RETURN-NUMBER-RECEIVED} \issue{THE-VALUES:RETURN-NUMBER-RECEIVED} \issue{THE-AMBIGUITY:FOR-DECLARATION} \issue{JUN90-TRIVIAL-ISSUES:27} \issue{THE-VALUES:RETURN-NUMBER-RECEIVED}
