====== Function Y-OR-N-P, YES-OR-NO-P ======

====Syntax====

**{y-or-n-p} {''&optional'' control** //\rest} arguments// → //generalized-boolean// **{yes-or-no-p} {''&optional'' control** //\rest} arguments// → //generalized-boolean//

====Arguments and Values====

//control// - a //[[CL:Glossary:format control]]//.

//arguments// - //[[CL:Glossary:format arguments]]// for //control//.

//generalized-boolean// - a //[[CL:Glossary:generalized boolean]]//.

====Description====

These functions ask a question and parse a response from the user. They return //[[CL:Glossary:true]]// if the answer is affirmative, or //[[CL:Glossary:false]]// if the answer is negative.

**[[CL:Functions:y-or-n-p]]** is for asking the user a question whose answer is either "yes" or "no." It is intended that the reply require the user to answer a yes-or-no question with a single character.

**[[CL:Functions:yes-or-no-p]]** is also for asking the user a question whose answer is either "Yes" or "No." It is intended that the reply require the user to take more action than just a single keystroke, such as typing the full word ''yes'' or ''no'' followed by a newline.

**[[CL:Functions:y-or-n-p]]** types out a message (if supplied), reads an answer in some //[[CL:Glossary:implementation-dependent]]// manner (intended to be short and simple, such as reading a single character such as ''Y'' or ''N''). **[[CL:Functions:yes-or-no-p]]** types out a message (if supplied), attracts the user's attention (for example, by ringing the terminal's bell), and reads an answer in some //[[CL:Glossary:implementation-dependent]]// manner (intended to be multiple characters, such as ''YES'' or ''NO'').

If //format-control// is supplied and not **[[CL:Constant Variables:nil]]**, then a **[[CL:Functions:fresh-line]]** operation is performed; then a message is printed as if //format-control// and //arguments// were given to **[[CL:Functions:format]]**. In any case, **[[CL:Functions:yes-or-no-p]]** and **[[CL:Functions:y-or-n-p]]** will provide a prompt such as ""(Y or N)'''' or ""(Yes or No)'''' if appropriate.

All input and output are performed using //[[CL:Glossary:query I/O]]//.

====Examples====

<blockquote> (y-or-n-p "(t or nil) given by")
▷ (t or nil) given by (Y or N) \IN{Y} → //[[CL:Glossary:true]]// (yes-or-no-p "a ~S message" 'frightening)
▷ a FRIGHTENING message (Yes or No) \IN{no} → //[[CL:Glossary:false]]// (y-or-n-p "Produce listing file?")
▷ Produce listing file?
▷ Please respond with Y or N. \IN{n} → //[[CL:Glossary:false]]// </blockquote>

====Side Effects====

Output to and input from //[[CL:Glossary:query I/O]]// will occur.

====Affected By====

**[[CL:Variables:*query-io*]]**.

====Exceptional Situations====

None.

====See Also====

**[[CL:Functions:format]]**

====Notes====

**[[CL:Functions:yes-or-no-p]]** and **[[CL:Functions:yes-or-no-p]]** do not add question marks to the end of the prompt string, so any desired question mark or other punctuation should be explicitly included in the text query.

\issue{FORMAT-STRING-ARGUMENTS:SPECIFY}
