====== Function PROVIDE, REQUIRE ======

====Syntax====

\DefunWithValues provide {module-name} {//[[CL:Glossary:implementation-dependent]]//}

**require {module-name** //\opt} pathname-list// → ////[[CL:Glossary:implementation-dependent]]////

====Arguments and Values====

//module-name// - a //[[CL:Glossary:\symbolnamedesignator]]//.

//pathname-list// - **[[CL:Constant Variables:nil]]**, or a //[[CL:Glossary:designator]]// for a //[[CL:Glossary:non-empty]]// //[[CL:Glossary:list]]// of //[[CL:Glossary:pathname designators]]//. The default is **[[CL:Constant Variables:nil]]**.

====Description====

**[[CL:Functions:provide]]** adds the //module-name// to the //[[CL:Glossary:list]]// held by **[[CL:Variables:*modules*]]**, if such a name is not already present.

**[[CL:Functions:require]]** tests for the presence of the //module-name// in the //[[CL:Glossary:list]]// held by **[[CL:Variables:*modules*]]**. If it is present, **[[CL:Functions:require]]** immediately returns.

Otherwise, an attempt is made to load an appropriate set of //[[CL:Glossary:files]]// as follows: The //pathname-list// argument, if //[[CL:Glossary:non-nil]]//, specifies a list of //[[CL:Glossary:pathnames]]// to be loaded in order, from left to right. If the //pathname-list// is **[[CL:Constant Variables:nil]]**, an //[[CL:Glossary:implementation-dependent]]// mechanism will be invoked in an attempt to load the module named //module-name//; if no such module can be loaded, an error of type **[[CL:Types:error]]** is signaled.

Both functions use **[[CL:Functions:string=]]** to test for the presence of a //module-name//.

====Examples====

<blockquote> ;;; This illustrates a nonportable use of REQUIRE, because it ;;; depends on the implementation-dependent file-loading mechanism.

(require "CALCULUS")

;;; This use of REQUIRE is nonportable because of the literal ;;; physical pathname.

(require "CALCULUS" "/usr/lib/lisp/calculus")

;;; One form of portable usage involves supplying a logical pathname, ;;; with appropriate translations defined elsewhere.

(require "CALCULUS" "lib:calculus")

;;; Another form of portable usage involves using a variable or ;;; table lookup function to determine the pathname, which again ;;; must be initialized elsewhere.

(require "CALCULUS" *calculus-module-pathname*) </blockquote>

====Side Effects====

**[[CL:Functions:provide]]** modifies **[[CL:Variables:*modules*]]**.

====Affected By====

The specific action taken by **[[CL:Functions:require]]** is affected by calls to **[[CL:Functions:provide]]** (or, in general, any changes to \thevalueof{*modules*}).

====Exceptional Situations====

Should signal an error of type type-error if //module-name// is not a //[[CL:Glossary:\symbolnamedesignator]]//.

If **[[CL:Functions:require]]** fails to perform the requested operation due to a problem while interacting with the //[[CL:Glossary:file system]]//, an error of type **[[CL:Types:file-error]]** is signaled.

An error of type **[[CL:Types:file-error]]** might be signaled if any //[[CL:Glossary:pathname]]// in //pathname-list// is a //[[CL:Glossary:designator]]// for a //[[CL:Glossary:wild]]// //[[CL:Glossary:pathname]]//.

====See Also====

**[[CL:Variables:*modules*]]**,

{\secref\PathnamesAsFilenames}

====Notes====

The functions **[[CL:Functions:provide]]** and **[[CL:Functions:require]]** are deprecated.

If a module consists of a single //[[CL:Glossary:package]]//, it is customary for the package and module names to be the same.

\issue{REQUIRE-PATHNAME-DEFAULTS-AGAIN:X3J13-DEC-91} \issue{REQUIRE-PATHNAME-DEFAULTS-YET-AGAIN:RESTORE-ARGUMENT} \issue{REQUIRE-PATHNAME-DEFAULTS-YET-AGAIN:RESTORE-ARGUMENT} \issue{REQUIRE-PATHNAME-DEFAULTS-YET-AGAIN:RESTORE-ARGUMENT} \issue{REQUIRE-PATHNAME-DEFAULTS-YET-AGAIN:RESTORE-ARGUMENT} \issue{FILE-OPEN-ERROR:SIGNAL-FILE-ERROR} \issue{FILE-OPEN-ERROR:SIGNAL-FILE-ERROR} \issue{FILE-OPEN-ERROR:SIGNAL-FILE-ERROR}
