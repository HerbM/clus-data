====== System Class COMPLEX ======

====Class Precedence List==== **[[CL:Types:complex]]**, **[[CL:Types:number]]**, **[[CL:Types:t]]**

====Description====

The type **[[CL:Types:complex]]** includes all mathematical complex numbers other than those included in the type **[[CL:Types:rational]]**. //[[CL:Glossary:Complexes]]// are

expressed in Cartesian form with a real part and an imaginary part, each of which is a //[[CL:Glossary:real]]//. The real part and imaginary part are either both //[[CL:Glossary:rational]]// or both of the same //[[CL:Glossary:float]]// //[[CL:Glossary:type]]//. The imaginary part can be a //[[CL:Glossary:float]]// zero, but can never be a //[[CL:Glossary:rational]]// zero, for such a number is always represented by \clisp\ as a //[[CL:Glossary:rational]]// rather than a //[[CL:Glossary:complex]]//.

====Compound Type Specifier Kind====

Specializing.

====Compound Type Specifier Syntax====

\Deftype{complex}{\ttbrac{typespec | **[[CL:Types:wildcard|*]]**}}

====Compound Type Specifier Arguments====

//typespec// - a //[[CL:Glossary:type specifier]]// that denotes a subtype of **[[CL:Types:real]]**.

====Compound Type Specifier Description====

\editornote{KMP: If you ask me, this definition is a complete mess. Looking at issue ARRAY-TYPE-ELEMENT-TYPE-SEMANTICS:UNIFY-UPGRADING does not help me figure it out, either. Anyone got any suggestions?}

Every element of this //[[CL:Glossary:type]]// is a //[[CL:Glossary:complex]]// whose real part and imaginary part are each of type

''(upgraded-complex-part-type //typespec//)''.

This //[[CL:Glossary:type]]// encompasses those //[[CL:Glossary:complexes]]// that can result by giving numbers of //[[CL:Glossary:type]]// //typespec// to **[[CL:Functions:complex]]**.

**[[CL:Functions:(complex //type-specifier//)]]** refers to all //[[CL:Glossary:complexes]]// that can result from giving //[[CL:Glossary:numbers]]// of //type// //type-specifier// to the function **[[CL:Functions:complex]]**, plus all other //[[CL:Glossary:complexes]]// of the same specialized representation.



====See Also====

{\secref\RuleOfCanonRepForComplexRationals}, {\secref\NumsFromTokens}, {\secref\PrintingComplexes}

====Notes====

The input syntax for a //[[CL:Glossary:complex]]// with real part ''r'' and imaginary part ''i'' is ''#C(''r'' ''i'')''. For further details, see section {\secref\StandardMacroChars}.

For every //[[CL:Glossary:float]]//, ''n'', there is a //[[CL:Glossary:complex]]// which represents the same mathematical number and which can be obtained by **[[CL:Functions:(COERCE ''n'' 'COMPLEX)]]**.

\issue{ARRAY-TYPE-ELEMENT-TYPE-SEMANTICS:UNIFY-UPGRADING} \issue{ARRAY-TYPE-ELEMENT-TYPE-SEMANTICS:UNIFY-UPGRADING}
