====== Function INVOKE-RESTART ======

====Syntax====

\DefunWithValues invoke-restart {restart ''&rest'' arguments} {\starparam{result}}

====Arguments and Values====

//restart// - a //[[CL:Glossary:restart designator]]//.

//argument// - an //[[CL:Glossary:object]]//.

//results// - the //[[CL:Glossary:values]]// returned by the //[[CL:Glossary:function]]// associated with //restart//, if that //[[CL:Glossary:function]]// returns.

====Description====

Calls the //[[CL:Glossary:function]]// associated with //restart//, passing //arguments// to it. //Restart// must be valid in the current //[[CL:Glossary:dynamic environment]]//.

====Examples==== <blockquote> (defun add3 (x) (check-type x number) (+ x 3))

(foo 'seven)
▷ Error: The value SEVEN was not of type NUMBER.
▷ To continue, type :CONTINUE followed by an option number:
▷ 1: Specify a different value to use.
▷ 2: Return to Lisp Toplevel.
▷ Debug> \IN{(invoke-restart 'store-value 7)} → 10 </blockquote>

====Side Effects====

A non-local transfer of control might be done by the restart.

====Affected By====

Existing restarts.

====Exceptional Situations====

If //restart// is not valid, an error of type **[[CL:Types:control-error]]** is signaled.

====See Also====

**[[CL:Functions:find-restart]]**, **[[CL:Macros:restart-bind]]**, **[[CL:Macros:restart-case]]**, **[[CL:Functions:invoke-restart-interactively]]**

====Notes====

The most common use for **[[CL:Functions:invoke-restart]]** is in a //[[CL:Glossary:handler]]//. It might be used explicitly, or implicitly through **[[CL:Functions:invoke-restart-interactively]]** or a //[[CL:Glossary:restart function]]//.

//[[CL:Glossary:Restart functions]]// call **[[CL:Functions:invoke-restart]]**, not vice versa. That is, //[[CL:Glossary:invoke-restart]]// provides primitive functionality, and //[[CL:Glossary:restart functions]]// are non-essential "syntactic sugar."

