====== Type BASE-CHAR ======

====Supertypes====

**[[CL:Types:base-char]]**, **[[CL:Types:character]]**, **[[CL:Types:t]]**

====Description====

The type **[[CL:Types:base-char]]** is defined as the //[[CL:Glossary:upgraded array element type]]// of **[[CL:Types:standard-char]]**. An //[[CL:Glossary:implementation]]// can support additional subtypes of **[[CL:Types:character]]** (besides the ones listed in this standard) that might or might not be \supertypesof{base-char}. In addition, an //[[CL:Glossary:implementation]]// can define **[[CL:Types:base-char]]** to be the //[[CL:Glossary:same]]// //[[CL:Glossary:type]]// as **[[CL:Types:character]]**.

//[[CL:Glossary:Base characters]]// are distinguished in the following respects:

\itemitem{1.} The type **[[CL:Types:standard-char]]** is a //[[CL:Glossary:subrepertoire]]// of the type **[[CL:Types:base-char]]**. \itemitem{2.} The selection of //[[CL:Glossary:base characters]]// that are not //[[CL:Glossary:standard characters]]// is implementation defined. \itemitem{3.} Only //[[CL:Glossary:object|objects]]// of the type **[[CL:Types:base-char]]** can be //[[CL:Glossary:element|elements]]// of a //[[CL:Glossary:base string]]//. \itemitem{4.} No upper bound is specified for the number of characters in the **[[CL:Types:base-char]]** //[[CL:Glossary:repertoire]]//; the size of that //[[CL:Glossary:repertoire]]// is //[[CL:Glossary:implementation-defined]]//. The lower bound is~96, the number of //[[CL:Glossary:standard characters]]//.





Whether a character is a //[[CL:Glossary:base character]]// depends on the way that an //[[CL:Glossary:implementation]]// represents //[[CL:Glossary:strings]]//, and not any other properties of the //[[CL:Glossary:implementation]]// or the host operating system. For example, one implementation might encode all //[[CL:Glossary:strings]]// as characters having 16-bit encodings, and another might have two kinds of //[[CL:Glossary:strings]]//: those with characters having 8-bit encodings and those with characters having 16-bit encodings. In the first //[[CL:Glossary:implementation]]//, the type **[[CL:Types:base-char]]** is equivalent to the type **[[CL:Types:character]]**: there is only one kind of //[[CL:Glossary:string]]//. In the second //[[CL:Glossary:implementation]]//, the //[[CL:Glossary:base characters]]// might be those //[[CL:Glossary:characters]]// that could be stored in a //[[CL:Glossary:string]]// of //[[CL:Glossary:characters]]// having 8-bit encodings. In such an implementation, the type **[[CL:Types:base-char]]** is a //[[CL:Glossary:proper subtype]]// of the type **[[CL:Types:character]]**.



The type **[[CL:Types:standard-char]]** is a

subtype of **[[CL:Types:base-char]]**.

\issue{CHARACTER-VS-CHAR:LESS-INCONSISTENT-SHORT} \issue{CHARACTER-PROPOSAL:2-3-1}
