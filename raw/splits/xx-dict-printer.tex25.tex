====== Variable *PRINT-PPRINT-DISPATCH* ======

====Value Type====

a //[[CL:Glossary:pprint dispatch table]]//.

====Initial Value====

//[[CL:Glossary:implementation-dependent]]//, but the initial entries all use a special class of priorities that have the property that they are less than every priority that can be specified using **[[CL:Functions:set-pprint-dispatch]]**, so that the initial contents of any entry can be overridden.

====Description====

The //[[CL:Glossary:pprint dispatch table]]// which currently controls the //[[CL:Glossary:pretty printer]]//.

====Examples====

None.

====See Also====

**[[CL:Variables:*print-pretty*]]**, {\secref\PPrintDispatchTables}

====Notes====

The intent is that the initial //[[CL:Glossary:value]]// of this //[[CL:Glossary:variable]]// should cause `traditional' //[[CL:Glossary:pretty printing]]// of //[[CL:Glossary:code]]//.

In general, however, you can put a value in **[[CL:Variables:*print-pprint-dispatch*]]** that makes pretty-printed output look exactly like non-pretty-printed output.

Setting **[[CL:Variables:*print-pretty*]]** to //[[CL:Glossary:true]]// just causes the functions contained in the //[[CL:Glossary:current pprint dispatch table]]// to have priority over normal **[[CL:Functions:print-object]]** methods; it has no magic way of enforcing that those functions actually produce pretty output. For details, see section {\secref\PPrintDispatchTables}.

\issue{PRETTY-PRINT-INTERFACE} \issue{GENERALIZE-PRETTY-PRINTER:UNIFY}
