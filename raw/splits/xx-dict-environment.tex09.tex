====== Macro STEP ======

====Syntax====

\DefmacWithValues step {form} {\starparam{result}}

====Arguments and Values====

//form// - a //[[CL:Glossary:form]]//; \evalspecial.

//results// - the //[[CL:Glossary:values]]// returned by the //form//.

====Description====

**[[CL:Macros:step]]** implements a debugging paradigm wherein the programmer is allowed to //[[CL:Glossary:step]]// through the //[[CL:Glossary:evaluation]]// of a //[[CL:Glossary:form]]//. The specific nature of the interaction,

including which I/O streams are used and whether the stepping has lexical or dynamic scope,

is //[[CL:Glossary:implementation-defined]]//.

**[[CL:Macros:step]]** evaluates //form// in the current //[[CL:Glossary:environment]]//. A call to **[[CL:Macros:step]]** can be compiled, but it is acceptable for an implementation to interactively step through only those parts of the computation that are interpreted.

It is technically permissible for a //[[CL:Glossary:conforming implementation]]// to take no action at all other than normal //[[CL:Glossary:execution]]// of the //form//. In such a situation, ''(step ''form'')'' is equivalent to, for example, ''(let () ''form'')''. In implementations where this is the case, the associated documentation should mention that fact.

====Examples====

None.

====Affected By====

None.

====Exceptional Situations====

None.

====See Also====

**[[CL:Macros:trace]]**

====Notes====

//[[CL:Glossary:Implementations]]// are encouraged to respond to the typing of ''?'' or the pressing of a "help key" by providing help including a list of commands.

\issue{EVALHOOK-STEP-CONFUSION:X3J13-NOV-89} \issue{EVALHOOK-STEP-CONFUSION:X3J13-NOV-89} \issue{STEP-ENVIRONMENT:CURRENT} \issue{STEP-MINIMAL:PERMIT-PROGN} \issue{EVALHOOK-STEP-CONFUSION:X3J13-NOV-89}
