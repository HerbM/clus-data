\begincom{string=, string/=, string<, string>, string<=, string>=, string-equal, string-not-equal, string-lessp, string-greaterp, string-not-greaterp, string-not-lessp}\ftype{Function}

====Syntax====

\DefunWithValues {string''=''} {string1 string2 ''&key'' start1 end1 start2 end2} {generalized-boolean}

\DefunMultiWithValues {string1 string2 ''&key'' start1 end1 start2 end2} {mismatch-index} {string/''='' string''<'' string''>'' string''<='' string''>=''}

\DefunWithValues {string-equal} {string1 string2 ''&key'' start1 end1 start2 end2} {generalized-boolean}

\DefunMultiWithValues {string1 string2 ''&key'' start1 end1 start2 end2} {mismatch-index} {string-not-equal string-lessp string-greaterp string-not-greaterp string-not-lessp}

====Arguments and Values====

//string1// - a //[[CL:Glossary:string designator]]//.

//string2// - a //[[CL:Glossary:string designator]]//.

//start1//, //end1// - //[[CL:Glossary:bounding index designators]]// of //string1//. \Defaults{//start// and //end//}{''0'' and **[[CL:Constant Variables:nil]]**}

//start2//, //end2// - //[[CL:Glossary:bounding index designators]]// of //string2//. \Defaults{//start// and //end//}{''0'' and **[[CL:Constant Variables:nil]]**}

//generalized-boolean// - a //[[CL:Glossary:generalized boolean]]//.

//mismatch-index// - a //[[CL:Glossary:bounding index]]// of //string1//, or **[[CL:Constant Variables:nil]]**.

====Description====

These functions perform lexicographic comparisons on //string1// and //string2//. **[[CL:Functions:string=]]** and **[[CL:Functions:string-equal]]** are called equality functions; the others are called inequality functions. The comparison operations these //[[CL:Glossary:functions]]// perform are restricted to the subsequence of //string1// //[[CL:Glossary:bounded]]// by //[[CL:Glossary:start1]]// and //end1// and to the subsequence of //string2// //[[CL:Glossary:bounded]]// by //[[CL:Glossary:start2]]// and //end2//.

A string ''a'' is equal to a string ''b'' if it contains the same number of characters, and the corresponding characters are the //[[CL:Glossary:same]]//

under **[[CL:Functions:char=]]** or **[[CL:Functions:char-equal]]**, as appropriate.

A string ''a'' is less than a string ''b'' if in the first position in which they differ the character of ''a'' is less than the corresponding character of ''b'' according to **[[CL:Functions:char<]]** or **[[CL:Functions:char-lessp]]** as appropriate, or if string ''a'' is a proper prefix of string ''b'' (of shorter length and matching in all the characters of ''a'').

The equality functions return a //generalized boolean// that is //[[CL:Glossary:true]]// if the strings are equal, or //[[CL:Glossary:false]]// otherwise.

The inequality functions return a //mismatch-index// that is //[[CL:Glossary:true]]// if the strings are not equal, or //[[CL:Glossary:false]]// otherwise. When the //mismatch-index// is //[[CL:Glossary:true]]//, it is an //[[CL:Glossary:integer]]// representing the first character position at which the two substrings differ, as an offset from the beginning of //string1//.

The comparison has one of the following results:

\beginlist

\itemitem{**[[CL:Functions:string=]]**}

**[[CL:Functions:string=]]** is //[[CL:Glossary:true]]// if the supplied substrings are of the same length and contain the //[[CL:Glossary:same]]// characters in corresponding positions; otherwise it is //[[CL:Glossary:false]]//.

\itemitem{**[[CL:Functions:string/=]]**}

**[[CL:Functions:string/=]]** is //[[CL:Glossary:true]]// if the supplied substrings are different; otherwise it is //[[CL:Glossary:false]]//.

\itemitem{**[[CL:Functions:string-equal]]**}

**[[CL:Functions:string-equal]]** is just like **[[CL:Functions:string=]]** except that differences in case are ignored; two characters are considered to be the same if **[[CL:Functions:char-equal]]** is //[[CL:Glossary:true]]// of them.

\itemitem{**[[CL:Functions:string<]]**}

**[[CL:Functions:string<]]** is //[[CL:Glossary:true]]// if substring1 is less than substring2; otherwise it is //[[CL:Glossary:false]]//.

\itemitem{**[[CL:Functions:string>]]**}

**[[CL:Functions:string>]]** is //[[CL:Glossary:true]]// if substring1 is greater than substring2; otherwise it is //[[CL:Glossary:false]]//.

\itemitem{**[[CL:Functions:string-lessp]]**, **[[CL:Functions:string-greaterp]]**}

**[[CL:Functions:string-lessp]]** and **[[CL:Functions:string-greaterp]]** are exactly like **[[CL:Functions:string<]]** and **[[CL:Functions:string>]]**, respectively, except that distinctions between uppercase and lowercase letters are ignored. It is as if **[[CL:Functions:char-lessp]]** were used instead of **[[CL:Functions:char<]]** for comparing characters.

\itemitem{**[[CL:Functions:string<=]]**}

**[[CL:Functions:string<=]]** is //[[CL:Glossary:true]]// if substring1 is less than or equal to substring2; otherwise it is //[[CL:Glossary:false]]//.

\itemitem{**[[CL:Functions:string>=]]**}

**[[CL:Functions:string>=]]** is //[[CL:Glossary:true]]// if substring1 is greater than or equal to substring2; otherwise it is //[[CL:Glossary:false]]//.

\itemitem{**[[CL:Functions:string-not-greaterp]]**, **[[CL:Functions:string-not-lessp]]**}

**[[CL:Functions:string-not-greaterp]]** and **[[CL:Functions:string-not-lessp]]** are exactly like **[[CL:Functions:string<=]]** and **[[CL:Functions:string>=]]**, respectively, except that distinctions between uppercase and lowercase letters are ignored. It is as if **[[CL:Functions:char-lessp]]** were used instead of **[[CL:Functions:char<]]** for comparing characters.

\endlist


====Examples====

<blockquote> (string= "foo" "foo") → //[[CL:Glossary:true]]// (string= "foo" "Foo") → //[[CL:Glossary:false]]// (string= "foo" "bar") → //[[CL:Glossary:false]]// (string= "together" "frog" :start1 1 :end1 3 :start2 2) → //[[CL:Glossary:true]]// (string-equal "foo" "Foo") → //[[CL:Glossary:true]]// (string= "abcd" "01234abcd9012" :start2 5 :end2 9) → //[[CL:Glossary:true]]// (string< "aaaa" "aaab") → 3 (string>= "aaaaa" "aaaa") → 4 (string-not-greaterp "Abcde" "abcdE") → 5 (string-lessp "012AAAA789" "01aaab6" :start1 3 :end1 7 :start2 2 :end2 6) → 6 (string-not-equal "AAAA" "aaaA") → //[[CL:Glossary:false]]// </blockquote>

====Side Effects====

None.

====Affected By====

None.

====Exceptional Situations====

None.

====See Also====

**[[CL:Functions:char=]]**

====Notes====

**[[CL:Functions:equal]]** calls **[[CL:Functions:string=]]** if applied to two //[[CL:Glossary:strings]]//.

\issue{STRING-COERCION:MAKE-CONSISTENT} \issue{SUBSEQ-OUT-OF-BOUNDS} \issue{RANGE-OF-START-AND-END-PARAMETERS:INTEGER-AND-INTEGER-NIL}
