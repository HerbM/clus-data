====== Function READ-FROM-STRING ======

====Syntax====

\DefunWithValuesNewline read-from-string {string ''&optional'' eof-error-p eof-value ''&key'' start end preserve-whitespace} {object, position}

====Arguments and Values====

//string// - a //[[CL:Glossary:string]]//.

//eof-error-p// - a //[[CL:Glossary:generalized boolean]]//. The default is //[[CL:Glossary:true]]//.

//eof-value// - an //[[CL:Glossary:object]]//.

The default is **[[CL:Constant Variables:nil]]**.

//start//, //end// - //[[CL:Glossary:bounding index designators]]// of //string//. \Defaults{//start// and //end//}{''0'' and **[[CL:Constant Variables:nil]]**}

//preserve-whitespace// - a //[[CL:Glossary:generalized boolean]]//. The default is //[[CL:Glossary:false]]//.

//object// - an //[[CL:Glossary:object]]// (parsed by the //[[CL:Glossary:Lisp reader]]//) or the //eof-value//.

//position// - an //[[CL:Glossary:integer]]// greater than or equal to zero, and less than or equal to one more than the //[[CL:Glossary:length]]// of the //string//.

====Description====

Parses the printed representation of an //[[CL:Glossary:object]]// from the subsequence of //string// //[[CL:Glossary:bounded]]// by //start// and //end//, as if **[[CL:Functions:read]]** had been called on an //[[CL:Glossary:input]]// //[[CL:Glossary:stream]]// containing those same //[[CL:Glossary:characters]]//.

If //preserve-whitespace// is //[[CL:Glossary:true]]//, the operation will preserve //[[CL:Glossary:whitespace]]// as **[[CL:Functions:read-preserving-whitespace]]** would do.

If an //[[CL:Glossary:object]]// is successfully parsed, the //[[CL:Glossary:primary value]]//, //object//, is the //[[CL:Glossary:object]]// that was parsed. If //eof-error-p// is //[[CL:Glossary:false]]// and if the end of the //substring// is reached, //eof-value// is returned.

The //[[CL:Glossary:secondary value]]//, //position//, is the index of the first //[[CL:Glossary:character]]// in the //[[CL:Glossary:bounded]]// //string// that was not read. The //position// may depend upon the value of //preserve-whitespace//.

If the entire //string// was read, the //position// returned is either the //length// of the //string// or one greater than the //length// of the //string//.

====Examples====

<blockquote> (read-from-string " 1 3 5" t nil :start 2) → 3, 5 (read-from-string "(a b c)") → (A B C), 7 </blockquote>

====Side Effects====

None.

====Affected By====

None.

====Exceptional Situations====

If the end of the supplied substring occurs before an //[[CL:Glossary:object]]// can be read, an error is signaled if //eof-error-p// is //[[CL:Glossary:true]]//. An error is signaled if the end of the //substring// occurs in the middle of an incomplete //[[CL:Glossary:object]]//.

====See Also====

**[[CL:Functions:read]]**, **[[CL:Functions:read-preserving-whitespace]]**

====Notes====


The reason that //position// is allowed to be beyond the //[[CL:Glossary:length]]// of the //string// is to permit (but not require) the //[[CL:Glossary:implementation]]// to work by simulating the effect of a trailing delimiter at the end of the //[[CL:Glossary:bounded]]// //string//. When //preserve-whitespace// is //[[CL:Glossary:true]]//, the //position// might count the simulated delimiter.

\issue{ARGUMENTS-UNDERSPECIFIED:SPECIFY} \issue{SUBSEQ-OUT-OF-BOUNDS} \issue{RANGE-OF-START-AND-END-PARAMETERS:INTEGER-AND-INTEGER-NIL}
