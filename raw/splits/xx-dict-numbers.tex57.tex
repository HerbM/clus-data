====== Function INTEGER-LENGTH ======

====Syntax====

**integer-length** //integer// → //number-of-bits//

====Arguments and Values====

//integer// - an //[[CL:Glossary:integer]]//.

//number-of-bits// - a non-negative //[[CL:Glossary:integer]]//.

====Description====

Returns the number of bits needed to represent //integer// in binary two's-complement format.

====Examples====

<blockquote> (integer-length 0) → 0 (integer-length 1) → 1 (integer-length 3) → 2 (integer-length 4) → 3 (integer-length 7) → 3 (integer-length -1) → 0 (integer-length -4) → 2 (integer-length -7) → 3 (integer-length -8) → 3 (integer-length (expt 2 9)) → 10 (integer-length (1- (expt 2 9))) → 9 (integer-length (- (expt 2 9))) → 9 (integer-length (- (1+ (expt 2 9)))) → 10 </blockquote>

====Side Effects====

None.

====Affected By====

None.

====Exceptional Situations====

Should signal an error of type type-error if //integer// is not an //[[CL:Glossary:integer]]//.

====See Also====

None.

====Notes====

This function could have been defined by:

<blockquote> (defun integer-length (integer) (ceiling (log (if (minusp integer) (- integer) (1+ integer)) 2))) </blockquote>

If //integer// is non-negative, then its value can be represented in unsigned binary form in a field whose width in bits is no smaller than **[[CL:Functions:(integer-length //integer//)]]**. Regardless of the sign of //integer//, its value can be represented in signed binary two's-complement form in a field whose width in bits is no smaller than **[[CL:Functions:(+ (integer-length //integer//) 1)]]**.

