====== Standard Generic Function MAKE-LOAD-FORM ======


====Syntax====

\DefgenWithValues {make-load-form} {object ''&optional'' environment} {creation-form[, initialization-form]}

====Method Signatures====

\Defmeth make-load-form {\specparam{object}{standard-object} ''&optional'' environment} \Defmeth make-load-form {\specparam{object}{structure-object} ''&optional'' environment} \Defmeth make-load-form {\specparam{object}{condition} ''&optional'' environment} \Defmeth make-load-form {\specparam{object}{class} ''&optional'' environment}

====Arguments and Values====

//object// - an //[[CL:Glossary:object]]//.


//environment// - an //[[CL:Glossary:environment object]]//.

//creation-form// - a //[[CL:Glossary:form]]//.

//initialization-form// - a //[[CL:Glossary:form]]//.

====Description====

\TheGF{make-load-form} creates and returns one or two //[[CL:Glossary:forms]]//, a //creation-form// and an //initialization-form//, that enable **[[CL:Functions:load]]** to construct an //[[CL:Glossary:object]]// equivalent to //object//.

//Environment// is an //[[CL:Glossary:environment object]]// corresponding to the //[[CL:Glossary:lexical environment]]// in which the //[[CL:Glossary:forms]]// will be processed.

The //[[CL:Glossary:file compiler]]// calls **[[CL:Functions:make-load-form]]** to process certain //[[CL:Glossary:classes]]// of //[[CL:Glossary:literal objects]]//; see section {\secref\CallingMakeLoadForm}.

//[[CL:Glossary:Conforming programs]]// may call **[[CL:Functions:make-load-form]]** directly, providing //object// is a //[[CL:Glossary:generalized instance]]// of

**[[CL:Types:standard-object]]**, **[[CL:Types:structure-object]]**, or **[[CL:Types:condition]]**.

The creation form is a //[[CL:Glossary:form]]// that, when evaluated at **[[CL:Functions:load]]** time, should return an //[[CL:Glossary:object]]// that is equivalent to //object//. The exact meaning of equivalent depends on the //[[CL:Glossary:type]]// of //[[CL:Glossary:object]]// and is up to the programmer who defines a //[[CL:Glossary:method]]// for **[[CL:Functions:make-load-form]]**; see section {\secref\LiteralsInCompiledFiles}.


The initialization form is a //[[CL:Glossary:form]]// that, when evaluated at **[[CL:Functions:load]]** time, should perform further initialization of the //[[CL:Glossary:object]]//. The value returned by the initialization form is ignored.

If **[[CL:Functions:make-load-form]]** returns only one value, the initialization form is **[[CL:Constant Variables:nil]]**, which has no effect. If //object// appears as a constant in the initialization form, at **[[CL:Functions:load]]** time it will be replaced by the equivalent //[[CL:Glossary:object]]// constructed by the creation form; this is how the further initialization gains access to the //[[CL:Glossary:object]]//.


Both the //creation-form// and the //initialization-form// may contain references to any //[[CL:Glossary:externalizable object]]//. However, there must not be any circular dependencies in creation forms. An example of a circular dependency is when the creation form for the object ''X'' contains a reference to the object ''Y'', and the creation form for the object ''Y'' contains a reference to the object ''X''.

Initialization forms are not subject to any restriction against circular dependencies, which is the reason that initialization forms exist; see the example of circular data structures below.


The creation form for an //[[CL:Glossary:object]]// is always //[[CL:Glossary:evaluated]]// before the initialization form for that //[[CL:Glossary:object]]//. When either the creation form or the initialization form references other //[[CL:Glossary:object|objects]]// that have not been referenced earlier in the //[[CL:Glossary:file]]// being //[[CL:Glossary:compiled]]//, the //[[CL:Glossary:compiler]]// ensures that all of the referenced //[[CL:Glossary:object|objects]]// have been created before //[[CL:Glossary:evaluating]]// the referencing //[[CL:Glossary:form]]//. When the referenced //[[CL:Glossary:object]]// is of a //[[CL:Glossary:type]]// which

the //[[CL:Glossary:file compiler]]// processes using **[[CL:Functions:make-load-form]]**, this involves //[[CL:Glossary:evaluating]]// the creation form returned for it. (This is the reason for the prohibition against circular references among creation forms).

Each initialization form is //[[CL:Glossary:evaluated]]// as soon as possible after its associated creation form, as determined by data flow. If the initialization form for an //[[CL:Glossary:object]]// does not reference any other //[[CL:Glossary:object|objects]]// not referenced earlier in the //[[CL:Glossary:file]]// and processed by

the //[[CL:Glossary:file compiler]]// using **[[CL:Functions:make-load-form]]**, the initialization form is evaluated immediately after the creation form. If a creation or initialization form ''F'' does contain references to such //[[CL:Glossary:object|objects]]//, the creation forms for those other objects are evaluated before ''F'', and the initialization forms for those other //[[CL:Glossary:object|objects]]// are also evaluated before ''F'' whenever they do not depend on the //[[CL:Glossary:object]]// created or initialized by ''F''. Where these rules do not uniquely determine an order of //[[CL:Glossary:evaluation]]// between two creation/initialization forms, the order of //[[CL:Glossary:evaluation]]// is unspecified.

While these creation and initialization forms are being evaluated, the //[[CL:Glossary:object|objects]]// are possibly in an uninitialized state, analogous to the state of an //[[CL:Glossary:object]]// between the time it has been created by **[[CL:Functions:allocate-instance]]** and it has been processed fully by **[[CL:Functions:initialize-instance]]**. Programmers writing //[[CL:Glossary:methods]]// for **[[CL:Functions:make-load-form]]** must take care in manipulating //[[CL:Glossary:object|objects]]// not to depend on //[[CL:Glossary:slots]]// that have not yet been initialized.

It is //[[CL:Glossary:implementation-dependent]]// whether **[[CL:Functions:load]]** calls **[[CL:Functions:eval]]** on the //[[CL:Glossary:forms]]// or does some other operation that has an equivalent effect. For example, the //[[CL:Glossary:forms]]// might be translated into different but equivalent //[[CL:Glossary:forms]]// and then evaluated, they might be compiled and the resulting functions called by **[[CL:Functions:load]]**, or they might be interpreted by a special-purpose function different from **[[CL:Functions:eval]]**. All that is required is that the effect be equivalent to evaluating the //[[CL:Glossary:forms]]//.

The //[[CL:Glossary:method]]// //[[CL:Glossary:specialized]]// on **[[CL:Types:class]]** returns a creation //[[CL:Glossary:form]]// using the //[[CL:Glossary:name]]// of the //[[CL:Glossary:class]]// if the //[[CL:Glossary:class]]// has a //[[CL:Glossary:proper name]]// in //environment//, signaling an error of type **[[CL:Types:error]]** if it does not have a //[[CL:Glossary:proper name]]//. //[[CL:Glossary:Evaluation]]// of the creation //[[CL:Glossary:form]]// uses the //[[CL:Glossary:name]]// to find the //[[CL:Glossary:class]]// with that //[[CL:Glossary:name]]//, as if by //[[CL:Glossary:calling]]// **[[CL:Functions:find-class]]**. If a //[[CL:Glossary:class]]// with that //[[CL:Glossary:name]]// has not been defined, then a //[[CL:Glossary:class]]// may be computed in an //[[CL:Glossary:implementation-defined]]// manner. If a //[[CL:Glossary:class]]// cannot be returned as the result of //[[CL:Glossary:evaluating]]// the creation //[[CL:Glossary:form]]//, then an error of type **[[CL:Types:error]]** is signaled.

Both //[[CL:Glossary:conforming implementations]]// and //[[CL:Glossary:conforming programs]]// may further //[[CL:Glossary:specialize]]// **[[CL:Functions:make-load-form]]**.

====Examples====

<blockquote> (defclass obj () ((x :initarg :x :reader obj-x) (y :initarg :y :reader obj-y) (dist :accessor obj-dist))) → #<STANDARD-CLASS OBJ 250020030> (defmethod shared-initialize :after ((self obj) slot-names &rest keys) (declare (ignore slot-names keys)) (unless (slot-boundp self 'dist) ([[CL:Macros:setf]] (obj-dist self) (sqrt (+ (expt (obj-x self) 2) (expt (obj-y self) 2)))))) → #<STANDARD-METHOD SHARED-INITIALIZE (:AFTER) (OBJ T) 26266714> (defmethod make-load-form ((self obj) &optional environment) (declare (ignore environment)) ;; Note that this definition only works because X and Y do not ;; contain information which refers back to the object itself. ;; For a more general solution to this problem, see revised example below. `(make-instance ',(class-of self) :x ',(obj-x self) :y ',(obj-y self))) → #<STANDARD-METHOD MAKE-LOAD-FORM (OBJ) 26267532> ([[CL:Macros:defparameter]] obj1 (make-instance 'obj :x 3.0 :y 4.0)) → #<OBJ 26274136> (obj-dist obj1) → 5.0 (make-load-form obj1) → (MAKE-INSTANCE 'OBJ :X '3.0 :Y '4.0) </blockquote>

In the above example, an equivalent //[[CL:Glossary:instance]]// of ''obj'' is reconstructed by using the values of two of its //[[CL:Glossary:slots]]//. The value of the third //[[CL:Glossary:slot]]// is derived from those two values.

\medbreak Another way to write the **[[CL:Functions:make-load-form]]** //[[CL:Glossary:method]]// in that example is to use **[[CL:Functions:make-load-form-saving-slots]]**. The code it generates might yield a slightly different result from the **[[CL:Functions:make-load-form]]** //[[CL:Glossary:method]]// shown above, but the operational effect will be the same. For example:

\smallbreak <blockquote> ;; Redefine method defined above. (defmethod make-load-form ((self obj) &optional environment) (make-load-form-saving-slots self :slot-names '(x y) :environment environment)) → #<STANDARD-METHOD MAKE-LOAD-FORM (OBJ) 42755655> ;; Try MAKE-LOAD-FORM on object created above. (make-load-form obj1) → (ALLOCATE-INSTANCE '#<STANDARD-CLASS OBJ 250020030>), (PROGN (SETF (SLOT-VALUE '#<OBJ 26274136> 'X) '3.0) (SETF (SLOT-VALUE '#<OBJ 26274136> 'Y) '4.0) (INITIALIZE-INSTANCE '#<OBJ 26274136>)) </blockquote>

\medbreak In the following example, //[[CL:Glossary:instances]]// of ''my-frob'' are "interned" in some way. An equivalent //[[CL:Glossary:instance]]// is reconstructed by using the value of the name slot as a key for searching existing //[[CL:Glossary:object|objects]]//. In this case the programmer has chosen to create a new //[[CL:Glossary:object]]// if no existing //[[CL:Glossary:object]]// is found; alternatively an error could have been signaled in that case.

\smallbreak <blockquote> (defclass my-frob () ((name :initarg :name :reader my-name))) (defmethod make-load-form ((self my-frob) &optional environment) (declare (ignore environment)) `(find-my-frob ',(my-name self) :if-does-not-exist :create)) </blockquote>

\medbreak In the following example, the data structure to be dumped is circular, because each parent has a list of its children and each child has a reference back to its parent. If **[[CL:Functions:make-load-form]]** is called on one //[[CL:Glossary:object]]// in such a structure, the creation form creates an equivalent //[[CL:Glossary:object]]// and fills in the children slot, which forces creation of equivalent //[[CL:Glossary:object|objects]]// for all of its children, grandchildren, etc. At this point none of the parent //[[CL:Glossary:slots]]// have been filled in. The initialization form fills in the parent //[[CL:Glossary:slot]]//, which forces creation of an equivalent //[[CL:Glossary:object]]// for the parent if it was not already created. Thus the entire tree is recreated at **[[CL:Functions:load]]** time. At compile time, **[[CL:Functions:make-load-form]]** is called once for each //[[CL:Glossary:object]]// in the tree. All of the creation forms are evaluated, in //[[CL:Glossary:implementation-dependent]]// order, and then all of the initialization forms are evaluated, also in //[[CL:Glossary:implementation-dependent]]// order.

\smallbreak <blockquote> (defclass tree-with-parent () ((parent :accessor tree-parent) (children :initarg :children))) (defmethod make-load-form ((x tree-with-parent) &optional environment) (declare (ignore environment)) (values ;; creation form `(make-instance ',(class-of x) :children ',(slot-value x 'children)) ;; initialization form `([[CL:Macros:setf]] (tree-parent ',x) ',(slot-value x 'parent)))) </blockquote>

\medbreak In the following example, the data structure to be dumped has no special properties and an equivalent structure can be reconstructed simply by reconstructing the //[[CL:Glossary:slots]]//' contents.

\smallbreak <blockquote> (defstruct my-struct a b c) (defmethod make-load-form ((s my-struct) &optional environment) (make-load-form-saving-slots s :environment environment)) </blockquote>

====Affected By====

None.

====Exceptional Situations====

The //[[CL:Glossary:methods]]// //[[CL:Glossary:specialized]]// on **[[CL:Types:standard-object]]**, **[[CL:Types:structure-object]]**, and **[[CL:Types:condition]]** all signal an error of type **[[CL:Types:error]]**.


It is //[[CL:Glossary:implementation-dependent]]// whether //[[CL:Glossary:calling]]// **[[CL:Functions:make-load-form]]** on a //[[CL:Glossary:generalized instance]]// of a //[[CL:Glossary:system class]]// signals an error or returns creation and initialization //[[CL:Glossary:forms]]//.

====See Also====

**[[CL:Functions:compile-file]]**, **[[CL:Functions:make-load-form-saving-slots]]**, {\secref\CallingMakeLoadForm} {\secref\Evaluation}, {\secref\Compilation}

====Notes====

The //[[CL:Glossary:file compiler]]//

calls **[[CL:Functions:make-load-form]]** in specific circumstances detailed in \secref\CallingMakeLoadForm.

Some //[[CL:Glossary:implementations]]// may provide facilities for defining new //[[CL:Glossary:subclasses]]// of //[[CL:Glossary:classes]]// which are specified as //[[CL:Glossary:system classes]]//. (Some likely candidates include **[[CL:Types:generic-function]]**, **[[CL:Types:method]]**, and **[[CL:Types:stream]]**). Such //[[CL:Glossary:implementations]]// should document how the //[[CL:Glossary:file compiler]]// processes //[[CL:Glossary:instances]]// of such //[[CL:Glossary:classes]]// when encountered as //[[CL:Glossary:literal objects]]//, and should document any relevant //[[CL:Glossary:methods]]// for **[[CL:Functions:make-load-form]]**.

\issue{MAKE-LOAD-FORM-CONFUSION:REWRITE} \issue{LOAD-OBJECTS:MAKE-LOAD-FORM}
