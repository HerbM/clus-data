====== Type Specifier VALUES ======

====Compound Type Specifier Kind====

Specializing.

====Compound Type Specifier Syntax====

\Deftype{values}{\down{value-typespec}}

\reviewer{Barmar: Missing \keyref{key}}

\auxbnf{value-typespec}{\starparam{typespec} \ttbrac{''&optional'' {\starparam{typespec}}} \ttbrac{''&rest'' typespec} \ttbrac{\keyref{allow-other-keys}}}

====Compound Type Specifier Arguments====

//typespec// - a //[[CL:Glossary:type specifier]]//.

====Compound Type Specifier Description====

This //[[CL:Glossary:type specifier]]// can be used only as the //value-type// in a **[[CL:Types:function]]** //[[CL:Glossary:type specifier]]// or a \specref{the} //[[CL:Glossary:special form]]//. It is used to specify individual //[[CL:Glossary:types]]// when //[[CL:Glossary:multiple values]]// are involved. The \keyref{optional} and \keyref{rest} markers can appear in the //value-type// list; they indicate the parameter list of a //[[CL:Glossary:function]]// that, when given to \specref{multiple-value-call} along with the values,

would correctly receive those values.

The symbol **[[CL:Types:wildcard|*]]** may not be among the //value-types//.

The symbol \misc{values} is not valid as a //[[CL:Glossary:type specifier]]//; and, specifically, it is not an abbreviation for ''(values)''.

\issue{TYPE-SPECIFIER-ABBREVIATION:X3J13-JUN90-GUESS}
