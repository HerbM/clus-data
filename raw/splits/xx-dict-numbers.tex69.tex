====== Accessor LDB ======

====Syntax====

**ldb** //bytespec integer// → //byte//

(**setf** (**ldb** //bytespec place//) //new-byte//)

====Arguments and Values====

//bytespec// - a //[[CL:Glossary:byte specifier]]//.

//integer// - an //[[CL:Glossary:integer]]//.

//byte//, //new-byte// - a non-negative //[[CL:Glossary:integer]]//.

====Description====

**[[CL:Functions:ldb]]** extracts and returns the //[[CL:Glossary:byte]]// of //integer// specified by //bytespec//.

**[[CL:Functions:ldb]]** returns an //[[CL:Glossary:integer]]// in which the bits with weights ''2^{(''s''-1)}'' through ''2^{0}'' are the same as those in //integer// with weights ''2^{(''p''+''s''-1)}'' through ''2^''p'''', and all other bits zero; ''s'' is ''(byte-size //bytespec//)'' and ''p'' is ''(byte-position //bytespec//)''.

**[[CL:Macros:setf]]** may be used with **[[CL:Functions:ldb]]** to modify a byte within the //integer// that is stored in a given //place//.

The order of evaluation, when an **[[CL:Functions:ldb]]** form is supplied to **[[CL:Macros:setf]]**, is exactly left-to-right. \idxtext{order of evaluation}\idxtext{evaluation order}

The effect is to perform a **[[CL:Functions:dpb]]** operation and then store the result back into the //place//.

====Examples====

<blockquote> (ldb (byte 2 1) 10) → 1 ([[CL:Macros:defparameter]] a (list 8)) → (8) ([[CL:Macros:setf]] (ldb (byte 2 1) (car a)) 1) → 1 a → (10) </blockquote>

====Side Effects====

None.

====Affected By====

None.

====Exceptional Situations====

None.

====See Also====

**[[CL:Functions:byte]]**, **[[CL:Functions:byte-position]]**, **[[CL:Functions:byte-size]]**, **[[CL:Functions:dpb]]**

====Notes====

<blockquote> (logbitp //j// (ldb (byte //s// //p//) //n//)) ≡ (and (< //j// //s//) (logbitp (+ //j// //p//) //n//)) </blockquote>

In general,

<blockquote> (ldb (byte 0 //x//) //y//) → 0 </blockquote>

for all valid values of //x// and //y//.

Historically, the name "ldb" comes from a DEC PDP-10 assembly language instruction meaning "load byte."

\issue{PUSH-EVALUATION-ORDER:FIRST-ITEM}
