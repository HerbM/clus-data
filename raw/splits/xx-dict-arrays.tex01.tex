====== System Class ARRAY ======
An //[[CL:Glossary:array]]// contains //[[CL:Glossary:object|objects]]// arranged according to a Cartesian coordinate system.

An //[[CL:Glossary:array]]// provides mappings from a set of //[[CL:Glossary:fixnum|fixnums]]//

''πi<sub>0</sub>,i<sub>1</sub>,...,i<sub>r-1</sub>œ'' to corresponding //[[CL:Glossary:element|elements]]// of the //[[CL:Glossary:array]]//, where ''0 ≤ i<sub>j</sub> < d<sub>j</sub>'', ''r'' is the rank of the array, and ''d<sub>j</sub>'' is the size of //[[CL:Glossary:dimension]]// ''j'' of the array.

When an //[[CL:Glossary:array]]// is created, the program requesting its creation may declare that all //[[CL:Glossary:element|elements]]// are of a particular //[[CL:Glossary:type]]//, called the //[[CL:Glossary:expressed array element type]]//. The implementation is permitted to //[[CL:Glossary:upgrade]]// this type in order to produce the //[[CL:Glossary:actual array element type]]//, which is the //[[CL:Glossary:element type]]// for the //[[CL:Glossary:array]]// is actually //[[CL:Glossary:specialized]]//. See the //[[CL:Glossary:function]]// **[[CL:Functions:upgraded-array-element-type]]**.

====Class Precedence List====
**[[CL:Types:array]]**, **[[CL:Types:t]]**

====Compound Type Specifier Kind====
Specializing.

====Compound Type Specifier Syntax====
  * //**array** //[πelement-type | **[[CL:Types:wildcard|*]]**œ [dimension-spec]]////
  * ''dimension-spec'' := ''rank | **[[CL:Types:wildcard|*]]** | (πdimension | **[[CL:Types:wildcard|*]]**œ*)''

====Compound Type Specifier Arguments====
  * //dimension// - a //[[CL:Glossary:valid array dimension]]//.
  * //element-type// - a //[[CL:Glossary:type specifier]]//.
  * //rank// - a non-negative //[[CL:Glossary:fixnum]]//.

====Compound Type Specifier Description====
This denotes the set of //[[CL:Glossary:array|arrays]]// whose //[[CL:Glossary:element type]]//, //[[CL:Glossary:rank]]//, and //[[CL:Glossary:dimension|dimensions]]// match any given //element-type//, //rank//, and //dimensions//. Specifically:

If //element-type// is the //[[CL:Glossary:symbol]]// **[[CL:Types:wildcard|*]]**, //[[CL:Glossary:array|arrays]]// are not excluded on the basis of their //[[CL:Glossary:element type]]//. Otherwise, only those //arrays// are included whose //[[CL:Glossary:actual array element type]]// is the result of //[[CL:Glossary:upgrade|upgrading]]// //element-type//; see section {\secref\ArrayUpgrading}.

If the //dimension-spec// is a //rank//, the set includes only those //arrays// having that //[[CL:Glossary:rank]]//. If the //dimension-spec// is a //[[CL:Glossary:list]]// of //dimensions//, the set includes only those //arrays// having a //[[CL:Glossary:rank]]// given by the //[[CL:Glossary:length]]// of the //dimensions//, and having the indicated //dimensions//; in this case, **[[CL:Types:wildcard|*]]** matches any value for the corresponding //[[CL:Glossary:dimension]]//. If the //dimension-spec// is the //[[CL:Glossary:symbol]]// **[[CL:Types:wildcard|*]]**, the set is not restricted on the basis of //[[CL:Glossary:rank]]// or //[[CL:Glossary:dimension]]//.

====See Also====
**[[CL:Variables:*print-array*]]**, **[[CL:Functions:aref]]**, **[[CL:Functions:make-array]]**, **[[CL:Types:vector]]**, {\secref\SharpsignA}, {\secref\PrintingOtherArrays}

====Notes====
Note that the type **[[CL:Functions:(array t)]]** is a proper //[[CL:Glossary:subtype]]// of the type **[[CL:Functions:(array *)]]**. The reason is that the type **[[CL:Functions:(array t)]]** is the set of //[[CL:Glossary:array|arrays]]// that can hold any //[[CL:Glossary:object]]// (the //[[CL:Glossary:element|elements]]// are of type **[[CL:Types:t]]**, which includes all //[[CL:Glossary:object|objects]]//). On the other hand, the type **[[CL:Functions:(array *)]]** is the set of all //[[CL:Glossary:array|arrays]]// whatsoever, including for example //[[CL:Glossary:array|arrays]]// that can hold only //[[CL:Glossary:characters]]//.

The type **[[CL:Functions:(array character)]]** is not a //[[CL:Glossary:subtype]]// of the type **[[CL:Functions:(array t)]]**; the two sets are //[[CL:Glossary:disjoint]]// because the type **[[CL:Functions:(array character)]]** is not the set of all //[[CL:Glossary:array|arrays]]// that can hold //[[CL:Glossary:characters]]//, but rather the set of //[[CL:Glossary:array|arrays]]// that are specialized to hold precisely //[[CL:Glossary:characters]]// and no other //[[CL:Glossary:object|objects]]//.

\issue{ARRAY-TYPE-ELEMENT-TYPE-SEMANTICS:UNIFY-UPGRADING} 
\issue{ARRAY-DIMENSION-IMPLICATIONS:ALL-FIXNUM}
