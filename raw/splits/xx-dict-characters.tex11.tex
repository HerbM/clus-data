====== Function DIGIT-CHAR-P ======

====Syntax====

**digit-char-p {char** //\opt} radix// → //weight//

====Arguments and Values====

//char// - a //[[CL:Glossary:character]]//.

//radix// - a //[[CL:Glossary:radix]]//. The default is ''10''.

//weight// - either a non-negative //[[CL:Glossary:integer]]// less than //radix//, or //[[CL:Glossary:false]]//.

====Description====

Tests whether //char// is a digit in the specified //radix//

(i.e. with a weight less than //radix//). If it is a digit in that //radix//, its weight is returned as an //[[CL:Glossary:integer]]//; otherwise **[[CL:Constant Variables:nil]]** is returned.

====Examples====

<blockquote> (digit-char-p #\\5) → 5 (digit-char-p #\\5 2) → //[[CL:Glossary:false]]// (digit-char-p #\\A) → //[[CL:Glossary:false]]// (digit-char-p #\\a) → //[[CL:Glossary:false]]// (digit-char-p #\\A 11) → 10 (digit-char-p #\\a 11) → 10 (mapcar #'(lambda (radix) (map 'list #'(lambda (x) (digit-char-p x radix)) "059AaFGZ")) '(2 8 10 16 36)) → ((0 NIL NIL NIL NIL NIL NIL NIL) (0 5 NIL NIL NIL NIL NIL NIL) (0 5 9 NIL NIL NIL NIL NIL) (0 5 9 10 10 15 NIL NIL) (0 5 9 10 10 15 16 35)) </blockquote>

====Affected By====

None. (In particular, the results of this predicate are independent of any special syntax which might have been enabled in the //[[CL:Glossary:current readtable]]//.)

====Exceptional Situations====

None.

====See Also====

**[[CL:Functions:alphanumericp]]**

====Notes====

Digits are //[[CL:Glossary:graphic]]// //[[CL:Glossary:characters]]//.

