====== Type SIGNED-BYTE ======

====Supertypes====

**[[CL:Types:signed-byte]]**, **[[CL:Types:integer]]**, **[[CL:Types:rational]]**, **[[CL:Types:real]]**, **[[CL:Types:number]]**, **[[CL:Types:t]]**

====Description====

The atomic //[[CL:Glossary:type specifier]]// **[[CL:Types:signed-byte]]** denotes the same type as is denoted by the //[[CL:Glossary:type specifier]]// **[[CL:Types:integer]]**; however, the list forms of these two //[[CL:Glossary:type specifiers]]// have different semantics.

====Compound Type Specifier Kind====

Abbreviating.

====Compound Type Specifier Syntax====

\Deftype{signed-byte}{\ttbrac{s | **[[CL:Types:wildcard|*]]**}}

====Compound Type Specifier Arguments====

//s// - a positive //[[CL:Glossary:integer]]//.

====Compound Type Specifier Description====

This denotes the set of //[[CL:Glossary:integers]]// that can be represented in two's-complement form in a //[[CL:Glossary:byte]]// of //s// bits. This is equivalent to \f{(integer ''-2^{s-1}'' ''2^{s-1}-1'')}. The type **[[CL:Types:signed-byte]]** or the type ''(signed-byte *)'' is the same as the type **[[CL:Types:integer]]**.

\issue{REAL-NUMBER-TYPE:X3J13-MAR-89}
