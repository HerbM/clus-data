====== Function CLEAR-INPUT ======

====Syntax====

**{clear-input}** //''&optional'' input-stream// → //**[[CL:Constant Variables:nil]]**//

====Arguments and Values====

//input-stream// - an //[[CL:Glossary:input]]// //[[CL:Glossary:stream designator]]//. The default is //[[CL:Glossary:standard input]]//.

====Description====

Clears any available input from //input-stream//.

If **[[CL:Functions:clear-input]]** does not make sense for //input-stream//, then **[[CL:Functions:clear-input]]** does nothing.

====Examples==== <blockquote> ;; The exact I/O behavior of this example might vary from implementation ;; to implementation depending on the kind of interactive buffering that ;; occurs. (The call to SLEEP here is intended to help even out the ;; differences in implementations which do not do line-at-a-time buffering.)

(defun read-sleepily (&optional (clear-p nil) (zzz 0)) (list (progn (print '>) (read)) ;; Note that input typed within the first ZZZ seconds ;; will be discarded. (progn (print '>) (if zzz (sleep zzz)) (print '>>) (if clear-p (clear-input)) (read))))

(read-sleepily)
▷ > \IN{10}
▷ >
▷ >> \IN{20} → (10 20)

(read-sleepily t)
▷ > \IN{10}
▷ >
▷ >> \IN{20} → (10 20)

(read-sleepily t 10)
▷ > \IN{10}
▷ > \IN{20} ; Some implementations won't echo typeahead here.
▷ >> \IN{30} → (10 30) </blockquote>

====Side Effects====

The //input-stream// is modified.

====Affected By====

**[[CL:Variables:*standard-input*]]**

====Exceptional Situations====

Should signal an error of type type-error if //input-stream// is not a //[[CL:Glossary:stream designator]]//.

====See Also====

**[[CL:Functions:clear-output]]**

====Notes====

None.

