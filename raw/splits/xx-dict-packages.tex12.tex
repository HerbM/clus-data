====== Function MAKE-PACKAGE ======

====Syntax====

**make-package {package-name** //\key} nicknames use// → //package//

====Arguments and Values====

//package-name// - a //[[CL:Glossary:\packagenamedesignator]]//.

//nicknames// - a //[[CL:Glossary:list]]// of //[[CL:Glossary:\packagenamedesignators]]//. The default is the //[[CL:Glossary:empty list]]//.

//use// -

a //[[CL:Glossary:list]]// of //[[CL:Glossary:package designators]]//.

The default is //[[CL:Glossary:implementation-defined]]//.

//package// - a //[[CL:Glossary:package]]//.

====Description====

Creates a new //[[CL:Glossary:package]]// with the name //package-name//.

//Nicknames// are additional //[[CL:Glossary:names]]// which may be used to refer to the new //[[CL:Glossary:package]]//.

//use// specifies zero or more //[[CL:Glossary:packages]]// the //[[CL:Glossary:external symbols]]// of which are to be inherited by the new //[[CL:Glossary:package]]//. See the //[[CL:Glossary:function]]// **[[CL:Functions:use-package]]**.

====Examples====

<blockquote> (make-package 'temporary :nicknames '("TEMP" "temp")) → #<PACKAGE "TEMPORARY"> (make-package "OWNER" :use '("temp")) → #<PACKAGE "OWNER"> (package-used-by-list 'temp) → (#<PACKAGE "OWNER">) (package-use-list 'owner) → (#<PACKAGE "TEMPORARY">) </blockquote>

====Side Effects====

None.

====Affected By====

The existence of other //[[CL:Glossary:packages]]// in the system.

====Exceptional Situations====

The consequences are unspecified if //[[CL:Glossary:packages]]// denoted by //use// do not exist.

A //[[CL:Glossary:correctable]]// error is signaled if the //package-name// or any of the //nicknames// is already the //[[CL:Glossary:name]]// or //[[CL:Glossary:nickname]]// of an existing //[[CL:Glossary:package]]//.

====See Also====

**[[CL:Macros:defpackage]]**, **[[CL:Functions:use-package]]**

====Notes====

In situations where the //[[CL:Glossary:packages]]// to be used contain symbols which would conflict, it is necessary to first create the package with '':use '()'', then to use **[[CL:Functions:shadow]]** or **[[CL:Functions:shadowing-import]]** to address the conflicts, and then after that to use **[[CL:Functions:use-package]]** once the conflicts have been addressed.

When packages are being created as part of the static definition of a program rather than dynamically by the program, it is generally considered more stylistically appropriate to use **[[CL:Macros:defpackage]]** rather than **[[CL:Functions:make-package]]**.

\issue{PACKAGE-FUNCTION-CONSISTENCY:MORE-PERMISSIVE} \issue{MAKE-PACKAGE-USE-DEFAULT:IMPLEMENTATION-DEPENDENT}
