====== Special Operator EVAL-WHEN ======

====Syntax====

\DefspecWithValues eval-when {\paren{\starparam{situation}} \starparam{form}} {\starparam{result}}

====Arguments and Values====

//situation// - One of the //[[CL:Glossary:symbols]]// **'':compile-toplevel''**\idxkwd{compile-toplevel}, **'':load-toplevel''**\idxkwd{load-toplevel}, **'':execute''**\idxkwd{execute}, \misc{compile}\idxref{compile}, \misc{load}\idxref{load}, or \misc{eval}\idxref{eval}.

The use of \misc{eval}, \misc{compile}, and \misc{load} is deprecated.

//forms// - an //[[CL:Glossary:implicit progn]]//.

//results// - the //[[CL:Glossary:values]]// of the //[[CL:Glossary:forms]]// if they are executed, or **[[CL:Constant Variables:nil]]** if they are not.

====Description====

The body of an \specref{eval-when} form is processed as an //[[CL:Glossary:implicit progn]]//, but only in the //situations// listed.



The use of the //situations// **'':compile-toplevel''** (or **[[CL:Functions:compile]]**) and **'':load-toplevel''** (or **[[CL:Functions:load]]**) controls whether and when //[[CL:Glossary:evaluation]]// occurs when \specref{eval-when} appears as a //[[CL:Glossary:top level form]]// in code processed by **[[CL:Functions:compile-file]]**. see section {\secref\FileCompilation}.

The use of the //situation// **'':execute''** (or **[[CL:Functions:eval]]**) controls whether evaluation occurs for other \specref{eval-when} //[[CL:Glossary:forms]]//; that is, those that are not //[[CL:Glossary:top level forms]]//, or those in code processed by **[[CL:Functions:eval]]** or **[[CL:Variables:compile]]**. If the **'':execute''** situation is specified in such a //[[CL:Glossary:form]]//, then the body //forms// are processed as an //[[CL:Glossary:implicit progn]]//; otherwise, the \specref{eval-when} //[[CL:Glossary:form]]// returns **[[CL:Constant Variables:nil]]**.


\specref{eval-when} normally appears as a //[[CL:Glossary:top level form]]//, but it is meaningful for it to appear as a //[[CL:Glossary:non-top-level form]]//. However, the compile-time side effects described in {\secref\Compilation} only take place when \specref{eval-when} appears as a //[[CL:Glossary:top level form]]//.



====Examples====

One example of the use of \specref{eval-when} is that for the compiler to be able to read a file properly when it uses user-defined //[[CL:Glossary:reader macros]]//, it is necessary to write

<blockquote> (eval-when (:compile-toplevel :load-toplevel :execute) (set-macro-character #\\$ #'(lambda (stream char) (declare (ignore char)) (list 'dollar (read stream))))) → T </blockquote> This causes the call to **[[CL:Functions:set-macro-character]]** to be executed in the compiler's execution environment, thereby modifying its reader syntax table.

<blockquote> ;;; The EVAL-WHEN in this case is not at toplevel, so only the :EXECUTE ;;; keyword is considered. At compile time, this has no effect. ;;; At load time (if the LET is at toplevel), or at execution time ;;; (if the LET is embedded in some other form which does not execute ;;; until later) this sets (SYMBOL-FUNCTION 'FOO1) to a function which ;;; returns 1. (let ((x 1)) (eval-when (:execute :load-toplevel :compile-toplevel) ([[CL:Macros:setf]] (symbol-function 'foo1) #'(lambda () x))))

;;; If this expression occurs at the toplevel of a file to be compiled, ;;; it has BOTH a compile time AND a load-time effect of setting ;;; (SYMBOL-FUNCTION 'FOO2) to a function which returns 2. (eval-when (:execute :load-toplevel :compile-toplevel) (let ((x 2)) (eval-when (:execute :load-toplevel :compile-toplevel) ([[CL:Macros:setf]] (symbol-function 'foo2) #'(lambda () x)))))

;;; If this expression occurs at the toplevel of a file to be compiled, ;;; it has BOTH a compile time AND a load-time effect of setting the ;;; function cell of FOO3 to a function which returns 3. (eval-when (:execute :load-toplevel :compile-toplevel) ([[CL:Macros:setf]] (symbol-function 'foo3) #'(lambda () 3)))

;;; #4: This always does nothing. It simply returns NIL. (eval-when (:compile-toplevel) (eval-when (:compile-toplevel) (print 'foo4)))

;;; If this form occurs at toplevel of a file to be compiled, FOO5 is ;;; printed at compile time. If this form occurs in a non-top-level ;;; position, nothing is printed at compile time. Regardless of context, ;;; nothing is ever printed at load time or execution time. (eval-when (:compile-toplevel) (eval-when (:execute) (print 'foo5)))

;;; If this form occurs at toplevel of a file to be compiled, FOO6 is ;;; printed at compile time. If this form occurs in a non-top-level ;;; position, nothing is printed at compile time. Regardless of context, ;;; nothing is ever printed at load time or execution time. (eval-when (:execute :load-toplevel) (eval-when (:compile-toplevel) (print 'foo6))) </blockquote>

====Affected By====

None.

====Exceptional Situations====

None.

====See Also====

**[[CL:Functions:compile-file]]**, {\secref\Compilation}

====Notes====

The following effects are logical consequences of the definition of \specref{eval-when}:

\beginlist \itemitem{\bull} Execution of a single \specref{eval-when} expression executes the body code at most once.

\itemitem{\bull} //[[CL:Glossary:Macros]]// intended for use in //[[CL:Glossary:top level forms]]// should be written so that side-effects are done by the //[[CL:Glossary:forms]]// in the macro expansion. The macro-expander itself should not do the side-effects.

For example:

Wrong:

<blockquote> (defmacro foo () (really-foo) `(really-foo)) </blockquote>

Right:

<blockquote> (defmacro foo () `(eval-when (:compile-toplevel :execute :load-toplevel) (really-foo))) </blockquote>

Adherence to this convention means that such //[[CL:Glossary:macros]]// behave intuitively when appearing as //[[CL:Glossary:non-top-level forms]]//.

\itemitem{\bull} Placing a variable binding around an \specref{eval-when} reliably captures the binding because the compile-time-too mode cannot occur (i.e. introducing a variable binding means that the \specref{eval-when} is not a //[[CL:Glossary:top level form]]//). For example,

<blockquote> (let ((x 3)) (eval-when (:execute :load-toplevel :compile-toplevel) (print x))) </blockquote>

prints **[[CL:Functions:3]]** at execution (i.e. load) time, and does not print anything at compile time. This is important so that expansions of **[[CL:Macros:defun]]** and **[[CL:Macros:defmacro]]** can be done in terms of \specref{eval-when} and can correctly capture the //[[CL:Glossary:lexical environment]]//.

<blockquote> (defun bar (x) (defun foo () (+ x 3))) </blockquote>

might expand into

<blockquote> (defun bar (x) (progn (eval-when (:compile-toplevel) (compiler::notice-function-definition 'foo '(x))) (eval-when (:execute :load-toplevel) ([[CL:Macros:setf]] (symbol-function 'foo) #'(lambda () (+ x 3)))))) </blockquote>

which would be treated by the above rules the same as

<blockquote> (defun bar (x) ([[CL:Macros:setf]] (symbol-function 'foo) #'(lambda () (+ x 3)))) </blockquote>

when the definition of ''bar'' is not a //[[CL:Glossary:top level form]]//. \endlist




\issue{EVAL-WHEN-NON-TOP-LEVEL:GENERALIZE-EVAL-NEW-KEYWORDS} \issue{EVAL-WHEN-NON-TOP-LEVEL:GENERALIZE-EVAL-NEW-KEYWORDS} \issue{EVAL-WHEN-OBSOLETE-KEYWORDS:X3J13-MAR-1993} \issue{EVAL-WHEN-NON-TOP-LEVEL:GENERALIZE-EVAL-NEW-KEYWORDS} \issue{DEFINING-MACROS-NON-TOP-LEVEL:ALLOW}
