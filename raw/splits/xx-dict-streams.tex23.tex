====== Function WRITE-STRING, WRITE-LINE ======

====Syntax====

**{write-string} {string ''&optional'' output-stream** //\key} start end// → //string// **{write-line} {string ''&optional'' output-stream** //\key} start end// → //string//

====Arguments and Values====

//string// - a //[[CL:Glossary:string]]//.

//output-stream// -- an //[[CL:Glossary:output]]// //[[CL:Glossary:stream designator]]//. The default is //[[CL:Glossary:standard output]]//.

//start//, //end// - //[[CL:Glossary:bounding index designators]]// of //string//. \Defaults{//start// and //end//}{''0'' and **[[CL:Constant Variables:nil]]**}

====Description====

**[[CL:Functions:write-string]]** writes the //[[CL:Glossary:characters]]// of the subsequence of //string// //[[CL:Glossary:bounded]]// by //start// and //end// to //output-stream//. **[[CL:Functions:write-line]]** does the same thing, but then outputs a newline afterwards.

====Examples====

<blockquote> (prog1 (write-string "books" nil :end 4) (write-string "worms"))
▷ bookworms → "books" (progn (write-char #\\*) (write-line "test12" *standard-output* :end 5) (write-line "*test2") (write-char #\\*) nil)
▷ *test1
▷ *test2
▷ * → NIL </blockquote>

====Side Effects====

None.

====Affected By====

**[[CL:Variables:*standard-output*]]**, **[[CL:Variables:*terminal-io*]]**.

====Exceptional Situations====

None.

====See Also====

**[[CL:Functions:read-line]]**, **[[CL:Functions:write-char]]**

====Notes====

**[[CL:Functions:write-line]]** and **[[CL:Functions:write-string]]** return //string//, not the substring //[[CL:Glossary:bounded]]// by //start// and //end//.

<blockquote> (write-string string) ≡ (dotimes (i (length string) (write-char (char string i)))

(write-line string) ≡ (prog1 (write-string string) (terpri)) </blockquote>

\issue{SUBSEQ-OUT-OF-BOUNDS} \issue{RANGE-OF-START-AND-END-PARAMETERS:INTEGER-AND-INTEGER-NIL} \issue{READ-AND-WRITE-BYTES:NEW-FUNCTIONS}
