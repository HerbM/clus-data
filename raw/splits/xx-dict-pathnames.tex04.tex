====== Function MAKE-PATHNAME ======

====Syntax====

\DefunWithValuesNewline make-pathname {''&key'' host device directory name type version defaults case} {pathname}

====Arguments and Values====

//host// - a //[[CL:Glossary:valid physical pathname host]]//.

\HairyDefault.

//device// - a //[[CL:Glossary:valid pathname device]]//.

\HairyDefault.

//directory// - a //[[CL:Glossary:valid pathname directory]]//.

\HairyDefault.

//name// - a //[[CL:Glossary:valid pathname name]]//.

\HairyDefault.

//type// - a //[[CL:Glossary:valid pathname type]]//.

\HairyDefault.

//version// - a //[[CL:Glossary:valid pathname version]]//.

\HairyDefault.

//defaults// - a //[[CL:Glossary:pathname designator]]//. \Default{a //[[CL:Glossary:pathname]]// whose host component is the same as the host component of \thevalueof{*default-pathname-defaults*}, and whose other components are all **[[CL:Constant Variables:nil]]**}

//case// - one of **'':common''** or **'':local''**. The default is **'':local''**.

//pathname// - a //[[CL:Glossary:pathname]]//.

====Description====

Constructs and returns a //[[CL:Glossary:pathname]]// from the supplied keyword arguments.

After the components supplied explicitly by //host//, //device//, //directory//, //name//, //type//, and //version// are filled in, the merging rules used by **[[CL:Functions:merge-pathnames]]** are used to fill in any

unsupplied components from the defaults supplied by //defaults//.

Whenever a //[[CL:Glossary:pathname]]// is constructed the components may be canonicalized if appropriate.

For the explanation of the arguments that can be supplied for each component, see section {\secref\PathnameComponents}.

If //case// is supplied, it is treated as described in \secref\PathnameComponentCase.

The resulting //pathname// is a //[[CL:Glossary:logical pathname]]// if and only its host component

is a //[[CL:Glossary:logical host]]// or a //[[CL:Glossary:string]]// that names a defined //[[CL:Glossary:logical host]]//.

If the //directory// is a //[[CL:Glossary:string]]//, it should be the name of a top level directory, and should not contain any punctuation characters; that is, specifying a //[[CL:Glossary:string]]//, ''str'', is equivalent to specifying the list ''(:absolute ''str'')''. Specifying the symbol **'':wild''** is equivalent to specifying the list ''(:absolute :wild-inferiors)'', or ''(:absolute :wild)'' in a file system that does not support **'':wild-inferiors''**.


====Examples====

<blockquote> ;; Implementation A -- an implementation with access to a single ;; Unix file system. This implementation happens to never display ;; the `host' information in a namestring, since there is only one host. (make-pathname :directory '(:absolute "public" "games") :name "chess" :type "db") → #P"/public/games/chess.db" \medbreak ;; Implementation B -- an implementation with access to one or more ;; VMS file systems. This implementation displays `host' information ;; in the namestring only when the host is not the local host. ;; It uses a double colon to separate a host name from the host's local ;; file name. (make-pathname :directory '(:absolute "PUBLIC" "GAMES") :name "CHESS" :type "DB") → #P"SYS''DISK:[PUBLIC.GAMES]CHESS.DB" (make-pathname :host "BOBBY" :directory '(:absolute "PUBLIC" "GAMES") :name "CHESS" :type "DB") → #P"BOBBY::SYS''DISK:[PUBLIC.GAMES]CHESS.DB" \medbreak ;; Implementation C -- an implementation with simultaneous access to ;; multiple file systems from the same Lisp image. In this ;; implementation, there is a convention that any text preceding the ;; first colon in a pathname namestring is a host name. (dolist (case '(:common :local)) (dolist (host '("MY-LISPM" "MY-VAX" "MY-UNIX")) (print (make-pathname :host host :case case :directory '(:absolute "PUBLIC" "GAMES") :name "CHESS" :type "DB"))))
▷ #P"MY-LISPM:>public>games>chess.db"
▷ #P"MY-VAX:SYS''DISK:[PUBLIC.GAMES]CHESS.DB"
▷ #P"MY-UNIX:/public/games/chess.db"
▷ #P"MY-LISPM:>public>games>chess.db"
▷ #P"MY-VAX:SYS''DISK:[PUBLIC.GAMES]CHESS.DB"
▷ #P"MY-UNIX:/PUBLIC/GAMES/CHESS.DB" → NIL </blockquote>

====Affected By====

The //[[CL:Glossary:file system]]//.

====Exceptional Situations====

None.

====See Also====

**[[CL:Functions:merge-pathnames]]**, **[[CL:Types:pathname]]**, **[[CL:Types:logical-pathname]]**,{\secref\FileSystemConcepts},

{\secref\PathnamesAsFilenames}

====Notes====

Portable programs should not supply **'':unspecific''** for any component. see section {\secref\UnspecificComponent}.

\issue{PATHNAME-UNSPECIFIC-COMPONENT:NEW-TOKEN} \issue{PATHNAME-COMPONENT-CASE:KEYWORD-ARGUMENT} \issue{PATHNAME-COMPONENT-CASE:KEYWORD-ARGUMENT} \issue{PATHNAME-HOST-PARSING:RECOGNIZE-LOGICAL-HOST-NAMES} \issue{PATHNAME-LOGICAL:ADD} \issue{PATHNAME-SUBDIRECTORY-LIST:NEW-REPRESENTATION} \issue{PATHNAME-LOGICAL:ADD} \issue{FILE-OPEN-ERROR:SIGNAL-FILE-ERROR}
