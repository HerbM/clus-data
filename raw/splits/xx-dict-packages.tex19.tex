====== Macro DEFPACKAGE ======

====Syntax====

\DefmacWithValues defpackage {defined-package-name \interleave{\down{option}}} {package}

\auxbnf{option}{\starparen{**'':nicknames''** \starparam{nickname}} | \CR \paren{**'':documentation''** ''string''} | \CR \starparen{**'':use''** \starparam{package-name}} | \CR \starparen{**'':shadow''** \stardown{symbol-name}} | \CR \starparen{**'':shadowing-import-from''** //package-name// \stardown{symbol-name}} | \CR \starparen{**'':import-from''** //package-name// \stardown{symbol-name}} | \CR \starparen{**'':export''** \stardown{symbol-name}} | \CR \starparen{**'':intern''** \stardown{symbol-name}} | \CR \paren{**'':size''** //[[CL:Glossary:integer]]//}}

====Arguments and Values====

//defined-package-name// - a //[[CL:Glossary:\packagenamedesignator]]//.

//package-name// - a //[[CL:Glossary:package designator]]//.

//nickname// - a //[[CL:Glossary:\packagenamedesignator]]//.

//symbol-name// - a //[[CL:Glossary:\symbolnamedesignator]]//.

//package// - the //[[CL:Glossary:package]]// named //package-name//.

====Description====

**[[CL:Macros:defpackage]]** creates a //[[CL:Glossary:package]]// as specified and returns the //[[CL:Glossary:package]]//.

If //defined-package-name// already refers to an existing //[[CL:Glossary:package]]//, the name-to-package mapping for that name is not changed. If the new definition is at variance with the current state of that //[[CL:Glossary:package]]//, the consequences are undefined; an implementation might choose to modify the existing //[[CL:Glossary:package]]// to reflect the new definition. If //defined-package-name// is a //[[CL:Glossary:symbol]]//, its //[[CL:Glossary:name]]// is used.

The standard ''options'' are described below.

\beginlist \itemitem{**'':nicknames''**}

The arguments to **'':nicknames''** set the //[[CL:Glossary:package]]//'s nicknames to the supplied names.



\itemitem{**'':documentation''**}

The argument to **'':documentation''** specifies a //[[CL:Glossary:documentation string]]//; it is attached as a //[[CL:Glossary:documentation string]]// to the //[[CL:Glossary:package]]//. At most one **'':documentation''** option can appear in a single **[[CL:Macros:defpackage]]** //[[CL:Glossary:form]]//.

\itemitem{**'':use''**}

The arguments to **'':use''** set the //[[CL:Glossary:packages]]// that the //[[CL:Glossary:package]]// named by //package-name// will inherit from. If **'':use''** is not supplied,

it defaults to the same //[[CL:Glossary:implementation-dependent]]// value as \thekeyarg{use} to **[[CL:Functions:make-package]]**.


\itemitem{**'':shadow''**}

The arguments to **'':shadow''**, //symbol-names//, name //[[CL:Glossary:symbols]]// that are to be created in the //[[CL:Glossary:package]]// being defined. These //[[CL:Glossary:symbols]]// are added to the list of shadowing //[[CL:Glossary:symbols]]// effectively as if by **[[CL:Functions:shadow]]**.

\itemitem{**'':shadowing-import-from''**}

The //[[CL:Glossary:symbols]]// named by the argument //symbol-names// are found (involving a lookup as if by **[[CL:Functions:find-symbol]]**) in the specified //package-name//. The resulting //[[CL:Glossary:symbols]]// are //[[CL:Glossary:imported]]// into the //[[CL:Glossary:package]]// being defined, and placed on the shadowing symbols list as if by **[[CL:Functions:shadowing-import]]**. In no case are //[[CL:Glossary:symbols]]// created in any //[[CL:Glossary:package]]// other than the one being defined.

\itemitem{**'':import-from''**}

The //[[CL:Glossary:symbols]]// named by the argument //symbol-names// are found in the //[[CL:Glossary:package]]// named by //package-name// and they are //[[CL:Glossary:imported]]// into the //[[CL:Glossary:package]]// being defined. In no case are //[[CL:Glossary:symbols]]// created in any //[[CL:Glossary:package]]// other than the one being defined.

\itemitem{**'':export''**}

The //[[CL:Glossary:symbols]]// named by the argument //symbol-names// are found or created in the //[[CL:Glossary:package]]// being defined and //[[CL:Glossary:exported]]//. The **'':export''** option interacts with the **'':use''** option, since inherited //[[CL:Glossary:symbols]]// can be used rather than new ones created. The **'':export''** option interacts with the **'':import-from''** and **'':shadowing-import-from''** options, since //[[CL:Glossary:imported]]// symbols can be used rather than new ones created. If an argument to the **'':export''** option is //[[CL:Glossary:accessible]]// as an (inherited) //[[CL:Glossary:internal symbol]]// via **[[CL:Functions:use-package]]**, that the //[[CL:Glossary:symbol]]// named by //symbol-name// is first //[[CL:Glossary:imported]]// into the //[[CL:Glossary:package]]// being defined, and is then //[[CL:Glossary:exported]]// from that //[[CL:Glossary:package]]//.

\itemitem{**'':intern''**}

The //[[CL:Glossary:symbols]]// named by the argument //symbol-names// are found or created in the //[[CL:Glossary:package]]// being defined. The **'':intern''** option interacts with the **'':use''** option, since inherited //[[CL:Glossary:symbols]]// can be used rather than new ones created.

\itemitem{**'':size''**}

The argument to the **'':size''** option declares the approximate number of //[[CL:Glossary:symbols]]// expected in the //[[CL:Glossary:package]]//. This is an efficiency hint only and might be ignored by an implementation. \endlist

The order in which the options appear in a **[[CL:Macros:defpackage]]** form is irrelevant. The order in which they are executed is as follows: \beginlist \itemitem{1.} **'':shadow''** and **'':shadowing-import-from''**. \itemitem{2.} **'':use''**. \itemitem{3.} **'':import-from''** and **'':intern''**. \itemitem{4.} **'':export''**. \endlist Shadows are established first, since they might be necessary to block spurious name conflicts when the **'':use''** option is processed. The **'':use''** option is executed next so that **'':intern''** and **'':export''** options can refer to normally inherited //[[CL:Glossary:symbols]]//. The **'':export''** option is executed last so that it can refer to //[[CL:Glossary:symbols]]// created by any of the other options; in particular, //[[CL:Glossary:shadowing symbols]]// and //[[CL:Glossary:imported]]// //[[CL:Glossary:symbols]]// can be made external.


If a {defpackage} //[[CL:Glossary:form]]// appears as a //[[CL:Glossary:top level form]]//, all of the actions normally performed by this //[[CL:Glossary:macro]]// at load time must also be performed at compile time.


====Examples====

<blockquote> (defpackage "MY-PACKAGE" (:nicknames "MYPKG" "MY-PKG") (:use "COMMON-LISP") (:shadow "CAR" "CDR") (:shadowing-import-from "VENDOR-COMMON-LISP" "CONS") (:import-from "VENDOR-COMMON-LISP" "GC") (:export "EQ" "CONS" "FROBOLA") )


(defpackage my-package (:nicknames mypkg :MY-PKG) ; remember Common Lisp conventions for case (:use common-lisp) ; conversion on symbols (:shadow CAR :cdr #:cons) (:export "CONS") ; this is the shadowed one. ) </blockquote>

====Affected By====

Existing //[[CL:Glossary:packages]]//.

====Exceptional Situations====

If one of the supplied **'':nicknames''** already refers to an existing //[[CL:Glossary:package]]//, an error of type **[[CL:Types:package-error]]** is signaled.

An error of type **[[CL:Types:program-error]]** should be signaled if **'':size''** or **'':documentation''** appears more than once.

Since //[[CL:Glossary:implementations]]// might allow extended ''options'' an error of type **[[CL:Types:program-error]]** should be signaled if an ''option'' is present that is not actually supported in the host //[[CL:Glossary:implementation]]//.

The collection of //symbol-name// arguments given to the options **'':shadow''**, **'':intern''**, **'':import-from''**, and **'':shadowing-import-from''** must all be disjoint; additionally, the //symbol-name// arguments given to **'':export''** and **'':intern''** must be disjoint. Disjoint in this context is defined as no two of the //symbol-names// being **[[CL:Functions:string=]]** with each other. If either condition is violated, an error of type **[[CL:Types:program-error]]** should be signaled.

For the **'':shadowing-import-from''** and **'':import-from''** options, a //[[CL:Glossary:correctable]]// //[[CL:Glossary:error]]// of type **[[CL:Types:package-error]]** is signaled if no //[[CL:Glossary:symbol]]// is //[[CL:Glossary:accessible]]// in the //[[CL:Glossary:package]]// named by //package-name// for one of the argument //symbol-names//.

Name conflict errors are handled by the underlying calls to **[[CL:Functions:make-package]]**, **[[CL:Functions:use-package]]**, **[[CL:Functions:import]]**, and **[[CL:Functions:export]]**. see section {\secref\PackageConcepts}.

====See Also====

**[[CL:Functions:documentation]]**, {\secref\PackageConcepts}, {\secref\Compilation}

====Notes====

The **'':intern''** option is useful if an **'':import-from''** or a **'':shadowing-import-from''** option in a subsequent call to **[[CL:Macros:defpackage]]** (for some other //[[CL:Glossary:package]]//) expects to find these //[[CL:Glossary:symbols]]// //[[CL:Glossary:accessible]]// but not necessarily external.


It is recommended that the entire //[[CL:Glossary:package]]// definition is put in a single place, and that all the //[[CL:Glossary:package]]// definitions of a program are in a single file. This file can be //[[CL:Glossary:loaded]]// before //[[CL:Glossary:loading]]// or compiling anything else that depends on those //[[CL:Glossary:packages]]//. Such a file can be read in \thepackage{common-lisp-user}, avoiding any initial state issues.

**[[CL:Macros:defpackage]]** cannot be used to create two ``mutually recursive'' packages, such as:

<blockquote> (defpackage my-package (:use common-lisp your-package) ;requires your-package to exist first (:export "MY-FUN")) (defpackage your-package (:use common-lisp) (:import-from my-package "MY-FUN") ;requires my-package to exist first (:export "MY-FUN")) </blockquote>

However, nothing prevents the user from using the //[[CL:Glossary:package]]//-affecting functions such as **[[CL:Functions:use-package]]**, **[[CL:Functions:import]]**, and **[[CL:Functions:export]]** to establish such links after a more standard use of **[[CL:Macros:defpackage]]**.

The macroexpansion of **[[CL:Macros:defpackage]]** could usefully canonicalize the names into //[[CL:Glossary:strings]]//, so that even if a source file has random //[[CL:Glossary:symbols]]// in the **[[CL:Macros:defpackage]]** form, the compiled file would only contain //[[CL:Glossary:strings]]//.

Frequently additional //[[CL:Glossary:implementation-dependent]]// options take the form of a //[[CL:Glossary:keyword]]// standing by itself as an abbreviation for a list **[[CL:Functions:(keyword T)]]**; this syntax should be properly reported as an unrecognized option in implementations that do not support it.


\issue{DEFPACKAGE:ADDITION} \issue{DOCUMENTATION-FUNCTION-BUGS:FIX} \issue{PACKAGE-FUNCTION-CONSISTENCY:MORE-PERMISSIVE} \issue{DOCUMENTATION-FUNCTION-BUGS:FIX} \issue{MAKE-PACKAGE-USE-DEFAULT:IMPLEMENTATION-DEPENDENT} \issue{COMPILE-FILE-HANDLING-OF-TOP-LEVEL-FORMS:CLARIFY}
