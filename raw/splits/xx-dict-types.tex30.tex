====== Function TYPE-ERROR-DATUM, TYPE-ERROR-EXPECTED-TYPE ======

====Syntax====

**type-error-datum** //condition// → //datum// **type-error-expected-type** //condition// → //expected-type//

====Arguments and Values====

//condition// - a //[[CL:Glossary:condition]]// of type **[[CL:Types:type-error]]**.

//datum// - an //[[CL:Glossary:object]]//.

//expected-type// - a //[[CL:Glossary:type specifier]]//.

====Description====

**[[CL:Functions:type-error-datum]]** returns the offending datum in the //[[CL:Glossary:situation]]// represented by the //condition//.

**[[CL:Functions:type-error-expected-type]]** returns the expected type of the offending datum in the //[[CL:Glossary:situation]]// represented by the //condition//.

====Examples====

<blockquote> (defun fix-digits (condition) (check-type condition type-error) (let* ((digits '(zero one two three four five six seven eight nine)) (val (position (type-error-datum condition) digits))) (if (and val (subtypep 'fixnum (type-error-expected-type condition))) (store-value 7))))

(defun foo (x) (handler-bind ((type-error #'fix-digits)) (check-type x number) (+ x 3)))

(foo 'seven) → 10 </blockquote>

====Side Effects====

None.

====Affected By====

None.

====Exceptional Situations====

None.

====See Also====

**[[CL:Types:type-error]]**,{\secref\Conditions}

====Notes====

None.

