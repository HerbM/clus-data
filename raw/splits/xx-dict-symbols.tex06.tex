====== Function COPY-SYMBOL ======

====Syntax====

**copy-symbol {symbol** //\opt} copy-properties// → //new-symbol//

====Arguments and Values====

//symbol// - a //[[CL:Glossary:symbol]]//.

//copy-properties// - a //[[CL:Glossary:generalized boolean]]//. The default is //[[CL:Glossary:false]]//.

//new-symbol// - a //[[CL:Glossary:fresh]]//, //[[CL:Glossary:uninterned]]// //[[CL:Glossary:symbol]]//.

====Description====

**[[CL:Functions:copy-symbol]]** returns a //[[CL:Glossary:fresh]]//, //[[CL:Glossary:uninterned]]// //[[CL:Glossary:symbol]]//, the //[[CL:Glossary:name]]// of which is **[[CL:Functions:string=]]** to and possibly the //[[CL:Glossary:same]]// as the //[[CL:Glossary:name]]// of the given //symbol//.

If //copy-properties// is //[[CL:Glossary:false]]//, the //new-symbol// is neither //[[CL:Glossary:bound]]// nor //[[CL:Glossary:fbound]]// and has a //[[CL:Glossary:null]]// //[[CL:Glossary:property list]]//. If //copy-properties// is //[[CL:Glossary:true]]//, then the initial //[[CL:Glossary:value]]// of //new-symbol// is the //[[CL:Glossary:value]]// of //symbol//, the initial //[[CL:Glossary:function]]// definition of //new-symbol// is the //[[CL:Glossary:functional value]]// of //symbol//, and the //[[CL:Glossary:property list]]// of //new-symbol// is

a //[[CL:Glossary:copy]]// of the //[[CL:Glossary:property list]]// of //symbol//.

====Examples====

<blockquote> ([[CL:Macros:defparameter]] fred 'fred-smith) → FRED-SMITH ([[CL:Macros:setf]] (symbol-value fred) 3) → 3 ([[CL:Macros:defparameter]] fred-clone-1a (copy-symbol fred nil)) → #:FRED-SMITH ([[CL:Macros:defparameter]] fred-clone-1b (copy-symbol fred nil)) → #:FRED-SMITH ([[CL:Macros:defparameter]] fred-clone-2a (copy-symbol fred t)) → #:FRED-SMITH ([[CL:Macros:defparameter]] fred-clone-2b (copy-symbol fred t)) → #:FRED-SMITH (eq fred fred-clone-1a) → //[[CL:Glossary:false]]// (eq fred-clone-1a fred-clone-1b) → //[[CL:Glossary:false]]// (eq fred-clone-2a fred-clone-2b) → //[[CL:Glossary:false]]// (eq fred-clone-1a fred-clone-2a) → //[[CL:Glossary:false]]// (symbol-value fred) → 3 (boundp fred-clone-1a) → //[[CL:Glossary:false]]// (symbol-value fred-clone-2a) → 3 ([[CL:Macros:setf]] (symbol-value fred-clone-2a) 4) → 4 (symbol-value fred) → 3 (symbol-value fred-clone-2a) → 4 (symbol-value fred-clone-2b) → 3 (boundp fred-clone-1a) → //[[CL:Glossary:false]]// ([[CL:Macros:setf]] (symbol-function fred) #'(lambda (x) x)) → #<FUNCTION anonymous> (fboundp fred) → //[[CL:Glossary:true]]// (fboundp fred-clone-1a) → //[[CL:Glossary:false]]// (fboundp fred-clone-2a) → //[[CL:Glossary:false]]// </blockquote>

====Side Effects====

None.

====Affected By====

None.

====Exceptional Situations====

Should signal an error of type type-error if //symbol// is not a //[[CL:Glossary:symbol]]//.

====See Also====

**[[CL:Functions:make-symbol]]**

====Notes====

Implementors are encouraged not to copy the //[[CL:Glossary:string]]// which is the //[[CL:Glossary:symbol]]//'s //[[CL:Glossary:name]]// unnecessarily. Unless there is a good reason to do so, the normal implementation strategy is for the //new-symbol//'s //[[CL:Glossary:name]]// to be //[[CL:Glossary:identical]]// to the given //symbol//'s //[[CL:Glossary:name]]//.

\issue{COPY-SYMBOL-PRINT-NAME:EQUAL} \issue{COPY-SYMBOL-COPY-PLIST:COPY-LIST}
