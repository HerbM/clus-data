====== Function NUMERATOR, DENOMINATOR ======

====Syntax====

**numerator ** //rational// → //numerator// **denominator** //rational// → //denominator//

====Arguments and Values====

//rational// - a //[[CL:Glossary:rational]]//.

//numerator// - an //[[CL:Glossary:integer]]//.

//denominator// - a positive //[[CL:Glossary:integer]]//.

====Description====

**[[CL:Functions:numerator]]** and **[[CL:Functions:denominator]]** reduce //rational// to canonical form and compute the numerator or denominator of that number.

**[[CL:Functions:numerator]]** and **[[CL:Functions:denominator]]** return the numerator or denominator of the canonical form of //rational//.

If //rational// is an //[[CL:Glossary:integer]]//, **[[CL:Functions:numerator]]** returns //rational// and **[[CL:Functions:denominator]]** returns 1.

====Examples==== <blockquote> (numerator 1/2) → 1 (denominator 12/36) → 3 (numerator -1) → -1 (denominator (/ -33)) → 33 (numerator (/ 8 -6)) → -4 (denominator (/ 8 -6)) → 3 </blockquote>

====Side Effects====

None.

====Affected By====

None.

====Exceptional Situations====

None.

====See Also====

**[[CL:Functions:/]]**

====Notes==== <blockquote> (gcd (numerator x) (denominator x)) → 1 </blockquote>

