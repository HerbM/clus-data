====== Function PPRINT-FILL, PPRINT-LINEAR, PPRINT-TABULAR ======

====Syntax====

**pprint-fill {stream object** //\opt} colon-p at-sign-p// → //**[[CL:Constant Variables:nil]]**// **pprint-linear {stream object** //\opt} colon-p at-sign-p// → //**[[CL:Constant Variables:nil]]**// **pprint-tabular {stream object** //\opt} colon-p at-sign-p tabsize// → //**[[CL:Constant Variables:nil]]**//

====Arguments and Values====

//stream// - an //[[CL:Glossary:output]]// //[[CL:Glossary:stream designator]]//.

//object// - an //[[CL:Glossary:object]]//.

//colon-p// - a //[[CL:Glossary:generalized boolean]]//. The default is //[[CL:Glossary:true]]//.

//at-sign-p// - a //[[CL:Glossary:generalized boolean]]//. The default is //[[CL:Glossary:implementation-dependent]]//.

//tabsize// - a non-negative //[[CL:Glossary:integer]]//. The default is ''16''.

====Description====

The functions **[[CL:Functions:pprint-fill]]**, **[[CL:Functions:pprint-linear]]**, and **[[CL:Functions:pprint-tabular]]** specify particular ways of //[[CL:Glossary:pretty printing]]// a //[[CL:Glossary:list]]// to //stream//. Each function prints parentheses around the output if and only if //colon-p// is //[[CL:Glossary:true]]//. Each function ignores its //at-sign-p// argument. (Both arguments are included even though only one is needed so that these functions can be used via \formatOp{/.../} and as **[[CL:Functions:set-pprint-dispatch]]** functions, as well as directly.) Each function handles abbreviation and the detection of circularity and sharing correctly, and uses **[[CL:Functions:write]]** to print //object// when it is a //[[CL:Glossary:non-list]]//.

If //object// is a //[[CL:Glossary:list]]// and if \thevalueof{*print-pretty*} is //[[CL:Glossary:false]]//, each of these functions prints //object// using a minimum of //[[CL:Glossary:whitespace]]//, as described in \secref\PrintingListsAndConses. Otherwise (if //object// is a //[[CL:Glossary:list]]// and if \thevalueof{*print-pretty*} is //[[CL:Glossary:true]]//):

\beginlist

\itemitem{\bull} The //[[CL:Glossary:function]]// **[[CL:Functions:pprint-linear]]** prints a //[[CL:Glossary:list]]// either all on one line, or with each //[[CL:Glossary:element]]// on a separate line.

\itemitem{\bull} The //[[CL:Glossary:function]]// **[[CL:Functions:pprint-fill]]** prints a //[[CL:Glossary:list]]// with as many //[[CL:Glossary:element|elements]]// as possible on each line.

\itemitem{\bull} The //[[CL:Glossary:function]]// **[[CL:Functions:pprint-tabular]]** is the same as **[[CL:Functions:pprint-fill]]** except that it prints the //[[CL:Glossary:element|elements]]// so that they line up in columns. The //tabsize// specifies the column spacing in //[[CL:Glossary:ems]]//,

which is the total spacing from the leading edge of one column to the leading edge of the next. \endlist

====Examples====

Evaluating the following with a line length of ''25'' produces the output shown.

<blockquote> (progn (princ "Roads ") (pprint-tabular *standard-output* '(elm main maple center) nil nil 8)) Roads ELM MAIN MAPLE CENTER </blockquote>

====Side Effects====

Performs output to the indicated //[[CL:Glossary:stream]]//.

====Affected By====

The cursor position on the indicated //[[CL:Glossary:stream]]//, if it can be determined.

====Exceptional Situations====

None.

====See Also====

None.

====Notes====

The //[[CL:Glossary:function]]// **[[CL:Functions:pprint-tabular]]** could be defined as follows:

<blockquote> (defun pprint-tabular (s list &optional (colon-p t) at-sign-p (tabsize nil)) (declare (ignore at-sign-p)) (when (null tabsize) ([[CL:Macros:defparameter]] tabsize 16)) (pprint-logical-block (s list :prefix (if colon-p "(" "") :suffix (if colon-p ")" "")) (pprint-exit-if-list-exhausted) (loop (write (pprint-pop) :stream s) (pprint-exit-if-list-exhausted) (write-char #\\Space s) (pprint-tab :section-relative 0 tabsize s) (pprint-newline :fill s)))) </blockquote>

Note that it would have been inconvenient to specify this function using **[[CL:Functions:format]]**, because of the need to pass its //tabsize// argument through to a \formatdirective{:T} nested within an iteration over a list.

\issue{PRETTY-PRINT-INTERFACE} \issue{GENERALIZE-PRETTY-PRINTER:UNIFY}
