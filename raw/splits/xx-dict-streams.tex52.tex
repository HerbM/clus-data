====== Macro WITH-INPUT-FROM-STRING ======

====Syntax====

\DefmacWithValuesNewline with-input-from-string {\paren{var string ''&key'' index start end} \starparam{declaration} \starparam{form}} {\starparam{result}}

====Arguments and Values====

//var// - a //[[CL:Glossary:variable]]// //[[CL:Glossary:name]]//.

//string// - a //[[CL:Glossary:form]]//; evaluated to produce a //[[CL:Glossary:string]]//.

//index// - a //[[CL:Glossary:place]]//.

//start//, //end// - //[[CL:Glossary:bounding index designators]]// of //string//. \Defaults{//start// and //end//}{''0'' and **[[CL:Constant Variables:nil]]**}

//declaration// - a \misc{declare} //[[CL:Glossary:expression]]//; \noeval.

//forms// - an //[[CL:Glossary:implicit progn]]//.

//result// - the //[[CL:Glossary:values]]// returned by the //forms//.

====Description====

Creates an

//[[CL:Glossary:input]]// //[[CL:Glossary:string stream]]//,

provides an opportunity to perform operations on the //[[CL:Glossary:stream]]// (returning zero or more //[[CL:Glossary:values]]//), and then closes the //[[CL:Glossary:string stream]]//.

//String// is evaluated first, and //var// is bound to a character //[[CL:Glossary:input]]// //[[CL:Glossary:string stream]]// that supplies //[[CL:Glossary:characters]]// from the subsequence of the resulting //[[CL:Glossary:string]]// //[[CL:Glossary:bounded]]// by //start// and //end//. The body is executed as an //[[CL:Glossary:implicit progn]]//.

The //[[CL:Glossary:input]]// //[[CL:Glossary:string stream]]// is automatically closed on exit from **[[CL:Macros:with-input-from-string]]**, no matter whether the exit is normal or abnormal.

The //[[CL:Glossary:input]]// //[[CL:Glossary:string stream]]// to which the //[[CL:Glossary:variable]]// //var// is //[[CL:Glossary:bound]]// has //[[CL:Glossary:dynamic extent]]//; its //[[CL:Glossary:extent]]// ends when the //[[CL:Glossary:form]]// is exited.

The //index// is a pointer within the //string// to be advanced. If **[[CL:Macros:with-input-from-string]]** is exited normally, then //index// will have

as its //[[CL:Glossary:value]]// the index into the //string// indicating the first character not read which is **[[CL:Functions:(length //string//)]]** if all characters were used. The place specified by //index// is not updated as reading progresses, but only at the end of the operation.

//start// and //index// may both specify the same variable, which is a pointer within the //string// to be advanced, perhaps repeatedly by some containing loop.

The consequences are undefined if an attempt is made to //[[CL:Glossary:assign]]// the //[[CL:Glossary:variable]]// //var//.

====Examples====

<blockquote> (with-input-from-string (s "XXX1 2 3 4xxx" :index ind :start 3 :end 10) (+ (read s) (read s) (read s))) → 6 ind → 9 (with-input-from-string (s "Animal Crackers" :index j :start 6) (read s)) → CRACKERS </blockquote> The variable ''j'' is set to ''15''.

====Side Effects====

The //[[CL:Glossary:value]]// of the //[[CL:Glossary:place]]// named by //index//, if any, is modified.

====Affected By====

None.

====Exceptional Situations====

None.

====See Also====

**[[CL:Functions:make-string-input-stream]]**,

{\secref\TraversalRules}

====Notes====

None.

\issue{DECLS-AND-DOC} \issue{SUBSEQ-OUT-OF-BOUNDS:IS-AN-ERROR} \issue{STREAM-ACCESS:ADD-TYPES-ACCESSORS} \issue{WITH-OPEN-FILE-STREAM-EXTENT:DYNAMIC-EXTENT} \issue{WITH-OPEN-FILE-SETQ:EXPLICITLY-VAGUE} \issue{MAPPING-DESTRUCTIVE-INTERACTION:EXPLICITLY-VAGUE}
