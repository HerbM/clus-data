====== Function LOGTEST ======

====Syntax====

**logtest** //integer-1 integer-2// → //generalized-boolean//

====Arguments and Values====

//integer-1// - an //[[CL:Glossary:integer]]//.

//integer-2// - an //[[CL:Glossary:integer]]//.

//generalized-boolean// - a //[[CL:Glossary:generalized boolean]]//.

====Description====

Returns //[[CL:Glossary:true]]// if any of the bits designated by the 1's in //integer-1// is 1 in //integer-2//; otherwise it is //[[CL:Glossary:false]]//. //integer-1// and //integer-2// are treated as if they were binary.

Negative //integer-1// and //integer-2// are treated as if they were represented in two's-complement binary.

====Examples====

<blockquote> (logtest 1 7) → //[[CL:Glossary:true]]// (logtest 1 2) → //[[CL:Glossary:false]]// (logtest -2 -1) → //[[CL:Glossary:true]]// (logtest 0 -1) → //[[CL:Glossary:false]]// </blockquote>

====Side Effects====

None.

====Affected By====

None.

====Exceptional Situations====

Should signal an error of type type-error if //integer-1// is not an //[[CL:Glossary:integer]]//. Should signal an error of type type-error if //integer-2// is not an //[[CL:Glossary:integer]]//.

====See Also====

None.

====Notes====

<blockquote> (logtest //x// //y//) ≡ (not (zerop (logand //x// //y//))) </blockquote>

