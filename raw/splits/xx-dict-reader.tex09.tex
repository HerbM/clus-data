====== Function SET-DISPATCH-MACRO-CHARACTER, GET-DISPATCH-MACRO-CHARACTER ======

====Syntax====

\DefunWithValues get-dispatch-macro-character {disp-char sub-char ''&optional'' readtable} {function}

\DefunWithValues set-dispatch-macro-character {disp-char sub-char new-function ''&optional'' readtable} {\t}

====Arguments and Values====

//disp-char// - a //[[CL:Glossary:character]]//.

//sub-char// - a //[[CL:Glossary:character]]//.

//readtable// - a //[[CL:Glossary:readtable designator]]//.

The default is the //[[CL:Glossary:current readtable]]//.

//function// - a //[[CL:Glossary:function designator]]// or **[[CL:Constant Variables:nil]]**.

//new-function// - a //[[CL:Glossary:function designator]]//.

====Description====

**[[CL:Functions:set-dispatch-macro-character]]** causes //new-function// to be called when //disp-char// followed by //sub-char// is read.

If //sub-char// is a lowercase letter, it is converted to its uppercase equivalent. It is an error if //sub-char// is one of the ten decimal digits.

**[[CL:Functions:set-dispatch-macro-character]]** installs a //new-function// to be called when a particular //[[CL:Glossary:dispatching macro character]]// pair is read. //New-function// is installed as the dispatch function to be called when //readtable// is in use and when //disp-char// is followed by //sub-char//.

For more information about how the //new-function// is invoked, see section {\secref\MacroChars}.

**[[CL:Functions:get-dispatch-macro-character]]** retrieves the dispatch function associated with //disp-char// and //sub-char// in //readtable//.

**[[CL:Functions:get-dispatch-macro-character]]** returns the macro-character function for //sub-char// under //disp-char//, or **[[CL:Constant Variables:nil]]** if there is no function associated with //sub-char//. If //sub-char// is a decimal digit, **[[CL:Functions:get-dispatch-macro-character]]** returns **[[CL:Constant Variables:nil]]**.

====Examples====

<blockquote> (get-dispatch-macro-character #\# #\\{) → NIL (set-dispatch-macro-character #\# #\\{ ;dispatch on #{ #'(lambda(s c n) (let ((list (read s nil (values) t))) ;list is object after #n{ (when (consp list) ;return nth element of list (unless (and n (< 0 n (length list))) ([[CL:Macros:defparameter]] n 0)) ([[CL:Macros:defparameter]] list (nth n list))) list))) → T #{(1 2 3 4) → 1 #3{(0 1 2 3) → 3 #{123 → 123 </blockquote>

If it is desired that ''#\''''foo'''' : as if it were ''(dollars ''foo'')''.

<blockquote> (defun |#''-reader| (stream subchar arg) (declare (ignore subchar arg)) (list 'dollars (read stream t nil t))) → |#''-reader| (set-dispatch-macro-character #\# #\\\'' #'|#\$-reader|) → T </blockquote>

====See Also====

{\secref\MacroChars}

====Side Effects====

The //readtable// is modified.

====Affected By====

**[[CL:Variables:*readtable*]]**.

====Exceptional Situations====

For either function, an error is signaled if //disp-char// is not a //[[CL:Glossary:dispatching macro character]]// in //readtable//.

====See Also====

**[[CL:Variables:*readtable*]]**

====Notes==== It is necessary to use **[[CL:Functions:make-dispatch-macro-character]]** to set up the dispatch character before specifying its sub-characters.

\issue{GET-MACRO-CHARACTER-READTABLE:NIL-STANDARD}
