====== Macro DEFMACRO ======

====Syntax====

\DefmacWithValuesNewline defmacro {name lambda-list {\DeclsAndDoc} \starparam{form}} {name}

====Arguments and Values====

//name// - a //[[CL:Glossary:symbol]]//.


//lambda-list// - a //[[CL:Glossary:macro lambda list]]//.

//declaration// - a \misc{declare} //[[CL:Glossary:expression]]//; \noeval.

//documentation// - a //[[CL:Glossary:string]]//; \noeval.

//form// - a //[[CL:Glossary:form]]//.

====Description====

Defines //name// as a //[[CL:Glossary:macro]]// by associating a //[[CL:Glossary:macro function]]// with that //name// in the global environment.

The //[[CL:Glossary:macro function]]// is defined in the same //[[CL:Glossary:lexical environment]]// in which the **[[CL:Macros:defmacro]]** //[[CL:Glossary:form]]// appears.



The parameter variables in //lambda-list// are bound to destructured portions of the macro call.

The expansion function accepts two arguments, a //[[CL:Glossary:form]]// and an //[[CL:Glossary:environment]]//. The expansion function returns a //[[CL:Glossary:form]]//. The body of the expansion function is specified by //forms//. //Forms// are executed in order. The value of the last //form// executed is returned as the expansion of the //[[CL:Glossary:macro]]//.

The body //forms// of the expansion function (but not the //lambda-list//)

are implicitly enclosed in a //[[CL:Glossary:block]]// whose name is //name//.


The //lambda-list// conforms to the requirements described in \secref\MacroLambdaLists.


//Documentation// is attached as a //[[CL:Glossary:documentation string]]// to //name// (as kind \specref{function}) and to the //[[CL:Glossary:macro function]]//.



**[[CL:Macros:defmacro]]** can be used to redefine a //[[CL:Glossary:macro]]// or to replace a //[[CL:Glossary:function]]// definition with a //[[CL:Glossary:macro]]// definition.




Recursive expansion of the //[[CL:Glossary:form]]// returned must terminate, including the expansion of other //[[CL:Glossary:macros]]// which are //[[CL:Glossary:subforms]]// of other //[[CL:Glossary:forms]]// returned.

The consequences are undefined if the result of fully macroexpanding

a //[[CL:Glossary:form]]// contains any //[[CL:Glossary:circular]]// //[[CL:Glossary:list structure]]// except in //[[CL:Glossary:literal objects]]//.



If a **[[CL:Macros:defmacro]]** //[[CL:Glossary:form]]// appears as a //[[CL:Glossary:top level form]]//, the //[[CL:Glossary:compiler]]// must store the //[[CL:Glossary:macro]]// definition at compile time, so that occurrences of the macro later on in the file can be expanded correctly. Users must ensure that the body of the //[[CL:Glossary:macro]]// can be evaluated at compile time if it is referenced within the //[[CL:Glossary:file]]// being //[[CL:Glossary:compiled]]//.

====Examples====

<blockquote> (defmacro mac1 (a b) "Mac1 multiplies and adds" `(+ ,a (* ,b 3))) → MAC1 (mac1 4 5) → 19 (documentation 'mac1 'function) → "Mac1 multiplies and adds" (defmacro mac2 (&optional (a 2 b) (c 3 d) &rest x) `'(,a ,b ,c ,d ,x)) → MAC2 (mac2 6) → (6 T 3 NIL NIL) (mac2 6 3 8) → (6 T 3 T (8)) (defmacro mac3 (&whole r a &optional (b 3) &rest x &key c (d a)) `'(,r ,a ,b ,c ,d ,x)) → MAC3 (mac3 1 6 :d 8 :c 9 :d 10) → ((MAC3 1 6 :D 8 :C 9 :D 10) 1 6 9 8 (:D 8 :C 9 :D 10)) </blockquote>








The stipulation that an embedded //[[CL:Glossary:destructuring lambda list]]// is permitted only where //[[CL:Glossary:ordinary lambda list]]// syntax would permit a parameter name but not a //[[CL:Glossary:list]]// is made to prevent ambiguity. For example, the following is not valid:

<blockquote> (defmacro loser (x &optional (a b &rest c) &rest z) ...) </blockquote> because //[[CL:Glossary:ordinary lambda list]]// syntax does permit a //[[CL:Glossary:list]]// following \optional; the list ''(a b \&rest c)'' would be interpreted as describing an optional parameter named ''a'' whose default value is that of the form ''b'', with a supplied-p parameter named \keyref{rest} (not valid), and an extraneous symbol ''c'' in the list (also not valid). An almost correct way to express this is

<blockquote> (defmacro loser (x &optional ((a b &rest c)) &rest z) ...) </blockquote> The extra set of parentheses removes the ambiguity. However, the definition is now incorrect because a macro call such as ''(loser (car pool))'' would not provide any argument form for the lambda list ''(a b \&rest c)'', and so the default value against which to match the //[[CL:Glossary:lambda list]]// would be **[[CL:Constant Variables:nil]]** because no explicit default value was specified. The consequences of this are unspecified since the empty list, **[[CL:Constant Variables:nil]]**, does not have //[[CL:Glossary:forms]]// to satisfy the parameters ''a'' and ''b''. The fully correct definition would be either

<blockquote> (defmacro loser (x &optional ((a b &rest c) '(nil nil)) &rest z) ...) </blockquote> or

<blockquote> (defmacro loser (x &optional ((&optional a b &rest c)) &rest z) ...) </blockquote> These differ slightly: the first requires that if the macro call specifies ''a'' explicitly then it must also specify ''b'' explicitly, whereas the second does not have this requirement. For example,

<blockquote> (loser (car pool) ((+ x 1))) </blockquote> would be a valid call for the second definition but not for the first.



<blockquote> (defmacro dm1a (&whole x) `',x) (macroexpand '(dm1a)) → (QUOTE (DM1A)) (macroexpand '(dm1a a)) is an error.

(defmacro dm1b (&whole x a &optional b) `'(,x ,a ,b)) (macroexpand '(dm1b)) is an error. (macroexpand '(dm1b q)) → (QUOTE ((DM1B Q) Q NIL)) (macroexpand '(dm1b q r)) → (QUOTE ((DM1B Q R) Q R)) (macroexpand '(dm1b q r s)) is an error. </blockquote>

<blockquote> (defmacro dm2a (&whole form a b) `'(form ,form a ,a b ,b)) (macroexpand '(dm2a x y)) → (QUOTE (FORM (DM2A X Y) A X B Y)) (dm2a x y) → (FORM (DM2A X Y) A X B Y)

(defmacro dm2b (&whole form a (&whole b (c . d) &optional (e 5)) &body f &environment env) ``(,',form ,,a ,',b ,',(macroexpand c env) ,',d ,',e ,',f)) ;Note that because backquote is involved, implementations may differ ;slightly in the nature (though not the functionality) of the expansion. (macroexpand '(dm2b x1 (((incf x2) x3 x4)) x5 x6)) → (LIST* '(DM2B X1 (((INCF X2) X3 X4)) X5 X6) X1 '((((INCF X2) X3 X4)) (SETQ X2 (+ X2 1)) (X3 X4) 5 (X5 X6))), T (let ((x1 5)) (macrolet ((segundo (x) `(cadr ,x))) (dm2b x1 (((segundo x2) x3 x4)) x5 x6))) → ((DM2B X1 (((SEGUNDO X2) X3 X4)) X5 X6) 5 (((SEGUNDO X2) X3 X4)) (CADR X2) (X3 X4) 5 (X5 X6)) </blockquote>



====Affected By====

None.

\label Exceptional Situations:\none.

====See Also====

**[[CL:Macros:define-compiler-macro]]**,

**[[CL:Macros:destructuring-bind]]**, **[[CL:Functions:documentation]]**, **[[CL:Functions:macroexpand]]**, **[[CL:Variables:*macroexpand-hook*]]**, \specref{macrolet}, **[[CL:Functions:macro-function]]**, {\secref\Evaluation}, {\secref\Compilation}, {\secref\DocVsDecls}

====Notes====

None.


\issue{COMPILE-FILE-HANDLING-OF-TOP-LEVEL-FORMS:CLARIFY} \issue{DEFMACRO-LAMBDA-LIST:TIGHTEN-DESCRIPTION} \issue{DEFINE-COMPILER-MACRO:X3J13-NOV89} \issue{DECLS-AND-DOC} \issue{DEFINING-MACROS-NON-TOP-LEVEL:ALLOW} \issue{FLET-IMPLICIT-BLOCK:YES} \issue{DEFMACRO-BLOCK-SCOPE:EXCLUDES-BINDINGS} \issue{DOCUMENTATION-FUNCTION-BUGS:FIX} \issue{RECURSIVE-DEFTYPE:EXPLICITLY-VAGUE}
