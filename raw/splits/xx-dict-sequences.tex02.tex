====== Function COPY-SEQ ======

====Syntax====

**copy-seq** //sequence// → //copied-sequence//

====Arguments and Values====

//sequence// - a //[[CL:Glossary:proper sequence]]//.

//copied-sequence// - a //[[CL:Glossary:proper sequence]]//.

====Description====

Creates a copy of //sequence//. The //[[CL:Glossary:element|elements]]// of the new //[[CL:Glossary:sequence]]// are the //[[CL:Glossary:same]]// as the corresponding //[[CL:Glossary:element|elements]]// of the given //sequence//.

If //sequence// is a //[[CL:Glossary:vector]]//, the result is a //[[CL:Glossary:fresh]]// //[[CL:Glossary:simple array]]// of //[[CL:Glossary:rank]]// one that has the same //[[CL:Glossary:actual array element type]]// as //sequence//. If //sequence// is a //[[CL:Glossary:list]]//, the result is a //[[CL:Glossary:fresh]]// //[[CL:Glossary:list]]//.

====Examples==== <blockquote> ([[CL:Macros:defparameter]] str "a string") → "a string" (equalp str (copy-seq str)) → //[[CL:Glossary:true]]// (eql str (copy-seq str)) → //[[CL:Glossary:false]]// </blockquote>

====Side Effects====

None.

====Affected By====

None.

====Exceptional Situations====

Should be prepared to signal an error of type type-error if //sequence// is not a //[[CL:Glossary:proper sequence]]//.

====See Also====

**[[CL:Functions:copy-list]]**

====Notes====

From a functional standpoint, <blockquote> (copy-seq x) ≡ (subseq x 0) </blockquote> However, the programmer intent is typically very different in these two cases.

