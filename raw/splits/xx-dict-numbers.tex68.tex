====== Function DPB ======

====Syntax====

**dpb** //newbyte bytespec integer// → //result-integer//

====Arguments and Values====

//newbyte// - an //[[CL:Glossary:integer]]//.

//bytespec// - a //[[CL:Glossary:byte specifier]]//.

//integer// - an //[[CL:Glossary:integer]]//.

//result-integer// - an //[[CL:Glossary:integer]]//.

====Description====

**[[CL:Functions:dpb]]** (deposit byte) is used to replace a field of bits within //integer//. **[[CL:Functions:dpb]]** returns an //[[CL:Glossary:integer]]// that is the same as //integer// except in the bits specified by //bytespec//.

Let ''s'' be the size specified by //bytespec//; then the low ''s'' bits of //newbyte// appear in the result in the byte specified by //bytespec//. //Newbyte// is interpreted as being right-justified, as if it were the result of **[[CL:Functions:ldb]]**.

====Examples====

<blockquote> (dpb 1 (byte 1 10) 0) → 1024 (dpb -2 (byte 2 10) 0) → 2048 (dpb 1 (byte 2 10) 2048) → 1024 </blockquote>

====Side Effects====

None.

====Affected By====

None.

====Exceptional Situations====

None.

====See Also====

**[[CL:Functions:byte]]**, **[[CL:Functions:deposit-field]]**, **[[CL:Functions:ldb]]**

====Notes====

<blockquote> (logbitp //j// (dpb //m// (byte //s// //p//) //n//)) ≡ (if (and (>= //j// //p//) (< //j// (+ //p// //s//))) (logbitp (- //j// //p//) //m//) (logbitp //j// //n//)) </blockquote>

In general,

<blockquote> (dpb //x// (byte 0 //y//) //z//) → //z// </blockquote>

for all valid values of //x//, //y//, and //z//.

Historically, the name "dpb" comes from a DEC PDP-10 assembly language instruction meaning "deposit byte."

