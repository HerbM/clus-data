====== Local Macro LOOP-FINISH ======

====Syntax====

\DefmacNoReturn loop-finish {\noargs}

====Description====

\Themacro{loop-finish} can be used lexically within

an extended **[[CL:Macros:loop]]** //[[CL:Glossary:form]]// to terminate that //[[CL:Glossary:form]]// "normally." That is, it transfers control to the loop epilogue

of the lexically innermost extended **[[CL:Macros:loop]]** //[[CL:Glossary:form]]//. This permits execution of any **[[CL:Macros:finally]]** clause (for effect) and

the return of any accumulated result.

====Examples====

<blockquote> ;; Terminate the loop, but return the accumulated count. (loop for i in '(1 2 3 stop-here 4 5 6) when (symbolp i) do (loop-finish) count i) → 3

;; The preceding loop is equivalent to: (loop for i in '(1 2 3 stop-here 4 5 6) until (symbolp i) count i) → 3

;; While LOOP-FINISH can be used can be used in a variety of ;; situations it is really most needed in a situation where a need ;; to exit is detected at other than the loop's `top level' ;; (where UNTIL or WHEN often work just as well), or where some ;; computation must occur between the point where a need to exit is ;; detected and the point where the exit actually occurs. For example: (defun tokenize-sentence (string) (macrolet ((add-word (wvar svar) `(when ,wvar (push (coerce (nreverse ,wvar) 'string) ,svar) ([[CL:Macros:defparameter]] ,wvar nil)))) (loop with word = '() and sentence = '() and endpos = nil for i below (length string) do (let ((char (aref string i))) (case char (#\\Space (add-word word sentence)) (#\\. ([[CL:Macros:defparameter]] endpos (1+ i)) (loop-finish)) (otherwise (push char word)))) finally (add-word word sentence) (return (values (nreverse sentence) endpos))))) → TOKENIZE-SENTENCE

(tokenize-sentence "this is a sentence. this is another sentence.") → ("this" "is" "a" "sentence"), 19

(tokenize-sentence "this is a sentence") → ("this" "is" "a" "sentence"), NIL

</blockquote>

====Side Effects====

Transfers control.

====Affected By====

None.

====Exceptional Situations====

Whether or not **[[CL:Macros:loop-finish]]** is //[[CL:Glossary:fbound]]// in the //[[CL:Glossary:global environment]]// is //[[CL:Glossary:implementation-dependent]]//; however, the restrictions on redefinition and //[[CL:Glossary:shadowing]]// of **[[CL:Macros:loop-finish]]** are the same as for //[[CL:Glossary:symbols]]// in \thepackage{common-lisp} which are //[[CL:Glossary:fbound]]// in the //[[CL:Glossary:global environment]]//. The consequences of attempting to use **[[CL:Macros:loop-finish]]** outside of **[[CL:Macros:loop]]** are undefined.

====See Also====

**[[CL:Macros:loop]]**, {\secref\LoopFacility}

====Notes====

\issue{LEXICAL-CONSTRUCT-GLOBAL-DEFINITION:UNDEFINED}
