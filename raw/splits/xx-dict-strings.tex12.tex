====== Function MAKE-STRING ======

====Syntax====

**make-string {size** //\key} initial-element element-type// → //string//

====Arguments and Values====

//size// - a //[[CL:Glossary:valid array dimension]]//.

//initial-element// - a //[[CL:Glossary:character]]//.

The default is //[[CL:Glossary:implementation-dependent]]//.

//element-type// - a //[[CL:Glossary:type specifier]]//. The default is **[[CL:Types:character]]**.

//string// - a //[[CL:Glossary:simple string]]//.

====Description====

**[[CL:Functions:make-string]]** returns a //[[CL:Glossary:simple string]]// of length //size// whose elements have been initialized to //initial-element//.

The //element-type// names the //[[CL:Glossary:type]]// of the //[[CL:Glossary:element|elements]]// of the //[[CL:Glossary:string]]//; a //[[CL:Glossary:string]]// is constructed of the most //[[CL:Glossary:specialized]]// //[[CL:Glossary:type]]// that can accommodate //[[CL:Glossary:element|elements]]// of the given //[[CL:Glossary:type]]//.

====Examples====

<blockquote> (make-string 10 :initial-element #\\5) → "5555555555" (length (make-string 10)) → 10 </blockquote>

====Affected By====

The //[[CL:Glossary:implementation]]//.

====Exceptional Situations====

None.

====See Also====

None.

====Notes====

None.

\issue{ARGUMENTS-UNDERSPECIFIED:SPECIFY} \issue{ARGUMENTS-UNDERSPECIFIED:SPECIFY} \issue{CHARACTER-PROPOSAL:2-3-6} \issue{CHARACTER-PROPOSAL:2-3-6}
