====== Function CERROR ======

====Syntax====

**cerror {continue-format-control datum** //\rest} arguments// → //**[[CL:Constant Variables:nil]]**//

====Arguments and Values====

//Continue-format-control// - a //[[CL:Glossary:format control]]//.

\reviewer{Barmar: What is continue-format-control used for??}

//datum//, //arguments// - //[[CL:Glossary:designators]]// for a //[[CL:Glossary:condition]]// of default type **[[CL:Types:simple-error]]**.

====Description====

**[[CL:Functions:cerror]]** effectively invokes **[[CL:Functions:error]]** on the //[[CL:Glossary:condition]]// named by //datum//. As with any function that implicitly calls **[[CL:Functions:error]]**, if the //[[CL:Glossary:condition]]// is not handled, ''(invoke-debugger ''condition'')'' is executed. While signaling is going on, and while in the debugger if it is reached, it is possible to continue code execution (i.e. to return from **[[CL:Functions:cerror]]**) using \therestart{continue}.

If //datum// is a //[[CL:Glossary:condition]]//, //arguments// can be supplied, but are used only in conjunction with the //continue-format-control//.

====Examples====

<blockquote> (defun real-sqrt (n) (when (minusp n) ([[CL:Macros:defparameter]] n (- n)) (cerror "Return sqrt(~D) instead." "Tried to take sqrt(-~D)." n)) (sqrt n))

(real-sqrt 4) → 2.0

(real-sqrt -9)
▷ Correctable error in REAL-SQRT: Tried to take sqrt(-9).
▷ Restart options:
▷ 1: Return sqrt(9) instead.
▷ 2: Top level.
▷ Debug> \IN{:continue 1} → 3.0

(define-condition not-a-number (error) ((argument :reader not-a-number-argument :initarg :argument)) (:report (lambda (condition stream) (format stream "~S is not a number." (not-a-number-argument condition)))))

(defun assure-number (n) (loop (when (numberp n) (return n)) (cerror "Enter a number." 'not-a-number :argument n) (format t "~&Type a number: ") ([[CL:Macros:defparameter]] n (read)) (fresh-line)))

(assure-number 'a)
▷ Correctable error in ASSURE-NUMBER: A is not a number.
▷ Restart options:
▷ 1: Enter a number.
▷ 2: Top level.
▷ Debug> \IN{:continue 1}
▷ Type a number: \IN{1/2} → 1/2

(defun assure-large-number (n) (loop (when (and (numberp n) (> n 73)) (return n)) (cerror "Enter a number~:[~; a bit larger than ~D~]." "~*~A is not a large number." (numberp n) n) (format t "~&Type a large number: ") ([[CL:Macros:defparameter]] n (read)) (fresh-line)))

(assure-large-number 10000) → 10000

(assure-large-number 'a)
▷ Correctable error in ASSURE-LARGE-NUMBER: A is not a large number.
▷ Restart options:
▷ 1: Enter a number.
▷ 2: Top level.
▷ Debug> \IN{:continue 1}
▷ Type a large number: \IN{88} → 88

(assure-large-number 37)
▷ Correctable error in ASSURE-LARGE-NUMBER: 37 is not a large number.
▷ Restart options:
▷ 1: Enter a number a bit larger than 37.
▷ 2: Top level.
▷ Debug> \IN{:continue 1}
▷ Type a large number: \IN{259} → 259

(define-condition not-a-large-number (error) ((argument :reader not-a-large-number-argument :initarg :argument)) (:report (lambda (condition stream) (format stream "~S is not a large number." (not-a-large-number-argument condition)))))

(defun assure-large-number (n) (loop (when (and (numberp n) (> n 73)) (return n)) (cerror "Enter a number~3*~:[~; a bit larger than ~*~D~]." 'not-a-large-number :argument n :ignore (numberp n) :ignore n :allow-other-keys t) (format t "~&Type a large number: ") ([[CL:Macros:defparameter]] n (read)) (fresh-line)))


(assure-large-number 'a)
▷ Correctable error in ASSURE-LARGE-NUMBER: A is not a large number.
▷ Restart options:
▷ 1: Enter a number.
▷ 2: Top level.
▷ Debug> \IN{:continue 1}
▷ Type a large number: \IN{88} → 88

(assure-large-number 37)
▷ Correctable error in ASSURE-LARGE-NUMBER: A is not a large number.
▷ Restart options:
▷ 1: Enter a number a bit larger than 37.
▷ 2: Top level.
▷ Debug> \IN{:continue 1}
▷ Type a large number: \IN{259} → 259 </blockquote>

====Affected By====

**[[CL:Variables:*break-on-signals*]]**.

Existing handler bindings.

====Exceptional Situations====

None.

====See Also====

**[[CL:Functions:error]]**, **[[CL:Functions:format]]**, **[[CL:Macros:handler-bind]]**, **[[CL:Variables:*break-on-signals*]]**, **[[CL:Types:simple-type-error]]**

====Notes====

If //datum// is a //[[CL:Glossary:condition]]// //[[CL:Glossary:type]]// rather than a //[[CL:Glossary:string]]//, the **[[CL:Functions:format]]** directive **[[CL:Functions:~*]]** may be especially useful in the //continue-format-control// in order to ignore the //[[CL:Glossary:keywords]]// in the //[[CL:Glossary:initialization argument list]]//. For example:

<blockquote> (cerror "enter a new value to replace ~*~s" 'not-a-number :argument a) </blockquote>


\issue{FORMAT-STRING-ARGUMENTS:SPECIFY}
