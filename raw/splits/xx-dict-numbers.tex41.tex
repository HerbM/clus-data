====== Function RANDOM ======

====Syntax====

**random {limit** //\opt} random-state// → //random-number//

====Arguments and Values====

//limit// - a positive //[[CL:Glossary:integer]]//, or a positive //[[CL:Glossary:float]]//.

//random-state// - a //[[CL:Glossary:random state]]//. The default is the //[[CL:Glossary:current random state]]//.

//random-number// - a non-negative //[[CL:Glossary:number]]// less than //limit// and of the same //[[CL:Glossary:type]]// as //limit//.

====Description====

Returns a pseudo-random number that is a non-negative //[[CL:Glossary:number]]// less than //limit// and of the same //[[CL:Glossary:type]]// as //limit//.

The //random-state//, which is modified by this function, encodes the internal state maintained by the random number generator.

An approximately uniform choice distribution is used. If //limit// is an //[[CL:Glossary:integer]]//, each of the possible results occurs with (approximate) probability 1///limit//.

====Examples====

<blockquote> (<= 0 (random 1000) 1000) → //[[CL:Glossary:true]]// (let ((state1 (make-random-state)) (state2 (make-random-state))) (= (random 1000 state1) (random 1000 state2))) → //[[CL:Glossary:true]]// </blockquote>

====Side Effects====

The //random-state// is modified.

====Affected By====

None.

====Exceptional Situations====

Should signal an error of type type-error if //limit// is not a positive //[[CL:Glossary:integer]]// or a positive //[[CL:Glossary:real]]//.

====See Also====

**[[CL:Functions:make-random-state]]**, **[[CL:Variables:*random-state*]]**

====Notes====

See \CLtL\ for information about generating random numbers.

