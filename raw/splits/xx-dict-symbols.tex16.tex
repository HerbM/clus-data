====== Function REMPROP ======

====Syntax====

**remprop** //symbol indicator// → //generalized-boolean//

====Arguments and Values====

//symbol// - a //[[CL:Glossary:symbol]]//.

//indicator// - an //[[CL:Glossary:object]]//.

//generalized-boolean// - a //[[CL:Glossary:generalized boolean]]//.

====Description====

**[[CL:Functions:remprop]]** removes from the //[[CL:Glossary:property list]]// of //symbol// a //[[CL:Glossary:property]]// with a //[[CL:Glossary:property indicator]]//

//[[CL:Glossary:identical]]// to //indicator//.

If there are multiple //[[CL:Glossary:properties]]// with the //[[CL:Glossary:identical]]// key, **[[CL:Macros:remprop]]** only removes the first such //[[CL:Glossary:property]]//.

**[[CL:Macros:remprop]]** returns //[[CL:Glossary:false]]// if no such //[[CL:Glossary:property]]// was found, or //[[CL:Glossary:true]]// if a property was found.

The //[[CL:Glossary:property indicator]]// and the corresponding //[[CL:Glossary:property value]]// are removed in an undefined order by destructively splicing the property list.

The permissible side-effects correspond to those permitted for **[[CL:Macros:remf]]**, such that:

<blockquote> (remprop ''x'' ''y'') ≡ (remf (symbol-plist ''x'') ''y'') </blockquote>

====Examples====

<blockquote> ([[CL:Macros:defparameter]] test (make-symbol "PSEUDO-PI")) → #:PSEUDO-PI (symbol-plist test) → () ([[CL:Macros:setf]] (get test 'constant) t) → T ([[CL:Macros:setf]] (get test 'approximation) 3.14) → 3.14 ([[CL:Macros:setf]] (get test 'error-range) 'noticeable) → NOTICEABLE (symbol-plist test) → (ERROR-RANGE NOTICEABLE APPROXIMATION 3.14 CONSTANT T) ([[CL:Macros:setf]] (get test 'approximation) nil) → NIL (symbol-plist test) → (ERROR-RANGE NOTICEABLE APPROXIMATION NIL CONSTANT T) (get test 'approximation) → NIL (remprop test 'approximation) → //[[CL:Glossary:true]]// (get test 'approximation) → NIL (symbol-plist test) → (ERROR-RANGE NOTICEABLE CONSTANT T) (remprop test 'approximation) → NIL (symbol-plist test) → (ERROR-RANGE NOTICEABLE CONSTANT T) (remprop test 'error-range) → //[[CL:Glossary:true]]// ([[CL:Macros:setf]] (get test 'approximation) 3) → 3 (symbol-plist test) → (APPROXIMATION 3 CONSTANT T) </blockquote>

====Side Effects====

The //[[CL:Glossary:property list]]// of //symbol// is modified.

====Affected By====

None.

====Exceptional Situations====

Should signal an error of type type-error if //symbol// is not a //[[CL:Glossary:symbol]]//.

====See Also====

**[[CL:Macros:remf]]**, **[[CL:Functions:symbol-plist]]**

====Notes====

//[[CL:Glossary:Numbers]]// and //[[CL:Glossary:characters]]// are not recommended for use as //indicators// in portable code since **[[CL:Functions:remprop]]** tests with **[[CL:Functions:eq]]** rather than **[[CL:Functions:eql]]**, and consequently the effect of using such //indicators// is //[[CL:Glossary:implementation-dependent]]//. Of course, if you've gotten as far as needing to remove such a //[[CL:Glossary:property]]//, you don't have much choice---the time to have been thinking about this was when you used **[[CL:Macros:setf]]** of **[[CL:Macros:get]]** to establish the //[[CL:Glossary:property]]//.

\issue{PLIST-DUPLICATES:ALLOW} \issue{REMF-DESTRUCTION-UNSPECIFIED:X3J13-MAR-89}
