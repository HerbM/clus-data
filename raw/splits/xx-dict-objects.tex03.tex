====== Standard Generic Function ALLOCATE-INSTANCE ======


====Syntax====

\DefgenWithValues allocate-instance {class ''&rest'' initargs ''&key'' {\allowotherkeys}} {new-instance}

====Method Signatures====

\Defmeth {allocate-instance} {\specparam{class}{standard-class} ''&rest'' initargs}

\Defmeth {allocate-instance} {\specparam{class}{structure-class} ''&rest'' initargs}

====Arguments and Values====

//class// - a //[[CL:Glossary:class]]//.

//initargs// - a //[[CL:Glossary:list]]// of //[[CL:Glossary:keyword/value pairs]]// (initialization argument //[[CL:Glossary:names]]// and //[[CL:Glossary:values]]//).

//new-instance// - an //[[CL:Glossary:object]]// whose //[[CL:Glossary:class]]// is //class//.

====Description====

The generic function **[[CL:Functions:allocate-instance]]** creates and returns a new instance of the //class//, without initializing it. When the //class// is a //[[CL:Glossary:standard class]]//, this means that the //[[CL:Glossary:slots]]// are //[[CL:Glossary:unbound]]//; when the //[[CL:Glossary:class]]// is a //[[CL:Glossary:structure class]]//, this means the //[[CL:Glossary:slots]]//' //[[CL:Glossary:values]]// are unspecified.

The caller of **[[CL:Functions:allocate-instance]]** is expected to have already checked the initialization arguments.

The //[[CL:Glossary:generic function]]// **[[CL:Functions:allocate-instance]]** is called by **[[CL:Functions:make-instance]]**, as described in \secref\ObjectCreationAndInit.

====Affected By====

None.

====Exceptional Situations====

None.

====See Also====

**[[CL:Macros:defclass]]**, **[[CL:Functions:make-instance]]**, **[[CL:Functions:class-of]]**, {\secref\ObjectCreationAndInit}

====Notes====

The consequences of adding //[[CL:Glossary:methods]]// to **[[CL:Functions:allocate-instance]]** is unspecified. This capability might be added by the //[[CL:Glossary:Metaobject Protocol]]//.

\issue{ALLOCATE-INSTANCE:ADD} \issue{INITIALIZATION-FUNCTION-KEYWORD-CHECKING}
