====== Macro DEFCLASS ======

====Syntax====


\DefmacWithValuesNewline defclass {//class-name// \paren{\star{{//superclass-name//}}} \paren{\star{\curly{''slot-specifier''}}} ''\lbrack\!\lbrack\downarrow\!''class-option''\,\rbrack\!\rbrack''} {new-class}

\settabs\+\hskip≤ftskip&\cr \+&\cleartabs''slot-specifier''::''='' &''slot-name'' ''\vert'' (''slot-name'' ''\lbrack\!\lbrack\downarrow\!''slot-option''\,\rbrack\!\rbrack'')\cr \Vskip 1pc! \+&''slot-name''::''='' //[[CL:Glossary:symbol]]//\cr \Vskip 1pc! \+&\cleartabs''slot-option''::''='' &\star{\curly{**'':reader''** //reader-function-name//}} ''\vert'' \cr \+&&\star{\curly{**'':writer''** //writer-function-name//}} ''\vert'' \cr \+&&\star{\curly{**'':accessor''** //reader-function-name//}} ''\vert'' \cr \+&&\curly{**'':allocation''** //allocation-type//} ''\vert'' \cr \+&&\star{\curly{**'':initarg''** //initarg-name//}} ''\vert'' \cr \+&&\curly{**'':initform''** //form//} ''\vert'' \cr \+&&\curly{**'':type''** //type-specifier//} ''\vert'' \cr \+&&\curly{**'':documentation''** //[[CL:Glossary:string]]//} \cr \Vskip 1pc! \+&''function-name''::''='' \curly{//[[CL:Glossary:symbol]]// ''\vert'' **[[CL:Functions:(setf //[[CL:Glossary:symbol]]**)]]//}\cr \Vskip 1pc! \+&\cleartabs//class-option//::''='' &(**'':default-initargs''** ''.'' //initarg-list//) ''\vert'' \cr \+&&(**'':documentation''** //[[CL:Glossary:string]]//) ''\vert'' \cr \+&&(**'':metaclass''** //class-name//) \cr \Vskip 1pc!

====Arguments and Values====


//Class-name// - a //[[CL:Glossary:non-nil]]// //[[CL:Glossary:symbol]]//.

//Superclass-name//--a //[[CL:Glossary:non-nil]]// //[[CL:Glossary:symbol]]//.

//Slot-name//--a //[[CL:Glossary:symbol]]//. The //slot-name// argument is

a //[[CL:Glossary:symbol]]// that is syntactically valid for use as a variable name.

//Reader-function-name// - a //[[CL:Glossary:non-nil]]// //[[CL:Glossary:symbol]]//. **'':reader''** can be supplied more than once for a given //[[CL:Glossary:slot]]//.

//Writer-function-name// - a //[[CL:Glossary:generic function]]// name. **'':writer''** can be supplied more than once for a given //[[CL:Glossary:slot]]//.

//Reader-function-name// - a //[[CL:Glossary:non-nil]]// //[[CL:Glossary:symbol]]//. **'':accessor''** can be supplied more than once for a given //[[CL:Glossary:slot]]//.

//Allocation-type// - (member **'':instance''** **'':class''**). **'':allocation''** can be supplied once at most for a given //[[CL:Glossary:slot]]//.

//Initarg-name// - a //[[CL:Glossary:symbol]]//. **'':initarg''** can be supplied more than once for a given //[[CL:Glossary:slot]]//.

//Form// - a //[[CL:Glossary:form]]//. **'':init-form''** can be supplied once at most for a given //[[CL:Glossary:slot]]//.

//Type-specifier// - a //[[CL:Glossary:type specifier]]//. **'':type''** can be supplied once at most for a given //[[CL:Glossary:slot]]//.

//Class-option// - refers to the //[[CL:Glossary:class]]// as a whole or to all class //[[CL:Glossary:slots]]//.

//Initarg-list// - a //[[CL:Glossary:list]]// of alternating initialization argument //[[CL:Glossary:names]]// and default initial value //[[CL:Glossary:forms]]//. **'':default-initargs''** can be supplied at most once.

//Class-name// - a //[[CL:Glossary:non-nil]]// //[[CL:Glossary:symbol]]//. **'':metaclass''** can be supplied once at most.















//new-class// - the new //[[CL:Glossary:class]]// //[[CL:Glossary:object]]//.

====Description====

The macro **[[CL:Macros:defclass]]** defines a new named //[[CL:Glossary:class]]//. It returns the new //[[CL:Glossary:class]]// //[[CL:Glossary:object]]// as its result.

The syntax of **[[CL:Macros:defclass]]** provides options for specifying initialization arguments for //[[CL:Glossary:slots]]//, for specifying default initialization values for //[[CL:Glossary:slots]]//, and for requesting that //[[CL:Glossary:methods]]// on specified //[[CL:Glossary:generic functions]]// be automatically generated for reading and writing the values of //[[CL:Glossary:slots]]//. No reader or writer functions are defined by default; their generation must be explicitly requested. However, //[[CL:Glossary:slots]]// can always be //[[CL:Glossary:accessed]]// using **[[CL:Functions:slot-value]]**.

Defining a new //[[CL:Glossary:class]]// also causes a //[[CL:Glossary:type]]// of the same name to be defined. The predicate ''(typep //object// //class-name//)'' returns true if the //[[CL:Glossary:class]]// of the given //object// is the //[[CL:Glossary:class]]// named by //class-name// itself or a subclass of the class //class-name//. A //[[CL:Glossary:class]]// //[[CL:Glossary:object]]// can be used as a //[[CL:Glossary:type specifier]]//. Thus ''(typep //object// //class//)'' returns //[[CL:Glossary:true]]// if the //[[CL:Glossary:class]]// of the //object// is //class// itself or a subclass of //class//.

The //class-name// argument specifies the //[[CL:Glossary:proper name]]// of the new //[[CL:Glossary:class]]//.

If a //[[CL:Glossary:class]]// with the same //[[CL:Glossary:proper name]]// already exists and that //[[CL:Glossary:class]]// is an //[[CL:Glossary:instance]]// of **[[CL:Types:standard-class]]**, and if the **[[CL:Macros:defclass]]** form for the definition of the new //[[CL:Glossary:class]]// specifies a //[[CL:Glossary:class]]// of //[[CL:Glossary:class]]// **[[CL:Types:standard-class]]**,the existing //[[CL:Glossary:class]]// is redefined, and instances of it (and its //[[CL:Glossary:subclasses]]//) are updated to the new definition at the time that they are next //[[CL:Glossary:accessed]]//. For details, see section {\secref\ClassReDef}.


Each //superclass-name// argument specifies a direct //[[CL:Glossary:superclass]]// of the new //[[CL:Glossary:class]]//.

If the //[[CL:Glossary:superclass]]// list is empty, then the //[[CL:Glossary:superclass]]// defaults depending on the //[[CL:Glossary:metaclass]]//, with **[[CL:Types:standard-object]]** being the default for **[[CL:Types:standard-class]]**.

The new //[[CL:Glossary:class]]// will inherit //[[CL:Glossary:slots]]// and //[[CL:Glossary:methods]]// from each of its direct //[[CL:Glossary:superclasses]]//, from their direct //[[CL:Glossary:superclasses]]//, and so on. For a discussion of how //[[CL:Glossary:slots]]// and //[[CL:Glossary:methods]]// are inherited, see section {\secref\Inheritance}.



The following slot options are available:

\beginlist

\itemitem{\bull} The **'':reader''** slot option specifies that an //[[CL:Glossary:unqualified method]]// is to be defined on the //[[CL:Glossary:generic function]]// named //reader-function-name// to read the value of the given //[[CL:Glossary:slot]]//.

\itemitem{\bull} The **'':writer''** slot option specifies that an //[[CL:Glossary:unqualified method]]// is to be defined on the //[[CL:Glossary:generic function]]// named //writer-function-name// to write the value of the //[[CL:Glossary:slot]]//.

\itemitem{\bull} The **'':accessor''** slot option specifies that an //[[CL:Glossary:unqualified method]]// is to be defined on the generic function named //reader-function-name// to read the value of the given //[[CL:Glossary:slot]]// and that an //[[CL:Glossary:unqualified method]]// is to be defined on the //[[CL:Glossary:generic function]]// named ''(setf //reader-function-name//)'' to be used with **[[CL:Macros:setf]]** to modify the value of the //[[CL:Glossary:slot]]//.

\itemitem{\bull} The **'':allocation''** slot option is used to specify where storage is to be allocated for the given //[[CL:Glossary:slot]]//. Storage for a //[[CL:Glossary:slot]]// can be located in each instance or in the //[[CL:Glossary:class]]// //[[CL:Glossary:object]]// itself. The value of the //allocation-type// argument can be either the keyword **'':instance''** or the keyword **'':class''**. If the **'':allocation''** slot option is not specified, the effect is the same as specifying '':allocation :instance''. \beginlist \itemitem{--} If //allocation-type// is **'':instance''**, a //[[CL:Glossary:local slot]]// of the name //slot-name// is allocated in each instance of the //[[CL:Glossary:class]]//.

\itemitem{--} If //allocation-type// is **'':class''**, a shared //[[CL:Glossary:slot]]// of the given name is allocated in the //[[CL:Glossary:class]]// //[[CL:Glossary:object]]// created by this **[[CL:Macros:defclass]]** form. The value of the //[[CL:Glossary:slot]]// is shared by all //[[CL:Glossary:instances]]// of the //[[CL:Glossary:class]]//. If a class ''C\sub1'' defines such a //[[CL:Glossary:shared slot]]//, any subclass ''C\sub2'' of ''C\sub1'' will share this single //[[CL:Glossary:slot]]// unless the **[[CL:Macros:defclass]]** form for ''C\sub2'' specifies a //[[CL:Glossary:slot]]// of the same //[[CL:Glossary:name]]// or there is a superclass of ''C\sub2'' that precedes ''C\sub1'' in the class precedence list of ''C\sub2'' and that defines a //[[CL:Glossary:slot]]// of the same //[[CL:Glossary:name]]//. \endlist \itemitem{\bull} The **'':initform''** slot option is used to provide a default initial value form to be used in the initialization of the //[[CL:Glossary:slot]]//. This //[[CL:Glossary:form]]// is evaluated every time it is used to initialize the //[[CL:Glossary:slot]]//. The lexical environment in which this //[[CL:Glossary:form]]// is evaluated is the lexical environment in which the **[[CL:Macros:defclass]]** form was evaluated. Note that the lexical environment refers both to variables and to functions. For //[[CL:Glossary:local slots]]//, the dynamic environment is the dynamic environment in which **[[CL:Functions:make-instance]]** is called; for shared //[[CL:Glossary:slots]]//, the dynamic environment is the dynamic environment in which the **[[CL:Macros:defclass]]** form was evaluated. see section {\secref\ObjectCreationAndInit}.

No implementation is permitted to extend the syntax of **[[CL:Macros:defclass]]** to allow ''(//slot-name// //form//)'' as an abbreviation for ''(//slot-name// :initform //form//)''. \reviewer{Barmar: Can you extend this to mean something else?}

\itemitem{\bull} The **'':initarg''** slot option declares an initialization argument named //initarg-name// and specifies that this initialization argument initializes the given //[[CL:Glossary:slot]]//. If the initialization argument has a value in the call to **[[CL:Functions:initialize-instance]]**, the value will be stored into the given //[[CL:Glossary:slot]]//, and the slot's **'':initform''** slot option, if any, is not evaluated. If none of the initialization arguments specified for a given //[[CL:Glossary:slot]]// has a value, the //[[CL:Glossary:slot]]// is initialized according to the **'':initform''** slot option, if specified.

\itemitem{\bull} The **'':type''** slot option specifies that the contents of the //[[CL:Glossary:slot]]// will always be of the specified data type. It effectively declares the result type of the reader generic function when applied to an //[[CL:Glossary:object]]// of this //[[CL:Glossary:class]]//. The consequences of attempting to store in a //[[CL:Glossary:slot]]// a value that does not satisfy the type of the //[[CL:Glossary:slot]]// are undefined. The **'':type''** slot option is further discussed in \secref\SlotInheritance.

\itemitem{\bull} The **'':documentation''** slot option provides a //[[CL:Glossary:documentation string]]// for the //[[CL:Glossary:slot]]//. **'':documentation''** can be supplied once at most for a given //[[CL:Glossary:slot]]//. \reviewer{Barmar: How is this retrieved?} \endlist

Each class option is an option that refers to the //[[CL:Glossary:class]]// as a whole.

The following class options are available:

\beginlist \itemitem{\bull} The **'':default-initargs''** class option is followed by a list of alternating initialization argument //[[CL:Glossary:names]]// and default initial value forms. If any of these initialization arguments does not appear in the initialization argument list supplied to **[[CL:Functions:make-instance]]**, the corresponding default initial value form is evaluated, and the initialization argument //[[CL:Glossary:name]]// and the //[[CL:Glossary:form]]//'s value are added to the end of the initialization argument list before the instance is created; see section {\secref\ObjectCreationAndInit}. The default initial value form is evaluated each time it is used. The lexical environment in which this //[[CL:Glossary:form]]// is evaluated is the lexical environment in which the **[[CL:Macros:defclass]]** form was evaluated. The dynamic environment is the dynamic environment in which **[[CL:Functions:make-instance]]** was called. If an initialization argument //[[CL:Glossary:name]]// appears more than once in a **'':default-initargs''** class option, an error is signaled.


\itemitem{\bull}

The **'':documentation''** class option causes a //[[CL:Glossary:documentation string]]// to be attached with the //[[CL:Glossary:class]]// //[[CL:Glossary:object]]//, and attached with kind \misc{type} to the //class-name//. **'':documentation''** can be supplied once at most.


\itemitem{\bull} The **'':metaclass''** class option is used to specify that instances of the //[[CL:Glossary:class]]// being defined are to have a different metaclass than the default provided by the system (\theclass{standard-class}).

\endlist

Note the following rules of **[[CL:Macros:defclass]]** for //[[CL:Glossary:standard classes]]//:

\beginlist

\itemitem{\bull} It is not required that the //[[CL:Glossary:superclasses]]// of a //[[CL:Glossary:class]]// be defined before the **[[CL:Macros:defclass]]** form for that //[[CL:Glossary:class]]// is evaluated.

\itemitem{\bull} All the //[[CL:Glossary:superclasses]]// of a //[[CL:Glossary:class]]// must be defined before an //[[CL:Glossary:instance]]// of the //[[CL:Glossary:class]]// can be made.

\itemitem{\bull} A //[[CL:Glossary:class]]// must be defined before it can be used as a parameter specializer in a **[[CL:Macros:defmethod]]** form.

\endlist

The \OS\ can be extended to cover situations where these rules are not obeyed.

Some slot options are inherited by a //[[CL:Glossary:class]]// from its //[[CL:Glossary:superclasses]]//, and some can be shadowed or altered by providing a local slot description. No class options except **'':default-initargs''** are inherited. For a detailed description of how //[[CL:Glossary:slots]]// and slot options are inherited, see section {\secref\SlotInheritance}.

The options to **[[CL:Macros:defclass]]** can be extended. It is required that all implementations signal an error if they observe a class option or a slot option that is not implemented locally.

It is valid to specify more than one reader, writer, accessor, or initialization argument for a //[[CL:Glossary:slot]]//. No other slot option can appear more than once in a single slot description, or an error is signaled.

If no reader, writer, or accessor is specified for a //[[CL:Glossary:slot]]//, the //[[CL:Glossary:slot]]// can only be //[[CL:Glossary:accessed]]// by the function **[[CL:Functions:slot-value]]**.














If a **[[CL:Macros:defclass]]** //[[CL:Glossary:form]]// appears as a //[[CL:Glossary:top level form]]//, the //[[CL:Glossary:compiler]]// must make the //[[CL:Glossary:class]]// //[[CL:Glossary:name]]// be recognized as a valid //[[CL:Glossary:type]]// //[[CL:Glossary:name]]// in subsequent declarations (as for **[[CL:Macros:deftype]]**) and be recognized as a valid //[[CL:Glossary:class]]// //[[CL:Glossary:name]]// for **[[CL:Macros:defmethod]]** //[[CL:Glossary:parameter specializers]]// and for use as the **'':metaclass''** option of a subsequent **[[CL:Macros:defclass]]**. The //[[CL:Glossary:compiler]]// must make

the //[[CL:Glossary:class]]// definition available to be returned by **[[CL:Functions:find-class]]** when its //environment// //[[CL:Glossary:argument]]// is a value received as the //[[CL:Glossary:environment parameter]]// of a //[[CL:Glossary:macro]]//.

====Examples====

None.

====Affected By====

None.

====Exceptional Situations====

If there are any duplicate slot names, an error of type **[[CL:Types:program-error]]** is signaled.

If an initialization argument //[[CL:Glossary:name]]// appears more than once in **'':default-initargs''** class option, an error of type **[[CL:Types:program-error]]** is signaled.

If any of the following slot options appears more than once in a single slot description, an error of type **[[CL:Types:program-error]]** is signaled: **'':allocation''**, **'':initform''**, **'':type''**, **'':documentation''**.

It is required that all implementations signal an error of type **[[CL:Types:program-error]]** if they observe a class option or a slot option that is not implemented locally.



====See Also====

**[[CL:Functions:documentation]]**, **[[CL:Functions:initialize-instance]]**, **[[CL:Functions:make-instance]]**, **[[CL:Functions:slot-value]]**, {\secref\Classes}, {\secref\Inheritance}, {\secref\ClassReDef}, {\secref\DeterminingtheCPL}, {\secref\ObjectCreationAndInit}

====Notes====

None.

\issue{DOCUMENTATION-FUNCTION-BUGS:FIX} \issue{COMPILE-FILE-HANDLING-OF-TOP-LEVEL-FORMS:CLARIFY}
