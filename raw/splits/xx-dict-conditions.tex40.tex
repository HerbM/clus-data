====== Macro WITH-SIMPLE-RESTART ======

====Syntax====

\DefmacWithValuesNewline with-simple-restart {\paren{name format-control \starparam{format-argument}} \starparam{form}} {\starparam{result}}

====Arguments and Values====

//name// - a //[[CL:Glossary:symbol]]//.

//format-control// - a //[[CL:Glossary:format control]]//.

//format-argument// - an //[[CL:Glossary:object]]// (i.e. a //[[CL:Glossary:format argument]]//).

//forms// - an //[[CL:Glossary:implicit progn]]//.

//results// - in the normal situation, the //[[CL:Glossary:values]]// returned by the //forms//; in the exceptional situation where the //[[CL:Glossary:restart]]// named //name// is invoked, two values---**[[CL:Constant Variables:nil]]** and \t.

====Description====

**[[CL:Macros:with-simple-restart]]** establishes a restart.

If the restart designated by //name// is not invoked while executing //forms//, all values returned by the last of //forms// are returned. If the restart designated by //name// is invoked, control is transferred to **[[CL:Macros:with-simple-restart]]**, which returns two values, **[[CL:Constant Variables:nil]]** and \t.

If //name// is **[[CL:Constant Variables:nil]]**, an anonymous restart is established.

The //format-control// and //format-arguments// are used report the //[[CL:Glossary:restart]]//.

====Examples====

<blockquote> (defun read-eval-print-loop (level) (with-simple-restart (abort "Exit command level ~D." level) (loop (with-simple-restart (abort "Return to command level ~D." level) (let ((form (prog2 (fresh-line) (read) (fresh-line)))) (prin1 (eval form))))))) → READ-EVAL-PRINT-LOOP (read-eval-print-loop 1) (+ 'a 3)
▷ Error: The argument, A, to the function + was of the wrong type.
▷ The function expected a number.
▷ To continue, type :CONTINUE followed by an option number:
▷ 1: Specify a value to use this time.
▷ 2: Return to command level 1.
▷ 3: Exit command level 1.
▷ 4: Return to Lisp Toplevel. </blockquote>

<blockquote> (defun compute-fixnum-power-of-2 (x) (with-simple-restart (nil "Give up on computing 2{\hat}~D." x) (let ((result 1)) (dotimes (i x result) ([[CL:Macros:defparameter]] result (* 2 result)) (unless (fixnump result) (error "Power of 2 is too large.")))))) COMPUTE-FIXNUM-POWER-OF-2 (defun compute-power-of-2 (x) (or (compute-fixnum-power-of-2 x) 'something big)) COMPUTE-POWER-OF-2 (compute-power-of-2 10) 1024 (compute-power-of-2 10000)
▷ Error: Power of 2 is too large.
▷ To continue, type :CONTINUE followed by an option number.
▷ 1: Give up on computing 2{\hat}10000.
▷ 2: Return to Lisp Toplevel
▷ Debug> \IN{:continue 1} → SOMETHING-BIG </blockquote>

====Side Effects====

None.

====Affected By====

None.

====Exceptional Situations====

None.

====See Also====

**[[CL:Macros:restart-case]]**

====Notes====

**[[CL:Macros:with-simple-restart]]** is shorthand for one of the most common uses of **[[CL:Macros:restart-case]]**.

**[[CL:Macros:with-simple-restart]]** could be defined by:

<blockquote> (defmacro with-simple-restart ((restart-name format-control &rest format-arguments) &body forms) `(restart-case (progn ,@forms) (,restart-name () :report (lambda (stream) (format stream ,format-control ,@format-arguments)) (values nil t)))) </blockquote>

Because the second return value is \t\ in the exceptional case, it is common (but not required) to arrange for the second return value in the normal case to be missing or **[[CL:Constant Variables:nil]]** so that the two situations can be distinguished.

\issue{FORMAT-STRING-ARGUMENTS:SPECIFY} \issue{FORMAT-STRING-ARGUMENTS:SPECIFY}
