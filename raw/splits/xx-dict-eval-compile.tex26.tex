====== Declaration OPTIMIZE ======

====Syntax====

\f{(optimize \star{{//quality// | (//quality// //value//)}})} \idxref{compilation-speed} \idxref{debug} \idxref{safety} \idxref{space} \idxref{speed}

====Arguments====

//quality// - an //[[CL:Glossary:optimize quality]]//.

//value// - one of the //[[CL:Glossary:integers]]// ''0'', ''1'', ''2'', or ''3''.

====Valid Context====

//[[CL:Glossary:declaration]]// or //[[CL:Glossary:proclamation]]//

====Binding Types Affected====

None.

====Description====

Advises the compiler that each //quality// should be given attention according to the specified corresponding //value//. Each //quality// must be a //[[CL:Glossary:symbol]]// naming an //[[CL:Glossary:optimize quality]]//; the names and meanings of the standard //optimize qualities// are shown in \thenextfigure.

\tablefigtwo{Optimize qualities}{Name}{Meaning}{ \declref{compilation-speed} & speed of the compilation process \cr \declref{debug} & ease of debugging \cr \declref{safety} & run-time error checking \cr \declref{space} & both code size and run-time space \cr \declref{speed} & speed of the object code \cr }

There may be other, //[[CL:Glossary:implementation-defined]]// //[[CL:Glossary:optimize qualities]]//.

A //value// ''0'' means that the corresponding //quality// is totally unimportant, and ''3'' that the //quality// is extremely important; ''1'' and ''2'' are intermediate values, with ''1'' the

neutral value. ''(//quality// 3)'' can be abbreviated to //quality//.

Note that //[[CL:Glossary:code]]// which has the optimization ''(safety 3)'', or just **[[CL:Types:safety]]**,is called //[[CL:Glossary:safe]]// //[[CL:Glossary:code]]//.

The consequences are unspecified if a //quality// appears more than once with //[[CL:Glossary:different]]// //values//.

====Examples====

<blockquote> (defun often-used-subroutine (x y) (declare (optimize (safety 2))) (error-check x y) (hairy-setup x) (do ((i 0 (+ i 1)) (z x (cdr z))) ((null z)) ;; This inner loop really needs to burn. (declare (optimize speed)) (declare (fixnum i)) )) </blockquote>

====See Also====

\misc{declare}, **[[CL:Macros:declaim]]**, **[[CL:Functions:proclaim]]**, {\secref\DeclScope}

====Notes====

An \declref{optimize} declaration never applies to either a //[[CL:Glossary:variable]]// or a //[[CL:Glossary:function]]// //[[CL:Glossary:binding]]//. An \declref{optimize} declaration can only be a //[[CL:Glossary:free declaration]]//. For more information, see section {\secref\DeclScope}.

\issue{OPTIMIZE-DEBUG-INFO:NEW-QUALITY}
