====== Function FLOAT ======

====Syntax====

**float {number** //\opt} prototype// → //float//

====Arguments and Values====

//number// - a //[[CL:Glossary:real]]//.

//prototype// - a //[[CL:Glossary:float]]//.

//float// - a //[[CL:Glossary:float]]//.

====Description====

**[[CL:Functions:float]]** converts a

//[[CL:Glossary:real]]//

number to a //[[CL:Glossary:float]]//.

If a //prototype// is supplied, a //[[CL:Glossary:float]]// is returned that is mathematically equal to //number// but has the same format as //prototype//.

If //prototype// is not supplied, then if the //number// is already a //[[CL:Glossary:float]]//, it is returned; otherwise, a //[[CL:Glossary:float]]// is returned that is mathematically equal to //number// but is a //[[CL:Glossary:single float]]//.

====Examples====

<blockquote> (float 0) → 0.0 (float 1 .5) → 1.0 (float 1.0) → 1.0 (float 1/2) → 0.5 → 1.0d0 //or// → 1.0 (eql (float 1.0 1.0d0) 1.0d0) → //[[CL:Glossary:true]]// </blockquote>

====Side Effects====

None.

====Affected By====

None.

====Exceptional Situations====

None.

====See Also====

**[[CL:Functions:coerce]]**

====Notes====

None.

\issue{REAL-NUMBER-TYPE:X3J13-MAR-89}
