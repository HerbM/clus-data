====== Variable *BREAK-ON-SIGNALS* ======

====Value Type====

a //[[CL:Glossary:type specifier]]//.

====Initial Value====

**[[CL:Constant Variables:nil]]**.

====Description====

When ''(typep ''condition'' *break-on-signals*)'' returns //[[CL:Glossary:true]]//, calls to **[[CL:Functions:signal]]**, and to other //[[CL:Glossary:operators]]// such as **[[CL:Functions:error]]** that implicitly call **[[CL:Functions:signal]]**, enter the debugger prior to //[[CL:Glossary:signaling]]// the //[[CL:Glossary:condition]]//.

\Therestart{continue} can be used to continue with the normal //[[CL:Glossary:signaling]]// process when a break occurs process due to **[[CL:Variables:*break-on-signals*]]**.

====Examples====

<blockquote> *break-on-signals* → NIL (ignore-errors (error 'simple-error :format-control "Fooey!")) → NIL, #<SIMPLE-ERROR 32207172>

(let ((*break-on-signals* 'error)) (ignore-errors (error 'simple-error :format-control "Fooey!")))
▷ Break: Fooey!
▷ BREAK entered because of *BREAK-ON-SIGNALS*.
▷ To continue, type :CONTINUE followed by an option number:
▷ 1: Continue to signal.
▷ 2: Top level.
▷ Debug> \IN{:CONTINUE 1}
▷ Continue to signal. → NIL, #<SIMPLE-ERROR 32212257>

(let ((*break-on-signals* 'error)) (error 'simple-error :format-control "Fooey!"))
▷ Break: Fooey!
▷ BREAK entered because of *BREAK-ON-SIGNALS*.
▷ To continue, type :CONTINUE followed by an option number:
▷ 1: Continue to signal.
▷ 2: Top level.
▷ Debug> \IN{:CONTINUE 1}
▷ Continue to signal.
▷ Error: Fooey!
▷ To continue, type :CONTINUE followed by an option number:
▷ 1: Top level.
▷ Debug> \IN{:CONTINUE 1}
▷ Top level. </blockquote>

====Affected By====

None.

====See Also====

**[[CL:Functions:break]]**, **[[CL:Functions:signal]]**, **[[CL:Functions:warn]]**, **[[CL:Functions:error]]**, **[[CL:Functions:typep]]**, {\secref\ConditionSystemConcepts}

====Notes====

**[[CL:Variables:*break-on-signals*]]** is intended primarily for use in debugging code that does signaling. When setting **[[CL:Variables:*break-on-signals*]]**, the user is encouraged to choose the most restrictive specification that suffices. Setting **[[CL:Variables:*break-on-signals*]]** effectively violates the modular handling of //[[CL:Glossary:condition]]// signaling. In practice, the complete effect of setting **[[CL:Variables:*break-on-signals*]]** might be unpredictable in some cases since the user might not be aware of the variety or number of calls to **[[CL:Functions:signal]]** that are used in code called only incidentally.

**[[CL:Variables:*break-on-signals*]]** enables an early entry to the debugger but such an entry does not preclude an additional entry to the debugger in the case of operations such as **[[CL:Functions:error]]** and **[[CL:Functions:cerror]]**.

\issue{FORMAT-STRING-ARGUMENTS:SPECIFY} \issue{BREAK-ON-WARNINGS-OBSOLETE:REMOVE}
