====== Function SET-SYNTAX-FROM-CHAR ======

====Syntax====

\DefunWithValues set-syntax-from-char {to-char from-char ''&optional'' to-readtable from-readtable} {\t}

====Arguments and Values====

//to-char// - a //[[CL:Glossary:character]]//.

//from-char// - a //[[CL:Glossary:character]]//.

//to-readtable// - a //[[CL:Glossary:readtable]]//. The default is the //[[CL:Glossary:current readtable]]//.

//from-readtable// - a //[[CL:Glossary:readtable designator]]//. The default is the //[[CL:Glossary:standard readtable]]//.

====Description====

**[[CL:Functions:set-syntax-from-char]]** makes the syntax of //to-char// in //to-readtable// be the same as the syntax of //from-char// in //from-readtable//.

**[[CL:Functions:set-syntax-from-char]]** copies the //[[CL:Glossary:syntax types]]// of //from-char//. If //from-char// is a //[[CL:Glossary:macro character]]//, its //[[CL:Glossary:reader macro function]]// is copied also. If the character is a //[[CL:Glossary:dispatching macro character]]//, its entire dispatch table of //[[CL:Glossary:reader macro functions]]// is copied. The //[[CL:Glossary:constituent traits]]// of //from-char// are not copied.

A macro definition from a character such as ''"'' can be copied to another character; the standard definition for ''"'' looks for another character that is the same as the character that invoked it. The definition of ''('' can not be meaningfully copied to \f{{}, on the other hand. The result is that //[[CL:Glossary:lists]]// are of the form \f{{a b c)}, not \f{{a b c}}, because the definition always looks for a closing parenthesis, not a closing brace.

====Examples==== <blockquote> (set-syntax-from-char #\\7 #\\;) → T 123579 → 1235 </blockquote>

====Side Effects====

The //to-readtable// is modified.

====Affected By====

The existing values in the //from-readtable//.

====Exceptional Situations====

None.

====See Also====

**[[CL:Functions:set-macro-character]]**, **[[CL:Functions:make-dispatch-macro-character]]**, {\secref\CharacterSyntaxTypes}

====Notes====

The //[[CL:Glossary:constituent traits]]// of a //[[CL:Glossary:character]]// are "hard wired" into the parser for extended //[[CL:Glossary:tokens]]//. For example, if the definition of ''S'' is copied to ''*'', then ''*'' will become a //[[CL:Glossary:constituent]]// that is //[[CL:Glossary:alphabetic]]// but that cannot be used as a //[[CL:Glossary:short float]]// //[[CL:Glossary:exponent marker]]//. For further information, see section {\secref\ConstituentTraits}.

\issue{RETURN-VALUES-UNSPECIFIED:SPECIFY} \issue{ARGUMENTS-UNDERSPECIFIED:SPECIFY}
