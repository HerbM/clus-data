====== Variable *PRINT-PRETTY* ======

====Value Type====

a //[[CL:Glossary:generalized boolean]]//.

====Initial Value====

//[[CL:Glossary:implementation-dependent]]//.

====Description====

Controls whether the //[[CL:Glossary:Lisp printer]]// calls the //[[CL:Glossary:pretty printer]]//.

If it is //[[CL:Glossary:false]]//, the //[[CL:Glossary:pretty printer]]// is not used and

a minimum

of //[[CL:Glossary:whitespace]]// is output when printing an expression.

If it is //[[CL:Glossary:true]]//, the //[[CL:Glossary:pretty printer]]// is used, and the //[[CL:Glossary:Lisp printer]]// will endeavor to insert extra //[[CL:Glossary:whitespace]]// where appropriate to make //[[CL:Glossary:expressions]]// more readable.

**[[CL:Variables:*print-pretty*]]** has an effect even when \thevalueof{*print-escape*} is //[[CL:Glossary:false]]//.

====Examples====

<blockquote> ([[CL:Macros:defparameter]] *print-pretty* 'nil) → NIL (progn (write '(let ((a 1) (b 2) (c 3)) (+ a b c))) nil)
▷ (LET ((A 1) (B 2) (C 3)) (+ A B C)) → NIL (let ((*print-pretty* t)) (progn (write '(let ((a 1) (b 2) (c 3)) (+ a b c))) nil))
▷ (LET ((A 1)
▷ (B 2)
▷ (C 3))
▷ (+ A B C)) → NIL ;; Note that the first two expressions printed by this next form ;; differ from the second two only in whether escape characters are printed. ;; In all four cases, extra whitespace is inserted by the pretty printer. (flet ((test (x) (let ((*print-pretty* t)) (print x) (format t "~ (terpri) (princ x) (princ " ") (format t "~ (test '#'(lambda () (list "a" #\b 'c #'d))))
▷ #'(LAMBDA ()
▷ (LIST "a" #\b 'C #'D))
▷ #'(LAMBDA ()
▷ (LIST "a" #\b 'C #'D))
▷ #'(LAMBDA ()
▷ (LIST a b 'C #'D))
▷ #'(LAMBDA ()
▷ (LIST a b 'C #'D)) → NIL </blockquote>

====Affected By====

None.

====See Also====

**[[CL:Functions:write]]**

====Notes====

None.

\issue{PRINTER-WHITESPACE:JUST-ONE-SPACE} \issue{FORMAT-PRETTY-PRINT:YES}
