====== Variable *PRINT-ESCAPE* ======

====Value Type====

a //[[CL:Glossary:generalized boolean]]//.

====Initial Value====

//[[CL:Glossary:true]]//.

====Description====

If //[[CL:Glossary:false]]//, escape characters and //[[CL:Glossary:package prefixes]]// are not output when an expression is printed.


If //[[CL:Glossary:true]]//, an attempt is made to print an //[[CL:Glossary:expression]]// in such a way that it can be read again to produce an **[[CL:Functions:equal]]** //[[CL:Glossary:expression]]//. (This is only a guideline; not a requirement. See **[[CL:Variables:*print-readably*]]**.)

For more specific details of how \thevalueof{*print-escape*} affects the printing of certain //[[CL:Glossary:types]]//, see section {\secref\DefaultPrintObjMeths}.

====Examples==== <blockquote> (let ((*print-escape* t)) (write #\\a))
▷ #\\a → #\\a (let ((*print-escape* nil)) (write #\\a))
▷ a → #\\a </blockquote>

====Affected By====

**[[CL:Functions:princ]]**, **[[CL:Functions:prin1]]**, **[[CL:Functions:format]]**

====See Also====

**[[CL:Functions:write]]**, **[[CL:Functions:readtable-case]]**

====Notes====

**[[CL:Functions:princ]]** effectively binds **[[CL:Variables:*print-escape*]]** to //[[CL:Glossary:false]]//. **[[CL:Functions:prin1]]** effectively binds **[[CL:Variables:*print-escape*]]** to //[[CL:Glossary:true]]//.


