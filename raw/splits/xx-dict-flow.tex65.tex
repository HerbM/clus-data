====== Macro SHIFTF ======

====Syntax====

\DefmacWithValues shiftf {\plusparam{place} newvalue} {old-value-1}

====Arguments and Values====

//place// - a //[[CL:Glossary:place]]//.

//newvalue// - a //[[CL:Glossary:form]]//; \eval.

//old-value-1// - an //[[CL:Glossary:object]]// (the old //[[CL:Glossary:value]]// of the first //place//).

====Description====

**[[CL:Macros:shiftf]]** modifies the values of each //place// by storing //newvalue// into the last //place//, and shifting the values of the second through the last //place// into the remaining //places//.

If //newvalue// produces more values than there are store variables, the extra values are ignored. If //newvalue// produces fewer values than there are store variables, the missing values are set to **[[CL:Constant Variables:nil]]**.

In the form **[[CL:Functions:(shiftf ''place1'' ''place2'' ... ''placen'' ''newvalue'')]]**, the values in ''place1'' through ''placen'' are //[[CL:Glossary:read]]// and saved, and ''newvalue'' is evaluated, for a total of ''n''+1 values in all. Values 2 through ''n''+1 are then stored into ''place1'' through ''placen'', respectively. It is as if all the //places// form a shift register; the //newvalue// is shifted in from the right, all values shift over to the left one place, and the value shifted out of ''place1'' is returned.

For information about the //[[CL:Glossary:evaluation]]// of //[[CL:Glossary:subforms]]// of //places//, see section {\secref\GenRefSubFormEval}.

====Examples====

<blockquote> ([[CL:Macros:defparameter]] x (list 1 2 3) y 'trash) → TRASH (shiftf y x (cdr x) '(hi there)) → TRASH x → (2 3) y → (1 HI THERE)

([[CL:Macros:defparameter]] x (list 'a 'b 'c)) → (A B C) (shiftf (cadr x) 'z) → B x → (A Z C) (shiftf (cadr x) (cddr x) 'q) → Z x → (A (C) . Q) ([[CL:Macros:defparameter]] n 0) → 0 ([[CL:Macros:defparameter]] x (list 'a 'b 'c 'd)) → (A B C D) (shiftf (nth ([[CL:Macros:defparameter]] n (+ n 1)) x) 'z) → B x → (A Z C D) </blockquote>

====Affected By====

**[[CL:Macros:define-setf-expander]]**, **[[CL:Macros:defsetf]]**, **[[CL:Variables:*macroexpand-hook*]]**

====Exceptional Situations====

None.

====See Also====

**[[CL:Macros:setf]]**, **[[CL:Macros:rotatef]]**, {\secref\GeneralizedReference}

====Notes====

The effect of ''(shiftf //place1// //place2// ... //placen// //newvalue//)'' is roughly equivalent to

<blockquote> (let ((var1 //place1//) (var2 //place2//) ... (varn //placen//) (var0 //newvalue//)) ([[CL:Macros:setf]] //place1// var2) ([[CL:Macros:setf]] //place2// var3) ... ([[CL:Macros:setf]] //placen// var0) var1) </blockquote> except that the latter would evaluate any //[[CL:Glossary:subforms]]// of each ''place'' twice, whereas **[[CL:Macros:shiftf]]** evaluates them once. For example,

<blockquote> ([[CL:Macros:defparameter]] n 0) → 0 ([[CL:Macros:defparameter]] x (list 'a 'b 'c 'd)) → (A B C D) (prog1 (nth ([[CL:Macros:defparameter]] n (+ n 1)) x) ([[CL:Macros:setf]] (nth ([[CL:Macros:defparameter]] n (+ n 1)) x) 'z)) → B x → (A B Z D) </blockquote>

\issue{SETF-MULTIPLE-STORE-VARIABLES:ALLOW} \issue{PUSH-EVALUATION-ORDER:FIRST-ITEM}
