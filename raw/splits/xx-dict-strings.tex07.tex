====== Function STRING ======

====Syntax====

**string** //x// → //string//

====Arguments and Values====

//x// - a //[[CL:Glossary:string]]//, a //[[CL:Glossary:symbol]]//, or a //[[CL:Glossary:character]]//.

//string// - a //[[CL:Glossary:string]]//.

====Description====

Returns a //[[CL:Glossary:string]]// described by //x//; specifically:

\beginlist \item{\bull} If //x// is a //[[CL:Glossary:string]]//, it is returned. \item{\bull} If //x// is a //[[CL:Glossary:symbol]]//, its //[[CL:Glossary:name]]// is returned. \item{\bull}

If //x// is a //[[CL:Glossary:character]]//,

then a //[[CL:Glossary:string]]// containing that one //[[CL:Glossary:character]]// is returned. \item{\bull}

**[[CL:Functions:string]]** might perform additional, //[[CL:Glossary:implementation-defined]]// conversions.

\endlist

====Examples====

<blockquote> (string "already a string") → "already a string" (string 'elm) → "ELM" (string #\\c) → "c" </blockquote>

====Affected By====

None.

====Exceptional Situations====

In the case where a conversion is defined neither by this specification nor by the //[[CL:Glossary:implementation]]//, an error of type **[[CL:Types:type-error]]** is signaled.

====See Also====

**[[CL:Functions:coerce]]**, **[[CL:Types:string]]** (//[[CL:Glossary:type]]//).

====Notes====

**[[CL:Functions:coerce]]** can be used to convert a //[[CL:Glossary:sequence]]// of //[[CL:Glossary:characters]]// to a //[[CL:Glossary:string]]//.

**[[CL:Functions:prin1-to-string]]**, **[[CL:Functions:princ-to-string]]**, **[[CL:Functions:write-to-string]]**, or **[[CL:Functions:format]]** (with a first argument of **[[CL:Constant Variables:nil]]**) can be used to get a //[[CL:Glossary:string]]// representation of a //[[CL:Glossary:number]]// or any other //[[CL:Glossary:object]]//.

\issue{CHARACTER-PROPOSAL:2-1-1} \issue{CHARACTER-PROPOSAL:2-1-1} \issue{STRING-COERCION:MAKE-CONSISTENT} \issue{STRING-COERCION:MAKE-CONSISTENT}
