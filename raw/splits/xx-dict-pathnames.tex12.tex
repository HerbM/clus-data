====== Function PARSE-NAMESTRING ======

====Syntax====

\DefunWithValuesNewline parse-namestring {thing ''&optional'' host default-pathname ''&key'' start end junk-allowed} {pathname, position}

====Arguments and Values====

//thing// - a //[[CL:Glossary:string]]//, a //[[CL:Glossary:pathname]]//, or a //[[CL:Glossary:stream associated with a file]]//.

//host// - a //[[CL:Glossary:valid pathname host]]//, a //[[CL:Glossary:logical host]]//, or **[[CL:Constant Variables:nil]]**.

//default-pathname// - a //[[CL:Glossary:pathname designator]]//. \Default{\thevalueof{*default-pathname-defaults*}}

//start//, //end// - //[[CL:Glossary:bounding index designators]]// of //thing//. \Defaults{//start// and //end//}{''0'' and **[[CL:Constant Variables:nil]]**}

//junk-allowed// - a //[[CL:Glossary:generalized boolean]]//. The default is //[[CL:Glossary:false]]//.

//pathname// - a //[[CL:Glossary:pathname]]//, or **[[CL:Constant Variables:nil]]**.

//position// - a //[[CL:Glossary:bounding index designator]]// for //thing//.

====Description====

Converts //thing// into a //[[CL:Glossary:pathname]]//.

The //host// supplies a host name with respect to which the parsing occurs.

If //thing// is a //[[CL:Glossary:stream associated with a file]]//, processing proceeds as if the //[[CL:Glossary:pathname]]// used to open that //[[CL:Glossary:file]]// had been supplied instead.

If //thing// is a //[[CL:Glossary:pathname]]//, the //host// and the host component of //thing// are compared.

If they match, two values are immediately returned: //thing// and //start//; otherwise (if they do not match), an error is signaled.

Otherwise (if //thing// is a //[[CL:Glossary:string]]//), **[[CL:Functions:parse-namestring]]** parses the name of a //[[CL:Glossary:file]]// within the substring of //thing// bounded by //start// and //end//.

If //thing// is a //[[CL:Glossary:string]]// then

the substring of //thing// //[[CL:Glossary:bounded]]// by //start// and //end// is parsed into a //[[CL:Glossary:pathname]]// as follows:

\beginlist

\itemitem{\bull} If //host// is a //[[CL:Glossary:logical host]]// then //thing// is parsed as a //[[CL:Glossary:logical pathname]]// //[[CL:Glossary:namestring]]//

on the //host//.

\itemitem{\bull} If //host// is **[[CL:Constant Variables:nil]]** and //thing// is a syntactically valid //[[CL:Glossary:logical pathname]]// //[[CL:Glossary:namestring]]// containing an explicit host, then it is parsed as a //[[CL:Glossary:logical pathname]]// //[[CL:Glossary:namestring]]//.

\itemitem{\bull} If //host// is **[[CL:Constant Variables:nil]]**, //default-pathname// is a //[[CL:Glossary:logical pathname]]//, and //thing// is a syntactically valid //[[CL:Glossary:logical pathname]]// //[[CL:Glossary:namestring]]// without an explicit host, then it is parsed as a //[[CL:Glossary:logical pathname]]// //[[CL:Glossary:namestring]]//

on the host that is the host component of //default-pathname//.

\itemitem{\bull} Otherwise, the parsing of //thing// is //[[CL:Glossary:implementation-defined]]//.

\endlist

In the first

of these

cases, the host portion of the //[[CL:Glossary:logical pathname]]// namestring and its following //[[CL:Glossary:colon]]// are optional.

If the host portion of the namestring and //host// are both present and do not match, an error is signaled.

If //junk-allowed// is //[[CL:Glossary:true]]//, then the //[[CL:Glossary:primary value]]// is the //[[CL:Glossary:pathname]]// parsed or, if no syntactically correct //[[CL:Glossary:pathname]]// was seen, **[[CL:Constant Variables:nil]]**.

If //junk-allowed// is //[[CL:Glossary:false]]//, then the entire substring is scanned, and the //[[CL:Glossary:primary value]]// is the //[[CL:Glossary:pathname]]// parsed.

In either case, the //[[CL:Glossary:secondary value]]// is the index into //thing// of the delimiter that terminated the parse, or the index beyond the substring if the parse terminated at the end of the substring (as will always be the case if //junk-allowed// is //[[CL:Glossary:false]]//).

Parsing a //[[CL:Glossary:null]]// //[[CL:Glossary:string]]// always succeeds, producing a //[[CL:Glossary:pathname]]// with all components (except the host) equal to **[[CL:Constant Variables:nil]]**.

If //thing// contains an explicit host name and no explicit device name, then it is //[[CL:Glossary:implementation-defined]]// whether **[[CL:Functions:parse-namestring]]** will supply the standard default device for that host as the device component of the resulting //[[CL:Glossary:pathname]]//.

====Examples====

<blockquote> ([[CL:Macros:defparameter]] q (parse-namestring "test")) → #S(PATHNAME :HOST NIL :DEVICE NIL :DIRECTORY NIL :NAME "test" :TYPE NIL :VERSION NIL) (pathnamep q) → //[[CL:Glossary:true]]// (parse-namestring "test") → #S(PATHNAME :HOST NIL :DEVICE NIL :DIRECTORY NIL :NAME "test" :TYPE NIL :VERSION NIL), 4 ([[CL:Macros:defparameter]] s (open ''xxx'')) → #<Input File Stream...> (parse-namestring s) → #S(PATHNAME :HOST NIL :DEVICE NIL :DIRECTORY NIL :NAME ''xxx'' :TYPE NIL :VERSION NIL), 0 (parse-namestring "test" nil nil :start 2 :end 4 ) → #S(PATHNAME ...), 15 (parse-namestring "foo.lisp") → #P"foo.lisp" </blockquote>

====Affected By====

None.

====Exceptional Situations====

If //junk-allowed// is //[[CL:Glossary:false]]//, an error of type **[[CL:Types:parse-error]]** is signaled if //thing// does not consist entirely of the representation of a //[[CL:Glossary:pathname]]//, possibly surrounded on either side by //[[CL:Glossary:whitespace]]// characters if that is appropriate to the cultural conventions of the implementation.

If //host// is supplied and not **[[CL:Constant Variables:nil]]**, and //thing// contains a manifest host name, an error of type **[[CL:Types:error]]** is signaled if the hosts do not match.

If //thing// is a //[[CL:Glossary:logical pathname]]// namestring and if the host portion of the namestring and //host// are both present and do not match, an error of type **[[CL:Types:error]]** is signaled.

====See Also====

**[[CL:Types:pathname]]**, **[[CL:Types:logical-pathname]]**,{\secref\FileSystemConcepts},

{\secref\UnspecificComponent},

{\secref\PathnamesAsFilenames}

====Notes====

None.

\issue{PATHNAME-PRINT-READ:SHARPSIGN-P} \issue{PATHNAME-LOGICAL:ADD} \issue{PATHNAME-LOGICAL:ADD} \issue{FILE-OPEN-ERROR:SIGNAL-FILE-ERROR} \issue{PATHNAME-SYMBOL} \issue{PATHNAME-LOGICAL:ADD} \issue{PATHNAME-LOGICAL:ADD} \issue{PATHNAME-UNSPECIFIC-COMPONENT:NEW-TOKEN} \issue{SUBSEQ-OUT-OF-BOUNDS} \issue{RANGE-OF-START-AND-END-PARAMETERS:INTEGER-AND-INTEGER-NIL} \issue{PATHNAME-STREAM} \issue{PATHNAME-SYMBOL} \issue{PATHNAME-LOGICAL:ADD} \issue{PATHNAME-HOST-PARSING:RECOGNIZE-LOGICAL-HOST-NAMES}
