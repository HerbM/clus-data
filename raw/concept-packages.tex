




\beginsubSection{Introduction to Packages}



A //[[CL:Glossary:package]]// establishes a mapping from names to //[[CL:Glossary:symbols]]//. 
At any given time, one //[[CL:Glossary:package]]// is current.
The //[[CL:Glossary:current package]]// is the one that is \thevalueof{*package*}.
When using the //[[CL:Glossary:Lisp reader]]//,
it is possible to refer to //[[CL:Glossary:symbols]]// in //[[CL:Glossary:packages]]// 
other than the current one through the use of //[[CL:Glossary:package prefixes]]// in the 
printed representation of the //[[CL:Glossary:symbol]]//.


\Thenextfigure\ lists some //[[CL:Glossary:defined names]]// that are applicable
to //[[CL:Glossary:packages]]//.



Where an //[[CL:Glossary:operator]]// 
takes an argument that is either a //[[CL:Glossary:symbol]]// or a //[[CL:Glossary:list]]// 
of //[[CL:Glossary:symbols]]//,
an argument of \nil\ is treated as an empty //[[CL:Glossary:list]]// of //[[CL:Glossary:symbols]]//.
Any //package// argument may be either a //[[CL:Glossary:string]]//, a //[[CL:Glossary:symbol]]//, or
a //[[CL:Glossary:package]]//.  If a //[[CL:Glossary:symbol]]// is supplied, its name will be used
as the //[[CL:Glossary:package]]// name.
\issue{REQUIRE-PATHNAME-DEFAULTS:ELIMINATE}




\displaythree{Some Defined Names related to Packages}{
*modules*&import&provide\cr
*package*&in-package&rename-package\cr
defpackage&intern&require\cr
do-all-symbols&list-all-packages&shadow\cr
do-external-symbols&make-package&shadowing-import\cr
do-symbols&package-name&unexport\cr
export&package-nicknames&unintern\cr
find-all-symbols&package-shadowing-symbols&unuse-package\cr
find-package&package-use-list&use-package\cr
find-symbol&package-used-by-list&\cr
}


\beginsubsubsection{Package Names and Nicknames}


Each //[[CL:Glossary:package]]// has a //[[CL:Glossary:name]]// (a //[[CL:Glossary:string]]//) and perhaps some //[[CL:Glossary:nicknames]]//
(also //[[CL:Glossary:strings]]//).
These are assigned when the //[[CL:Glossary:package]]// is created and can be changed later.  


There is a single namespace for //[[CL:Glossary:packages]]//.  
\Thefunction{find-package} translates a package
//[[CL:Glossary:name]]// or //[[CL:Glossary:nickname]]// into the associated //[[CL:Glossary:package]]//.  
\Thefunction{package-name} returns the //[[CL:Glossary:name]]// of a //[[CL:Glossary:package]]//.  
\Thefunction{package-nicknames} returns 
a //[[CL:Glossary:list]]// of all //[[CL:Glossary:nicknames]]// for a //[[CL:Glossary:package]]//.
**[[CL:Functions:rename-package]]** removes a //[[CL:Glossary:package]]//'s
current //[[CL:Glossary:name]]// and //[[CL:Glossary:nicknames]]// and replaces them with new ones
specified by the caller.

\endsubsubsection%{Package Names and Nicknames}

\beginsubsubsection{Symbols in a Package}

\beginsubsubsubsection{Internal and External Symbols}


The mappings in a //[[CL:Glossary:package]]// are divided into two classes, external and internal.
The //[[CL:Glossary:symbols]]// targeted by these different mappings 
are called //[[CL:Glossary:external symbols]]// and //[[CL:Glossary:internal symbols]]//\idxterm{internal symbol} of the
//[[CL:Glossary:package]]//. Within a //[[CL:Glossary:package]]//, a name refers to one
//[[CL:Glossary:symbol]]// or to none; if it does refer
to a //[[CL:Glossary:symbol]]//, then it is either external or internal in that
//[[CL:Glossary:package]]//, but not both.

\newtermidx{External symbols}{external symbol}
are part of the package's public interface to other //[[CL:Glossary:packages]]//.
//[[CL:Glossary:Symbols]]// become //[[CL:Glossary:external symbols]]// of a given
//[[CL:Glossary:package]]// if they have been //[[CL:Glossary:exported]]// from that //[[CL:Glossary:package]]//.


A //[[CL:Glossary:symbol]]// has the same //[[CL:Glossary:name]]// no matter what //[[CL:Glossary:package]]// 
it is //[[CL:Glossary:present]]// in, but it might be an //[[CL:Glossary:external symbol]]// of some //[[CL:Glossary:packages]]//
and an //[[CL:Glossary:internal symbol]]// of others. 

\endsubsubsubsection%{Internal and External Symbols}

\beginsubsubsubsection{Package Inheritance}


//[[CL:Glossary:Packages]]// can be built up in layers.  From one point of view,
a //[[CL:Glossary:package]]// is a single collection
of mappings from //[[CL:Glossary:strings]]// into //[[CL:Glossary:internal symbols]]// and 
//[[CL:Glossary:external symbols]]//.
However, some of these mappings might be established within the //[[CL:Glossary:package]]// 
itself, while other mappings are inherited from other //[[CL:Glossary:packages]]// 
via **[[CL:Functions:use-package]]**.
A //[[CL:Glossary:symbol]]// is said to be //[[CL:Glossary:present]]// in a //[[CL:Glossary:package]]// 
if the mapping is in the //[[CL:Glossary:package]]// itself and is
not inherited from somewhere else.


There is no way to inherit the //[[CL:Glossary:internal symbols]]// of another //[[CL:Glossary:package]]//;
to refer to an //[[CL:Glossary:internal symbol]]// using the //[[CL:Glossary:Lisp reader]]//, 
    a //[[CL:Glossary:package]]// containing the //[[CL:Glossary:symbol]]// 
     must be made to be the //[[CL:Glossary:current package]]//,
    a //[[CL:Glossary:package prefix]]// must be used,
 or the //[[CL:Glossary:symbol]]// must be //[[CL:Glossary:imported]]// into the //[[CL:Glossary:current package]]//.

\endsubsubsubsection%{Package Inheritance}

\beginsubsubsubsection{Accessibility of Symbols in a Package}

A //[[CL:Glossary:symbol]]// becomes //[[CL:Glossary:accessible]]// in a //[[CL:Glossary:package]]// 
    if that is its //[[CL:Glossary:home package]]// when it is created,
 or if it is //[[CL:Glossary:imported]]// into that //[[CL:Glossary:package]]//,
 or by inheritance via **[[CL:Functions:use-package]]**.

If a //[[CL:Glossary:symbol]]// is //[[CL:Glossary:accessible]]// in a //[[CL:Glossary:package]]//,
it can be referred to when using the //[[CL:Glossary:Lisp reader]]//
without a //[[CL:Glossary:package prefix]]// when that //[[CL:Glossary:package]]// is the //[[CL:Glossary:current package]]//,
regardless of whether it is //[[CL:Glossary:present]]// or inherited.







//[[CL:Glossary:Symbols]]// from one //[[CL:Glossary:package]]// can be made //[[CL:Glossary:accessible]]// 
in another //[[CL:Glossary:package]]// in two ways.


\beginlist 
\itemitem{--}
Any individual //[[CL:Glossary:symbol]]// can be added to a //[[CL:Glossary:package]]// by use
of **[[CL:Functions:import]]**.  After the call to **[[CL:Functions:import]]** the
//[[CL:Glossary:symbol]]// is //[[CL:Glossary:present]]// in the importing //[[CL:Glossary:package]]//.
The status of the //[[CL:Glossary:symbol]]// in the //[[CL:Glossary:package]]// 
it came from (if any) is unchanged, and the //[[CL:Glossary:home package]]// for
this //[[CL:Glossary:symbol]]// is unchanged.
Once //[[CL:Glossary:imported]]//, a //[[CL:Glossary:symbol]]// is //[[CL:Glossary:present]]// in the
importing //[[CL:Glossary:package]]//
and can be removed only by calling **[[CL:Functions:unintern]]**.


A //[[CL:Glossary:symbol]]// is //[[CL:Glossary:shadowed]]//\meaning{3} by another //[[CL:Glossary:symbol]]// 
in some //[[CL:Glossary:package]]// if the first //[[CL:Glossary:symbol]]// would be //[[CL:Glossary:accessible]]//
by inheritance if not for the presence of the second //[[CL:Glossary:symbol]]//.
See **[[CL:Functions:shadowing-import]]**.



\itemitem{--}
The second mechanism for making //[[CL:Glossary:symbols]]// from one //[[CL:Glossary:package]]//
//[[CL:Glossary:accessible]]// in another is provided by **[[CL:Functions:use-package]]**.  
All of the //[[CL:Glossary:external symbols]]// of the used //[[CL:Glossary:package]]// are inherited
by the using //[[CL:Glossary:package]]//.
\Thefunction{unuse-package} undoes the effects of a previous **[[CL:Functions:use-package]]**.  
\endlist

\endsubsubsubsection%{Accessibility of Symbols in a Package}

\beginsubsubsubsection{Locating a Symbol in a Package}


When a //[[CL:Glossary:symbol]]// is to be located in a given //[[CL:Glossary:package]]// 
the following occurs:
\beginlist 
\itemitem{--} The //[[CL:Glossary:external symbols]]// and //[[CL:Glossary:internal symbols]]// of the 
//[[CL:Glossary:package]]// are searched for the //[[CL:Glossary:symbol]]//.
\itemitem{--} The //[[CL:Glossary:external symbols]]// of the used //[[CL:Glossary:packages]]// are 
searched
in some unspecified order.  The
order does not matter; see the rules for handling name
conflicts listed below. 
\endlist

\endsubsubsubsection%{Locating a Symbol in a Package}

\beginsubsubsubsection{Prevention of Name Conflicts in Packages}


Within one //[[CL:Glossary:package]]//, any particular name can refer to at most one
//[[CL:Glossary:symbol]]//.  A name conflict is said to occur when there would be more than
one candidate //[[CL:Glossary:symbol]]//.  Any time a name conflict is about to occur,
a //[[CL:Glossary:correctable]]// //[[CL:Glossary:error]]// is signaled.  

The following rules apply to name conflicts:

\beginlist

\itemitem{--}
Name conflicts are detected when they become possible, that is, when the
package structure is altered.  Name
conflicts are not checked during every name lookup.

\itemitem{--}
If the //[[CL:Glossary:same]]// //[[CL:Glossary:symbol]]// is //[[CL:Glossary:accessible]]// to a //[[CL:Glossary:package]]// 
through more than one path, there is no name conflict.
A //[[CL:Glossary:symbol]]// cannot conflict with itself. 
Name conflicts occur only between //[[CL:Glossary:distinct]]// //[[CL:Glossary:symbols]]// with
the same name (under **[[CL:Functions:string=]]**).


\itemitem{--} Every //[[CL:Glossary:package]]// has a list of shadowing //[[CL:Glossary:symbols]]//.  
A shadowing //[[CL:Glossary:symbol]]// takes precedence over any other //[[CL:Glossary:symbol]]// of
the same name that would otherwise be //[[CL:Glossary:accessible]]// in the //[[CL:Glossary:package]]//.  
A name conflict involving a shadowing symbol is always resolved in favor of
the shadowing //[[CL:Glossary:symbol]]//, without signaling an error (except for one
exception involving **[[CL:Functions:import]]**).
See **[[CL:Functions:shadow]]** and **[[CL:Functions:shadowing-import]]**.


\itemitem{--} 
The functions **[[CL:Functions:use-package]]**, **[[CL:Functions:import]]**, and 
**[[CL:Functions:export]]** check for name conflicts.  


\itemitem{--} 
**[[CL:Functions:shadow]]** and **[[CL:Functions:shadowing-import]]** 
never signal a name-conflict error.


\itemitem{--} 




**[[CL:Functions:unuse-package]]** and **[[CL:Functions:unexport]]**
do not need to do any name-conflict checking.
**[[CL:Functions:unintern]]** does name-conflict checking only when a //[[CL:Glossary:symbol]]// 
being //[[CL:Glossary:uninterned]]// is a //[[CL:Glossary:shadowing symbol]]//\idxterm{shadowing symbol}.


\itemitem{--} 
Giving a shadowing symbol to **[[CL:Functions:unintern]]** 
can uncover a name conflict that had
previously been resolved by the shadowing.  






  \itemitem{--} 
  Package functions signal name-conflict errors \oftype{package-error} before making any
  change to the package structure.  When multiple changes are to be made,
  it is
  permissible for the implementation to process each change separately.
  For example, when **[[CL:Functions:export]]** is given a 
//[[CL:Glossary:list]]// of 
//[[CL:Glossary:symbols]]//,
  aborting from a name
  conflict caused by the second //[[CL:Glossary:symbol]]// 
  in the //[[CL:Glossary:list]]// might still export the
  first //[[CL:Glossary:symbol]]// in the //[[CL:Glossary:list]]//.  
  However, a name-conflict error caused by **[[CL:Functions:export]]**
  of a single //[[CL:Glossary:symbol]]// will be signaled before
  that //[[CL:Glossary:symbol]]//'s //[[CL:Glossary:accessibility]]// in any //[[CL:Glossary:package]]// is changed.


\itemitem{--} 
Continuing from a name-conflict error must offer the user a chance to
resolve the name conflict in favor of either of the candidates.  The
//[[CL:Glossary:package]]// 
structure should be altered to reflect the resolution of the
name conflict, via **[[CL:Functions:shadowing-import]]**, 
**[[CL:Functions:unintern]]**,

or **[[CL:Functions:unexport]]**.


\itemitem{--} 
A name conflict in **[[CL:Functions:use-package]]** between a //[[CL:Glossary:symbol]]// 

//[[CL:Glossary:present]]// in the using //[[CL:Glossary:package]]// and an //[[CL:Glossary:external symbol]]// of the used 
//[[CL:Glossary:package]]// is resolved in favor of the first //[[CL:Glossary:symbol]]// by making it a
shadowing //[[CL:Glossary:symbol]]//, or in favor of the second //[[CL:Glossary:symbol]]// by uninterning
the first //[[CL:Glossary:symbol]]// from the using //[[CL:Glossary:package]]//. 


\itemitem{--} 
A name conflict in **[[CL:Functions:export]]** or **[[CL:Functions:unintern]]** 
due to a //[[CL:Glossary:package]]//'s inheriting two //[[CL:Glossary:distinct]]// //[[CL:Glossary:symbols]]// 
with the //[[CL:Glossary:same]]// //[[CL:Glossary:name]]// (under **[[CL:Functions:string=]]**)
from two other //[[CL:Glossary:packages]]// can be resolved in
favor of either //[[CL:Glossary:symbol]]// by importing it into the using
//[[CL:Glossary:package]]// and making it a //[[CL:Glossary:shadowing symbol]]//\idxterm{shadowing symbol},
just as with **[[CL:Functions:use-package]]**.
\endlist

\endsubsubsubsection%{Prevention of Name Conflicts in Packages}

\endsubsubsection%{Symbols in a Package}

\endsubSection%{Introduction to Packages}

\beginsubSection{Standardized Packages}


This section describes the //[[CL:Glossary:packages]]// that are available
in every //[[CL:Glossary:conforming implementation]]//.  A summary of the
//[[CL:Glossary:names]]// and //[[CL:Glossary:nicknames]]// of those //[[CL:Glossary:standardized]]// //[[CL:Glossary:packages]]// 
is given in \thenextfigure.

\tablefigtwo{Standardized Package Names}{Name}{Nicknames}{
\packref{common-lisp}&\packref{cl}\cr
\packref{common-lisp-user}&\packref{cl-user}\cr
\packref{keyword}&\i{none}\cr
}

\issue{LISP-PACKAGE-NAME:COMMON-LISP}




\issue{PACKAGE-CLUTTER:REDUCE}




\beginsubsubsection{The COMMON-LISP Package} 
\idxpackref{common-lisp}\idxpackref{cl}

\issue{LISP-PACKAGE-NAME:COMMON-LISP}
 
\Thepackage{common-lisp} contains the primitives of the \clisp\ system as
defined by this specification.  Its //[[CL:Glossary:external]]// //[[CL:Glossary:symbols]]// include
all of the //[[CL:Glossary:defined names]]// (except for //[[CL:Glossary:defined names]]// in
\thepackage{keyword}) that are present in the \clisp\ system, 
such as **[[CL:Functions:car]]**, **[[CL:Functions:cdr]]**,  \varref{*package*}, etc.
\Thepackage{common-lisp} has the //[[CL:Glossary:nickname]]// \packref{cl}.
 
\issue{PACKAGE-CLUTTER:REDUCE}
\Thepackage{common-lisp} has as //[[CL:Glossary:external]]// //[[CL:Glossary:symbols]]// those 
symbols enumerated in the figures in \secref\CLsymbols, and no others.
These //[[CL:Glossary:external]]// //[[CL:Glossary:symbols]]// are //[[CL:Glossary:present]]// in \thepackage{common-lisp}
but their //[[CL:Glossary:home package]]// need not be \thepackage{common-lisp}.

For example, the symbol \f{HELP} cannot be an //[[CL:Glossary:external symbol]]// of
\thepackage{common-lisp} because it is not mentioned in \secref\CLsymbols.
In contrast, the //[[CL:Glossary:symbol]]// \misc{variable}
must be an //[[CL:Glossary:external symbol]]// of \thepackage{common-lisp} 
even though it has no definition
because it is listed in that section
(to support its use as a valid second //[[CL:Glossary:argument]]// to \thefunction{documentation}). 


\Thepackage{common-lisp} can have additional //[[CL:Glossary:internal symbols]]//.


\beginsubsubsubsection{Constraints on the COMMON-LISP Package for Conforming Implementations}

\issue{PACKAGE-CLUTTER:REDUCE}
In a //[[CL:Glossary:conforming implementation]]//,
an //[[CL:Glossary:external]]// //[[CL:Glossary:symbol]]// of \thepackage{common-lisp} can have
   a //[[CL:Glossary:function]]//, //[[CL:Glossary:macro]]//, or //[[CL:Glossary:special operator]]// definition, 

   a //[[CL:Glossary:global variable]]// definition
   (or other status as a //[[CL:Glossary:dynamic variable]]// 
    due to a **[[CL:Declarations:special]]** //[[CL:Glossary:proclamation]]//),
or a //[[CL:Glossary:type]]// definition
only if explicitly permitted in this standard.


For example,
  **[[CL:Functions:fboundp]]** //[[CL:Glossary:yields]]// //[[CL:Glossary:false]]// 
  for any //[[CL:Glossary:external symbol]]// of \thepackage{common-lisp} 
  that is not the //[[CL:Glossary:name]]// of a //[[CL:Glossary:standardized]]// 
   //[[CL:Glossary:function]]//, //[[CL:Glossary:macro]]// or //[[CL:Glossary:special operator]]//,
and
  **[[CL:Functions:boundp]]** returns //[[CL:Glossary:false]]// 
  for any //[[CL:Glossary:external symbol]]// of \thepackage{common-lisp} 
  that is not the //[[CL:Glossary:name]]// of a //[[CL:Glossary:standardized]]// //[[CL:Glossary:global variable]]//.
It also follows that
  //[[CL:Glossary:conforming programs]]// can use //[[CL:Glossary:external symbols]]// of \thepackage{common-lisp} 
  as the //[[CL:Glossary:names]]// of local //[[CL:Glossary:lexical variables]]// 
  with confidence that those //[[CL:Glossary:names]]// have not been //[[CL:Glossary:proclaimed]]// **[[CL:Declarations:special]]** 
  by the //[[CL:Glossary:implementation]]//
  unless those //[[CL:Glossary:symbols]]// are
    //[[CL:Glossary:names]]// of //[[CL:Glossary:standardized]]// //[[CL:Glossary:global variables]]//.









A //[[CL:Glossary:conforming implementation]]// must not place any //[[CL:Glossary:property]]// on
an //[[CL:Glossary:external symbol]]// of \thepackage{common-lisp} using a //[[CL:Glossary:property indicator]]//
that is either an //[[CL:Glossary:external symbol]]// of any //[[CL:Glossary:standardized]]// //[[CL:Glossary:package]]//
or a //[[CL:Glossary:symbol]]// that is otherwise //[[CL:Glossary:accessible]]// in \thepackage{common-lisp-user}.
 






\endsubsubsubsection%{Constraints on the COMMON-LISP Package for Conforming Implementations}
 
\beginsubsubsubsection{Constraints on the COMMON-LISP Package for Conforming Programs}
\idxtext{redefinition}
\issue{LISP-SYMBOL-REDEFINITION:MAR89-X3J13}
Except where explicitly allowed, the consequences are undefined if any
of the following actions are performed on an //[[CL:Glossary:external symbol]]// 
of \thepackage{common-lisp}:

\beginlist

\itemitem{1.} //[[CL:Glossary:Binding]]// or altering its value (lexically or dynamically).
	      (Some exceptions are noted below.)

\itemitem{2.} Defining, 
\issue{LISP-SYMBOL-REDEFINITION-AGAIN:MORE-FIXES}
	      undefining, 

	  or //[[CL:Glossary:binding]]// it as a //[[CL:Glossary:function]]//.
	      (Some exceptions are noted below.)

\itemitem{3.} Defining,
\issue{LISP-SYMBOL-REDEFINITION-AGAIN:MORE-FIXES}
	      undefining, 

	   or //[[CL:Glossary:binding]]// it as a //[[CL:Glossary:macro]]//
\issue{DEFINE-COMPILER-MACRO:X3J13-NOV89}
	      or //[[CL:Glossary:compiler macro]]//.

	      (Some exceptions are noted below.)


\itemitem{4.} Defining it as a //[[CL:Glossary:type specifier]]// 
	      (via \macref{defstruct}, 
		   \macref{defclass},
		   \macref{deftype},
		   \macref{define-condition}).

\itemitem{5.} Defining it as a structure (via \macref{defstruct}).

\itemitem{6.} Defining it as a //[[CL:Glossary:declaration]]// 
	      with a **[[CL:Declarations:declaration]]** //[[CL:Glossary:proclamation]]//.

\itemitem{7.} Defining it as a //[[CL:Glossary:symbol macro]]//.





\itemitem{8.} Altering its //[[CL:Glossary:home package]]//.

\itemitem{9.} Tracing it  (via \macref{trace}).


\itemitem{10.} Declaring or proclaiming it


	       **[[CL:Declarations:special]]**
	       (via \misc{declare},
\issue{PROCLAIM-ETC-IN-COMPILE-FILE:NEW-MACRO}
		    \specref{declaim},

		 or **[[CL:Functions:proclaim]]**).

\itemitem{11.} Declaring or proclaiming its **[[CL:Declarations:type]]** or **[[CL:Declarations:ftype]]**
	       (via \misc{declare},
\issue{PROCLAIM-ETC-IN-COMPILE-FILE:NEW-MACRO}
		    \macref{declaim},

		 or **[[CL:Functions:proclaim]]**).
	       (Some exceptions are noted below.)

\itemitem{12.} Removing it from \thepackage{common-lisp}.

\issue{LISP-SYMBOL-REDEFINITION-AGAIN:MORE-FIXES}

\itemitem{13.} Defining a //[[CL:Glossary:setf expander]]// for it 
	       (via **[[CL:Functions:defsetf]]** or **[[CL:Functions:define-setf-method]]**).

\itemitem{14.} Defining, undefining, or binding its //[[CL:Glossary:setf function name]]//.

\itemitem{15.} Defining it as a //[[CL:Glossary:method combination]]// type 
		(via **[[CL:Functions:define-method-combination]]**).

\itemitem{16.} Using it as the class-name argument 
	       to **[[CL:Functions:setf]]** of **[[CL:Functions:find-class]]**.

\itemitem{17.} Binding it as a //[[CL:Glossary:catch tag]]//.

\itemitem{18.} Binding it as a //[[CL:Glossary:restart]]// //[[CL:Glossary:name]]//.

\itemitem{19.} Defining a //[[CL:Glossary:method]]// 
	       for a //[[CL:Glossary:standardized]]// //[[CL:Glossary:generic function]]// 
	       which is //[[CL:Glossary:applicable]]// when all of the //[[CL:Glossary:arguments]]//
      	       are //[[CL:Glossary:direct instances]]// of //[[CL:Glossary:standardized]]// //[[CL:Glossary:classes]]//.



\endlist 

\beginsubsubsubsubsection{Some Exceptions to Constraints on the COMMON-LISP Package for Conforming Programs}

If an //[[CL:Glossary:external symbol]]// of \thepackage{common-lisp}
is not globally defined as a //[[CL:Glossary:standardized]]// //[[CL:Glossary:dynamic variable]]// 
					      or //[[CL:Glossary:constant variable]]//,
it is allowed to lexically //[[CL:Glossary:bind]]// it 
          and to declare the **[[CL:Declarations:type]]** of that //[[CL:Glossary:binding]]//, 
and
it is allowed to locally //[[CL:Glossary:establish]]// it as a //[[CL:Glossary:symbol macro]]// 
(\eg with \specref{symbol-macrolet}).




Unless explicitly specified otherwise,
if an //[[CL:Glossary:external symbol]]// of \thepackage{common-lisp} 
is globally defined as a //[[CL:Glossary:standardized]]// //[[CL:Glossary:dynamic variable]]//,
it is permitted to //[[CL:Glossary:bind]]// or //[[CL:Glossary:assign]]// that //[[CL:Glossary:dynamic variable]]//
provided that the ``Value Type'' constraints on the //[[CL:Glossary:dynamic variable]]// 
are maintained, and that the new //[[CL:Glossary:value]]// of the //[[CL:Glossary:variable]]// 
is consistent with the stated purpose of the //[[CL:Glossary:variable]]//.

If an //[[CL:Glossary:external symbol]]// of \thepackage{common-lisp} is not defined
as a //[[CL:Glossary:standardized]]// //[[CL:Glossary:function]]//, //[[CL:Glossary:macro]]//, or //[[CL:Glossary:special operator]]//,
it is allowed to lexically //[[CL:Glossary:bind]]// it as a //[[CL:Glossary:function]]// (\eg with \specref{flet}),
              to declare the **[[CL:Declarations:ftype]]** of that //[[CL:Glossary:binding]]//, 
          and 


              (in //[[CL:Glossary:implementations]]// which provide the ability to do so)
	      to \macref{trace} that //[[CL:Glossary:binding]]//.

If an //[[CL:Glossary:external symbol]]// of \thepackage{common-lisp} is not defined
as a //[[CL:Glossary:standardized]]// //[[CL:Glossary:function]]//, //[[CL:Glossary:macro]]//, or //[[CL:Glossary:special operator]]//,
it is allowed to lexically //[[CL:Glossary:bind]]// it as a //[[CL:Glossary:macro]]// (\eg with \specref{macrolet}).




\issue{LISP-SYMBOL-REDEFINITION-AGAIN:MORE-FIXES}
If an //[[CL:Glossary:external symbol]]// of \thepackage{common-lisp} is not defined 
as a //[[CL:Glossary:standardized]]// //[[CL:Glossary:function]]//, //[[CL:Glossary:macro]]//, or //[[CL:Glossary:special operator]]//,
it is allowed to lexically //[[CL:Glossary:bind]]// its //[[CL:Glossary:setf function name]]//
as a //[[CL:Glossary:function]]//,
and to declare the **[[CL:Declarations:ftype]]** of that //[[CL:Glossary:binding]]//.


\endsubsubsubsubsection%{Some Exceptions to Constraints on the COMMON-LISP Package for Conforming Programs}

\endsubsubsubsection%{Constraints on the COMMON-LISP Package for Conforming Programs}

\endsubsubsection%{The COMMON-LISP Package} 

\beginsubsubsection{The COMMON-LISP-USER Package}
\idxpackref{common-lisp-user}\idxpackref{cl-user}

\Thepackage{common-lisp-user} is the //[[CL:Glossary:current package]]// when 
a \clisp\ system starts up.  This //[[CL:Glossary:package]]// //[[CL:Glossary:uses]]// \thepackage{common-lisp}.
\Thepackage{common-lisp-user} has the //[[CL:Glossary:nickname]]// \packref{cl-user}.
\issue{PACKAGE-CLUTTER:REDUCE}
\Thepackage{common-lisp-user} can have additional //[[CL:Glossary:symbols]]// //[[CL:Glossary:interned]]// within it;
it can //[[CL:Glossary:use]]// other //[[CL:Glossary:implementation-defined]]// //[[CL:Glossary:packages]]//.

 
\endsubsubsection%{The COMMON-LISP-USER Package}

\beginsubsubsection{The KEYWORD Package}
\idxpackref{keyword}








\Thepackage{keyword} contains //[[CL:Glossary:symbols]]//, called //[[CL:Glossary:keywords]]//\meaning{1},
that are typically used as special markers in //[[CL:Glossary:programs]]// 
and their associated data //[[CL:Glossary:expressions]]//\meaning{1}.

//[[CL:Glossary:Symbol]]// //[[CL:Glossary:tokens]]// that start with a //[[CL:Glossary:package marker]]// 
are parsed by the //[[CL:Glossary:Lisp reader]]// as //[[CL:Glossary:symbols]]// 
in \thepackage{keyword}; \seesection\SymbolTokens.
This makes it notationally convenient to use //[[CL:Glossary:keywords]]//
when communicating between programs in different //[[CL:Glossary:packages]]//.  
For example, the mechanism for passing //[[CL:Glossary:keyword parameters]]// in a //[[CL:Glossary:call]]// uses 
//[[CL:Glossary:keywords]]//\meaning{1} to name the corresponding //[[CL:Glossary:arguments]]//;
\seesection\OrdinaryLambdaLists.

//[[CL:Glossary:Symbols]]// in \thepackage{keyword} are, by definition, \oftype{keyword}.

\beginsubsubsubsection{Interning a Symbol in the KEYWORD Package}

\Thepackage{keyword} is treated differently than other //[[CL:Glossary:packages]]//
in that special actions are taken when a //[[CL:Glossary:symbol]]// is //[[CL:Glossary:interned]]// in it.
In particular, when a //[[CL:Glossary:symbol]]// is //[[CL:Glossary:interned]]// in \thepackage{keyword},
 it is automatically made to be an //[[CL:Glossary:external symbol]]// 
and is automatically made to be a //[[CL:Glossary:constant variable]]// with itself as a //[[CL:Glossary:value]]//.





\endsubsubsubsection%{Interning a Symbol in the KEYWORD Package}

\beginsubsubsubsection{Notes about The KEYWORD Package}






It is generally best to confine the use of //[[CL:Glossary:keywords]]// to situations in which
there are a finitely enumerable set of names to be selected between.  For example,
if there were two states of a light switch, they might be called **'':on''** and **'':off''**.

In situations where the set of names is not finitely enumerable
(\ie where name conflicts might arise)
it is frequently best to use //[[CL:Glossary:symbols]]// in some //[[CL:Glossary:package]]//
other than \packref{keyword} so that conflicts will be naturally avoided.
For example, it is generally not wise for a //[[CL:Glossary:program]]// to use a //[[CL:Glossary:keyword]]//\meaning{1} 
as a //[[CL:Glossary:property indicator]]//, since if there were ever another //[[CL:Glossary:program]]//
that did the same thing, each would clobber the other's data.

\endsubsubsubsection%{Notes about The KEYWORD Package}

\endsubsubsection%{The KEYWORD Package}

\beginsubsubsection{Implementation-Defined Packages} 

\issue{PACKAGE-CLUTTER:REDUCE}
Other, //[[CL:Glossary:implementation-defined]]// //[[CL:Glossary:packages]]// might be present
in the initial \clisp\ environment.





It is recommended, but not required, that the documentation for a
//[[CL:Glossary:conforming implementation]]// contain a full list of all //[[CL:Glossary:package]]// names
initially present in that //[[CL:Glossary:implementation]]// but not specified in this specification.
(See also the //[[CL:Glossary:function]]// **[[CL:Functions:list-all-packages]]**.)

\endsubsubsection%{Implementation-Defined Packages} 

\endsubSection%{Standardized Packages}

