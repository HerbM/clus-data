

\beginsubsection{Array Elements}
\DefineSection{ArrayElements}

An //[[CL:Glossary:array]]// contains a set of //[[CL:Glossary:objects]]// called //[[CL:Glossary:elements]]//
that can be referenced individually according to a rectilinear coordinate system.

\beginsubsubsection{Array Indices}


An //[[CL:Glossary:array]]// //[[CL:Glossary:element]]// is referred to by a (possibly empty) series of indices.
The length of the series must equal the //[[CL:Glossary:rank]]// of the //[[CL:Glossary:array]]//.
\issue{ARRAY-DIMENSION-LIMIT-IMPLICATIONS:ALL-FIXNUM}
Each index must be a non-negative //[[CL:Glossary:fixnum]]// 


less than the corresponding //[[CL:Glossary:array]]// //[[CL:Glossary:dimension]]//.
//[[CL:Glossary:Array]]// indexing is zero-origin.

\endsubsubsection%{Array Indices}

\beginsubsubsection{Array Dimensions}

An axis of an //[[CL:Glossary:array]]// is called a //[[CL:Glossary:dimension]]//.

Each //[[CL:Glossary:dimension]]// is a non-negative 
\issue{ARRAY-DIMENSION-LIMIT-IMPLICATIONS:ALL-FIXNUM}
//[[CL:Glossary:fixnum]]//;

 if any dimension of an //[[CL:Glossary:array]]// is zero, the //[[CL:Glossary:array]]// has no elements.


It is permissible for a //[[CL:Glossary:dimension]]// to be zero, 
in which case the //[[CL:Glossary:array]]// has no elements, 
and any attempt to //[[CL:Glossary:access]]// an //[[CL:Glossary:element]]//
is an error.  However, other properties of the //[[CL:Glossary:array]]//,  
such as the //[[CL:Glossary:dimensions]]// themselves, may be used.

\beginsubsubsubsection{Implementation Limits on Individual Array Dimensions}

An //[[CL:Glossary:implementation]]// may impose a limit on //[[CL:Glossary:dimensions]]// of an //[[CL:Glossary:array]]//,
but there is a minimum requirement on that limit.  \Seevar{array-dimension-limit}.

\endsubsubsubsection%{Implementation Limits on Individual Array Dimensions}

\endsubsubsection%{Array Dimensions}

\beginsubsubsection{Array Rank}




An //[[CL:Glossary:array]]// can have any number of //[[CL:Glossary:dimensions]]// (including zero).
The number of //[[CL:Glossary:dimensions]]// is called the //[[CL:Glossary:rank]]//.

If the rank of an //[[CL:Glossary:array]]// is zero then the //[[CL:Glossary:array]]// is said to have
no //[[CL:Glossary:dimensions]]//, and the product of the dimensions (see **[[CL:Functions:array-total-size]]**)
is then 1; a zero-rank //[[CL:Glossary:array]]// therefore has a single element.

\beginsubsubsubsection{Vectors}

An //[[CL:Glossary:array]]// of //[[CL:Glossary:rank]]// one (\ie a one-dimensional //[[CL:Glossary:array]]//)
is called a //[[CL:Glossary:vector]]//.

\beginsubsubsubsubsection{Fill Pointers}

A //[[CL:Glossary:fill pointer]]// is a non-negative //[[CL:Glossary:integer]]// no
larger than the total number of //[[CL:Glossary:elements]]// in a //[[CL:Glossary:vector]]//.
Not all //[[CL:Glossary:vectors]]// have //[[CL:Glossary:fill pointers]]//.
\Seefuns{make-array} and **[[CL:Functions:adjust-array]]**.

An //[[CL:Glossary:element]]// of a //[[CL:Glossary:vector]]// is said to be //[[CL:Glossary:active]]// if it has
an index that is greater than or equal to zero, 
but less than the //[[CL:Glossary:fill pointer]]// (if any).
For an //[[CL:Glossary:array]]// that has no //[[CL:Glossary:fill pointer]]//,
all //[[CL:Glossary:elements]]// are considered //[[CL:Glossary:active]]//.


Only //[[CL:Glossary:vectors]]// may have //[[CL:Glossary:fill pointers]]//; 
multidimensional //[[CL:Glossary:arrays]]// may not.
A multidimensional //[[CL:Glossary:array]]// that is displaced to a //[[CL:Glossary:vector]]// 
that has a //[[CL:Glossary:fill pointer]]// can be created.

\endsubsubsubsubsection%{Fill Pointers}

\endsubsubsubsection%{Vectors}

\beginsubsubsubsection{Multidimensional Arrays}

\beginsubsubsubsubsection{Storage Layout for Multidimensional Arrays}


Multidimensional //[[CL:Glossary:arrays]]// store their components in row-major order;
that is, internally a multidimensional //[[CL:Glossary:array]]// is stored as a
one-dimensional //[[CL:Glossary:array]]//, with the multidimensional index sets
ordered lexicographically, last index varying fastest.  
 
\endsubsubsubsubsection%{Storage Layout for Multidimensional Arrays}

\beginsubsubsubsubsection{Implementation Limits on Array Rank}

An //[[CL:Glossary:implementation]]// may impose a limit on the //[[CL:Glossary:rank]]// of an //[[CL:Glossary:array]]//,
but there is a minimum requirement on that limit.  \Seevar{array-rank-limit}.

\endsubsubsubsubsection%{Implementation Limits on Array Rank}

\endsubsubsubsection%{Multidimensional Arrays}

\endsubsubsection%{Array Rank}

\endsubsection%{Array Elements}

\beginsubsection{Specialized Arrays}


An //[[CL:Glossary:array]]// can be a //[[CL:Glossary:general]]// //[[CL:Glossary:array]]//, 
    meaning each //[[CL:Glossary:element]]// may be any //[[CL:Glossary:object]]//,
or it may be a //[[CL:Glossary:specialized]]// //[[CL:Glossary:array]]//,
    meaning that each //[[CL:Glossary:element]]// must be of a restricted //[[CL:Glossary:type]]//.

The phrasing "an //[[CL:Glossary:array]]// //[[CL:Glossary:specialized]]// to //[[CL:Glossary:type]]// \metavar{type}"
is sometimes used to emphasize the //[[CL:Glossary:element type]]// of an //[[CL:Glossary:array]]//.
This phrasing is tolerated even when the \metavar{type} is \typeref{t},
even though an //[[CL:Glossary:array]]// //[[CL:Glossary:specialized]]// to //[[CL:Glossary:type]]// //[[CL:Glossary:t]]//
is a //[[CL:Glossary:general]]// //[[CL:Glossary:array]]//, not a //[[CL:Glossary:specialized]]// //[[CL:Glossary:array]]//.

\Thenextfigure\ lists some //[[CL:Glossary:defined names]]// that are applicable to //[[CL:Glossary:array]]// 
creation, //[[CL:Glossary:access]]//, and information operations.



\displaythree{General Purpose Array-Related Defined Names}{
adjust-array&array-has-fill-pointer-p&make-array\cr
adjustable-array-p&array-in-bounds-p&svref\cr
aref&array-rank&upgraded-array-element-type\cr
array-dimension&array-rank-limit&upgraded-complex-part-type\cr
array-dimension-limit&array-row-major-index&vector\cr
array-dimensions&array-total-size&vector-pop\cr
array-displacement&array-total-size-limit&vector-push\cr
array-element-type&fill-pointer&vector-push-extend\cr
}

\beginsubsubsection{Array Upgrading}
\DefineSection{ArrayUpgrading}

\issue{ARRAY-TYPE-ELEMENT-TYPE-SEMANTICS:UNIFY-UPGRADING}





The //[[CL:Glossary:upgraded array element type]]// of a //[[CL:Glossary:type]]// $T\sub 1$
is a //[[CL:Glossary:type]]// $T\sub 2$ that is a //[[CL:Glossary:supertype]]// of $T\sub 1$
and that is used instead of $T\sub 1$ whenever $T\sub 1$
is used as an //[[CL:Glossary:array element type]]// 
for object creation or type discrimination.

During creation of an //[[CL:Glossary:array]]//,
the //[[CL:Glossary:element type]]// that was requested 
is called the //[[CL:Glossary:expressed array element type]]//.
The //[[CL:Glossary:upgraded array element type]]// of the //[[CL:Glossary:expressed array element type]]//
becomes the //[[CL:Glossary:actual array element type]]// of the //[[CL:Glossary:array]]// that is created.


//[[CL:Glossary:Type]]// //[[CL:Glossary:upgrading]]// implies a movement upwards in the type hierarchy lattice.
A //[[CL:Glossary:type]]// is always a //[[CL:Glossary:subtype]]// of its //[[CL:Glossary:upgraded array element type]]//.
Also, if a //[[CL:Glossary:type]]// $T\sub x$ is a //[[CL:Glossary:subtype]]// of another //[[CL:Glossary:type]]// $T\sub y$,
then
the //[[CL:Glossary:upgraded array element type]]// of $T\sub x$ 
must be a //[[CL:Glossary:subtype]]// of
the //[[CL:Glossary:upgraded array element type]]// of $T\sub y$.
Two //[[CL:Glossary:disjoint]]// //[[CL:Glossary:types]]// can be //[[CL:Glossary:upgraded]]// to the same //[[CL:Glossary:type]]//.

The //[[CL:Glossary:upgraded array element type]]// $T\sub 2$ of a //[[CL:Glossary:type]]// $T\sub 1$
is a function only of $T\sub 1$ itself;
that is, it is independent of any other property of the //[[CL:Glossary:array]]// 
for which $T\sub 2$ will be used,
such as //[[CL:Glossary:rank]]//, //[[CL:Glossary:adjustability]]//, //[[CL:Glossary:fill pointers]]//, or displacement.




\Thefunction{upgraded-array-element-type} 
can be used by //[[CL:Glossary:conforming programs]]// to predict how the //[[CL:Glossary:implementation]]//
will //[[CL:Glossary:upgrade]]// a given //[[CL:Glossary:type]]//.



\endsubsubsection%{Array Upgrading}

\beginsubsubsection{Required Kinds of Specialized Arrays}
\DefineSection{RequiredSpecializedArrays}


//[[CL:Glossary:Vectors]]// whose //[[CL:Glossary:elements]]// are restricted to //[[CL:Glossary:type]]//
\issue{CHARACTER-PROPOSAL:2-3-2}
\typeref{character} or a //[[CL:Glossary:subtype]]// of \typeref{character}

are called \newtermidx{strings}{string}. 
//[[CL:Glossary:Strings]]// are \oftype{string}.


\Thenextfigure\ lists some //[[CL:Glossary:defined names]]// related to //[[CL:Glossary:strings]]//.

//[[CL:Glossary:Strings]]// are //[[CL:Glossary:specialized]]// //[[CL:Glossary:arrays]]// 
and might logically have been included in this chapter.
However, for purposes of readability
most information about //[[CL:Glossary:strings]]// does not appear in this chapter;
see instead \chapref\Strings.






\displaythree{Operators that Manipulate Strings}{
char&string-equal&string-upcase\cr
make-string&string-greaterp&string{\tt /=}\cr
nstring-capitalize&string-left-trim&string{\tt <}\cr
nstring-downcase&string-lessp&string{\tt <=}\cr
nstring-upcase&string-not-equal&string{\tt =}\cr
schar&string-not-greaterp&string{\tt >}\cr
string&string-not-lessp&string{\tt >=}\cr
string-capitalize&string-right-trim&\cr
string-downcase&string-trim&\cr
}

//[[CL:Glossary:Vectors]]// whose //[[CL:Glossary:elements]]// are restricted to //[[CL:Glossary:type]]//
\typeref{bit} are called \newtermidx{bit vectors}{bit vector}.
//[[CL:Glossary:Bit vectors]]// are \oftype{bit-vector}.
\Thenextfigure\ lists some //[[CL:Glossary:defined names]]// for operations on //[[CL:Glossary:bit arrays]]//.

\displaythree{Operators that Manipulate Bit Arrays}{
bit&bit-ior&bit-orc2\cr
bit-and&bit-nand&bit-xor\cr
bit-andc1&bit-nor&sbit\cr
bit-andc2&bit-not&\cr
bit-eqv&bit-orc1&\cr
}

\endsubsubsection%{Required Kinds of Specialized Arrays}

\endsubsection%{Specialized Arrays}
