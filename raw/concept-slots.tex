

\beginsubSection{Introduction to Slots}
                     An //[[CL:Glossary:object]]// \ofmetaclass{standard-class} has zero or more named //[[CL:Glossary:slots]]//.  The //[[CL:Glossary:slots]]// of an //[[CL:Glossary:object]]// are determined  by the //[[CL:Glossary:class]]// of the //[[CL:Glossary:object]]//.  Each //[[CL:Glossary:slot]]// can hold one value. \reviewer{Barmar: All symbols are valid variable names.  Perhaps this means
                  to preclude the use of named constants?  We have a terminology 		  problem to solve.}%!!! The //[[CL:Glossary:name]]// of a //[[CL:Glossary:slot]]// is a //[[CL:Glossary:symbol]]// that is syntactically valid for use as a variable name.

When a //[[CL:Glossary:slot]]// does not have a value, the //[[CL:Glossary:slot]]// is said to be  //[[CL:Glossary:unbound]]//.  When an unbound //[[CL:Glossary:slot]]// is read, \reviewer{Barmar: from an object whose metaclass is standard-class?} the //[[CL:Glossary:generic function]]// **[[CL:Functions:slot-unbound]]** is invoked. The  system-supplied primary //[[CL:Glossary:method]]//  for **[[CL:Functions:slot-unbound]]** 

on //[[CL:Glossary:class]]// \typeref{t} signals an error. \issue{SLOT-MISSING-VALUES:SPECIFY} If **[[CL:Functions:slot-unbound]]** returns, its //[[CL:Glossary:primary value]]//  is used that time as the //[[CL:Glossary:value]]// of the //[[CL:Glossary:slot]]//.

The default initial value form for a //[[CL:Glossary:slot]]// is defined by the **'':initform''** slot option.  When the **'':initform''** form is used to supply a value, it is evaluated in the lexical environment in which the \macref{defclass} form was evaluated. The **'':initform''** along with the lexical environment in which the \macref{defclass} form was evaluated is called a //[[CL:Glossary:captured initialization form]]//.  For more details, \seesection\ObjectCreationAndInit.
              A //[[CL:Glossary:local slot]]// is defined to be a //[[CL:Glossary:slot]]// that is

//[[CL:Glossary:accessible]]// to exactly one //[[CL:Glossary:instance]]//,  namely the one in which the //[[CL:Glossary:slot]]// is allocated.   A //[[CL:Glossary:shared slot]]// is defined to be a //[[CL:Glossary:slot]]// that is visible to more than one //[[CL:Glossary:instance]]// of a given //[[CL:Glossary:class]]// and its //[[CL:Glossary:subclasses]]//.

A //[[CL:Glossary:class]]// is said to define a //[[CL:Glossary:slot]]// with a given //[[CL:Glossary:name]]// when the \macref{defclass} form for that //[[CL:Glossary:class]]// contains a //[[CL:Glossary:slot specifier]]// with that //[[CL:Glossary:name]]//.  Defining a //[[CL:Glossary:local slot]]// does not immediately create  a //[[CL:Glossary:slot]]//; it causes a //[[CL:Glossary:slot]]// to be created each time  an //[[CL:Glossary:instance]]// of the //[[CL:Glossary:class]]// is created. Defining a //[[CL:Glossary:shared slot]]// immediately creates a //[[CL:Glossary:slot]]//.
                                                     The **'':allocation''** slot option to \macref{defclass} controls the kind of //[[CL:Glossary:slot]]// that is defined.  If the value of the **'':allocation''** slot option is **'':instance''**, a //[[CL:Glossary:local slot]]// is created.  If the value of **'':allocation''** is **'':class''**, a //[[CL:Glossary:shared slot]]// is created.

A //[[CL:Glossary:slot]]// is said to be //[[CL:Glossary:accessible]]// in an //[[CL:Glossary:instance]]//  of a //[[CL:Glossary:class]]// if the //[[CL:Glossary:slot]]// is defined by the //[[CL:Glossary:class]]//  of the //[[CL:Glossary:instance]]// or is inherited from a //[[CL:Glossary:superclass]]// of that //[[CL:Glossary:class]]//.   At most one //[[CL:Glossary:slot]]// of a given //[[CL:Glossary:name]]// can be //[[CL:Glossary:accessible]]// in an //[[CL:Glossary:instance]]//.   A //[[CL:Glossary:shared slot]]// defined by a //[[CL:Glossary:class]]// is //[[CL:Glossary:accessible]]// in all //[[CL:Glossary:instances]]//  of that //[[CL:Glossary:class]]//.   A detailed explanation of the inheritance of //[[CL:Glossary:slots]]// is given in  \secref\SlotInheritance.

\endsubSection%{Slots} \beginsubSection{Accessing Slots}

//[[CL:Glossary:Slots]]// can be //[[CL:Glossary:accessed]]// in two ways: by use of the primitive function **[[CL:Functions:slot-value]]** and by use of //[[CL:Glossary:generic functions]]// generated by the \macref{defclass} form.

\Thefunction{slot-value} can be used with any of the //[[CL:Glossary:slot]]// names specified in the \macref{defclass} form to //[[CL:Glossary:access]]// a specific //[[CL:Glossary:slot]]// //[[CL:Glossary:accessible]]// in an //[[CL:Glossary:instance]]// of the given //[[CL:Glossary:class]]//.

The macro \macref{defclass} provides syntax for generating //[[CL:Glossary:methods]]// to read and write //[[CL:Glossary:slots]]//.  If a reader //[[CL:Glossary:method]]// is requested,  a //[[CL:Glossary:method]]// is automatically generated for reading the value of the //[[CL:Glossary:slot]]//, but no //[[CL:Glossary:method]]// for storing a value into it is generated. If a writer //[[CL:Glossary:method]]// is requested, a //[[CL:Glossary:method]]// is automatically  generated for storing a value into the //[[CL:Glossary:slot]]//, but no //[[CL:Glossary:method]]//  for reading its value is generated.  If an accessor //[[CL:Glossary:method]]// is  requested, a //[[CL:Glossary:method]]// for reading the value of the //[[CL:Glossary:slot]]// and a //[[CL:Glossary:method]]// for storing a value into the //[[CL:Glossary:slot]]// are automatically generated.  Reader and writer //[[CL:Glossary:methods]]// are implemented using **[[CL:Functions:slot-value]]**.

When a reader or writer //[[CL:Glossary:method]]// is specified for a //[[CL:Glossary:slot]]//, the name of the //[[CL:Glossary:generic function]]// to which the generated //[[CL:Glossary:method]]// belongs is directly specified.  If the //[[CL:Glossary:name]]// specified for the writer //[[CL:Glossary:method]]// is the symbol \f{name}, the //[[CL:Glossary:name]]// of the //[[CL:Glossary:generic function]]// for writing the //[[CL:Glossary:slot]]// is the symbol \f{name}, and the //[[CL:Glossary:generic function]]// takes two arguments: the new value and the //[[CL:Glossary:instance]]//, in that order.  If the //[[CL:Glossary:name]]// specified for the accessor //[[CL:Glossary:method]]// is the symbol \f{name}, the //[[CL:Glossary:name]]// of the //[[CL:Glossary:generic function]]// for reading the //[[CL:Glossary:slot]]// is the symbol  \f{name}, and the //[[CL:Glossary:name]]// of the //[[CL:Glossary:generic function]]// for writing  the //[[CL:Glossary:slot]]// is the list \f{(setf name)}.

A //[[CL:Glossary:generic function]]// created or modified by supplying **'':reader''**, **'':writer''**, or **'':accessor''** //[[CL:Glossary:slot]]// options can be treated exactly as an ordinary //[[CL:Glossary:generic function]]//.
            Note that **[[CL:Functions:slot-value]]** can be used to read or write the value of a //[[CL:Glossary:slot]]// whether or not reader or writer //[[CL:Glossary:methods]]// exist for that //[[CL:Glossary:slot]]//.  When **[[CL:Functions:slot-value]]** is used, no reader or writer //[[CL:Glossary:methods]]// are invoked.

The macro \macref{with-slots} can be used to establish a  //[[CL:Glossary:lexical environment]]// in which specified //[[CL:Glossary:slots]]// are lexically available as if they were variables.  The macro \macref{with-slots}  invokes \thefunction{slot-value} to //[[CL:Glossary:access]]// the specified //[[CL:Glossary:slots]]//.

The macro \macref{with-accessors} can be used to establish a lexical environment in which specified //[[CL:Glossary:slots]]// are lexically available through their accessors as if they were variables.  The macro \macref{with-accessors} invokes the appropriate accessors to //[[CL:Glossary:access]]// the specified //[[CL:Glossary:slots]]//. 

\endsubSection%{Accessing Slots} \beginsubsection{Inheritance of Slots and Slot Options} \DefineSection{SlotInheritance}

The set of the //[[CL:Glossary:names]]// of all //[[CL:Glossary:slots]]// //[[CL:Glossary:accessible]]//  in an //[[CL:Glossary:instance]]// of a //[[CL:Glossary:class]]// $C$ is the union of  the sets of //[[CL:Glossary:names]]// of //[[CL:Glossary:slots]]// defined by $C$ and its //[[CL:Glossary:superclasses]]//. The structure of an //[[CL:Glossary:instance]]// is the set of //[[CL:Glossary:names]]// of //[[CL:Glossary:local slots]]// in that //[[CL:Glossary:instance]]//.

In the simplest case, only one //[[CL:Glossary:class]]// among $C$ and its //[[CL:Glossary:superclasses]]// defines a //[[CL:Glossary:slot]]// with a given //[[CL:Glossary:slot]]// name.   If a //[[CL:Glossary:slot]]// is defined by a //[[CL:Glossary:superclass]]// of $C$\negthinspace,  the //[[CL:Glossary:slot]]// is said to be inherited.  The characteristics  of the //[[CL:Glossary:slot]]// are determined by the //[[CL:Glossary:slot specifier]]// of the defining //[[CL:Glossary:class]]//. Consider the defining //[[CL:Glossary:class]]// for a slot $S$\negthinspace.  If the value of the **'':allocation''**  slot option is **'':instance''**, then $S$ is a //[[CL:Glossary:local slot]]// and each  //[[CL:Glossary:instance]]// of $C$ has its own //[[CL:Glossary:slot]]// named $S$ that stores its own value.  If the value of the **'':allocation''** slot  option is **'':class''**, then $S$ is a //[[CL:Glossary:shared slot]]//, the //[[CL:Glossary:class]]//  that defined $S$ stores the value, and all //[[CL:Glossary:instances]]// of $C$ can //[[CL:Glossary:access]]// that single //[[CL:Glossary:slot]]//.   If the **'':allocation''** slot option is omitted, **'':instance''** is used.

In general, more than one //[[CL:Glossary:class]]// among $C$ and its  //[[CL:Glossary:superclasses]]// can define a //[[CL:Glossary:slot]]// with a given //[[CL:Glossary:name]]//.   In such cases, only one //[[CL:Glossary:slot]]// with the given name is //[[CL:Glossary:accessible]]// in an //[[CL:Glossary:instance]]//  of $C$\negthinspace, and the characteristics of that //[[CL:Glossary:slot]]// are  a combination of the several //[[CL:Glossary:slot]]// specifiers, computed as follows:

\beginlist

\itemitem{\bull} All the //[[CL:Glossary:slot specifiers]]// for a given //[[CL:Glossary:slot]]// name are ordered from most specific to least specific, according to the order in $C$'s //[[CL:Glossary:class precedence list]]// of the //[[CL:Glossary:classes]]// that define them. All references to the specificity of //[[CL:Glossary:slot specifiers]]// immediately below refers to this ordering.

\itemitem{\bull} The allocation of a //[[CL:Glossary:slot]]// is controlled by the most  specific //[[CL:Glossary:slot specifier]]//.  If the most specific //[[CL:Glossary:slot specifier]]//  does not contain an **'':allocation''** slot option, **'':instance''** is used. Less specific //[[CL:Glossary:slot specifiers]]// do not affect the allocation.

\itemitem{\bull} The default initial value form for a //[[CL:Glossary:slot]]//  is the value of the **'':initform''** slot option in the most specific //[[CL:Glossary:slot specifier]]// that contains one.  If no //[[CL:Glossary:slot specifier]]// contains an **'':initform''** slot option, the //[[CL:Glossary:slot]]//  has no default initial value form.

\itemitem{\bull} The contents of a //[[CL:Glossary:slot]]// will always be of type  \f{(and $T\sub 1$ $\ldots$ $T\sub n$)} where $T\sub 1 \ldots T\sub n$ are the values of the **'':type''** slot options contained in all of the //[[CL:Glossary:slot specifiers]]//.  If no //[[CL:Glossary:slot specifier]]// contains the **'':type''** slot option, the contents of the //[[CL:Glossary:slot]]// will always be  \oftype{t}. The consequences of attempting to store in a //[[CL:Glossary:slot]]// a value that does not satisfy the //[[CL:Glossary:type]]// of the //[[CL:Glossary:slot]]// are undefined.

\itemitem{\bull} The set of initialization arguments that initialize a  given //[[CL:Glossary:slot]]// is the union of the initialization arguments declared in the **'':initarg''** slot options in all the //[[CL:Glossary:slot specifiers]]//.

\itemitem{\bull} The //[[CL:Glossary:documentation string]]// for a //[[CL:Glossary:slot]]// is the value of the **'':documentation''** slot option in the most specific //[[CL:Glossary:slot]]// specifier that contains one.  If no //[[CL:Glossary:slot specifier]]// contains a **'':documentation''** slot option, the //[[CL:Glossary:slot]]// has no //[[CL:Glossary:documentation string]]//.

\endlist

A consequence of the allocation rule is that a //[[CL:Glossary:shared slot]]// can be //[[CL:Glossary:shadowed]]//.  For example, if a class $C\sub 1$ defines  a //[[CL:Glossary:slot]]// named $S$ whose value for the **'':allocation''** slot option is **'':class''**, that //[[CL:Glossary:slot]]// is //[[CL:Glossary:accessible]]//  in //[[CL:Glossary:instances]]// of $C\sub 1$ and all of its //[[CL:Glossary:subclasses]]//.  However, if $C\sub 2$ is a //[[CL:Glossary:subclass]]//  of $C\sub 1$ and also defines a //[[CL:Glossary:slot]]// named $S$\negthinspace, $C\sub 1$'s  //[[CL:Glossary:slot]]// is not shared by //[[CL:Glossary:instances]]// of $C\sub 2$ and its //[[CL:Glossary:subclasses]]//. When a class $C\sub 1$ defines a //[[CL:Glossary:shared slot]]//, any subclass $C\sub 2$ of $C\sub 1$ will share this single //[[CL:Glossary:slot]]//  unless the \macref{defclass} form for $C\sub 2$ specifies a //[[CL:Glossary:slot]]// of the same  //[[CL:Glossary:name]]// or there is a //[[CL:Glossary:superclass]]// of $C\sub 2$ that precedes $C\sub 1$ in the //[[CL:Glossary:class precedence list]]// of $C\sub 2$ that defines a //[[CL:Glossary:slot]]// of the same name.

A consequence of the type rule is that the value of a //[[CL:Glossary:slot]]// satisfies the type constraint of each //[[CL:Glossary:slot specifier]]// that contributes to that //[[CL:Glossary:slot]]//.  Because the result of attempting to store in a //[[CL:Glossary:slot]]// a value that does not satisfy the type constraint for the //[[CL:Glossary:slot]]// is undefined, the value in a //[[CL:Glossary:slot]]// might fail to satisfy its type constraint.
      The **'':reader''**, **'':writer''**, and **'':accessor''** slot options create //[[CL:Glossary:methods]]// rather than define the characteristics of a //[[CL:Glossary:slot]]//. Reader and writer //[[CL:Glossary:methods]]// are inherited in the sense described in \secref\MethodInheritance.

//[[CL:Glossary:Methods]]// that //[[CL:Glossary:access]]// //[[CL:Glossary:slots]]// use only the name of the //[[CL:Glossary:slot]]// and the //[[CL:Glossary:type]]// of the //[[CL:Glossary:slot]]//'s value.  Suppose a //[[CL:Glossary:superclass]]// provides a //[[CL:Glossary:method]]// that expects to //[[CL:Glossary:access]]// a //[[CL:Glossary:shared slot]]// of a given //[[CL:Glossary:name]]//, and a //[[CL:Glossary:subclass]]// defines a //[[CL:Glossary:local slot]]// with the same //[[CL:Glossary:name]]//.  If the //[[CL:Glossary:method]]// provided  by the //[[CL:Glossary:superclass]]// is used on an //[[CL:Glossary:instance]]// of the //[[CL:Glossary:subclass]]//,  the //[[CL:Glossary:method]]// //[[CL:Glossary:accesses]]// the //[[CL:Glossary:local slot]]//.

\endsubsection%{Inheritance of Slots and Slot Options}

