

\beginsubsection{Loading}


To **[[CL:Functions:load]]** a //[[CL:Glossary:file]]// is to treat its contents as //[[CL:Glossary:code]]//
and //[[CL:Glossary:execute]]// that //[[CL:Glossary:code]]//.
The //[[CL:Glossary:file]]// may contain //[[CL:Glossary:source code]]// or //[[CL:Glossary:compiled code]]//.

A //[[CL:Glossary:file]]// containing //[[CL:Glossary:source code]]// is called a //[[CL:Glossary:source file]]//.
//[[CL:Glossary:Loading]]// a //[[CL:Glossary:source file]]// is accomplished essentially 
by sequentially //[[CL:Glossary:reading]]//\meaning{2} the //[[CL:Glossary:forms]]// in the file,
//[[CL:Glossary:evaluating]]// each immediately after it is //[[CL:Glossary:read]]//.


A //[[CL:Glossary:file]]// containing //[[CL:Glossary:compiled code]]// is called a //[[CL:Glossary:compiled file]]//.
//[[CL:Glossary:Loading]]// a //[[CL:Glossary:compiled file]]// is similar to //[[CL:Glossary:loading]]// a //[[CL:Glossary:source file]]//,
except that the //[[CL:Glossary:file]]// does not contain text but rather an
//[[CL:Glossary:implementation-dependent]]// representation of pre-digested //[[CL:Glossary:expressions]]//
created by the //[[CL:Glossary:compiler]]//.  Often, a //[[CL:Glossary:compiled file]]// can be //[[CL:Glossary:loaded]]//
more quickly than a //[[CL:Glossary:source file]]//.
\Seesection\Compilation.

The way in which a //[[CL:Glossary:source file]]// is distinguished from a //[[CL:Glossary:compiled file]]// 
is //[[CL:Glossary:implementation-dependent]]//.

\endsubsection%{Loading}

\beginsubsection{Features}
\DefineSection{Features}

A //[[CL:Glossary:feature]]// is an aspect or attribute
     of \clisp, 
     of the //[[CL:Glossary:implementation]]//,
  or of the //[[CL:Glossary:environment]]//.
A //[[CL:Glossary:feature]]// is identified by a //[[CL:Glossary:symbol]]//.

A //[[CL:Glossary:feature]]// is said to be //[[CL:Glossary:present]]// in a //[[CL:Glossary:Lisp image]]//
if and only if the //[[CL:Glossary:symbol]]// naming it is an //[[CL:Glossary:element]]// of the
//[[CL:Glossary:list]]// held by \thevariable{*features*}, 
which is called the //[[CL:Glossary:features list]]//.

\beginsubsubsection{Feature Expressions}
\DefineSection{FeatureExpressions}

Boolean combinations of //[[CL:Glossary:features]]//, called \newtermidx{feature expressions}{feature expression},
are used by the \f{\#+} and \f{\#-} //[[CL:Glossary:reader macros]]// in order to
direct conditional //[[CL:Glossary:reading]]// of //[[CL:Glossary:expressions]]// by the //[[CL:Glossary:Lisp reader]]//.

The rules for interpreting a //[[CL:Glossary:feature expression]]// are as follows:

\beginlist

\itemitem{//[[CL:Glossary:feature]]//}

If a //[[CL:Glossary:symbol]]// naming a //[[CL:Glossary:feature]]// is used as a //[[CL:Glossary:feature expression]]//,
the //[[CL:Glossary:feature expression]]// succeeds if that //[[CL:Glossary:feature]]// is //[[CL:Glossary:present]]//;
otherwise it fails.

\itemitem{\f{(not //feature-conditional//)}}

A \misc{not} //[[CL:Glossary:feature expression]]// succeeds 
if its argument //feature-conditional// fails;
otherwise, it succeeds.

\itemitem{\f{(and \starparam{feature-conditional})}}

An \misc{and} //[[CL:Glossary:feature expression]]// succeeds 
if all of its argument //feature-conditionals// succeed;
otherwise, it fails.

\itemitem{\f{(or \starparam{feature-conditional})}}

An \misc{or} //[[CL:Glossary:feature expression]]// succeeds 
if any of its argument //feature-conditionals// succeed;
otherwise, it fails.

\endlist

\beginsubsubsubsection{Examples of Feature Expressions}
\DefineSection{FeatureExpExamples}

For example, suppose that
 in //[[CL:Glossary:implementation]]// A, the //[[CL:Glossary:features]]// \f{spice} and \f{perq} are //[[CL:Glossary:present]]//,
			     but the //[[CL:Glossary:feature]]// \f{lispm} is not //[[CL:Glossary:present]]//;
 in //[[CL:Glossary:implementation]]// B, the feature \f{lispm} is //[[CL:Glossary:present]]//,
			     but the //[[CL:Glossary:features]]// \f{spice} and \f{perq} are
			      not //[[CL:Glossary:present]]//;
 and 
 in //[[CL:Glossary:implementation]]// C, none of the features \f{spice}, //[[CL:Glossary:lispm]]//, or \f{perq} are
			     //[[CL:Glossary:present]]//.
\Thenextfigure\ shows some sample //[[CL:Glossary:expressions]]//, and how they would be 
//[[CL:Glossary:read]]//\meaning{2} in these //[[CL:Glossary:implementations]]//.

\showtwo{Features examples}{
\f{(cons \#+spice "Spice" \#-spice "Lispm" x)} \span \cr
\noalign{\vskip 2pt}
\quad in //[[CL:Glossary:implementation]]// A $\ldots$ & \f{(CONS "Spice" X)} \cr
\quad in //[[CL:Glossary:implementation]]// B $\ldots$ & \f{(CONS "Lispm" X)} \cr
\quad in //[[CL:Glossary:implementation]]// C $\ldots$ & \f{(CONS "Lispm" X)} \cr
\noalign{\vskip 5pt}
\f{(cons \#+spice "Spice" \#+LispM "Lispm" x)} \span \cr
\noalign{\vskip 2pt}
\quad in //[[CL:Glossary:implementation]]// A $\ldots$ & \f{(CONS "Spice" X)} \cr
\quad in //[[CL:Glossary:implementation]]// B $\ldots$ & \f{(CONS "Lispm" X)} \cr
\quad in //[[CL:Glossary:implementation]]// C $\ldots$ & \f{(CONS X)} \cr
\noalign{\vskip 5pt}
\f{(setq a '(1 2 \#+perq 43 \#+(not perq) 27))} \span \cr
\noalign{\vskip 2pt}
\quad in //[[CL:Glossary:implementation]]// A $\ldots$ & \f{(SETQ A '(1 2 43))} \cr
\quad in //[[CL:Glossary:implementation]]// B $\ldots$ & \f{(SETQ A '(1 2 27))} \cr
\quad in //[[CL:Glossary:implementation]]// C $\ldots$ & \f{(SETQ A '(1 2 27))} \cr
\noalign{\vskip 5pt}
\f{(let ((a 3) \#+(or spice lispm) (b 3)) (foo a))} \span \cr
\noalign{\vskip 2pt}
\quad in //[[CL:Glossary:implementation]]// A $\ldots$ & \f{(LET ((A 3) (B 3)) (FOO A))} \cr
\quad in //[[CL:Glossary:implementation]]// B $\ldots$ & \f{(LET ((A 3) (B 3)) (FOO A))} \cr
\quad in //[[CL:Glossary:implementation]]// C $\ldots$ & \f{(LET ((A 3)) (FOO A))} \cr
\noalign{\vskip 5pt}
\f{(cons \#+Lispm "\#+Spice" \#+Spice "foo" \#-(or Lispm Spice) 7 x)} \span \cr
\noalign{\vskip 2pt}
\quad in //[[CL:Glossary:implementation]]// A $\ldots$ & \f{(CONS "foo" X)}      \cr
\quad in //[[CL:Glossary:implementation]]// B $\ldots$ & \f{(CONS "\#+Spice" X)} \cr
\quad in //[[CL:Glossary:implementation]]// C $\ldots$ & \f{(CONS 7 X)}          \cr
}
\endsubsubsubsection%{Examples of Feature Expressions}

\endsubsubsection%{Feature Expressions}

\endsubsection%{Features}
