\section{Previous work}

\subsection{ANSI CL Standard}

\subsubsection{Published standard}

The ANSI \cl{} standard\cite{ANSI:1994:standard} is the specification of the \cl{} language, published in 1994.

The specification itself is a written document of over one thousand pages of formatted text. Such a large amount of technical data was a natural candidate to be turned into a digital database browsable by humans.
\subsubsection{Derived work}
\paragraph{Common Lisp HyperSpec}
The most famous work derived from the ANSI CL standard is the \cl{} HyperSpec\footnote{\url{http://www.lispworks.com/documentation/common-lisp.html}} (henceforth abbreviated as CLHS). It is a hyperlinked Web version of the original standard, which allows for easy navigation of the standard. Its HTML form also allows searching and quick access using external search engines, such as Google or IRC\footnote{Internet Relay Chat.} bots.

The CLHS was itself created by an automated tool which converted TeX into HTML.

The CLHS is released under an essentially non-free license which allows verbatim copying of CLHS as a whole, but prohibits any changes to it or creating any derivative works based on it. It is therefore not possible to build a unified piece of \cl{} documentation based on the CLHS.

Further information about the history of creation of the standard and the CLHS is available in the work by Kent M. Pitman, \textit{Common Lisp: The Untold Story}\cite{kmp:2012:untold}.

\paragraph{Franz Online ANS}
Another notable work is the Franz Online ANS\footnote{\url{
http://franz.com/search/search-ansi-about.lhtml}}, created by Franz Inc. and also being ``a semi-mechanical translation of the original TeX into HTML''. The presentation of the language standard is copyrighted by Franz. I have not attempted to contact the owners of that presentation---I only learned of its existence very recently, at which point I had already done most of the parsing work.

\subsection{Lisp content aggregators}

The \cl{} documentation spans far and wide beyond the \cl{} standard. Even during the time of \cl{} standardization, many extensions to the language existed, with their respective pieces of documentation.

From this arose the obvious need of aggregating Lisp content, both with respect to code and documentation. Below, I will outline three contemporary services which provide Lisp users with content.

\subsubsection{l1sp.org}

The l1sp.org\footnote{\url{http://l1sp.org}} service is a content aggregation tool created by Zach Beane. Its main purpose is to enable lookup of symbols inside various pieces of documentation scattered around the web. It currently contains links to over 20 pieces of documentation, including the CLHS, documentation for various CL implementations and many commonly used language extensions and libraries, such as the Metaobject Protocol\footnote{\url{http://metamodular.com/CLOS-MOP/}}, ASDF\footnote{Another System Definition Facility, \url{https://common-lisp.net/project/asdf/}} and Alexandria\footnote{\url{https://common-lisp.net/project/alexandria/}}.

This redirection service is very useful and allows for easy lookup, but such an approach depends on the presence of all the pieces of documentation in their respective places all around the Web. Also, the pieces of documentation are not linked to each other; for example, it is impossible to reach the \cl{} reference from within the Metaobject Protocol reference and vice versa. The individual pieces of documentation also greatly vary in style: both the typesetting and graphical layout and the textual form in which the information is presented to the reader.

\subsubsection{Quicklisp}

\ql{}\footnote{\url{https://www.quicklisp.org/}}, created by Zach Beane, provides a centralized repository of Lisp libraries through a piece of Lisp code, which in turn allows the programmer to automatically resolve dependencies, download and compile a particular library on their Lisp system.

While \ql{} is invaluable as a library repository, it does not provide any sort of documentation service---it is outside the scope of the \ql{} project. Therefore it cannot be a direct aid in creating a Lisp documentation project.

\subsubsection{Quickdocs}

The Quickdocs service\footnote{\url{http://quickdocs.org/}} is a content aggregation tool created by Eitaro Fukamachi expressly for automated collection and generation of documentation for Common Lisp libraries; therefore, it aids the issue outlined in the paragraph above by expanding \ql{} with documentation capabilities. The documentation itself is generated automatically from the source code of libraries found in the \ql{} repositories. It consists of the description of a \ql{} system and a list of exported symbols along with the type of objects they refer to and any documentation strings they may contain.

Such automation provides a very good and aesthetically pleasing means of reading about the protocol of a given system. The issue with such automatic generation is that it forces the authors of libraries to follow a convention of documenting their libraries in a particular way, which must be recognizable by the tool parsing the \ql{} systems. Otherwise, the documentation will not be visible in Quickdocs. Additionally, \ql{} descriptions often contain little more than links to external websites documenting the code, which deprives Quickdocs of the ability to automatically generate documentation for it.
