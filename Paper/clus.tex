\documentclass{sig-alternate-05-2015}
\usepackage[utf8]{inputenc}

\def\inputfig#1{\input #1}
\def\inputtex#1{\input #1}
\def\inputal#1{\input #1}
\def\inputcode#1{\input #1}

\inputtex{logos.tex}
\inputtex{refmacros.tex}
\inputtex{other-macros.tex}

\begin{document}
\title{Common Lisp UltraSpec - A Project For \\ Modern Common Lisp Documentation}
\numberofauthors{1}
\author{\alignauthor
Michał Herda\\
\affaddr{Faculty of Mathematics and Computer Science}\\
\affaddr{Jagiellonian University}\\
\affaddr{ul. Łojasiewicza 6}\\
\affaddr{Kraków, Poland}
\email{phoe@openmailbox.org}}

\toappear{Permission to make digital or hard copies of all or part of
  this work for personal or classroom use is granted without fee
  provided that copies are not made or distributed for profit or
  commercial advantage and that copies bear this notice and the full
  citation on the first page. Copyrights for components of this work
  owned by others than the author(s) must be honored. Abstracting with
  credit is permitted. To copy otherwise, or republish, to post on
  servers or to redistribute to lists, requires prior specific
  permission and/or a fee. Request permissions from
  Permissions@acm.org.}

\maketitle

\begin{abstract}
The \cl{} programming language has many bodies of documentation---the language
specification \cite{ANSI:1994:standard} itself, language extensions and multiple libraries.
I outline issues with the current state of \cl{} documentation and propose an improvement
in form of \cl{} UltraSpec. It is a project of modernizing the \LaTeX sources of the draft
standard \cite{ANSI:1994:draft} of \cl{} and ultimately unifying it with other bodies of
\cl{} documentation.

This is achieved through semi-automatic parsing and formatting of the sources using a text
editor, regular expressions and other text processing tools. The processed text is then
displayed in a web browser by means of a wiki engine.

The project is currently in its early phase with a part of the specification parsed and
edited. A demo of the project is available.
\end{abstract}

\begin{CCSXML}
<ccs2012>
<concept>
<concept_id>10011007.10011074.10011111.10010913</concept_id>
<concept_desc>Software and its engineering~Documentation</concept_desc>
<concept_significance>500</concept_significance>
</concept>
</ccs2012>
\end{CCSXML}

\ccsdesc[500]{Software and its engineering~Documentation}

\printccsdesc

\keywords{\cl{}, Documentation, Specification}

\inputtex{sec-introduction.tex}
\inputtex{sec-previous-work.tex}
\inputtex{sec-my-work.tex}
\inputtex{sec-conclusions-and-future-work.tex}

\bibliographystyle{abbrv}
\bibliography{clus}
\end{document}
