\section{Conclusions and future work}

\subsection{Benefits/Disadvantages}

The benefits of my approach come as logical continuations of the slogans used in section \ref{requirements}.

The most obvious one, which is also the goal of the project, is the construction of a contemporary source of \cl{} documentation and a singular resource capable of containing most of the knowledge a \cl{} programmer might need.

Another upside is modernization of the specification by fixing its issues and bugs, expanding its examples sections, clarifying any inconsistencies and questions that have emerged since the creation of the standard and giving it a more aesthetically pleasing look.

A beneficial side effect of my approach is generation of a version of the \cl{} specification in a markup format. Such a format can then be easily parsed by automated tools to produce a document of any required typesetting qualities.

\rule{\linewidth}{0.5pt}

The disadvantages of my current approach occur on different layers.

First of all, it is easy to keep a single static website on the Web for years without any changes, but CLUS is far from static because of its design. The body of code that CLUS will turn into, as the time progresses, will require maintenance in order to stay clear and readable; it will require reviewers to check the input from anyone wanting to contribute to the CLUS repositories.

Second, although it does apply specifically to the dpANS sources, parsing and hyperlinking the chapters of the specification takes significant time. Additionally, because of the variety of forms other bodies of Lisp documentation have, it will be non-trivial to import them into CLUS - it will require separate effort to have them parsed and prepared for inclusion.

Third, the legal status and licensing issues of the various pieces of documentation will require separate thought. Creating a compilation work of all these elements will be essentially creating a derivative of them all and legal caution will need to be taken in case of documents with unknown or confusing legal status. It might be required to negotiate the terms of inclusion of particular pieces of work into CLUS with the respective holders of rights to them.

\subsubsection{Thoughts}

Among all the literature available for studying \cl{}, I would like to mention the dpANS source files as a valuable read from a non-technical point of view.

The standard was created before the era of ubiquitous versioning systems. Because of this, the draft source contains many comments, some of them timestamped. They show the technical problems and decisions the langauge specifiers faced and solved in the process of creating a formal standard for a programming language. They also outline the features which were deprecated and removed - or, on the contrary, created and added along the way, some of which I personally find quite enlightening. What I want to emphasize here, though, is that they show X3J13 as a group of human beings working on a common goal. The comments there show various aspects of their work: from communicating messages between particular people, through decision-making and commented-out pieces of specification itself, to the in-jokes and humor of the people.

In my opinion, studying the original sources for all three draft previews (all of which are available online) might be valuable for any person who wants to research specification development or software development in general from a more humane point of view as well as Lisp programmers who are interested in extending their background and the process through which \cl{} came to life.

\rule{\linewidth}{0.5pt}

Another thought that I would like to mention here is the fact that, in the beginning, I had imagined my work as simple translation of the sources from their \TeX{} format into wiki markup in order to let the DokuWiki engine format them into HTML. Reality has verified these ideas - I quickly realized that the standard itself has its share of inconsistencies, bugs and other issues. It is of course expected for such a huge body of documentation to have issues and these issues do not undermine the value of the specification as a whole, but I have unexpectedly found myself to be able to fix them as I progress through the sources.

Suddenly, from a simple translator, I had become an editor of the \cl{} standard itself. What I am creating right now is not the draft sources being translated into DokuWiki markup - it is an edited version which contains many improvements and fixes to many issues that were impossible to fix in the previous CL specifications based on the work of X3J13.

It is a very responsible role that has emerged - but also one that I consider very satisfying.

\subsubsection{Plans}

It is impossible to speak of future plans without mentioning the Lisp community here.

The \cl{} \us{} was meant from the start to be a community-based project, meaning that it belongs to the Lisp community and is meant to be utilized and expanded within it. I hope that other people will aid me in my process by suggesting changes, submitting patches, possibly integrating the documentation for respective \cl{} libraries into the code and maintaining them later on.

Once the specification is completely integrated, I intend on extending its scope to include common facilities and extensions included and/or used in most contemporary \cl{} implementations, such as the Metaobject Protocol, ASDF, \ql{} and the compatibility libraries which provide cross-platform functionalities not included in the standard such as concurrency or networking.

I want to create quality standards for the respective types of pages and enforce them in order to keep the quality of the documentation high and its style consistent across pages and modules.

\subsubsection{Acknowledgements}

I would like to thank professor Robert Strandh for general support during the creation of this work, including, but not limited to, proofreading, advice, moral support and allowing me to use his \TeX layout for papers.

I would additionally like to thank the people from the \texttt{\#lisp} IRC channel for a lot of support on this paper and the \cl{} \us{} project in general.
