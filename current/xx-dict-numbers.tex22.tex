====== Constant Variable PI ======

====Value====

an //[[CL:Glossary:implementation-dependent]]// //[[CL:Glossary:long float]]//.

====Description====

The best //[[CL:Glossary:long float]]// approximation to the mathematical constant ''\pi''.

====Examples====

<blockquote> ;; In each of the following computations, the precision depends ;; on the implementation. Also, if `long float' is treated by ;; the implementation as equivalent to some other float format ;; (e.g., `double float') the exponent marker might be the marker ;; for that equivalent (e.g., `D' instead of `L'). pi → 3.141592653589793L0 (cos pi) → -1.0L0

(defun sin-of-degrees (degrees) (let ((x (if (floatp degrees) degrees (float degrees pi)))) (sin (* x (/ (float pi x) 180))))) </blockquote>

====See Also====

None.

====Notes====

An approximation to ''\pi'' in some other precision can be obtained by writing ''(float pi x)'', where ''x'' is a //[[CL:Glossary:float]]// of the desired precision, or by writing ''(coerce pi ''type'')'', where ''type'' is the desired type, such as **[[CL:Types:short-float]]**.

