====== Special Form SETQ ======

====Syntax====
  * **setq** //{pair''*''}// → //result//
  
<blockquote>
//pair// ::= //var form//
</blockquote>

====Arguments and Values====
  * //var// - a //[[CL:Glossary:symbol]]// naming a //[[CL:Glossary:variable]]// other than a //[[CL:Glossary:constant variable]]//.
  * //form// - a //[[CL:Glossary:form]]//.
  * //result// - the //[[CL:Glossary:primary value]]// of the last //form//, or **[[CL:Constant Variables:nil]]** if no //pairs// were supplied.

====Description====
Assigns values to //[[CL:Glossary:variables]]//.

''(setq var1 form1 var2 form2 ...)'' is the simple variable assignment statement of Lisp. First //form1// is evaluated and the result is stored in the variable //var1//, then //form2// is evaluated and the result stored in //var2//, and so forth. **setq** may be used for assignment of both lexical and dynamic variables.

If any //var// refers to a //[[CL:Glossary:binding]]// made by **[[CL:Special Operators:symbol-macrolet]]**, then that //var// is treated as if **[[CL:Macros:setf]]** (not **setq**) had been used.

====Examples====
<blockquote>
;;; TODO condemn toplevel SETQ

:;; A simple use of SETQ to establish values for variables. 
(setq a 1 b 2 c 3) <r>3</r>
a <r>1 </r>
b <r>2</r>
c <r>3</r>

;:; Use of SETQ to update values by sequential assignment. 
(setq a (1+ b) b (1+ a) c (+ a b)) <r>7 a → 3 b → 4 c → 7</r>

;:; This illustrates the use of SETQ on a symbol macro. 
([[CL:Special Operators:let]] ((x ([[CL:Functions:list]] 10 20 30)))
  ([[CL:Special Operators:symbol-macrolet]] ((y ([[CL:Functions:car]] x)) (z ([[CL:Functions:cadr]] x))) 
    (setq y ([[CL:Functions:math-one-plus|1+]] z) z ([[CL:Functions:math-one-plus|1+]] y))
    ([[CL:Functions:list]] x y z))) <r>((21 22 30) 21 22)</r>
</blockquote>

====Side Effects====
The //[[CL:Glossary:primary value]]// of each //form// is assigned to the corresponding //var//.

====Affected By====
None.

====Exceptional Situations====
None.

====See Also====
**[[CL:Macros:psetq|Macro PSETQ]]**, **[[CL:Functions:set|Function SET]]**, **[[CL:Macros:setf|Macro SETF]]**

====Notes====
None.

\issue{SYMBOL-MACROLET-SEMANTICS:SPECIAL-FORM}
