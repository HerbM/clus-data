====== Macro DEFMETHOD ======


====Syntax====


\DefmacWithValuesNewline {defmethod} {\vtop{\hbox{''function-name'' \star{\curly{''method-qualifier''}} ''specialized-lambda-list''} \hbox{{\DeclsAndDoc} \starparam{form}}}} {new-method}

\Vskip1pc!\null ''function-name''::''='' \curly{//[[CL:Glossary:symbol]]// ''\vert'' **[[CL:Functions:(setf //[[CL:Glossary:symbol]]**)]]//} \Vskip1pc!\null ''method-qualifier''::''='' //[[CL:Glossary:non-list]]// \Vskip1pc!\null \settabs\+\hskip≤ftskip&\cr \+&''specialized-lambda-list''::''='' (&\star{\curly{//var// ''\vert'' {\rm (}{//var// ''parameter-specializer-name''}{\rm )}}} \cr \+&&\ttbrac{''&optional'' \star{\curly{//var// ''\vert'' {\rm (}var \ttbrac{//initform// {[//supplied-p-parameter//]} }{\rm )}}}} \cr \+&&\ttbrac{{\tt\&rest} //var//} \cr \+&&**[[CL:Functions:\lbracket]]**{\key{}}&\star{\curly{//var// ''\vert'' {\rm (}\curly{//var// ''\vert'' {\rm (}//[[CL:Glossary:keyword]]////var//{\rm )}} //[//initform// [//supplied-p-parameter//] ]//{\rm )}}}\cr \+&&&\brac{\keyref{allow-other-keys}} **[[CL:Functions:\rbracket]]** \cr \+&&\ttbrac{{\tt\&aux} \star{\curly{//var// ''\vert'' {\rm (}//var// [//initform//] {\rm )}}}} {\rm )} \cr \Vskip1pc!\null \+&''parameter-specializer-name''::''='' //[[CL:Glossary:symbol]]// ''\vert'' {\rm (}**[[CL:Functions:eql]]** //eql-specializer-form//{\rm )}\cr \Vskip 1pc!

====Arguments and Values====

//declaration// - a \misc{declare} //[[CL:Glossary:expression]]//; \noeval.

//documentation// - a //[[CL:Glossary:string]]//; \noeval.

//var// - a //[[CL:Glossary:variable]]// //[[CL:Glossary:name]]//.

//eql-specializer-form// - a //[[CL:Glossary:form]]//.

//Form// - a //[[CL:Glossary:form]]//.

//Initform// - a //[[CL:Glossary:form]]//.

//Supplied-p-parameter// - variable name.

//new-method// - the new //[[CL:Glossary:method]]// //[[CL:Glossary:object]]//.

====Description====

The macro **[[CL:Macros:defmethod]]** defines a //[[CL:Glossary:method]]// on a //[[CL:Glossary:generic function]]//.

If **[[CL:Functions:(fboundp ''function-name'')]]** is **[[CL:Constant Variables:nil]]**, a //[[CL:Glossary:generic function]]// is created with default values for the argument precedence order (each argument is more specific than the arguments to its right in the argument list), for the generic function class (\theclass{standard-generic-function}), for the method class (\theclass{standard-method}), and for the method combination type (the standard method combination type). The //[[CL:Glossary:lambda list]]// of the //[[CL:Glossary:generic function]]// is congruent with the //[[CL:Glossary:lambda list]]// of the //[[CL:Glossary:method]]// being defined; if the **[[CL:Macros:defmethod]]** form mentions keyword arguments, the //[[CL:Glossary:lambda list]]// of the //[[CL:Glossary:generic function]]// will mention **[[CL:Functions:&key]]** (but no keyword arguments). If ''function-name'' names an //[[CL:Glossary:ordinary function]]//, a //[[CL:Glossary:macro]]//, or a //[[CL:Glossary:special operator]]//, an error is signaled.

If a //[[CL:Glossary:generic function]]// is currently named by {\it function-name}, the //[[CL:Glossary:lambda list]]// of the //[[CL:Glossary:method]]// must be congruent with the //[[CL:Glossary:lambda list]]// of the //[[CL:Glossary:generic function]]//. If this condition does not hold, an error is signaled. For a definition of congruence in this context, see section {\secref\GFMethodLambdaListCongruency}.


Each ''method-qualifier'' argument is an //[[CL:Glossary:object]]// that is used by method combination to identify the given //[[CL:Glossary:method]]//. The method combination type might further restrict what a method //[[CL:Glossary:qualifier]]// can be. The standard method combination type allows for //[[CL:Glossary:unqualified methods]]// and //[[CL:Glossary:methods]]// whose sole //[[CL:Glossary:qualifier]]// is one of the keywords **'':before''**, **'':after''**, or **'':around''**.

The ''specialized-lambda-list'' argument is like an ordinary //[[CL:Glossary:lambda list]]// except that the //[[CL:Glossary:names]]// of required parameters can be replaced by specialized parameters. A specialized parameter is a list of the form ''(//var// ''parameter-specializer-name'')''. Only required parameters can be specialized. If ''parameter-specializer-name'' is a //[[CL:Glossary:symbol]]// it names a //[[CL:Glossary:class]]//; if it is a //[[CL:Glossary:list]]//, it is of the form ''(eql //eql-specializer-form//)''. The parameter specializer name ''(eql //eql-specializer-form//)'' indicates that the corresponding argument must be **[[CL:Functions:eql]]** to the //[[CL:Glossary:object]]// that is the value of //eql-specializer-form// for the //[[CL:Glossary:method]]// to be applicable.

The //eql-specializer-form// is evaluated at the time that the expansion of the **[[CL:Macros:defmethod]]** macro is evaluated.

If no //[[CL:Glossary:parameter specializer name]]// is specified for a given required parameter, the //[[CL:Glossary:parameter specializer]]// defaults to \theclass{t}. For further discussion, see section {\secref\IntroToMethods}.

The //form// arguments specify the method body. The body of the //[[CL:Glossary:method]]// is enclosed in an //[[CL:Glossary:implicit block]]//. If ''function-name'' is a //[[CL:Glossary:symbol]]//, this block bears the same //[[CL:Glossary:name]]// as the //[[CL:Glossary:generic function]]//. If ''function-name'' is a //[[CL:Glossary:list]]// of the form **[[CL:Functions:(setf ''symbol'')]]**, the //[[CL:Glossary:name]]// of the block is ''symbol''.

The //[[CL:Glossary:class]]// of the //[[CL:Glossary:method]]// //[[CL:Glossary:object]]// that is created is that given by the method class option of the //[[CL:Glossary:generic function]]// on which the //[[CL:Glossary:method]]// is defined.

If the //[[CL:Glossary:generic function]]// already has a //[[CL:Glossary:method]]// that agrees with the //[[CL:Glossary:method]]// being defined on //[[CL:Glossary:parameter specializers]]// and //[[CL:Glossary:qualifiers]]//, **[[CL:Macros:defmethod]]** replaces the existing //[[CL:Glossary:method]]// with the one now being defined. For a definition of agreement in this context. see section {\secref\SpecializerQualifierAgreement}.

The //[[CL:Glossary:parameter specializers]]// are derived from the //[[CL:Glossary:parameter specializer names]]// as described in \secref\IntroToMethods.

The expansion of the **[[CL:Macros:defmethod]]** macro "refers to" each specialized parameter (see the description of \declref{ignore} within the description of \misc{declare}). This includes parameters that have an explicit //[[CL:Glossary:parameter specializer name]]// of \t. This means that a compiler warning does not occur if the body of the //[[CL:Glossary:method]]// does not refer to a specialized parameter, while a warning might occur if the body of the //[[CL:Glossary:method]]// does not refer to an unspecialized parameter. For this reason, a parameter that specializes on \t\ is not quite synonymous with an unspecialized parameter in this context.


Declarations at the head of the method body that apply to the method's //[[CL:Glossary:lambda variables]]// are treated as //[[CL:Glossary:bound declarations]]// whose //[[CL:Glossary:scope]]// is the same as the corresponding //[[CL:Glossary:bindings]]//.

Declarations at the head of the method body that apply to the functional bindings of **[[CL:Functions:call-next-method]]** or **[[CL:Functions:next-method-p]]** apply to references to those functions within the method body //forms//. Any outer //[[CL:Glossary:bindings]]// of the //[[CL:Glossary:function names]]// **[[CL:Functions:call-next-method]]** and **[[CL:Functions:next-method-p]]**, and declarations associated with such //[[CL:Glossary:bindings]]// are //[[CL:Glossary:shadowed]]// within the method body //forms//.

The //[[CL:Glossary:scope]]// of //[[CL:Glossary:free declarations]]// at the head of the method body is the entire method body, which includes any implicit local function definitions but excludes //[[CL:Glossary:initialization forms]]// for the //[[CL:Glossary:lambda variables]]//.

**[[CL:Macros:defmethod]]** is not required to perform any compile-time side effects. In particular, the //[[CL:Glossary:methods]]// are not installed for invocation during compilation. An //[[CL:Glossary:implementation]]// may choose to store information about the //[[CL:Glossary:generic function]]// for the purposes of compile-time error-checking (such as checking the number of arguments on calls, or noting that a definition for the function name has been seen).

//Documentation// is attached as a //[[CL:Glossary:documentation string]]// to the //[[CL:Glossary:method]]// //[[CL:Glossary:object]]//.

====Examples====

None.

====Affected By====

The definition of the referenced //[[CL:Glossary:generic function]]//.

====Exceptional Situations====

If ''function-name'' names an //[[CL:Glossary:ordinary function]]//, a //[[CL:Glossary:macro]]//, or a //[[CL:Glossary:special operator]]//, an error of type **[[CL:Types:error]]** is signaled.

If a //[[CL:Glossary:generic function]]// is currently named by {\it function-name}, the //[[CL:Glossary:lambda list]]// of the //[[CL:Glossary:method]]// must be congruent with the //[[CL:Glossary:lambda list]]// of the //[[CL:Glossary:generic function]]//, or an error of type **[[CL:Types:error]]** is signaled.


====See Also====

**[[CL:Macros:defgeneric]]**, **[[CL:Functions:documentation]]**, {\secref\IntroToMethods}, {\secref\GFMethodLambdaListCongruency}, {\secref\SpecializerQualifierAgreement}, {\secref\DocVsDecls}

====Notes====

None.


\issue{DECLS-AND-DOC} \issue{DEFMETHOD-DECLARATION-SCOPE:CORRESPONDS-TO-BINDINGS} \issue{COMPILE-FILE-HANDLING-OF-TOP-LEVEL-FORMS:CLARIFY} \issue{DOCUMENTATION-FUNCTION-BUGS:FIX}
