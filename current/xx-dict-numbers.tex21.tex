====== Function ASIN, ACOS, ATAN ======

====Syntax====

**asin** //number// → //radians// **acos** //number// → //radians// **atan {number1** //\opt} number2// → //radians//

====Arguments and Values====

//number// - a //[[CL:Glossary:number]]//.

//number1// - a //[[CL:Glossary:number]]// if //number2// is not supplied, or a //[[CL:Glossary:real]]// if //number2// is supplied.

//number2// - a //[[CL:Glossary:real]]//.

//radians// - a //[[CL:Glossary:number]]// (of radians).

====Description====

**[[CL:Functions:asin]]**, **[[CL:Functions:acos]]**, and **[[CL:Functions:atan]]** compute the arc sine, arc cosine, and arc tangent respectively.

The arc sine, arc cosine, and arc tangent (with only //number1// supplied) functions can be defined mathematically for //number// or //number1// specified as ''x'' as in \thenextfigure.

\tablefigtwo{Mathematical definition of arc sine, arc cosine, and arc tangent} {Function}{Definition}{ Arc sine & '' -i\ \ff{log} \bigl(ix+ \sqrt{1-x^2} \bigr) '' \cr

Arc cosine & '' (\pi/2) - \ff{arcsin} x '' \cr Arc tangent & '' -i\ \ff{log} \bigl((1+ix)\ \sqrt{1/(1+x^2)} \bigr) '' \cr }

These formulae are mathematically correct, assuming completely accurate computation. They are not necessarily the simplest ones for real-valued computations.

If both //number1// and //number2// are supplied for **[[CL:Functions:atan]]**, the result is the arc tangent of //number1/////number2//. The value of **[[CL:Functions:atan]]** is always between ''-\pi'' (exclusive) and~''\pi'' (inclusive)

when minus zero is not supported. The range of the two-argument arc tangent when minus zero is supported includes ''-\pi''.


For a

//[[CL:Glossary:real]]//

//number1//, the result is

a //[[CL:Glossary:real]]//

and lies between ''-\pi/2'' and~''\pi/2'' (both exclusive). //number1// can be a //[[CL:Glossary:complex]]// if //number2// is not supplied. If both are supplied, //number2// can be zero provided //number1// is not zero. \reviewer{Barmar: Should add ``However, if the implementation distinguishes positive and negative zero, both may be signed zeros, and limits are used to define the result.''}

The following definition for arc sine determines the range and branch cuts:

'''' \ff{arcsin} z = -i\ \ff{log} \Bigl(iz+\sqrt{1-z^2}\Bigr) ''''

The branch cut for the arc sine function is in two pieces: one along the negative real axis to the left of~''-1'' (inclusive), continuous with quadrant II, and one along the positive real axis to the right of~''1'' (inclusive), continuous with quadrant IV. The range is that strip of the complex plane containing numbers whose real part is between ''-\pi/2'' and~''\pi/2''. A number with real part equal to ''-\pi/2'' is in the range if and only if its imaginary part is non-negative; a number with real part equal to ''\pi/2'' is in the range if and only if its imaginary part is non-positive.

The following definition for arc cosine determines the range and branch cuts:

'''' \ff{arccos} z = {\pi\over2}- \ff{arcsin} z''''

or, which are equivalent,

'''' \ff{arccos} z = -i\ \ff{log} \Bigl(z+i\ \sqrt{1-z^2}\Bigr) ''''

'''' \ff{arccos} z = {{2\ \ff{log} \bigl(\sqrt{(1+z)/2} + i\ \sqrt{(1-z)/2}\bigr)}\over{i}}''''

The branch cut for the arc cosine function is in two pieces: one along the negative real axis to the left of~''-1'' (inclusive), continuous with quadrant II, and one along the positive real axis to the right of~''1'' (inclusive), continuous with quadrant IV. This is the same branch cut as for arc sine. The range is that strip of the complex plane containing numbers whose real part is between 0 and~''\pi''. A number with real part equal to 0 is in the range if and only if its imaginary part is non-negative; a number with real part equal to ''\pi'' is in the range if and only if its imaginary part is non-positive.

The following definition for (one-argument) arc tangent determines the range and branch cuts:


'''' \ff{arctan} z = {{\ff{log} (1+iz) - \ff{log} (1-iz)}\over{2i}} ''''


Beware of simplifying this formula; "obvious" simplifications are likely to alter the branch cuts or the values on the branch cuts incorrectly. The branch cut for the arc tangent function is in two pieces: one along the positive imaginary axis above ''i'' (exclusive), continuous with quadrant II, and one along the negative imaginary axis below ''-i'' (exclusive), continuous with quadrant IV. The points ''i'' and~''-i'' are excluded from the domain. The range is that strip of the complex plane containing numbers whose real part is between ''-\pi/2'' and~''\pi/2''. A number with real part equal to ''-\pi/2'' is in the range if and only if its imaginary part is strictly positive; a number with real part equal to ''\pi/2'' is in the range if and only if its imaginary part is strictly negative. Thus the range of arc tangent is identical to that of arc sine with the points ''-\pi/2'' and~''\pi/2'' excluded.

For **[[CL:Functions:atan]]**, the signs of //number1// (indicated as ''x'') and //number2// (indicated as ''y'') are used to derive quadrant information. \Thenextfigure\ details various special cases.

The asterisk (*) indicates that the entry in the figure applies to implementations that support minus zero.

\def\Result{result} \def\starY{\hbox to 1pc{*\hfil}} \def\starN{\hbox to 1pc{\hfil}} \tablefigfour{Quadrant information for arc tangent} {\starN''y'' Condition}{''x'' Condition}{Cartesian locus}{Range of result}{ \starN'' y = 0 '' & '' x > 0 '' & Positive x-axis & '' 0'' \cr \starY'' y = +0 '' & '' x > 0 '' & Positive x-axis & ''+0'' \cr \starY'' y = -0 '' & '' x > 0 '' & Positive x-axis & ''-0'' \cr \starN'' y > 0 '' & '' x > 0 '' & Quadrant I & ''0 < \Result < \pi/2 '' \cr \starN'' y > 0 '' & '' x = 0 '' & Positive y-axis & ''\pi/2'' \cr \starN'' y > 0 '' & '' x < 0 '' & Quadrant II & ''\pi/2 < \Result < \pi'' \cr \starN'' y = 0 '' & '' x < 0 '' & Negative x-axis & '' \pi'' \cr \starY'' y = +0 '' & '' x < 0 '' & Negative x-axis & ''+\pi'' \cr \starY'' y = -0 '' & '' x < 0 '' & Negative x-axis & ''-\pi'' \cr \starN'' y < 0 '' & '' x < 0 '' & Quadrant III & ''-\pi < \Result < -\pi/2'' \cr \starN'' y < 0 '' & '' x = 0 '' & Negative y-axis & ''-\pi/2'' \cr \starN'' y < 0 '' & '' x > 0 '' & Quadrant IV & ''-\pi/2 < \Result < 0 '' \cr \starN'' y = 0 '' & '' x = 0 '' & Origin & undefined consequences \cr \starY'' y = +0 '' & '' x = +0 '' & Origin & ''+0'' \cr \starY'' y = -0 '' & '' x = +0 '' & Origin & ''-0'' \cr \starY'' y = +0 '' & '' x = -0 '' & Origin & ''+\pi'' \cr \starY'' y = -0 '' & '' x = -0 '' & Origin & ''-\pi'' \cr }

====Examples====

<blockquote> (asin 0) → 0.0 (acos #c(0 1)) → #C(1.5707963267948966 -0.8813735870195432) (/ (atan 1 (sqrt 3)) 6) → 0.087266 (atan #c(0 2)) → #C(-1.5707964 0.54930615) </blockquote>


====Affected By====

None.

====Exceptional Situations====

**[[CL:Functions:acos]]** and **[[CL:Functions:asin]]** \shouldchecktype{number}{a //[[CL:Glossary:number]]//} **[[CL:Functions:atan]]** should signal **[[CL:Types:type-error]]** if one argument is supplied and that argument is not a //[[CL:Glossary:number]]//, or if two arguments are supplied and both of those arguments are not //[[CL:Glossary:real|reals]]//.

**[[CL:Functions:acos]]**, **[[CL:Functions:asin]]**, and **[[CL:Functions:atan]]** might signal **[[CL:Types:arithmetic-error]]**.

====See Also====

**[[CL:Functions:log]]**, **[[CL:Functions:sqrt]]**, {\secref\FloatSubstitutability}

====Notes====

The result of either **[[CL:Functions:asin]]** or **[[CL:Functions:acos]]** can be a //[[CL:Glossary:complex]]// even if //number// is not a //[[CL:Glossary:complex]]//; this occurs when the absolute value of //number// is greater than one.

\issue{COMPLEX-ATANH-BOGUS-FORMULA:TWEAK-MORE} \issue{COMPLEX-ATAN-BRANCH-CUT:TWEAK} \issue{REAL-NUMBER-TYPE:X3J13-MAR-89} \issue{REAL-NUMBER-TYPE:X3J13-MAR-89} \issue{COMPLEX-ATANH-BOGUS-FORMULA:TWEAK-MORE} \issue{COMPLEX-ATAN-BRANCH-CUT:TWEAK} \issue{IEEE-ATAN-BRANCH-CUT:SPLIT} \issue{COMPLEX-ATAN-BRANCH-CUT:TWEAK}
