====== Function ASIN, ACOS, ATAN ======

====Syntax====
  * **asin** //number// → //radians// 
  * **acos** //number// → //radians// 
  * **atan** //number1 ''&optional'' number2// → //radians//

====Arguments and Values====
  * //number// - a //[[CL:Glossary:number]]//.
  * //number1// - a //[[CL:Glossary:number]]// if //number2// is not supplied, or a //[[CL:Glossary:real]]// if //number2// is supplied.
  * //number2// - a //[[CL:Glossary:real]]//.
  * //radians// - a //[[CL:Glossary:number]]// (of radians).

====Description====

**asin**, **acos**, and **atan** compute the arc sine, arc cosine, and arc tangent respectively.

The arc sine, arc cosine, and arc tangent (with only //number1// supplied) functions can be defined mathematically for //number// or //number1// specified as //x// as in the below table.

^ Function    ^ Definition                                           ^
| Arc sine    | ''-i * log(i * //x// + sqrt(1 - //x//<sup>2</sup>))''             |
| Arc cosine  | ''(π / 2) - arcsin(//x//)''                                       |
| Arc tangent | ''-i * log((1 + i * //x//) * sqrt(1 / (1 + //x//<sup>2</sup>)))'' |

These formulae are mathematically correct, assuming completely accurate computation. They are not necessarily the simplest ones for real-valued computations.

If both //number1// and //number2// are supplied for **atan**, the result is the arc tangent of //number1//<nowiki>/</nowiki>//number2//. The value of **atan** is always between ''-π'' (exclusive) and ''π'' (inclusive) when minus zero is not supported. The range of the two-argument arc tangent when minus zero is supported includes ''-π''.

For a //[[CL:Glossary:real]]// //number1//, the result is a //[[CL:Glossary:real]]// and lies between ''-π/2'' and~''π/2'' (both exclusive). //number1// can be a //[[CL:Glossary:complex]]// if //number2// is not supplied. If both are supplied, //number2// can be zero provided //number1// is not zero. 

The following definition for arc sine determines the range and branch cuts:

<blockquote>
arcsin(//z//) = -i * log(i * //z// + sqrt(1 - //z//<sup>2</sup>))
</blockquote>

The branch cut for the arc sine function is in two pieces: one along the negative real axis to the left of~''-1'' (inclusive), continuous with quadrant II, and one along the positive real axis to the right of~''1'' (inclusive), continuous with quadrant IV. The range is that strip of the complex plane containing numbers whose real part is between ''-π/2'' and~''π/2''. A number with real part equal to ''-π/2'' is in the range if and only if its imaginary part is non-negative; a number with real part equal to ''π/2'' is in the range if and only if its imaginary part is non-positive.

The following equivalent definitions for arc cosine determines the range and branch cuts:

<blockquote>
//arccos//(//z//) = π / 2 - arcsin(//z//)

//arccos//(//z//) = -i * log(z + i 8 sqrt(1 - z<sup>2</sup>))

//arccos//(//z//) = (2 * log(sqrt((1 + //z//) / 2) + i * sqrt((1 - //z//) / 2))) / i
</blockquote>

The branch cut for the arc cosine function is in two pieces: one along the negative real axis to the left of ''-1'' (inclusive), continuous with quadrant II, and one along the positive real axis to the right of ''1'' (inclusive), continuous with quadrant IV. This is the same branch cut as for arc sine. The range is that strip of the complex plane containing numbers whose real part is between 0 and~''π''. A number with real part equal to 0 is in the range if and only if its imaginary part is non-negative; a number with real part equal to ''π'' is in the range if and only if its imaginary part is non-positive.

The following definition for (one-argument) arc tangent determines the range and branch cuts:

<blockquote>
arctan(//z//) = (log(1 + i * z) - log(1 - i * z)) / 2 * i
</blockquote>

Beware of simplifying this formula; "obvious" simplifications are likely to alter the branch cuts or the values on the branch cuts incorrectly. The branch cut for the arc tangent function is in two pieces: one along the positive imaginary axis above ''i'' (exclusive), continuous with quadrant II, and one along the negative imaginary axis below ''-i'' (exclusive), continuous with quadrant IV. The points ''i'' and ''-i'' are excluded from the domain. The range is that strip of the complex plane containing numbers whose real part is between ''-π/2'' and ''π/2''. A number with real part equal to ''-π/2'' is in the range if and only if its imaginary part is strictly positive; a number with real part equal to ''π/2'' is in the range if and only if its imaginary part is strictly negative. Thus the range of arc tangent is identical to that of arc sine with the points ''-π/2'' and ''π/2'' excluded.

For **atan**, the signs of //number1// (indicated as ''x'') and //number2// (indicated as ''y'') are used to derive quadrant information. The next table details various special cases.

The asterisk (*) indicates that the entry in the figure applies to implementations that support minus zero.

^   ^ ''y'' Condition  ^ ''x'' Condition  ^ Cartesian locus ^ Range of result           ^
|   | ''y = 0''        | ''x > 0''        | Positive x-axis |'' 0''                     |
| * | ''y = +0''       | ''x > 0''        | Positive x-axis |''+0''                     |
| * | ''y = -0''       | ''x > 0''        | Positive x-axis |''-0''                     |
|   | ''y > 0''        | ''x > 0''        | Quadrant I      |''0 < //result// < π/2''   |
|   | ''y > 0''        | ''x = 0''        | Positive y-axis |''π/2''                    |
|   | ''y > 0''        | ''x < 0''        | Quadrant II     |''π/2 < //result// < π''   |
|   | ''y = 0''        | ''x < 0''        | Negative x-axis |'' π''                     |
| * | ''y = +0''       | ''x < 0''        | Negative x-axis |''+π''                     |
| * | ''y = -0''       | ''x < 0''        | Negative x-axis |''-π''                     |
|   | ''y < 0''        | ''x < 0''        | Quadrant III    |''-π < //result// < -π/2'' |
|   | ''y < 0''        | ''x = 0''        | Negative y-axis |''-π/2''                   |
|   | ''y < 0''        | ''x > 0''        | Quadrant IV     |''-π/2 < //result// < 0''  |
|   | ''y = 0''        | ''x = 0''        | Origin          | undefined consequences    |
| * | ''y = +0''       | ''x = +0''       | Origin          |''+0''                     |
| * | ''y = -0''       | ''x = +0''       | Origin          |''-0''                     |
| * | ''y = +0''       | ''x = -0''       | Origin          |''+π''                     |
| * | ''y = -0''       | ''x = -0''       | Origin          |''-π''                     |

====Examples====
<blockquote>
(asin 0) <r>0.0 </r>
(acos #c(0 1)) <r>#C(1.5707963267948966 -0.8813735870195432) </r>
(/ (atan 1 (sqrt 3)) 6) <r>0.087266 </r>
(atan #c(0 2)) <r>#C(-1.5707964 0.54930615) </r>
</blockquote>

====Affected By====
None.

====Exceptional Situations====
**acos** and **asin** should signal an //[[CL:Glossary:error]]// of type **[[CL:Types:type-error]]** if //number// is a //[[CL:Glossary:number]]//. **atan** should signal **[[CL:Types:type-error]]** if one argument is supplied and that argument is not a //[[CL:Glossary:number]]//, or if two arguments are supplied and both of those arguments are not //[[CL:Glossary:reals]]//.

**acos**, **asin**, and **atan** might signal **[[CL:Types:arithmetic-error]]**.

====See Also====
  * **[[CL:Functions:log|Function LOG]]**
  * **[[CL:Functions:sqrt|Function SQRT]]**
  * {\secref\FloatSubstitutability}

====Notes====
The result of either **asin** or **acos** can be a //[[CL:Glossary:complex]]// even if //number// is not a //[[CL:Glossary:complex]]//; this occurs when the absolute value of //number// is greater than one.

\issue{COMPLEX-ATANH-BOGUS-FORMULA:TWEAK-MORE} \issue{COMPLEX-ATAN-BRANCH-CUT:TWEAK} \issue{REAL-NUMBER-TYPE:X3J13-MAR-89} \issue{REAL-NUMBER-TYPE:X3J13-MAR-89} \issue{COMPLEX-ATANH-BOGUS-FORMULA:TWEAK-MORE} \issue{COMPLEX-ATAN-BRANCH-CUT:TWEAK} \issue{IEEE-ATAN-BRANCH-CUT:SPLIT} \issue{COMPLEX-ATAN-BRANCH-CUT:TWEAK}
