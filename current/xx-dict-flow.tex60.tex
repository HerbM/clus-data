====== Macro DEFINE-MODIFY-MACRO ======

====Syntax====

\DefmacWithValues define-modify-macro {name lambda-list function [documentation]} {name}

====Arguments and Values====
  * //name// - a //[[CL:Glossary:symbol]]//.
  * //lambda-list// - a //[[CL:Glossary:define-modify-macro lambda list]]//
  * //function// - a //[[CL:Glossary:symbol]]//.
  * //documentation// - a //[[CL:Glossary:string]]//; not evaluated.

====Description====

**[[CL:Macros:define-modify-macro]]** defines a //[[CL:Glossary:macro]]// named //name// to //[[CL:Glossary:read]]// and //[[CL:Glossary:write]]// a //[[CL:Glossary:place]]//.

The arguments to the new //[[CL:Glossary:macro]]// are a //[[CL:Glossary:place]]//, followed by the arguments that are supplied in //lambda-list//.

//[[CL:Glossary:macro|macros]]// defined with **[[CL:Macros:define-modify-macro]]** correctly pass the //[[CL:Glossary:environment parameter]]// to

**[[CL:Functions:get-setf-expansion]]**.

When the //[[CL:Glossary:macro]]// is invoked, //function// is applied to the old contents of the //[[CL:Glossary:place]]// and the //lambda-list// arguments to obtain the new value, and the //[[CL:Glossary:place]]// is updated to contain the result.

Except for the issue of avoiding multiple evaluation (see below), the expansion of a **[[CL:Macros:define-modify-macro]]** is equivalent to the following:

<blockquote> (defmacro //name// (reference . //lambda-list//) //documentation// \bq([[CL:Macros:setf]] ,reference (//function// ,reference ,''arg1'' ,''arg2'' ...))) </blockquote>

where ''arg1'', ''arg2'', ..., are the parameters appearing in //lambda-list//; appropriate provision is made for a //[[CL:Glossary:rest parameter]]//.

The //[[CL:Glossary:subforms]]// of the macro calls defined by **[[CL:Macros:define-modify-macro]]** are evaluated as specified in \secref\GenRefSubFormEval.

//Documentation// is attached as a //[[CL:Glossary:documentation string]]// to //name// (as kind **[[CL:Special Operators:function]]**) and to the //[[CL:Glossary:macro function]]//.

If a **[[CL:Macros:define-modify-macro]]** //[[CL:Glossary:form]]// appears as a //[[CL:Glossary:top level form]]//, the //[[CL:Glossary:compiler]]// must store the //[[CL:Glossary:macro]]// definition at compile time, so that occurrences of the macro later on in the file can be expanded correctly.

====Examples====

<blockquote> (define-modify-macro appendf (&rest args) append "Append onto list") → APPENDF ([[CL:Macros:defparameter]] x '(a b c) y x) → (A B C) (appendf x '(d e f) '(1 2 3)) → (A B C D E F 1 2 3) x → (A B C D E F 1 2 3) y → (A B C) (define-modify-macro new-incf (&optional (delta 1)) +) (define-modify-macro unionf (other-set &rest keywords) union) </blockquote>

====Side Effects====

A macro definition is assigned to //name//.

====Affected By====

None.

====Exceptional Situations====

None.

====See Also====

**[[CL:Macros:defsetf]]**,

**[[CL:Macros:define-setf-expander]]**,

**[[CL:Functions:documentation]]**, {\secref\DocVsDecls}

====Notes====

None.

\issue{GET-SETF-METHOD-ENVIRONMENT:ADD-ARG} \issue{SETF-METHOD-VS-SETF-METHOD:RENAME-OLD-TERMS} \issue{PUSH-EVALUATION-ORDER:FIRST-ITEM} \issue{DOCUMENTATION-FUNCTION-BUGS:FIX} \issue{COMPILE-FILE-HANDLING-OF-TOP-LEVEL-FORMS:CLARIFY} \issue{SETF-METHOD-VS-SETF-METHOD:RENAME-OLD-TERMS}
