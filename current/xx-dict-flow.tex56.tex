====== Macro NTH-VALUE ======

====Syntax====

**nth-value** //n form// → //object//

====Arguments and Values====

//n// - a non-negative //[[CL:Glossary:integer]]//; \eval.

//form// - a //[[CL:Glossary:form]]//; \evalspecial.

//object// - an //[[CL:Glossary:object]]//.

====Description====

Evaluates //n// and then //form//, returning as its only value the //n//th value //[[CL:Glossary:yielded]]// by //form//, or **[[CL:Constant Variables:nil]]** if //n// is greater than or equal to the number of //[[CL:Glossary:value|values]]// returned by //form//. (The first returned value is numbered ''0''.)

====Examples====

<blockquote> (nth-value 0 (values 'a 'b)) → A (nth-value 1 (values 'a 'b)) → B (nth-value 2 (values 'a 'b)) → NIL (let* ((x 83927472397238947423879243432432432) (y 32423489732) (a (nth-value 1 (floor x y))) (b (mod x y))) (values a b (= a b))) → 3332987528, 3332987528, //[[CL:Glossary:true]]// </blockquote>

====Side Effects====

None.

====Affected By====

None.

====Exceptional Situations====

None.

====See Also====

**[[CL:Macros:multiple-value-list]]**, **[[CL:Functions:nth]]**

====Notes====

Operationally, the following relationship is true, although \specref{nth-value} might be more efficient in some //[[CL:Glossary:implementations]]// because, for example, some //[[CL:Glossary:consing]]// might be avoided.

<blockquote> (nth-value //n// //form//) ≡ (nth //n// (multiple-value-list //form//)) </blockquote>

\issue{NTH-VALUE:ADD}
