====== Function RATIONAL, RATIONALIZE ======

====Syntax====
  * **rational ** //number// → //rational// **rationalize** //number// → //rational//

====Arguments and Values====
  * //number// - a //[[CL:Glossary:real]]//.
  * //rational// - a //[[CL:Glossary:rational]]//.

====Description====
**rational** and **rationalize** convert //[[CL:Glossary:reals]]// to //[[CL:Glossary:rationals]]//.

If //number// is already //[[CL:Glossary:rational]]//, it is returned.

If //number// is a //[[CL:Glossary:float]]//, **rational** returns a //[[CL:Glossary:rational]]// that is mathematically equal in value to the //[[CL:Glossary:float]]//. **rationalize** returns a //[[CL:Glossary:rational]]// that approximates the //[[CL:Glossary:float]]// to the accuracy of the underlying floating-point representation.

**rational** assumes that the //[[CL:Glossary:float]]// is completely accurate.

**rationalize** assumes that the //[[CL:Glossary:float]]// is accurate only to the precision of the floating-point representation.

====Examples====
<blockquote> 
(rational 0) <r>0 </r>
(rationalize -11/100) <r>-11/100 </r>
(rational .1) <r>13421773/134217728 ;implementation-dependent </r>
(rationalize .1) <r>1/10</r>
</blockquote>

====Side Effects====
None.

====Affected By====
The //[[CL:Glossary:implementation]]//.

====Exceptional Situations====
Should signal an error of type type-error if //number// is not a //[[CL:Glossary:real]]//. Might signal **[[CL:Types:arithmetic-error]]**.

====See Also====
None.

====Notes====
Rationalizing a //[[CL:Glossary:float]]// by either method and then converting it back to a //[[CL:Glossary:float]]// of the same format produces the original //number//.

<blockquote> 
(float (rational //x//) //x//) 
  ≡ (float (rationalize //x//) //x//) 
  ≡ //x// 
</blockquote>

\issue{REAL-NUMBER-TYPE:X3J13-MAR-89}
