====== Function BREAK ======

====Syntax====

**break {''&optional'' format-control** //\rest} format-arguments// → //**[[CL:Constant Variables:nil]]**//

====Arguments and Values====

//format-control// - a //[[CL:Glossary:format control]]//.

The default is //[[CL:Glossary:implementation-dependent]]//.

//format-arguments// - //[[CL:Glossary:format arguments]]// for the //format-control//.

====Description====

**[[CL:Functions:break]]** //[[CL:Glossary:formats]]// //format-control// and //format-arguments// and then goes directly into the debugger without allowing any possibility of interception by programmed error-handling facilities.

If \therestart{continue} is used while in the debugger, **[[CL:Functions:break]]** immediately returns **[[CL:Constant Variables:nil]]** without taking any unusual recovery action.

**[[CL:Functions:break]]** binds **[[CL:Variables:*debugger-hook*]]** to **[[CL:Constant Variables:nil]]** before attempting to enter the debugger.

====Examples====

<blockquote> (break "You got here with arguments: ~:S." '(FOO 37 A))
▷ BREAK: You got here with these arguments: FOO, 37, A.
▷ To continue, type :CONTINUE followed by an option number:
▷ 1: Return from BREAK.
▷ 2: Top level.
▷ Debug> :CONTINUE 1
▷ Return from BREAK. → NIL

</blockquote>

====Side Effects====

The debugger is entered.

====Affected By====

**[[CL:Variables:*debug-io*]]**.

====Exceptional Situations====

None.

====See Also====

**[[CL:Functions:error]]**, **[[CL:Functions:invoke-debugger]]**.

====Notes====

**[[CL:Functions:break]]** is used as a way of inserting temporary debugging "breakpoints" in a program, not as a way of signaling errors. For this reason, **[[CL:Functions:break]]** does not take the //continue-format-control// //[[CL:Glossary:argument]]// that **[[CL:Functions:cerror]]** takes. This and the lack of any possibility of interception by //[[CL:Glossary:condition]]// //[[CL:Glossary:handling]]// are the only program-visible differences between **[[CL:Functions:break]]** and **[[CL:Functions:cerror]]**.

The user interface aspects of **[[CL:Functions:break]]** and **[[CL:Functions:cerror]]** are permitted to vary more widely, in order to accomodate the interface needs of the //[[CL:Glossary:implementation]]//. For example, it is permissible for a //[[CL:Glossary:Lisp read-eval-print loop]]// to be entered by **[[CL:Functions:break]]** rather than the conventional debugger.

**[[CL:Functions:break]]** could be defined by:

<blockquote> (defun break (&optional (format-control "Break") &rest format-arguments) (with-simple-restart (continue "Return from BREAK.") (let ((*debugger-hook* nil)) (invoke-debugger (make-condition 'simple-condition :format-control format-control :format-arguments format-arguments)))) nil) </blockquote>

\issue{FORMAT-STRING-ARGUMENTS:SPECIFY} \issue{DEBUGGER-HOOK-VS-BREAK:CLARIFY} \issue{FORMAT-STRING-ARGUMENTS:SPECIFY}
