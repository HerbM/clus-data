====== Function LCM ======

====Syntax====

**lcm** //''&rest'' integers// → //least-common-multiple//

====Arguments and Values====

//integer// - an //[[CL:Glossary:integer]]//.

//least-common-multiple// - a non-negative //[[CL:Glossary:integer]]//.

====Description====

**[[CL:Functions:lcm]]** returns the least common multiple of the //integers//.

If no //integer// is supplied, the //[[CL:Glossary:integer]]// ''1'' is returned.

If only one //integer// is supplied, the absolute value of that //integer// is returned.

For two arguments that are not both zero,

<blockquote> (lcm a b) ≡ (/ (abs (* a b)) (gcd a b)) </blockquote>

If one or both arguments are zero,

<blockquote> (lcm a 0) ≡ (lcm 0 a) ≡ 0 </blockquote>

For three or more arguments,

<blockquote> (lcm a b c ... z) ≡ (lcm (lcm a b) c ... z) </blockquote>

====Examples====

<blockquote> (lcm 10) → 10 (lcm 25 30) → 150 (lcm -24 18 10) → 360 (lcm 14 35) → 70 (lcm 0 5) → 0 (lcm 1 2 3 4 5 6) → 60 </blockquote>

====Side Effects====

None.

====Affected By====

None.

====Exceptional Situations====

Should signal **[[CL:Types:type-error]]** if any argument is not an //[[CL:Glossary:integer]]//.

====See Also====

**[[CL:Functions:gcd]]**

====Notes====

None.

\issue{LCM-NO-ARGUMENTS:1}
