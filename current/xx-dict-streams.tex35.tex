====== Function LISTEN ======

====Syntax====

**{listen}** //''&optional'' input-stream// → //generalized-boolean//

====Arguments and Values====

//input-stream// - an //[[CL:Glossary:input]]// //[[CL:Glossary:stream designator]]//. The default is //[[CL:Glossary:standard input]]//.

//generalized-boolean// - a //[[CL:Glossary:generalized boolean]]//.

====Description====

Returns //[[CL:Glossary:true]]// if there is a character immediately available from //input-stream//; otherwise, returns //[[CL:Glossary:false]]//. On a non-interactive //input-stream//, **[[CL:Functions:listen]]** returns //[[CL:Glossary:true]]// except when at //[[CL:Glossary:end of file]]//. If an //[[CL:Glossary:end of file]]// is encountered, **[[CL:Functions:listen]]** returns //[[CL:Glossary:false]]//. **[[CL:Functions:listen]]** is intended to be used when //input-stream// obtains characters from an interactive device such as a keyboard.

====Examples====

<blockquote> (progn (unread-char (read-char)) (list (listen) (read-char)))
▷ \IN{1} → (T #\\1) (progn (clear-input) (listen)) → NIL ;Unless you're a very fast typist! </blockquote>

====Side Effects====

None.

====Affected By====

**[[CL:Variables:*standard-input*]]**

====Exceptional Situations====

None.

====See Also====

**[[CL:Functions:interactive-stream-p]]**, **[[CL:Functions:read-char-no-hang]]**

====Notes====

None.

