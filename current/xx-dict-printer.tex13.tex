====== Function SET-PPRINT-DISPATCH ======

====Syntax====
  * **set-pprint-dispatch** //type-specifier// //function// ''&optional'' //priority// //table// → **[[CL:Constant Variables:nil]]**

====Arguments and Values====
  * //type-specifier// - a //[[CL:Glossary:type specifier]]//.
  * //function// - a //[[CL:Glossary:function]]//, a //[[CL:Glossary:function name]]//, or **[[CL:Constant Variables:nil]]**.
  * //priority// - a //[[CL:Glossary:real]]//. The default is ''0''.
  * //table// - a //[[CL:Glossary:pprint dispatch table]]//. The default is the //[[CL:Glossary:value]]// of **[[CL:Variables:star-print-pprint-dispatch-star|*print-pprint-dispatch*]]**.

====Description====
Installs an entry into the //[[CL:Glossary:pprint dispatch table]]// which is //table//.

//type-specifier// is the //[[CL:Glossary:key]]// of the entry. The first action of **[[CL:Functions:set-pprint-dispatch]]** is to remove any pre-existing entry associated with //type-specifier//. This guarantees that there will never be two entries associated with the same //[[CL:Glossary:type specifier]]// in a given //[[CL:Glossary:pprint dispatch table]]//. Equality of //[[CL:Glossary:type specifier|type specifiers]]// is tested by **[[CL:Functions:equal]]**.

Two values are associated with each //[[CL:Glossary:type specifier]]// in a //[[CL:Glossary:pprint dispatch table]]//: a //function// and a //priority//. The //function// must accept two arguments: the //[[CL:Glossary:stream]]// to which output is sent and the //[[CL:Glossary:object]]// to be printed. The //function// should //[[CL:Glossary:pretty print]]// the //[[CL:Glossary:object]]// to the //stream//. The //function// can assume that object satisfies the //[[CL:Glossary:type]]// given by //[[CL:Glossary:type-specifier]]//. The //function// must obey **[[CL:Variables:star-print-readably-star|*print-readably*]]**.

Any values returned by the //function// are ignored.

//priority// is a priority to resolve conflicts when an object matches more than one entry.

It is permissible for //function// to be **[[CL:Constant Variables:nil]]**. In this situation, there will be no //type-specifier// entry in //table// after **set-pprint-dispatch** returns.

====Examples====
None.

====Side Effects====
None.

====Affected By====
None.

====Exceptional Situations====
An error is signaled if //priority// is not a //[[CL:Glossary:real]]//.

====See Also====
None.

====Notes====
Since //[[CL:Glossary:pprint dispatch tables]]// are often used to control the pretty printing of Lisp code, it is common for the //type-specifier// to be an //[[CL:Glossary:expression]]// of the form ''([[CL:Types:cons]] //car-type// //cdr-type//)''. This signifies that the corresponding object must be a cons cell whose //[[CL:Glossary:car]]// matches the //[[CL:Glossary:type specifier]]// //car-type// and whose //[[CL:Glossary:cdr]]// matches the //[[CL:Glossary:type specifier]]// //cdr-type//. The //cdr-type// can be omitted in which case it defaults to **[[CL:Types:t]]**.

\issue{PRETTY-PRINT-INTERFACE}
