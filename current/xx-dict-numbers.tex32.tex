====== Function GCD ======

====Syntax====

**gcd** //''&rest'' integers// → //greatest-common-denominator//

====Arguments and Values====

//integer// - an //[[CL:Glossary:integer]]//.

//greatest-common-denominator// - a non-negative //[[CL:Glossary:integer]]//.

====Description====

Returns the greatest common divisor of //integers//. If only one //integer// is supplied, its absolute value is returned. If no //integers// are given, **[[CL:Functions:gcd]]** returns ''0'', which is an identity for this operation.

====Examples====

<blockquote> (gcd) → 0 (gcd 60 42) → 6 (gcd 3333 -33 101) → 1 (gcd 3333 -33 1002001) → 11 (gcd 91 -49) → 7 (gcd 63 -42 35) → 7 (gcd 5) → 5 (gcd -4) → 4 </blockquote>

====Side Effects====

None.

====Affected By====

None.

====Exceptional Situations====

\Shouldcheckanytype{integer}{an //[[CL:Glossary:integer]]//}

====See Also====

**[[CL:Functions:lcm]]**

====Notes==== For three or more arguments,

<blockquote> (gcd b c ... z) ≡ (gcd (gcd a b) c ... z) </blockquote>

