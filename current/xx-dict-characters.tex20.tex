====== Function CHAR-NAME ======

====Syntax====

**char-name** //character// → //name//

====Arguments and Values====

//character// - a //[[CL:Glossary:character]]//.

//name// - a //[[CL:Glossary:string]]// or **[[CL:Constant Variables:nil]]**.

====Description====

Returns a //[[CL:Glossary:string]]// that is the //[[CL:Glossary:name]]// of the //character//, or **[[CL:Constant Variables:nil]]** if the //character// has no //[[CL:Glossary:name]]//.

All //[[CL:Glossary:non-graphic]]// characters are required to have //[[CL:Glossary:names]]// unless they have some //[[CL:Glossary:implementation-defined]]// //[[CL:Glossary:attribute]]// which is not //[[CL:Glossary:null]]//. Whether or not other //[[CL:Glossary:characters]]// have //[[CL:Glossary:names]]// is //[[CL:Glossary:implementation-dependent]]//.

The //[[CL:Glossary:standard characters]]// \NewlineChar\ and \SpaceChar\ have the respective names ''"Newline"'' and ''"Space"''.

The //[[CL:Glossary:semi-standard]]// //[[CL:Glossary:characters]]// \TabChar, \PageChar, \RuboutChar, \LinefeedChar, \ReturnChar, and \BackspaceChar\

(if they are supported by the //[[CL:Glossary:implementation]]//) have the respective names ''"Tab"'', ''"Page"'', ''"Rubout"'', ''"Linefeed"'', ''"Return"'', and ''"Backspace"'' (in the indicated case, even though name lookup by ""#\\'''' and by the function **[[CL:Functions:name-char]]** is not case sensitive).

====Examples====

<blockquote> (char-name #\\ ) → "Space" (char-name #\\Space) → "Space" (char-name #\\Page) → "Page"

(char-name #\\a) → NIL //or// → "LOWERCASE-a" //or// → "Small-A" //or// → "LA01"

(char-name #\\A) → NIL //or// → "UPPERCASE-A" //or// → "Capital-A" //or// → "LA02"

;; Even though its CHAR-NAME can vary, #\\A prints as #\\A (prin1-to-string (read-from-string (format nil "#\\\\~A" (or (char-name #\\A) "A")))) → "#\\\\A" </blockquote>

====Affected By====

None.

====Exceptional Situations====

Should signal an error of type type-error if //character// is not a //[[CL:Glossary:character]]//.

====See Also====

**[[CL:Functions:name-char]]**, {\secref\PrintingCharacters}

====Notes====

//[[CL:Glossary:Non-graphic]]// //[[CL:Glossary:characters]]// having //[[CL:Glossary:names]]// are written by the //[[CL:Glossary:Lisp printer]]// as ""#\\'''' followed by the their //[[CL:Glossary:name]]//; see section {\secref\PrintingCharacters}.

\issue{CHAR-NAME-CASE:X3J13-MAR-91}
