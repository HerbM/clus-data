====== Function MAKE-HASH-TABLE ======

====Syntax====

\DefunWithValues make-hash-table {''&key'' test size rehash-size rehash-threshold} {hash-table}

====Arguments and Values====

//test// - a //[[CL:Glossary:designator]]// for one of the //[[CL:Glossary:functions]]// **[[CL:Functions:eq]]**, **[[CL:Functions:eql]]**, **[[CL:Functions:equal]]**, or

**[[CL:Functions:equalp]]**.

The default is **[[CL:Functions:eql]]**.

//size// - a non-negative //[[CL:Glossary:integer]]//.

The default is //[[CL:Glossary:implementation-dependent]]//.

//rehash-size// - a //[[CL:Glossary:real]]// of //[[CL:Glossary:type]]// ''(or (integer 1 *) (float (1.0) *))''. The default is //[[CL:Glossary:implementation-dependent]]//.


//rehash-threshold// - a //[[CL:Glossary:real]]// of //[[CL:Glossary:type]]// ''(real 0 1)''. The default is //[[CL:Glossary:implementation-dependent]]//.



//hash-table// - a //[[CL:Glossary:hash table]]//.

====Description====

Creates and returns a new //[[CL:Glossary:hash table]]//.

//test// determines how //[[CL:Glossary:keys]]// are compared.

An //[[CL:Glossary:object]]// is said to be present in the //hash-table// if that //[[CL:Glossary:object]]// is the //[[CL:Glossary:same]]// under the //[[CL:Glossary:test]]// as the //[[CL:Glossary:key]]// for some entry in the //hash-table//.

//size// is a hint to the //[[CL:Glossary:implementation]]// about how much initial space to allocate in the //hash-table//.

This information, taken together with the //rehash-threshold//, controls the approximate number of entries which it should be possible to insert before the table has to grow.

The actual size might be rounded up from //size// to the next `good' size; for example, some //[[CL:Glossary:implementations]]// might round to the next prime number.

//rehash-size// specifies a minimum amount to increase the size of the //hash-table// when it becomes full enough to require rehashing; see //rehash-theshold// below.

If //rehash-size// is an //[[CL:Glossary:integer]]//, the expected growth rate for the table is additive and the //[[CL:Glossary:integer]]// is the number of entries to add; if it is a //[[CL:Glossary:float]]//, the expected growth rate for the table is multiplicative and the //[[CL:Glossary:float]]// is the ratio of the new size to the old size.

As with //size//, the actual size of the increase might be rounded up.

//rehash-threshold// specifies how full the //hash-table// can get before it must grow.

It specifies the maximum desired hash-table occupancy level.

The //[[CL:Glossary:values]]// of //rehash-size// and //rehash-threshold// do not constrain the //[[CL:Glossary:implementation]]// to use any particular method for computing when and by how much the size of //hash-table// should be enlarged. Such decisions are //[[CL:Glossary:implementation-dependent]]//, and these //[[CL:Glossary:values]]// only hints from the //[[CL:Glossary:programmer]]// to the //[[CL:Glossary:implementation]]//, and the //[[CL:Glossary:implementation]]// is permitted to ignore them.

====Examples====

<blockquote> ([[CL:Macros:defparameter]] table (make-hash-table)) → #<HASH-TABLE EQL 0/120 46142754> ([[CL:Macros:setf]] (gethash "one" table) 1) → 1 (gethash "one" table) → NIL, //[[CL:Glossary:false]]// ([[CL:Macros:defparameter]] table (make-hash-table :test 'equal)) → #<HASH-TABLE EQUAL 0/139 46145547> ([[CL:Macros:setf]] (gethash "one" table) 1) → 1 (gethash "one" table) → 1, T (make-hash-table :rehash-size 1.5 :rehash-threshold 0.7) → #<HASH-TABLE EQL 0/120 46156620> </blockquote>

====Affected By====

None.

====Exceptional Situations====

None.

====See Also====

**[[CL:Functions:gethash]]**, **[[CL:Types:hash-table]]**

====Notes====

None.

\issue{HASH-TABLE-TESTS:ADD-EQUALP} \issue{ARGUMENTS-UNDERSPECIFIED:SPECIFY} \issue{HASH-TABLE-SIZE:INTENDED-ENTRIES} \issue{HASH-TABLE-SIZE:INTENDED-ENTRIES} \issue{HASH-TABLE-REHASH-SIZE-INTEGER} \issue{HASH-TABLE-SIZE:INTENDED-ENTRIES} \issue{HASH-TABLE-REHASH-SIZE-INTEGER} \issue{HASH-TABLE-SIZE:INTENDED-ENTRIES}
