====== Function CHAR-INT ======

====Syntax====

**char-int** //character// → //integer//

====Arguments and Values====

//character// - a //[[CL:Glossary:character]]//.

//integer// - a non-negative //[[CL:Glossary:integer]]//.

====Description====

Returns a non-negative //[[CL:Glossary:integer]]// encoding the //character// object. The manner in which the //[[CL:Glossary:integer]]// is computed is //[[CL:Glossary:implementation-dependent]]//. In contrast to **[[CL:Functions:sxhash]]**, the result is not guaranteed to be independent of the particular //[[CL:Glossary:Lisp image]]//.

If //character// has no //[[CL:Glossary:implementation-defined]]// //[[CL:Glossary:attributes]]//, the results of **[[CL:Functions:char-int]]** and **[[CL:Functions:char-code]]** are the same.

<blockquote> (char= ''c1'' ''c2'') ≡ (= (char-int ''c1'') (char-int ''c2'')) </blockquote> for characters ''c1'' and ''c2''.

====Examples====

<blockquote> (char-int #\\A) → 65 ; implementation A (char-int #\\A) → 577 ; implementation B (char-int #\\A) → 262145 ; implementation C </blockquote>

====Affected By====

None.

====Exceptional Situations====

None.

====See Also====

**[[CL:Functions:char-code]]**

====Notes====

None.

\issue{CHARACTER-PROPOSAL:2-1-2}
