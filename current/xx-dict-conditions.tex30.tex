====== Function MAKE-CONDITION ======

====Syntax====

**make-condition {type** //\rest} slot-initializations// → //condition//

====Arguments and Values====

//type// - a //[[CL:Glossary:type specifier]]// (for a //[[CL:Glossary:subtype]]// of **[[CL:Types:condition]]**).

//slot-initializations// - an //[[CL:Glossary:initialization argument list]]//.

//condition// - a //[[CL:Glossary:condition]]//.

====Description====

Constructs and returns a //[[CL:Glossary:condition]]// of type //type// using //slot-initializations// for the initial values of the slots. The newly created //[[CL:Glossary:condition]]// is returned.

====Examples====

<blockquote> (defvar *oops-count* 0)

([[CL:Macros:defparameter]] a (make-condition 'simple-error :format-control "This is your ~:R error." :format-arguments (list (incf *oops-count*)))) → #<SIMPLE-ERROR 32245104>

(format t "~&~A~
▷ This is your first error. → NIL

(error a)
▷ Error: This is your first error.
▷ To continue, type :CONTINUE followed by an option number:
▷ 1: Return to Lisp Toplevel.
▷ Debug> </blockquote>

====Side Effects====

None.

====Affected By====

The set of defined //[[CL:Glossary:condition]]// //[[CL:Glossary:types]]//.

====Exceptional Situations====

None.

====See Also====

**[[CL:Macros:define-condition]]**, {\secref\ConditionSystemConcepts}

====Notes====

None.

\issue{FORMAT-STRING-ARGUMENTS:SPECIFY}
