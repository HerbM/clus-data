====== Function WRITE, PRIN1, PRINT, PPRINT, PRINC ======

====Syntax====
\DefunWithValuesNewline write {//object// ''&key'' \writekeys{stream}} {object}

**prin1 {object** //\opt} output-stream// → //object// **princ {object** //\opt} output-stream// → //object// **print {object** //\opt} output-stream// → //object// **pprint {object** //\opt} output-stream// → //\novalues//

====Arguments and Values====
//object// - an //[[CL:Glossary:object]]//.

//output-stream// - an //[[CL:Glossary:output]]// //[[CL:Glossary:stream designator]]//. The default is //[[CL:Glossary:standard output]]//.

\writekeydescriptions{//stream// - an //[[CL:Glossary:output]]// //[[CL:Glossary:stream designator]]//. The default is //[[CL:Glossary:standard output]]//.}

====Description====
**[[CL:Functions:write]]**, **[[CL:Functions:prin1]]**, **[[CL:Functions:princ]]**, **[[CL:Functions:print]]**, and **[[CL:Functions:pprint]]** write the printed representation of //object// to //output-stream//.

**[[CL:Functions:write]]** is the general entry point to the //[[CL:Glossary:Lisp printer]]//. For each explicitly supplied //[[CL:Glossary:keyword parameter]]// named in \thenextfigure, the corresponding //[[CL:Glossary:printer control variable]]// is dynamically bound to its //[[CL:Glossary:value]]// while printing goes on; for each //[[CL:Glossary:keyword parameter]]// in \thenextfigure\ that is not explicitly supplied, the value of the corresponding //[[CL:Glossary:printer control variable]]// is the same as it was at the time **[[CL:Functions:write]]** was invoked. Once the appropriate //[[CL:Glossary:bindings]]// are //[[CL:Glossary:established]]//, the //[[CL:Glossary:object]]// is output by the //[[CL:Glossary:Lisp printer]]//.

\tablefigtwo{Argument correspondences for the WRITE function.}{Parameter}{Corresponding Dynamic Variable}{ //array// & **[[CL:Variables:*print-array*]]** \cr //base// & **[[CL:Variables:*print-base*]]** \cr //case// & **[[CL:Variables:*print-case*]]** \cr //circle// & **[[CL:Variables:*print-circle*]]** \cr //escape// & **[[CL:Variables:*print-escape*]]** \cr //gensym// & **[[CL:Variables:*print-gensym*]]** \cr //length// & **[[CL:Variables:*print-length*]]** \cr //level// & **[[CL:Variables:*print-level*]]** \cr //lines// & **[[CL:Variables:*print-lines*]]** \cr //miser-width// & **[[CL:Variables:*print-miser-width*]]** \cr //pprint-dispatch// & **[[CL:Variables:*print-pprint-dispatch*]]** \cr //pretty// & **[[CL:Variables:*print-pretty*]]** \cr //radix// & **[[CL:Variables:*print-radix*]]** \cr //readably// & **[[CL:Variables:*print-readably*]]** \cr //right-margin// & **[[CL:Variables:*print-right-margin*]]** \cr }

**[[CL:Functions:prin1]]**, **[[CL:Functions:princ]]**, **[[CL:Functions:print]]**, and **[[CL:Functions:pprint]]** implicitly //[[CL:Glossary:bind]]// certain print parameters to particular values. The remaining parameter values are taken from **[[CL:Variables:*print-array*]]**, **[[CL:Variables:*print-base*]]**, **[[CL:Variables:*print-case*]]**, **[[CL:Variables:*print-circle*]]**, **[[CL:Variables:*print-escape*]]**, **[[CL:Variables:*print-gensym*]]**, **[[CL:Variables:*print-length*]]**, **[[CL:Variables:*print-level*]]**, **[[CL:Variables:*print-lines*]]**, **[[CL:Variables:*print-miser-width*]]**, **[[CL:Variables:*print-pprint-dispatch*]]**, **[[CL:Variables:*print-pretty*]]**, **[[CL:Variables:*print-radix*]]**, and **[[CL:Variables:*print-right-margin*]]**.

**[[CL:Functions:prin1]]** produces output suitable for input to **[[CL:Functions:read]]**. It binds **[[CL:Variables:*print-escape*]]** to //[[CL:Glossary:true]]//.

**[[CL:Functions:princ]]** is just like **[[CL:Functions:prin1]]** except that the output has no //[[CL:Glossary:escape]]// //[[CL:Glossary:characters]]//. It binds **[[CL:Variables:*print-escape*]]** to //[[CL:Glossary:false]]//

and **[[CL:Variables:*print-readably*]]** to //[[CL:Glossary:false]]//.

The general rule is that output from **[[CL:Functions:princ]]** is intended to look good to people, while output from **[[CL:Functions:prin1]]** is intended to be acceptable to **[[CL:Functions:read]]**.

**[[CL:Functions:print]]** is just like **[[CL:Functions:prin1]]** except that the printed representation of //object// is preceded by a newline and followed by a space.

**[[CL:Functions:pprint]]** is just like **[[CL:Functions:print]]** except that the trailing space is omitted and //object// is printed with the **[[CL:Variables:*print-pretty*]]** flag //[[CL:Glossary:non-nil]]// to produce pretty output.

//Output-stream// specifies the //[[CL:Glossary:stream]]// to which output is to be sent.

====Affected By====
**[[CL:Variables:*standard-output*]]**, **[[CL:Variables:*terminal-io*]]**, **[[CL:Variables:*print-escape*]]**, **[[CL:Variables:*print-radix*]]**, **[[CL:Variables:*print-base*]]**, **[[CL:Variables:*print-circle*]]**, **[[CL:Variables:*print-pretty*]]**, **[[CL:Variables:*print-level*]]**, **[[CL:Variables:*print-length*]]**, **[[CL:Variables:*print-case*]]**, **[[CL:Variables:*print-gensym*]]**, **[[CL:Variables:*print-array*]]**, **[[CL:Variables:*read-default-float-format*]]**.

====Exceptional Situations====
None.

====See Also====
**[[CL:Functions:readtable-case]]**, {\secref\FORMATPrinterOps}

====Notes====
\Thefunctions{prin1} and **[[CL:Functions:print]]** do not bind **[[CL:Variables:*print-readably*]]**.

<blockquote> (prin1 object output-stream) ≡ (write object :stream output-stream :escape t) </blockquote>

<blockquote> (princ object output-stream) ≡ (write object stream output-stream :escape nil :readably nil) </blockquote>

<blockquote> (print object output-stream) ≡ (progn (terpri output-stream) (write object :stream output-stream :escape t) (write-char #\\space output-stream)) </blockquote>

<blockquote> (pprint object output-stream) ≡ (write object :stream output-stream :escape t :pretty t) </blockquote>

\issue{PRETTY-PRINT-INTERFACE} \issue{PRINC-READABLY:X3J13-DEC-91} \issue{PRINC-READABLY:X3J13-DEC-91}
