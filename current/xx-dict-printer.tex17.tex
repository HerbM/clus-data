====== Variable *PRINT-BASE*, *PRINT-RADIX* ======

====Value Type====
**[[CL:Variables:*print-base*]]**---a //[[CL:Glossary:radix]]//. **[[CL:Variables:*print-radix*]]**---a //[[CL:Glossary:generalized boolean]]//.

====Initial Value====
The initial //[[CL:Glossary:value]]// of **[[CL:Variables:*print-base*]]** is ''10''.

The initial //[[CL:Glossary:value]]// of **[[CL:Variables:*print-radix*]]** is //[[CL:Glossary:false]]//.

====Description====
**[[CL:Variables:*print-base*]]** and **[[CL:Variables:*print-radix*]]** control the printing of //[[CL:Glossary:rationals]]//. The //[[CL:Glossary:value]]// of **[[CL:Variables:star-print-base-star|*print-base*]]** is called the //[[CL:Glossary:current output base]]//.

The //[[CL:Glossary:value]]// of **[[CL:Variables:star-print-base-star|*print-base*]]** is the //[[CL:Glossary:radix]]// in which the printer will print //[[CL:Glossary:rationals]]//. For radices above ''10'', letters of the alphabet are used to represent digits above ''9''.

If the //[[CL:Glossary:value]]// of **[[CL:Variables:star-print-radix-star|*print-radix*]]** is //[[CL:Glossary:true]]//, the printer will print a radix specifier to indicate the //[[CL:Glossary:radix]]// in which it is printing a //[[CL:Glossary:rational]]// number. The radix specifier is always printed using lowercase letters. If **[[CL:Variables:*print-base*]]** is ''2'', ''8'', or ''16'', then the radix specifier used is ''#b'', ''#o'', or ''#x'', respectively. For //[[CL:Glossary:integers]]//, base ten is indicated by a trailing decimal point instead of a leading radix specifier; for //[[CL:Glossary:ratios]]//, ''#10r'' is used.

====Examples====
<blockquote> (let ((*print-base* 24.) (*print-radix* t)) (print 23.))
▷ #24rN → 23 ([[CL:Macros:defparameter]] *print-base* 10) → 10 ([[CL:Macros:defparameter]] *print-radix* nil) → NIL (dotimes (i 35) (let ((*print-base* (+ i 2))) ;print the decimal number 40 (write 40) ;in each base from 2 to 36 (if (zerop (mod i 10)) (terpri) (format t " "))))
▷ 101000
▷ 1111 220 130 104 55 50 44 40 37 34
▷ 31 2C 2A 28 26 24 22 20 1J 1I
▷ 1H 1G 1F 1E 1D 1C 1B 1A 19 18
▷ 17 16 15 14 → NIL (dolist (pb '(2 3 8 10 16)) (let ((*print-radix* t) ;print the integer 10 and (*print-base* pb)) ;the ratio 1/10 in bases 2, (format t "~&~S ~S~
▷ #b1010 #b1/1010
▷ #3r101 #3r1/101
▷ #o12 #o1/12
▷ 10. #10r1/10
▷ #xA #x1/A → NIL </blockquote>

====Affected By====
Might be //[[CL:Glossary:bound]]// by **[[CL:Functions:format]]**, and **[[CL:Functions:write]]**, **[[CL:Functions:write-to-string]]**.

====See Also====
**[[CL:Functions:format]]**, **[[CL:Functions:write]]**, **[[CL:Functions:write-to-string]]**

====Notes====
None.

