====== Function ALPHANUMERICP ======

====Syntax====

**alphanumericp** //character// → //generalized-boolean//

====Arguments and Values====

//character// - a //[[CL:Glossary:character]]//.

//generalized-boolean// - a //[[CL:Glossary:generalized boolean]]//.

====Description====

\Predicate{character}{an //[[CL:Glossary:alphabetic]]// //[[CL:Glossary:character]]// or a //[[CL:Glossary:numeric]]// //[[CL:Glossary:character]]//}

====Examples====

<blockquote> (alphanumericp #\\Z) → //[[CL:Glossary:true]]// (alphanumericp #\\9) → //[[CL:Glossary:true]]// (alphanumericp #\\Newline) → //[[CL:Glossary:false]]// (alphanumericp #\#) → //[[CL:Glossary:false]]// </blockquote>

====Affected By====

None. (In particular, the results of this predicate are independent of any special syntax which might have been enabled in the //[[CL:Glossary:current readtable]]//.)

====Exceptional Situations====

Should signal an error of type type-error if //character// is not a //[[CL:Glossary:character]]//.

====See Also====

**[[CL:Functions:alpha-char-p]]**, **[[CL:Functions:graphic-char-p]]**, **[[CL:Functions:digit-char-p]]**

====Notes====

Alphanumeric characters are graphic as defined by **[[CL:Functions:graphic-char-p]]**. The alphanumeric characters are a subset of the graphic characters.

The standard characters ''A'' through ''Z'', ''a'' through ''z'', and ''0'' through ''9'' are alphanumeric characters.

<blockquote> (alphanumericp x) ≡ (or (alpha-char-p x) (not (null (digit-char-p x)))) </blockquote>

