====== Special Operator FUNCTION ======

====Syntax====

\DefspecWithValues function {name} {function}

====Arguments and Values====

//name// - a //[[CL:Glossary:function name]]// or //[[CL:Glossary:lambda expression]]//.

//function// - a //[[CL:Glossary:function]]// //[[CL:Glossary:object]]//.

====Description====

The //[[CL:Glossary:value]]// of **[[CL:Special Operators:function]]** is the //[[CL:Glossary:functional value]]// of //name// in the current //[[CL:Glossary:lexical environment]]//.

If //name// is a //[[CL:Glossary:function name]]//, the functional definition of that name

is that

established by the innermost lexically enclosing **[[CL:Special Operators:flet]]**, **[[CL:Special Operators:labels]]**, or **[[CL:Special Operators:macrolet]]** //[[CL:Glossary:form]]//,

if there is one. Otherwise the global functional definition of the //[[CL:Glossary:function name]]// is returned.

If //name// is a //[[CL:Glossary:lambda expression]]//, then a //[[CL:Glossary:lexical closure]]// is returned. In situations where a //[[CL:Glossary:closure]]// over the same set of //[[CL:Glossary:binding|bindings]]// might be produced more than once, the various resulting //[[CL:Glossary:closures]]// might or might not be **[[CL:Functions:eq]]**.

It is an error to use **[[CL:Special Operators:function]]** on a //[[CL:Glossary:function name]]// that does not denote a //[[CL:Glossary:function]]// in the lexical environment in which the **[[CL:Special Operators:function]]** form appears. Specifically, it is an error to use **[[CL:Special Operators:function]]** on a //[[CL:Glossary:symbol]]// that denotes a //[[CL:Glossary:macro]]// or //[[CL:Glossary:special form]]//. An implementation may choose not to signal this error for performance reasons, but implementations are forbidden from defining the failure to signal an error as a useful behavior.

====Examples====

<blockquote> (defun adder (x) (function (lambda (y) (+ x y)))) </blockquote> The result of ''(adder 3)'' is a function that adds ''3'' to its argument:

<blockquote> ([[CL:Macros:defparameter]] add3 (adder 3)) (funcall add3 5) → 8 </blockquote> This works because **[[CL:Special Operators:function]]** creates a //[[CL:Glossary:closure]]// of the //[[CL:Glossary:lambda expression]]// that is able to refer to the //[[CL:Glossary:value]]// ''3'' of the variable ''x'' even after control has returned from the function ''adder''.

====Side Effects====

None.

====Affected By====

None.

====Exceptional Situations====

None.

====See Also====

**[[CL:Macros:defun]]**, **[[CL:Functions:fdefinition]]**, **[[CL:Special Operators:flet]]**, **[[CL:Special Operators:labels]]**, **[[CL:Functions:symbol-function]]**, {\secref\SymbolsAsForms}, {\secref\SharpsignQuote}, {\secref\PrintingOtherObjects}

====Notes====

The notation **[[CL:Functions:#'//name//]]** may be used as an abbreviation for **[[CL:Functions:(function //name//)]]**.

\issue{FUNCTION-NAME:LARGE} \issue{FUNCTION-TYPE:X3J13-MARCH-88}
