====== Type SHORT-FLOAT, SINGLE-FLOAT, DOUBLE-FLOAT, LONG-FLOAT ======

====Supertypes====
  * **short-float**, **[[CL:Types:float]]**, **[[CL:Types:real]]**, **[[CL:Types:number]]**, **[[CL:Types:t]]**
  * **single-float**, **[[CL:Types:float]]**, **[[CL:Types:real]]**, **[[CL:Types:number]]**, **[[CL:Types:t]]**
  * **double-float**, **[[CL:Types:float]]**, **[[CL:Types:real]]**, **[[CL:Types:number]]**, **[[CL:Types:t]]**
  * **long-float**, **[[CL:Types:float]]**, **[[CL:Types:real]]**, **[[CL:Types:number]]**, **[[CL:Types:t]]**

====Description====
For the four defined subtypes of **[[CL:Types:float]]**, it is true that intermediate between the type **short-float** and the type **long-float** are the type **single-float** and the type **double-float**. The precise definition of these categories is //[[CL:Glossary:implementation-defined]]//. The precision (measured in "bits", computed as ''p * log<sub>2</sub>b'') and the exponent size (also measured in "bits," computed as ''log<sub>2</sub>(n + 1)'', where ''n'' is the maximum exponent value) is recommended to be at least as great as the values in the below table. Each of the defined subtypes of **[[CL:Types:float]]** might or might not have a minus zero.

^ Format ^ Minimum Precision ^ Minimum Exponent Size ^
| Short  | 13 bits           | 5 bits                |
| Single | 24 bits           | 8 bits                |
| Double | 50 bits           | 8 bits                |
| Long   | 50 bits           | 8 bits                |

There can be fewer than four internal representations for //[[CL:Glossary:floats]]//. If there are fewer distinct representations, the following rules apply: 
  * If there is only one, it is the type **single-float**. In this representation, an //[[CL:Glossary:object]]// is simultaneously of //[[CL:Glossary:type|types]]// **single-float**, **double-float**, **short-float**, and **long-float**. 
  * Two internal representations can be arranged in either of the following ways: 
    * Two //[[CL:Glossary:type|types]]// are provided: **single-float** and **short-float**. An //[[CL:Glossary:object]]// is simultaneously of //[[CL:Glossary:type|types]]// **single-float**, **double-float**, and **long-float**. 
    * Two //[[CL:Glossary:type|types]]// are provided: **single-float** and **double-float**. An //[[CL:Glossary:object]]// is simultaneously of //[[CL:Glossary:type|types]]// **single-float** and **short-float**, or **double-float** and **long-float**.
  * Three internal representations can be arranged in either of the following ways: 
    * Three //[[CL:Glossary:type|types]]// are provided: **short-float**, **single-float**, and **double-float**. An //[[CL:Glossary:object]]// can simultaneously be of //[[CL:Glossary:type]]// **double-float** and **long-float**. 
    *  Three //[[CL:Glossary:type|types]]// are provided: **single-float**, **double-float**,and **long-float**. An //[[CL:Glossary:object]]// can simultaneously be of //[[CL:Glossary:type|types]]// **single-float** and **short-float**.

====Compound Type Specifier Kind====
Abbreviating.

====Compound Type Specifier Syntax====
  * **short-float** [//short-lower-limit// [//short-upper-limit//]] 
  * **single-float** [//single-lower-limit// [//single-upper-limit//]] 
  * **double-float** [//double-lower-limit// [//double-upper-limit//]] 
  * **long-float** [//long-lower-limit// [//long-upper-limit//]]

====Compound Type Specifier Arguments====

//short-lower-limit//, //short-upper-limit// - //[[CL:Glossary:interval designator|interval designators]]// for //[[CL:Glossary:type]]// **short-float**. The default for each //lower-limit// and //upper-limit// is the //[[CL:Glossary:symbol]]// **[[CL:Types:wildcard|*]]**.

//single-lower-limit//, //single-upper-limit// - //[[CL:Glossary:interval designator|interval designators]]// for //[[CL:Glossary:type]]// **single-float**. The default for each //lower-limit// and //upper-limit// is the //[[CL:Glossary:symbol]]// **[[CL:Types:wildcard|*]]**.

//double-lower-limit//, //double-upper-limit// - //[[CL:Glossary:interval designator|interval designators]]// for //[[CL:Glossary:type]]// **double-float**. The default for each //lower-limit// and //upper-limit// is the //[[CL:Glossary:symbol]]// **[[CL:Types:wildcard|*]]**.

//long-lower-limit//, //long-upper-limit// - //[[CL:Glossary:interval designator|interval designators]]// for //[[CL:Glossary:type]]// **long-float**. The default for each //lower-limit// and //upper-limit// is the //[[CL:Glossary:symbol]]// **[[CL:Types:wildcard|*]]**.

====Compound Type Specifier Description====
Each of these denotes the set of //[[CL:Glossary:floats]]// of the indicated //[[CL:Glossary:type]]// that are on the interval specified by the //[[CL:Glossary:interval designator|interval designators]]//.

\issue{REAL-NUMBER-TYPE:X3J13-MAR-89} \issue{REAL-NUMBER-TYPE:X3J13-MAR-89} \issue{REAL-NUMBER-TYPE:X3J13-MAR-89} \issue{REAL-NUMBER-TYPE:X3J13-MAR-89}
