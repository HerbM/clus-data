====== Function DIGIT-CHAR ======

====Syntax====

**digit-char {weight** //\opt} radix// → //char//

====Arguments and Values====

//weight// - a non-negative //[[CL:Glossary:integer]]//.

//radix// - a //[[CL:Glossary:radix]]//. The default is ''10''.

//char// - a //[[CL:Glossary:character]]// or //[[CL:Glossary:false]]//.

====Description====

If //weight// is less than //radix//, **[[CL:Functions:digit-char]]** returns a //[[CL:Glossary:character]]// which has that //weight// when considered as a digit in the specified radix. If the resulting //[[CL:Glossary:character]]// is to be an //[[CL:Glossary:alphabetic]]// //[[CL:Glossary:character]]//, it will be an uppercase //[[CL:Glossary:character]]//.

If //weight// is greater than or equal to //radix//, **[[CL:Functions:digit-char]]** returns //[[CL:Glossary:false]]//.

====Examples====

<blockquote> (digit-char 0) → #\\0 (digit-char 10 11) → #\\A (digit-char 10 10) → //[[CL:Glossary:false]]// (digit-char 7) → #\\7 (digit-char 12) → //[[CL:Glossary:false]]// (digit-char 12 16) → #\\C ;not #\\c (digit-char 6 2) → //[[CL:Glossary:false]]// (digit-char 1 2) → #\\1 </blockquote>

====Affected By====

None.

====Exceptional Situations====

None.

====See Also====

**[[CL:Functions:digit-char-p]]**, **[[CL:Functions:graphic-char-p]]**, {\secref\CharacterSyntax}

====Notes====

\issue{CHARACTER-PROPOSAL:2-1-1}
