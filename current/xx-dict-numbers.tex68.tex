====== Function DPB ======

====Syntax====
  * **dpb** //newbyte bytespec integer// → //result-integer//

====Arguments and Values====
  * //newbyte// - an //[[CL:Glossary:integer]]//.
  * //bytespec// - a //[[CL:Glossary:byte specifier]]//.
  * //integer// - an //[[CL:Glossary:integer]]//.
  * //result-integer// - an //[[CL:Glossary:integer]]//.

====Description====
**dpb** (deposit byte) is used to replace a field of bits within //integer//. **dpb** returns an //[[CL:Glossary:integer]]// that is the same as //integer// except in the bits specified by //bytespec//.

Let ''s'' be the size specified by //bytespec//; then the low ''s'' bits of //newbyte// appear in the result in the byte specified by //bytespec//. //newbyte// is interpreted as being right-justified, as if it were the result of **[[CL:Functions:ldb]]**.

====Examples====
<blockquote>
(dpb 1 ([[CL:Functions:byte]] 1 10) 0) <r>1024</r>
(dpb -2 ([[CL:Functions:byte]] 2 10) 0) <r>2048</r>
(dpb 1 ([[CL:Functions:byte]] 2 10) 2048) <r>1024</r>
</blockquote>

====Side Effects====

None.

====Affected By====
None.

====Exceptional Situations====
None.

====See Also====
  * **[[CL:Functions:byte|Function BYTE]]**
  * **[[CL:Functions:deposit-field|Function DEPOSIT-FIELD]]**
  * **[[CL:Functions:ldb|Function LDB]]**

====Notes====
<blockquote>
([[CL:Functions:logbitp]] //j// (dpb //m// ([[CL:Functions:byte]] //s// //p//) //n//)) 
  ≡ ([[CL:Special Operators:if]] ([[CL:Macros:and]] ([[CL:Functions:math-not-less|>=]] //j// //p//) 
             ([[CL:Functions:math-less|<]] //j// (+ //p// //s//)))
        ([[CL:Functions:logbitp]] (- //j// //p//) //m//) 
        ([[CL:Functions:logbitp]] //j// //n//))
</blockquote>

In general:

<blockquote>
(dpb //x// ([[CL:Functions:byte]] 0 //y//) //z//) <r>//z//</r>
</blockquote>

for all valid values of //x//, //y//, and //z//.

Historically, the name "dpb" comes from a DEC PDP-10 assembly language instruction meaning "deposit byte."
