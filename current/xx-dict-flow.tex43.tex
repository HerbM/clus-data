====== Special Operator IF ======

====Syntax====

\DefspecWithValues if {//test-form// //then-form// [//else-form//]} {\starparam{result}}

====Arguments and Values====
  * //test-form// - a //[[CL:Glossary:form]]//.
  * //then-form// - a //[[CL:Glossary:form]]//.
  * //else-form// - a //[[CL:Glossary:form]]//. The default is **[[CL:Constant Variables:nil]]**.
  * //results// - if the //test-form// //[[CL:Glossary:yielded]]// //[[CL:Glossary:true]]//, the //[[CL:Glossary:value|values]]// returned by the //then-form//; otherwise, the //[[CL:Glossary:value|values]]// returned by the //else-form//.

====Description====

**[[CL:Special Operators:if]]** allows the execution of a //[[CL:Glossary:form]]// to be dependent on a single //test-form//.

First //test-form// is evaluated. If the result is //[[CL:Glossary:true]]//, then //then-form// is selected; otherwise //else-form// is selected. Whichever form is selected is then evaluated.

====Examples====

<blockquote> (if t 1) → 1 (if nil 1 2) → 2 (defun test () (dolist (truth-value '(t nil 1 (a b c))) (if truth-value (print 'true) (print 'false)) (prin1 truth-value))) → TEST (test)
▷ TRUE T
▷ FALSE NIL
▷ TRUE 1
▷ TRUE (A B C) → NIL </blockquote>

====Affected By====

None.

====Exceptional Situations====

None.

====See Also====

**[[CL:Macros:cond]]**, **[[CL:Macros:unless]]**, **[[CL:Macros:when]]**

====Notes====

<blockquote> (if //test-form// //then-form// //else-form//) ≡ (cond (//test-form// //then-form//) (t //else-form//)) </blockquote>

