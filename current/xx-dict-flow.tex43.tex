====== Special Operator IF ======

====Syntax====
  * **if** //test-form then-form [else-form]// → //result''*''//

====Arguments and Values====
  * //test-form// - a //[[CL:Glossary:form]]//.
  * //then-form// - a //[[CL:Glossary:form]]//.
  * //else-form// - a //[[CL:Glossary:form]]//. The default is **[[CL:Constant Variables:nil]]**.
  * //results// - if the //test-form// //[[CL:Glossary:yield|yielded]]// //[[CL:Glossary:true]]//, the //[[CL:Glossary:value|values]]// returned by the //then-form//; otherwise, the //[[CL:Glossary:value|values]]// returned by the //else-form//.

====Description====
**[[CL:Special Operators:if]]** allows the execution of a //[[CL:Glossary:form]]// to be dependent on a single //test-form//.

First //test-form// is evaluated. If the result is //[[CL:Glossary:true]]//, then //then-form// is selected; otherwise //else-form// is selected. Whichever form is selected is then evaluated.

====Examples====
<blockquote>
(if [[CL:Constant Variables:t]] 1) <r>1</r>
(if [[CL:Constant Variables:nil]] 1 2) <r>2</r>
([[CL:Macros:defun]] test () 
  ([[CL:Macros:dolist]] (truth-value '([[CL:Constant Variables:t]] [[CL:Constant Variables:nil]] 1 (a b c)))
    (if truth-value 
        ([[CL:Functions:print]] 'true)
        ([[CL:Functions:print]] 'false))
    ([[CL:Functions:prin1]] truth-value))) <r>TEST </r>
(test)
<o>TRUE [[CL:Constant Variables:t|T]]
FALSE [[CL:Constant Variables:nil|NIL]]
TRUE 1
TRUE (A B C) </o>
<r>[[CL:Constant Variables:NIL]]</r>
</blockquote>

====Affected By====
None.

====Exceptional Situations====
None.

====See Also====
  * **[[CL:Macros:cond|Macro COND]]**
  * **[[CL:Macros:unless|Macro UNLESS]]**
  * **[[CL:Macros:when|Macro WHEN]]**

====Notes====
<blockquote>
(if //test-form// //then-form// //else-form//) 
  ≡ (cond (//test-form// //then-form//) (t //else-form//))
</blockquote>

