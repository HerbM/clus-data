====== Standard Generic Function DESCRIBE-OBJECT ======


====Syntax====

\DefgenWithValues {describe-object} {object stream} {//[[CL:Glossary:implementation-dependent]]//}

====Method Signatures====

\Defmeth {describe-object} {(//object// standard-object) //stream//}

====Arguments and Values====

//object// - an //[[CL:Glossary:object]]//.

//stream// - a //[[CL:Glossary:stream]]//.

====Description====

The generic function **[[CL:Functions:describe-object]]** prints a description of //object// to a //stream//. **[[CL:Functions:describe-object]]** is called by **[[CL:Functions:describe]]**; it must not be called by the user.


Each implementation is required to provide a //[[CL:Glossary:method]]// on \theclass{standard-object} and //[[CL:Glossary:methods]]// on enough other //[[CL:Glossary:classes]]// so as to ensure that there is always an applicable //[[CL:Glossary:method]]//. Implementations are free to add //[[CL:Glossary:methods]]// for other //[[CL:Glossary:classes]]//. Users can write //[[CL:Glossary:methods]]// for **[[CL:Functions:describe-object]]** for their own //[[CL:Glossary:classes]]// if they do not wish to inherit an implementation-supplied //[[CL:Glossary:method]]//.

//[[CL:Glossary:Methods]]// on **[[CL:Functions:describe-object]]** can recursively call **[[CL:Functions:describe]]**. Indentation, depth limits, and circularity detection are all taken care of automatically, provided that each //[[CL:Glossary:method]]// handles exactly one level of structure and calls **[[CL:Functions:describe]]** recursively if there are more structural levels. The consequences are undefined if this rule is not obeyed.

In some implementations the //stream// argument passed to a **[[CL:Functions:describe-object]]** method is not the original //stream//, but is an intermediate //[[CL:Glossary:stream]]// that implements parts of **[[CL:Functions:describe]]**. //[[CL:Glossary:Methods]]// should therefore not depend on the identity of this //[[CL:Glossary:stream]]//.


====Examples====

<blockquote> (defclass spaceship () ((captain :initarg :captain :accessor spaceship-captain) (serial# :initarg :serial-number :accessor spaceship-serial-number)))

(defclass federation-starship (spaceship) ())

(defmethod describe-object ((s spaceship) stream) (with-slots (captain serial#) s (format stream "~&~S is a spaceship of type ~S,~ ~ and with serial number ~D.~ s (type-of s) captain serial#)))

(make-instance 'federation-starship :captain "Rachel Garrett" :serial-number "NCC-1701-C") → #<FEDERATION-STARSHIP 26312465>

(describe *)
▷ #<FEDERATION-STARSHIP 26312465> is a spaceship of type FEDERATION-STARSHIP,
▷ with Rachel Garrett at the helm and with serial number NCC-1701-C. → \novalues </blockquote>

====Affected By====

None.

====Exceptional Situations====

None.

====See Also====

**[[CL:Functions:describe]]**

====Notes====

The same implementation techniques that are applicable to **[[CL:Functions:print-object]]** are applicable to **[[CL:Functions:describe-object]]**.

The reason for making the return values for **[[CL:Functions:describe-object]]** unspecified is to avoid forcing users to include explicit ''(values)'' in all of their //[[CL:Glossary:methods]]//. **[[CL:Functions:describe]]** takes care of that.


\issue{DESCRIBE-UNDERSPECIFIED:DESCRIBE-OBJECT}
