====== Function EVERY, SOME, NOTEVERY, NOTANY ======

====Syntax====
  * **every ** //predicate ''&rest'' sequences''+''// → //generalized-boolean// 
  * **some ** //predicate ''&rest'' sequences''+''// → //result// 
  * **notevery** //predicate ''&rest'' sequences''+''// → //generalized-boolean// 
  * **notany ** //predicate ''&rest'' sequences''+''// → //generalized-boolean//

====Arguments and Values====
  * //predicate// - a //[[CL:Glossary:designator]]// for a //[[CL:Glossary:function]]// of as many //[[CL:Glossary:arguments]]// as there are //sequences//.
  * //sequence// - a //[[CL:Glossary:sequence]]//.
  * //result// - an //[[CL:Glossary:object]]//.
  * //generalized-boolean// - a //[[CL:Glossary:generalized boolean]]//.

====Description====
**[[CL:Functions:every]]**, **[[CL:Functions:some]]**, **[[CL:Functions:notevery]]**, and **[[CL:Functions:notany]]** test //[[CL:Glossary:element|elements]]// of //sequences// for satisfaction of a given //predicate//.

The first argument to //predicate// is an //[[CL:Glossary:element]]// of the first //sequence//; each succeeding argument is an //[[CL:Glossary:element]]// of a succeeding //sequence//.

//Predicate// is first applied to the elements with index ''0'' in each of the //sequences//, and possibly then to the elements with index ''1'', and so on, until a termination criterion is met or the end of the shortest of the //sequences// is reached.

**[[CL:Functions:every]]** returns //[[CL:Glossary:false]]// as soon as any invocation of //predicate// returns //[[CL:Glossary:false]]//. If the end of a //sequence// is reached, **[[CL:Functions:every]]** returns //[[CL:Glossary:true]]//. Thus, **[[CL:Functions:every]]** returns //[[CL:Glossary:true]]// if and only if every invocation of //predicate// returns //[[CL:Glossary:true]]//.

**[[CL:Functions:some]]** returns the first //[[CL:Glossary:non-nil]]// value which is returned by an invocation of //predicate//. If the end of a //sequence// is reached without any invocation of the //predicate// returning //[[CL:Glossary:true]]//, **[[CL:Functions:some]]** returns //[[CL:Glossary:false]]//. Thus, **[[CL:Functions:some]]** returns //[[CL:Glossary:true]]// if and only if some invocation of //predicate// returns //[[CL:Glossary:true]]//.

**[[CL:Functions:notany]]** returns //[[CL:Glossary:false]]// as soon as any invocation of //predicate// returns //[[CL:Glossary:true]]//. If the end of a //sequence// is reached, **[[CL:Functions:notany]]** returns //[[CL:Glossary:true]]//. Thus, **[[CL:Functions:notany]]** returns //[[CL:Glossary:true]]// if and only if it is not the case that any invocation of //predicate// returns //[[CL:Glossary:true]]//.

**[[CL:Functions:notevery]]** returns //[[CL:Glossary:true]]// as soon as any invocation of //predicate// returns //[[CL:Glossary:false]]//. If the end of a //sequence// is reached, **[[CL:Functions:notevery]]** returns //[[CL:Glossary:false]]//. Thus, **[[CL:Functions:notevery]]** returns //[[CL:Glossary:true]]// if and only if it is not the case that every invocation of //predicate// returns //[[CL:Glossary:true]]//.

====Examples====
<blockquote> (every #'characterp "abc") → //[[CL:Glossary:true]]// (some #'= '(1 2 3 4 5) '(5 4 3 2 1)) → //[[CL:Glossary:true]]// (notevery #'< '(1 2 3 4) '(5 6 7 8) '(9 10 11 12)) → //[[CL:Glossary:false]]// (notany #'> '(1 2 3 4) '(5 6 7 8) '(9 10 11 12)) → //[[CL:Glossary:true]]// </blockquote>

====Affected By====
None.

====Exceptional Situations====
Should signal **[[CL:Types:type-error]]** if its first argument is neither a //[[CL:Glossary:symbol]]// nor a //[[CL:Glossary:function]]// or if any subsequent argument is not a //[[CL:Glossary:proper sequence]]//.

Other exceptional situations are possible, depending on the nature of the //predicate//.

====See Also====
  * **[[CL:Macros:and|Macro AND]]**
  * **[[CL:Macros:or|Macro OR]]**,

{\secref\TraversalRules}

====Notes====
<blockquote> (notany //predicate// \starparam{sequence}) ≡ (not (some //predicate// \starparam{sequence})) (notevery //predicate// \starparam{sequence}) ≡ (not (every //predicate// \starparam{sequence})) </blockquote>

\issue{MAPPING-DESTRUCTIVE-INTERACTION:EXPLICITLY-VAGUE}
