====== Type Specifier SATISFIES ======

====Compound Type Specifier Kind====
Predicating.

====Compound Type Specifier Syntax====
//**satisfies** predicate-name//

====Compound Type Specifier Arguments====
//predicate-name// - a //[[CL:Glossary:symbol]]//.

====Compound Type Specifier Description====
This denotes the set of all //[[CL:Glossary:object|objects]]// that satisfy the //[[CL:Glossary:predicate]]// //predicate-name//, which must be a //[[CL:Glossary:symbol]]// whose global //[[CL:Glossary:function]]// definition is a one-argument predicate. A name is required for //predicate-name//; //[[CL:Glossary:lambda expressions]]// are not allowed. For example, the //[[CL:Glossary:type specifier]]// **[[CL:Functions:(and integer (satisfies evenp))]]** denotes the set of all even integers. The form **[[CL:Functions:(typep //x// '(satisfies //p//))]]** is equivalent to **[[CL:Functions:(if (//p// //x//) t nil)]]**.

The argument is required. The //[[CL:Glossary:symbol]]// **[[CL:Types:wildcard|*]]** can be the argument, but it denotes itself (the //[[CL:Glossary:symbol]]// **[[CL:Types:wildcard|*]]**), and does not represent an unspecified value.

The symbol **[[CL:Types:satisfies]]** is not valid as a //[[CL:Glossary:type specifier]]//.

\issue{TYPE-SPECIFIER-ABBREVIATION:X3J13-JUN90-GUESS}
