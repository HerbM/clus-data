====== Special Operator PROGV ======

====Syntax====

\DefspecWithValues progv {//symbols// //values// \starparam{form}} {\starparam{result}}

====Arguments and Values====

//symbols// - a //[[CL:Glossary:list]]// of //[[CL:Glossary:symbols]]//; \eval.

//values// - a //[[CL:Glossary:list]]// of //[[CL:Glossary:object|objects]]//; \eval.

//forms// - an //[[CL:Glossary:implicit progn]]//.

//results// - the //[[CL:Glossary:value|values]]// returned by the //[[CL:Glossary:forms]]//.

====Description====

\specref{progv} creates new dynamic variable //[[CL:Glossary:bindings]]// and executes each //form// using those //[[CL:Glossary:bindings]]//. Each //form// is evaluated in order.

\specref{progv} allows //[[CL:Glossary:binding]]// one or more dynamic variables whose names may be determined at run time. Each //form// is evaluated in order with the dynamic variables whose names are in //symbols// bound to corresponding //values//. If too few //values// are supplied, the remaining //[[CL:Glossary:symbols]]// are bound and then made to have no value. If too many //values// are supplied, the excess values are ignored. The //[[CL:Glossary:bindings]]// of the dynamic variables are undone on exit from \specref{progv}.

====Examples==== <blockquote> ([[CL:Macros:defparameter]] *x* 1) → 1 (progv '(*x*) '(2) *x*) → 2 *x* → 1

Assuming *x* is not globally special,

(let ((*x* 3)) (progv '(*x*) '(4) (list *x* (symbol-value '*x*)))) → (3 4) </blockquote>

====Affected By====

None.

====Exceptional Situations====

None.

====See Also====

\specref{let}, {\secref\Evaluation}

====Notes====

Among other things, \specref{progv} is useful when writing interpreters for languages embedded in Lisp; it provides a handle on the mechanism for //[[CL:Glossary:binding]]// //[[CL:Glossary:dynamic variables]]//.

