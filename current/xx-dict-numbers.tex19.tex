======Function FLOOR, FFLOOR, CEILING, FCEILING, TRUNCATE, FTRUNCATE, ROUND, FROUND======

====Syntax====
  * **floor** //number ''&optional'' divisor// → //quotient, remainder//
  * **ffloor** //number ''&optional'' divisor// → //quotient, remainder//
  * **ceiling** //number ''&optional'' divisor// → //quotient, remainder//
  * **fceiling** //number ''&optional'' divisor// → //quotient, remainder//
  * **truncate** //number ''&optional'' divisor// → //quotient, remainder//
  * **ftruncate** //number ''&optional'' divisor// → //quotient, remainder//
  * **round** //number ''&optional'' divisor// → //quotient, remainder//
  * **fround** //number ''&optional'' divisor// → //quotient, remainder//

====Arguments and Values====
  * //number// - a //[[CL:Glossary:real]]//.
  * //divisor// - a non-zero //[[CL:Glossary:real]]//. The default is the //[[CL:Glossary:integer]]// ''1''.
  * //quotient// - for **floor**, **ceiling**, **truncate**, and **round**: an //[[CL:Glossary:integer]]//; for **ffloor**, **fceiling**, **ftruncate**, and **fround**: a //[[CL:Glossary:float]]//.
  * //remainder// - a //[[CL:Glossary:real]]//.

====Description====
These functions divide //number// by //divisor//, returning a //quotient// and //remainder//, such that:

<blockquote>
//quotient// * //divisor// + //remainder// = //number//
</blockquote>

The //quotient// always represents a mathematical integer.

When more than one mathematical integer might be possible (i.e. when the remainder is not zero), the kind of rounding or truncation depends on the //[[CL:Glossary:operator]]//:

===floor, ffloor===
**floor** and **ffloor** produce a //quotient// that has been truncated toward negative infinity; that is, the //quotient// represents the largest mathematical integer that is not larger than the mathematical quotient.

===ceiling, fceiling===
**ceiling** and **fceiling** produce a //quotient// that has been truncated toward positive infinity; that is, the //quotient// represents the smallest mathematical integer that is not smaller than the mathematical result.

===truncate, ftruncate===
**truncate** and **ftruncate** produce a //quotient// that has been truncated towards zero; that is, the //quotient// represents the mathematical integer of the same sign as the mathematical quotient, and that has the greatest integral magnitude not greater than that of the mathematical quotient.

===round, fround===
**round** and **fround** produce a //quotient// that has been rounded to the nearest mathematical integer; if the mathematical quotient is exactly halfway between two integers, (that is, it has the form ''integer + 1/2''), then the //quotient// has been rounded to the even (divisible by two) integer.

----

All of these functions perform type conversion operations on //numbers//.

The //remainder// is an //[[CL:Glossary:integer]]// if both //x// and //y// are //[[CL:Glossary:integers]]//, is a //[[CL:Glossary:rational]]// if both //x// and //y// are //[[CL:Glossary:rationals]]//, and is a //[[CL:Glossary:float]]// if either //x// or //y// is a //[[CL:Glossary:float]]//.

**ffloor**, **fceiling**, **ftruncate**, and **fround** handle arguments of different //[[CL:Glossary:type|types]]// in the following way: If //number// is a //[[CL:Glossary:float]]//, and //divisor// is not a //[[CL:Glossary:float]]// of longer format, then the first result is a //[[CL:Glossary:float]]// of the same //[[CL:Glossary:type]]// as //number//. Otherwise, the first result is of the //[[CL:Glossary:type]]// determined by //[[CL:Glossary:contagion]]// rules; see section {\secref\NumericContagionRules}.

====Examples====
<blockquote>
(floor 3/2) <r>1
1/2 </r>
(ceiling 3 2) <r>2
-1 </r>
(ffloor 3 2) <r>1.0
1 </r>
(ffloor -4.7) <r>-5.0
0.3 </r>
(ffloor 3.5d0) <r>3.0d0
0.5d0 </r>
(fceiling 3/2) <r>2.0
-1/2 </r>
(truncate 1) <r>1
0 </r>
(truncate .5) <r>0
0.5 </r>
(round .5) <r>0
0.5 </r>
(ftruncate -7 2) <r>-3.0
-1 </r>
(fround -7 2) <r>-4.0
1 </r>
([[CL:Macros:dolist]] (n '(2.6 2.5 2.4 0.7 0.3 -0.3 -0.7 -2.4 -2.5 -2.6)) 
  ([[CL:Functions:format]] [[CL:Constant Variables:t]] "~&~4,1@F ~2,' D ~2,' D ~2,' D ~2,' D" 
          n (floor n) (ceiling n) (truncate n) (round n)))
<o>+2.6  2  3  2  3
+2.5  2  3  2  2
+2.4  2  3  2  2
+0.7  0  1  0  1
+0.3  0  1  0  0
-0.3 -1  0  0  0
-0.7 -1  0  0 -1
-2.4 -3 -2 -2 -2
-2.5 -3 -2 -2 -2
-2.6 -3 -2 -2 -3 </o>
<r>NIL </r>
</blockquote>

====Side Effects====
None.

====Affected By====
None.

====Exceptional Situations====
None.

====See Also====
None.

====Notes====
When only //number// is given, the two results are exact; the mathematical sum of the two results is always equal to the mathematical value of //number//.

''(function //number// //divisor//)'' and ''(function (/ //number// //divisor//))'' (where ''function'' is any of one of **floor**, **ceiling**, **ffloor**, **fceiling**, **truncate**, **round**, **ftruncate**, and **fround**) return the same first value, but they return different remainders as the second value. For example:

<blockquote>
(floor 5 2) <r>2
1</r>
(floor (/ 5 2)) <r>2
1/2</r>
</blockquote>

If an effect is desired that is similar to **round**, but that always rounds up or down (rather than toward the nearest even integer) if the mathematical quotient is exactly halfway between two integers, the programmer should consider a construction such as ''(floor ([[CL:Functions:math-add|+]] x 1/2))'' or ''(ceiling ([[CL:Functions:math-subtract|-]] x 1/2))''.
