====== Function INTEGER-LENGTH ======

====Syntax====
  * **integer-length** //integer// → //number-of-bits//

====Arguments and Values====
  * //integer// - an //[[CL:Glossary:integer]]//.
  * //number-of-bits// - a non-negative //[[CL:Glossary:integer]]//.

====Description====
Returns the number of bits needed to represent //integer// in binary two's-complement format.

====Examples====
<blockquote> 
(integer-length 0) <r>0 </r>
(integer-length 1) <r>1 </r>
(integer-length 3) <r>2 </r>
(integer-length 4) <r>3 </r>
(integer-length 7) <r>3 </r>
(integer-length -1) <r>0</r>
(integer-length -4) <r>2 </r>
(integer-length -7) <r>3 </r>
(integer-length -8) <r>3 </r>
(integer-length ([[CL:Functions:expt]] 2 9)) <r>10 </r>
(integer-length ([[CL:Functions:math-one-minus|1-]] ([[CL:Functions:expt]] 2 9))) <r>9 </r>
(integer-length ([[CL:Functions:math-subtract|-]] ([[CL:Functions:expt]] 2 9))) <r>9 </r>
(integer-length ([[CL:Functions:math-subtract|-]] ([[CL:Functions:math-one-plus|1+]] ([[CL:Functions:expt]] 2 9)))) <r>10 </r>
</blockquote>

====Side Effects====
None.

====Affected By====
None.

====Exceptional Situations====
Should signal an error of type type-error if //integer// is not an //[[CL:Glossary:integer]]//.

====See Also====
None.

====Notes====
This function could have been defined by:

<blockquote> 
([[CL:Macros:defun]] integer-length (integer) 
  ([[CL:Functions:ceiling]] ([[CL:Functions:log]] ([[CL:Special Operators:if]] ([[CL:Functions:minusp]] integer) 
                    ([[CL:Functions:math-subtract|-]] integer) 
                    ([[CL:Functions:math-one-plus|1+]] integer)) 
                2)))
</blockquote>

If //integer// is non-negative, then its value can be represented in unsigned binary form in a field whose width in bits is no smaller than ''(integer-length //integer//)''. Regardless of the sign of //integer//, its value can be represented in signed binary two's-complement form in a field whose width in bits is no smaller than **[[CL:Functions:(+ (integer-length //integer//) 1)]]**.

