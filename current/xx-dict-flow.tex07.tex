====== Function FUNCALL ======

====Syntax====

\DefunWithValues funcall {function ''&rest'' args} {\starparam{result}}

====Arguments and Values====

//function// - a //[[CL:Glossary:function designator]]//.

//args// - //[[CL:Glossary:arguments]]// to the //function//.

//results// - the //[[CL:Glossary:value|values]]// returned by the //function//.

====Description====

**[[CL:Functions:funcall]]** applies //function// to //args//.

If //function// is a //[[CL:Glossary:symbol]]//, it is coerced to a //[[CL:Glossary:function]]// as if by finding its //[[CL:Glossary:functional value]]// in the //[[CL:Glossary:global environment]]//.

====Examples====

<blockquote> (funcall #'+ 1 2 3) → 6 (funcall 'car '(1 2 3)) → 1 (funcall 'position 1 '(1 2 3 2 1) :start 1) → 4 (cons 1 2) → (1 . 2) (flet ((cons (x y) `(kons ,x ,y))) (let ((cons (symbol-function '+))) (funcall #'cons (funcall 'cons 1 2) (funcall cons 1 2)))) → (KONS (1 . 2) 3) </blockquote>

====Affected By====

None.

====Exceptional Situations====

An error of type **[[CL:Types:undefined-function]]** should be signaled if //function// is a //[[CL:Glossary:symbol]]// that does not have a global definition as a //[[CL:Glossary:function]]// or that has a global definition as a //[[CL:Glossary:macro]]// or a //[[CL:Glossary:special operator]]//.

====See Also====

**[[CL:Functions:apply]]**, **[[CL:Special Operators:function]]**, {\secref\Evaluation}

====Notes====

<blockquote> (funcall //function// //arg1// //arg2// ...) ≡ (apply //function// //arg1// //arg2// ... nil) ≡ (apply //function// (list //arg1// //arg2// ...)) </blockquote>

The difference between **[[CL:Functions:funcall]]** and an ordinary function call is that in the former case the //function// is obtained by ordinary //[[CL:Glossary:evaluation]]// of a //[[CL:Glossary:form]]//, and in the latter case it is obtained by the special interpretation of the function position that normally occurs.

\issue{FUNCTION-TYPE:X3J13-MARCH-88}
