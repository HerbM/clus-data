====== Variable *PRINT-READABLY* ======

====Value Type====
a //[[CL:Glossary:generalized boolean]]//.

====Initial Value====
//[[CL:Glossary:false]]//.

====Description====
If **[[CL:Variables:star-print-readably-star|*print-readably*]]** is //[[CL:Glossary:true]]//, some special rules for printing //[[CL:Glossary:object|objects]]// go into effect. Specifically, printing any //[[CL:Glossary:object]]// ''O\sub 1'' produces a printed representation that, when seen by the //[[CL:Glossary:Lisp reader]]//

while the //[[CL:Glossary:standard readtable]]// is in effect, will produce an //[[CL:Glossary:object]]// ''O\sub 2'' that is //[[CL:Glossary:similar]]// to ''O\sub 1''. The printed representation produced might or might not be the same as the printed representation produced when **[[CL:Variables:star-print-readably-star|*print-readably*]]** is //[[CL:Glossary:false]]//.

If printing an //[[CL:Glossary:object]]// //[[CL:Glossary:readably]]// is not possible, an error of type **[[CL:Types:print-not-readable]]** is signaled rather than using a syntax (//e.g.// the ""#<'''' syntax) that would not be readable by the same //[[CL:Glossary:implementation]]//. If the //[[CL:Glossary:value]]// of some other //[[CL:Glossary:printer control variable]]// is such that these requirements would be violated, the //[[CL:Glossary:value]]// of that other //[[CL:Glossary:variable]]// is ignored.

Specifically, if **[[CL:Variables:star-print-readably-star|*print-readably*]]** is //[[CL:Glossary:true]]//, printing proceeds as if **[[CL:Variables:star-print-escape-star|*print-escape*]]**, **[[CL:Variables:star-print-array-star|*print-array*]]**, and **[[CL:Variables:star-print-gensym-star|*print-gensym*]]** were also //[[CL:Glossary:true]]//, and as if **[[CL:Variables:star-print-length-star|*print-length*]]**, **[[CL:Variables:star-print-level-star|*print-level*]]**, and **[[CL:Variables:star-print-lines-star|*print-lines*]]** were //[[CL:Glossary:false]]//.

If **[[CL:Variables:star-print-readably-star|*print-readably*]]** is //[[CL:Glossary:false]]//, the normal rules for printing and the normal interpretations of other //[[CL:Glossary:printer control variables]]// are in effect.

Individual //[[CL:Glossary:method|methods]]// for **[[CL:Functions:print-object]]**, including user-defined //[[CL:Glossary:method|methods]]//, are responsible for implementing these requirements.

If **[[CL:Variables:star-read-eval-star|*read-eval*]]** is //[[CL:Glossary:false]]// and **[[CL:Variables:star-print-readably-star|*print-readably*]]** is //[[CL:Glossary:true]]//, any such method that would output a reference to the ""#.'''' //[[CL:Glossary:reader macro]]// will either output something else or will signal an error (as described above).

====Examples====
<blockquote> (let ((x (list "a" '\\a (gensym) '((a (b (c))) d e f g))) (*print-escape* nil) (*print-gensym* nil) (*print-level* 3) (*print-length* 3)) (write x) (let ((*print-readably* t)) (terpri) (write x) :done))
▷ (a a G4581 ((A #) D E ...))
▷ ("a" |a| #:G4581 ((A (B (C))) D E F G)) → :DONE

;; This is setup code is shared between the examples ;; of three hypothetical implementations which follow. ([[CL:Macros:defparameter]] table (make-hash-table)) → #<HASH-TABLE EQL 0/120 32005763> ([[CL:Macros:setf]] (gethash table 1) 'one) → ONE ([[CL:Macros:setf]] (gethash table 2) 'two) → TWO

;; Implementation A (let ((*print-readably* t)) (print table)) Error: Can't print #<HASH-TABLE EQL 0/120 32005763> readably.

;; Implementation B ;; No standardized #S notation for hash tables is defined, ;; but there might be an implementation-defined notation. (let ((*print-readably* t)) (print table))
▷ #S(HASH-TABLE :TEST EQL :SIZE 120 :CONTENTS (1 ONE 2 TWO)) → #<HASH-TABLE EQL 0/120 32005763>

;; Implementation C ;; Note that #. notation can only be used if *READ-EVAL* is true. ;; If *READ-EVAL* were false, this same implementation might have to ;; signal an error unless it had yet another printing strategy to fall ;; back on. (let ((*print-readably* t)) (print table))
▷ #.(LET ((HASH-TABLE (MAKE-HASH-TABLE)))
▷ (SETF (GETHASH 1 HASH-TABLE) ONE)
▷ (SETF (GETHASH 2 HASH-TABLE) TWO)
▷ HASH-TABLE) → #<HASH-TABLE EQL 0/120 32005763> </blockquote>

====Affected By====
None.

====See Also====
**[[CL:Functions:write]]**, **[[CL:Macros:print-unreadable-object]]**

====Notes====
The rules for "//[[CL:Glossary:similarity]]//" imply that ''#A'' or ''#('' syntax cannot be used for //[[CL:Glossary:array|arrays]]// of //[[CL:Glossary:element type]]// other than **[[CL:Types:t]]**. An implementation will have to use another syntax or signal an error of type **[[CL:Types:print-not-readable]]**.

\issue{DATA-IO:ADD-SUPPORT} \issue{PRINT-READABLY-BEHAVIOR:CLARIFY} \issue{PRINT-READABLY-BEHAVIOR:CLARIFY}
