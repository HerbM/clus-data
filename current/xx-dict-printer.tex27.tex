====== Variable *PRINT-READABLY* ======

====Value Type====
a //[[CL:Glossary:generalized boolean]]//.

====Initial Value====
//[[CL:Glossary:false]]//.

====Description====
If **<nowiki>*print-readably*</nowiki>** is //[[CL:Glossary:true]]//, some special rules for printing //[[CL:Glossary:object|objects]]// go into effect. Specifically, printing any //[[CL:Glossary:object]]// ''O<sub>1</sub>'' produces a printed representation that, when seen by the //[[CL:Glossary:Lisp reader]]// while the //[[CL:Glossary:standard readtable]]// is in effect, will produce an //[[CL:Glossary:object]]// ''O<sub>2</sub>'' that is //[[CL:Glossary:similar]]// to ''O<sub>1</sub>''. The printed representation produced might or might not be the same as the printed representation produced when **<nowiki>*print-readably*</nowiki>** is //[[CL:Glossary:false]]//.

If printing an //[[CL:Glossary:object]]// //[[CL:Glossary:readably]]// is not possible, an error of type **[[CL:Types:print-not-readable]]** is signaled rather than using a syntax (//e.g.// the ''#<'' syntax) that would not be readable by the same //[[CL:Glossary:implementation]]//. If the //[[CL:Glossary:value]]// of some other //[[CL:Glossary:printer control variable]]// is such that these requirements would be violated, the //[[CL:Glossary:value]]// of that other //[[CL:Glossary:variable]]// is ignored.

Specifically, if **<nowiki>*print-readably*</nowiki>** is //[[CL:Glossary:true]]//, printing proceeds as if **[[CL:Variables:star-print-escape-star|*print-escape*]]**, **[[CL:Variables:star-print-array-star|*print-array*]]**, and **[[CL:Variables:star-print-gensym-star|*print-gensym*]]** were also //[[CL:Glossary:true]]//, and as if **[[CL:Variables:star-print-length-star|*print-length*]]**, **[[CL:Variables:star-print-level-star|*print-level*]]**, and **[[CL:Variables:star-print-lines-star|*print-lines*]]** were //[[CL:Glossary:false]]//.

If **<nowiki>*print-readably*</nowiki>** is //[[CL:Glossary:false]]//, the normal rules for printing and the normal interpretations of other //[[CL:Glossary:printer control variables]]// are in effect.

Individual //[[CL:Glossary:method|methods]]// for **[[CL:Functions:print-object]]**, including user-defined //[[CL:Glossary:method|methods]]//, are responsible for implementing these requirements.

If **[[CL:Variables:star-read-eval-star|*read-eval*]]** is //[[CL:Glossary:false]]// and **<nowiki>*print-readably*</nowiki>** is //[[CL:Glossary:true]]//, any such method that would output a reference to the ''#.'' //[[CL:Glossary:reader macro]]// will either output something else or will signal an error (as described above).

====Examples====
<blockquote> 
([[CL:Special Operators:let]] ((x ([[CL:Functions:list]] "a" '\a ([[CL:Functions:gensym]]) '((a (b (c))) d e f g))) 
      ([[CL:Variables:star-print-escape-star|*print-escape*]] nil) 
      ([[CL:Variables:star-print-gensym-star|*print-gensym*]] nil)
      ([[CL:Variables:star-print-level-star|*print-level*]] 3) 
      ([[CL:Variables:star-print-length-star|*print-length*]] 3))
  ([[CL:Functions:write]] x) 
  ([[CL:Special Operators:let]] ((*print-readably* t)) 
    ([[CL:Functions:terpri]]) 
    ([[CL:Functions:write]] x) 
    :done))
<o>(a a G4581 ((A #) D E ...))
("a" |a| #:G4581 ((A (B (C))) D E F G))</o> 
<r>:DONE</r>
</blockquote>

------

This setup code is shared between the examples of three hypothetical implementations which follow. 

<blockquote>
([[CL:Macros:defparameter]] *table* ([[CL:Functions:make-hash-table]])) <r>#<HASH-TABLE EQL 0/120 32005763> </r>
([[CL:Macros:setf]] ([[CL:Functions:gethash]] *table* 1) 'one) <r>ONE</r>
([[CL:Macros:setf]] ([[CL:Functions:gethash]] *table* 2) 'two) <r>TWO</r>
</blockquote>

==Implementation A==
If an //[[CL:Glossary:implementation]]// cannot print the hash table readably, it will //[[CL:Glossary:signal]]// an //[[CL:Glossary:error]]//.
<blockquote>
([[CL:Special Operators:let]] ((*print-readably* t))
  ([[CL:Functions:print]] table))
<e>Error: Can't print #<HASH-TABLE EQL 0/120 32005763> readably.</e>
</blockquote>

==Implementation B==
No standardized ''#S'' notation for //[[CL:Glossary:hash table|hash tables]]// is defined, but there might be an //[[CL:Glossary:implementation-defined]]// notation.
<blockquote>
([[CL:Special Operators:let]] ((*print-readably* t))
  ([[CL:Functions:print]] table))
<o>#S(HASH-TABLE :TEST EQL :SIZE 120 :CONTENTS (1 ONE 2 TWO)) </o>
<r>#<HASH-TABLE EQL 0/120 32005763></r>
</blockquote>

==Implementation C==
Note that ''#.'' notation can only be used if **[[CL:Variables:star-read-eval-star|*read-eval*]]** is true. If **[[CL:Variables:star-read-eval-star|*read-eval*]]** were false, this same //[[CL:Glossary:implementation]]// might have to //[[CL:Glossary:signal]]// an //[[CL:Glossary:error]]// unless it had yet another printing strategy to fall back on. 
<blockquote>
([[CL:Special Operators:let]] ((*print-readably* t))
  ([[CL:Functions:print]] table))
<o>#.([[CL:Special Operators:LET]] ((HASH-TABLE ([[CL:Functions:MAKE-HASH-TABLE]])))
    ([[CL:Macros:SETF]] ([[CL:Functions:GETHASH]] 1 HASH-TABLE) ONE)
    ([[CL:Macros:SETF]] ([[CL:Functions:GETHASH]] 2 HASH-TABLE) TWO)
    HASH-TABLE) </o>
<r>#<HASH-TABLE EQL 0/120 32005763></r>
</blockquote>

====Affected By====
None.

====See Also====
  * **[[CL:Functions:write|Function WRITE]]**
  * **[[CL:Macros:print-unreadable-object|Macro PRINT-UNREADABLE-OBJECT]]**

====Notes====
The rules for "//[[CL:Glossary:similarity]]//" imply that ''#A'' or ''#('' syntax cannot be used for //[[CL:Glossary:array|arrays]]// of //[[CL:Glossary:element type]]// other than **[[CL:Types:t]]**. An implementation will have to use another syntax or signal an error of type **[[CL:Types:print-not-readable]]**.

\issue{DATA-IO:ADD-SUPPORT} \issue{PRINT-READABLY-BEHAVIOR:CLARIFY} \issue{PRINT-READABLY-BEHAVIOR:CLARIFY}
