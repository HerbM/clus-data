====== Macro DESTRUCTURING-BIND ======

====Syntax====
  *  **destructuring-bind** //lambda-list expression declaration''*'' form''*''// //result''*''//

====Arguments and Values====
  * //lambda-list// - a //[[CL:Glossary:destructuring lambda list]]//.
  * //expression// - a //[[CL:Glossary:form]]//.
  * //declaration// - a **[[CL:Special Operators:declare]]** //[[CL:Glossary:expression]]//; not evaluated.
  * //form//s - an //[[CL:Glossary:implicit progn]]//.
  * //result//s - the //[[CL:Glossary:value|values]]// returned by the //[[CL:Glossary:form|forms]]//.

====Description====
**destructuring-bind** binds the variables specified in //lambda-list// to the corresponding values in the tree structure resulting from the evaluation of //expression//; then **destructuring-bind** evaluates //form//s.

The //lambda-list// supports destructuring as described in \secref\DestructuringLambdaLists.

====Examples====
<blockquote>
;;; TODO better examples for this, d-bind really deserves more love

(defun iota (n) (loop for i from 1 to n collect i)) ; helper
(destructuring-bind ((a &optional (b 'bee)) one two three) 
    `((alpha) ,@(iota 3))
  (list a b three two one)) <r>(ALPHA BEE 3 2 1)</r>
</blockquote>

====Affected By====
None.

====Exceptional Situations====
If the result of evaluating the //expression// does not match the destructuring pattern, an error of type **[[CL:Types:error]]** should be signaled.

====See Also====
  * **[[CL:Special Operators:macrolet|Special Operator MACROLET]]**
  * **[[CL:Macros:defmacro|Macro DEFMACRO]]**

====Notes====
None.

\issue{DECLS-AND-DOC} \issue{DESTRUCTURING-BIND:NEW-MACRO}
