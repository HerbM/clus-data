====== Macro ROTATEF ======

====Syntax====

\DefmacWithValues rotatef {\starparam{place}} {**[[CL:Constant Variables:nil]]**}

====Arguments and Values====
  * //place// - a //[[CL:Glossary:place]]//.

====Description====

**[[CL:Macros:rotatef]]** modifies the values of each //place// by rotating values from one //place// into another.

If a //place// produces more values than there are store variables, the extra values are ignored. If a //place// produces fewer values than there are store variables, the missing values are set to **[[CL:Constant Variables:nil]]**.

In the form ''(rotatef ''place1'' ''place2'' ... ''placen'')'', the values in ''place1'' through ''placen'' are //[[CL:Glossary:read]]// and //[[CL:Glossary:written]]//. Values 2 through ''n'' and value 1 are then stored into ''place1'' through ''placen''. It is as if all the places form an end-around shift register that is rotated one place to the left, with the value of ''place1'' being shifted around the end to ''placen''.

For information about the //[[CL:Glossary:evaluation]]// of //[[CL:Glossary:subforms]]// of //places//, see section {\secref\GenRefSubFormEval}.


====Examples==== <blockquote> (let ((n 0) (x (list 'a 'b 'c 'd 'e 'f 'g))) (rotatef (nth (incf n) x) (nth (incf n) x) (nth (incf n) x)) x) → (A C D B E F G) </blockquote>

====Affected By====

None.

====Exceptional Situations====

None.

====See Also====

**[[CL:Macros:define-setf-expander]]**, **[[CL:Macros:defsetf]]**, **[[CL:Macros:setf]]**, **[[CL:Macros:shiftf]]**, **[[CL:Variables:*macroexpand-hook*]]**, {\secref\GeneralizedReference}

====Notes====

The effect of ''(rotatef //place1// //place2// ... //placen//)'' is roughly equivalent to

<blockquote> (psetf //place1// //place2// //place2// //place3// ... //placen// //place1//) </blockquote> except that the latter would evaluate any //[[CL:Glossary:subforms]]// of each ''place'' twice, whereas **[[CL:Macros:rotatef]]** evaluates them once.

\issue{SETF-MULTIPLE-STORE-VARIABLES:ALLOW} \issue{PUSH-EVALUATION-ORDER:FIRST-ITEM}
