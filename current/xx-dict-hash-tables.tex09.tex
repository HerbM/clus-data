====== Accessor GETHASH ======

====Syntax====
  * **gethash** //key hash-table ''&optional'' default// → //value, present-p// 
  * (**setf** (**gethash** //key hash-table ''&optional'' default//) //new-value//)

====Arguments and Values====
  * //key// - an //[[CL:Glossary:object]]//.
  * //hash-table// - a //[[CL:Glossary:hash table]]//.
  * //default// - an //[[CL:Glossary:object]]//. The default is **[[CL:Constant Variables:nil]]**.
  * //value// - an //[[CL:Glossary:object]]//.
  * //present-p// - a //[[CL:Glossary:generalized boolean]]//.

====Description====
//value// is the //[[CL:Glossary:object]]// in //hash-table// whose //[[CL:Glossary:key]]// is the //[[CL:Glossary:same]]// as //key// under the //hash-table//'s equivalence test. If there is no such entry, //value// is the //default//.

//present-p// is //[[CL:Glossary:true]]// if an entry is found; otherwise, it is //[[CL:Glossary:false]]//.

**[[CL:Macros:setf]]** may be used with **gethash** to modify the //[[CL:Glossary:value]]// associated with a given //[[CL:Glossary:key]]//, or to add a new entry.

When a **gethash** //[[CL:Glossary:form]]// is used as a **[[CL:Macros:setf]]** //place//, any //default// which is supplied is evaluated according to normal left-to-right evaluation rules, but its //[[CL:Glossary:value]]// is ignored.

====Examples====
<blockquote>
([[CL:Macros:defparameter]] *table* (make-hash-table)) <r>*TABLE*</r>
(gethash 1 *table*) <r>[[CL:Constant Variables:nil|NIL]]
//[[CL:Glossary:false]]// </r>
(gethash 1 *table* 2) <r>2
//[[CL:Glossary:false]]// </r>
([[CL:Macros:setf]] (gethash 1 *table*) "one") <r>"one" </r>
([[CL:Macros:setf]] (gethash 2 *table* "two") "two") <r>"two" </r>
(gethash 1 *table*) <r>"one"
//[[CL:Glossary:true]]// </r>
(gethash 2 *table*) <r>"two"
//[[CL:Glossary:true]]// </r>
(gethash [[CL:Constant Variables:nil]] *table*) <r>[[CL:Constant Variables:nil|NIL]]
//[[CL:Glossary:false]]// </r>
([[CL:Macros:setf]] (gethash [[CL:Constant Variables:nil]] *table*) [[CL:Constant Variables:nil]]) <r>[[CL:Constant Variables:nil|NIL]] </r>
(gethash [[CL:Constant Variables:nil]] *table*) <r>[[CL:Constant Variables:nil|NIL]]
//[[CL:Glossary:true]]// </r>
(defvar *counters* (make-hash-table)) <r>*COUNTERS* </r>
(gethash 'foo *counters*) <r>[[CL:Constant Variables:nil|NIL]]
//[[CL:Glossary:false]]// </r>
(gethash 'foo *counters* 0) <r>0
//[[CL:Glossary:false]]// </r>
(defmacro how-many (obj) 
  `(values (gethash ,obj *counters* 0))) <r>HOW-MANY </r>
(defun count-it (obj) 
  (incf (how-many obj))) <r>COUNT-IT </r>
(dolist (x '(bar foo foo bar bar baz)) 
  (count-it x)) <r>[[CL:Constant Variables:nil|NIL]]</r>
(how-many 'foo) <r>2 </r>
(how-many 'bar) <r>3 </r>
(how-many 'quux) <r>0 </r>
</blockquote>

====Side Effects====
None.

====Affected By====
None.

====Exceptional Situations====
None.

====See Also====
  * **[[CL:Functions:remhash|Function REMHASH]]**

====Notes====
The //[[CL:Glossary:secondary value]]//, //present-p//, can be used to distinguish the absence of an entry from the presence of an entry that has a value of //default//.

\issue{SETF-GET-DEFAULT:EVALUATED-BUT-IGNORED}
