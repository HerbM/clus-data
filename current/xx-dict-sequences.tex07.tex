====== Function MAP ======

====Syntax====

**map** //result-type function ''&rest'' sequences''+''// → //result//

====Arguments and Values====

//result-type// -- a **[[CL:Types:sequence]]** //[[CL:Glossary:type specifier]]//, or **[[CL:Constant Variables:nil]]**.

//function// - a //[[CL:Glossary:function designator]]//. //function// must take as many arguments as there are //sequences//.

//sequence// - a //[[CL:Glossary:proper sequence]]//.

//result// - if //result-type// is a //[[CL:Glossary:type specifier]]// other than **[[CL:Constant Variables:nil]]**, then a //[[CL:Glossary:sequence]]// of the //[[CL:Glossary:type]]// it denotes; otherwise (if the //result-type// is **[[CL:Constant Variables:nil]]**), **[[CL:Constant Variables:nil]]**.

====Description====

Applies //function// to successive sets of arguments in which one argument is obtained from each //[[CL:Glossary:sequence]]//.

The //function// is called first on all the elements with index ''0'', then on all those with index ''1'', and so on. The //result-type// specifies the //[[CL:Glossary:type]]// of the resulting //[[CL:Glossary:sequence]]//.

**[[CL:Functions:map]]** returns **[[CL:Constant Variables:nil]]** if //result-type// is **[[CL:Constant Variables:nil]]**.

Otherwise, **[[CL:Functions:map]]** returns a //[[CL:Glossary:sequence]]// such that element ''j'' is the result of applying //function// to element ''j'' of each of the //sequences//. The result //[[CL:Glossary:sequence]]// is as long as the shortest of the //sequences//. The consequences are undefined if the result of applying //function// to the successive elements of the //sequences// cannot be contained in a //[[CL:Glossary:sequence]]// of the //[[CL:Glossary:type]]// given by //result-type//.

If the //result-type// is a //[[CL:Glossary:subtype]]// of **[[CL:Types:list]]**,the result will be a //[[CL:Glossary:list]]//.

If the //result-type// is a //[[CL:Glossary:subtype]]// of **[[CL:Types:vector]]**,then if the implementation can determine the element type specified for the //result-type//, the element type of the resulting array is the result of //[[CL:Glossary:upgrade|upgrading]]// that element type; or, if the implementation can determine that the element type is unspecified (or ''*''), the element type of the resulting array is **[[CL:Types:t]]**; otherwise, an error is signaled.

====Examples====

<blockquote> (map 'string #'(lambda (x y) (char "01234567890ABCDEF" (mod (+ x y) 16))) '(1 2 3 4) '(10 9 8 7)) → "AAAA" ([[CL:Macros:defparameter]] seq '("lower" "UPPER" "" "123")) → ("lower" "UPPER" "" "123") (map nil #'nstring-upcase seq) → NIL seq → ("LOWER" "UPPER" "" "123") (map 'list #'- '(1 2 3 4)) → (-1 -2 -3 -4) (map 'string #'(lambda (x) (if (oddp x) #\\1 #\\0)) '(1 2 3 4)) → "1010" </blockquote>

<blockquote> (map '(vector * 4) #'cons "abc" "de") should signal an error </blockquote>

====Affected By====

None.

====Exceptional Situations====

An error of type **[[CL:Types:type-error]]** must be signaled if the //result-type// is not a //[[CL:Glossary:recognizable subtype]]// of **[[CL:Types:list]]**, not a //[[CL:Glossary:recognizable subtype]]// of **[[CL:Types:vector]]**, and not **[[CL:Constant Variables:nil]]**.

Should be prepared to signal an error of type type-error if any //sequence// is not a //[[CL:Glossary:proper sequence]]//.

An error of type **[[CL:Types:type-error]]** should be signaled if //result-type// specifies the number of elements and the minimum length of the //sequences// is different from that number.

====See Also====

{\secref\TraversalRules}

====Notes====

None.

\issue{SEQUENCE-TYPE-LENGTH:MUST-MATCH} \issue{CONCATENATE-SEQUENCE:SIGNAL-ERROR} \issue{SEQUENCE-TYPE-LENGTH:MUST-MATCH} \issue{MAPPING-DESTRUCTIVE-INTERACTION:EXPLICITLY-VAGUE} \issue{CONCATENATE-SEQUENCE:SIGNAL-ERROR} \issue{FUNCTION-TYPE:X3J13-MARCH-88} \issue{CONCATENATE-SEQUENCE:SIGNAL-ERROR}
