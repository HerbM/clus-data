====== Function COERCE ======

====Syntax====
**coerce** //object result-type// → //result//

====Arguments and Values====
//object// - an //[[CL:Glossary:object]]//.

//result-type// - a //[[CL:Glossary:type specifier]]//.

//result// - an //[[CL:Glossary:object]]//, of //[[CL:Glossary:type]]// //result-type// except in situations described in \secref\RuleOfCanonRepForComplexRationals.

====Description====
//[[CL:Glossary:Coerces]]// the //object// to //[[CL:Glossary:type]]// //result-type//.

If //object// is already of //[[CL:Glossary:type]]// //result-type//, the //object// itself is returned, regardless of whether it would have been possible in general to coerce an //[[CL:Glossary:object]]// of some other //[[CL:Glossary:type]]// to //result-type//.

Otherwise, the //object// is //[[CL:Glossary:coerced]]// to //[[CL:Glossary:type]]// //result-type// according to the following rules:

\beginlist

\itemitem{**[[CL:Types:sequence]]**}

If the //result-type// is a //[[CL:Glossary:recognizable subtype]]// of **[[CL:Types:list]]**,and the //[[CL:Glossary:object]]// is a //[[CL:Glossary:sequence]]//, then the //result// is a //[[CL:Glossary:list]]// that has the //[[CL:Glossary:same]]// //[[CL:Glossary:element|elements]]// as //object//.

If the //result-type// is a //[[CL:Glossary:recognizable subtype]]// of **[[CL:Types:vector]]**,and the //[[CL:Glossary:object]]// is a //[[CL:Glossary:sequence]]//, then the //result// is a //[[CL:Glossary:vector]]// that has the //[[CL:Glossary:same]]// //[[CL:Glossary:element|elements]]// as //object//. If //result-type// is a specialized //[[CL:Glossary:type]]//, the //result// has an //[[CL:Glossary:actual array element type]]// that is the result of //[[CL:Glossary:upgrade|upgrading]]// the element type part of that //[[CL:Glossary:specialized]]// //[[CL:Glossary:type]]//. If no element type is specified, the element type defaults to **[[CL:Types:t]]**. If the //[[CL:Glossary:implementation]]// cannot determine the element type, an error is signaled.

\itemitem{**[[CL:Types:character]]**}

If the //result-type// is **[[CL:Types:character]]** and the //[[CL:Glossary:object]]// is a //[[CL:Glossary:character designator]]//, the //result// is the //[[CL:Glossary:character]]// it denotes.

\itemitem{**[[CL:Types:complex]]**}

If the //result-type// is **[[CL:Types:complex]]**

and the //[[CL:Glossary:object]]// is a //[[CL:Glossary:real]]//, then the //result// is obtained by constructing a //[[CL:Glossary:complex]]// whose real part is the //[[CL:Glossary:object]]// and whose imaginary part is the result of //[[CL:Glossary:coercing]]// an //[[CL:Glossary:integer]]// zero to the //[[CL:Glossary:type]]// of the //[[CL:Glossary:object]]// (using **[[CL:Functions:coerce]]**). (If the real part is a //[[CL:Glossary:rational]]//, however, then the result must be represented as a //[[CL:Glossary:rational]]// rather than a //[[CL:Glossary:complex]]//; see section {\secref\RuleOfCanonRepForComplexRationals}. So, for example, ''(coerce 3 'complex)'' is permissible, but will return ''3'', which is not a //[[CL:Glossary:complex]]//.)

\itemitem{**[[CL:Types:float]]**}

If the //result-type// is any of **[[CL:Types:float]]**, **[[CL:Types:short-float]]**, **[[CL:Types:single-float]]**, **[[CL:Types:double-float]]**, **[[CL:Types:long-float]]**,and the //[[CL:Glossary:object]]// is a

//[[CL:Glossary:real]]//,

then the //result// is a //[[CL:Glossary:float]]// of //[[CL:Glossary:type]]// //result-type// which is equal in sign and magnitude to the //[[CL:Glossary:object]]// to whatever degree of representational precision is permitted by that //[[CL:Glossary:float]]// representation. (If the //result-type// is **[[CL:Types:float]]** and //object// is not already a //[[CL:Glossary:float]]//, then the //result// is a //[[CL:Glossary:single float]]//.)

\itemitem{**[[CL:Types:function]]**}

If the //result-type// is **[[CL:Types:function]]**,and //object// is any

//[[CL:Glossary:function name]]//

that is //[[CL:Glossary:fbound]]// but that is globally defined neither as a //[[CL:Glossary:macro name]]// nor as a //[[CL:Glossary:special operator]]//, then the //result// is the //[[CL:Glossary:functional value]]// of //object//.

If the //result-type// is **[[CL:Types:function]]**,and //object// is a //[[CL:Glossary:lambda expression]]//, then the //result// is a //[[CL:Glossary:closure]]// of //object// in the //[[CL:Glossary:null lexical environment]]//.

\itemitem{**[[CL:Types:t]]**}

Any //object// can be //[[CL:Glossary:coerced]]// to an //[[CL:Glossary:object]]// of type **[[CL:Types:t]]**. In this case, the //object// is simply returned.

\endlist

====Examples====
<blockquote> (coerce '(a b c) 'vector) → #(A B C) (coerce 'a 'character) → #\\A (coerce 4.56 'complex) → #C(4.56 0.0) (coerce 4.5s0 'complex) → #C(4.5s0 0.0s0) (coerce 7/2 'complex) → 7/2 (coerce 0 'short-float) → 0.0s0 (coerce 3.5L0 'float) → 3.5L0 (coerce 7/2 'float) → 3.5 (coerce (cons 1 2) t) → (1 . 2) </blockquote>

All the following //[[CL:Glossary:forms]]// should signal an error:

<blockquote> (coerce '(a b c) '(vector * 4)) (coerce #(a b c) '(vector * 4)) (coerce '(a b c) '(vector * 2)) (coerce #(a b c) '(vector * 2)) (coerce "foo" '(string 2)) (coerce #(#\\a #\\b #\\c) '(string 2)) (coerce '(0 1) '(simple-bit-vector 3)) </blockquote>

====Affected By====
None.

====Exceptional Situations====
If a coercion is not possible, an error of type **[[CL:Types:type-error]]** is signaled.

''(coerce x 'nil)'' always signals an error of type **[[CL:Types:type-error]]**.

An error

of type **[[CL:Types:error]]** is signaled if the //result-type// is **[[CL:Types:function]]** but //object// is a //[[CL:Glossary:symbol]]// that is not //[[CL:Glossary:fbound]]// or if the //[[CL:Glossary:symbol]]// names a //[[CL:Glossary:macro]]// or a //[[CL:Glossary:special operator]]//.

An error of type **[[CL:Types:type-error]]** should be signaled if //result-type// specifies the number of elements and //object// is of a different length.

====See Also====
**[[CL:Functions:rational]]**, **[[CL:Functions:floor]]**, **[[CL:Functions:char-code]]**, **[[CL:Functions:char-int]]**

====Notes====
Coercions from //[[CL:Glossary:floats]]// to //[[CL:Glossary:rationals]]// and from //[[CL:Glossary:ratios]]// to //[[CL:Glossary:integers]]// are not provided because of rounding problems.

<blockquote> (coerce x 't) ≡ (identity x) ≡ x </blockquote>

\issue{FUNCTION-TYPE:X3J13-MARCH-88} \issue{COERCING-SETF-NAME-TO-FUNCTION:ALL-FUNCTION-NAMES} \issue{SEQUENCE-TYPE-LENGTH:MUST-MATCH} \issue{SEQUENCE-TYPE-LENGTH:MUST-MATCH} \issue{CONCATENATE-SEQUENCE:SIGNAL-ERROR} \issue{CHARACTER-LOOSE-ENDS:FIX} \issue{REAL-NUMBER-TYPE:X3J13-MAR-89}
