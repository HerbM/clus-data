====== Macro CHECK-TYPE ======

====Syntax====

**check-type {place typespec** //[//string//]}// → //**[[CL:Constant Variables:nil]]**//

====Arguments and Values====

//place// - a //[[CL:Glossary:place]]//.

//typespec// - a //[[CL:Glossary:type specifier]]//.

//string// - a //[[CL:Glossary:string]]//; \eval.

====Description====

**[[CL:Functions:check-type]]** signals a //[[CL:Glossary:correctable]]// //[[CL:Glossary:error]]// of type **[[CL:Types:type-error]]** if the contents of //place// are not of the type //typespec//.

**[[CL:Macros:check-type]]** can return only if \therestart{store-value} is invoked, either explicitly from a handler or implicitly as one of the options offered by the debugger. If \therestart{store-value} is invoked, **[[CL:Macros:check-type]]** stores the new value that is the argument to the //[[CL:Glossary:restart]]// invocation (or that is prompted for interactively by the debugger) in //place// and starts over, checking the type of the new value and signaling another error if it is still not of the desired //[[CL:Glossary:type]]//.

The first time //place// is //[[CL:Glossary:evaluated]]//, it is //[[CL:Glossary:evaluated]]// by normal evaluation rules. It is later //[[CL:Glossary:evaluated]]// as a //[[CL:Glossary:place]]// if the type check fails and \therestart{store-value} is used; see section {\secref\GenRefSubFormEval}.

//[[CL:Glossary:string]]// should be an English description of the type, starting with an indefinite article ("a" or "an"). If //[[CL:Glossary:string]]// is not supplied, it is computed automatically from //typespec//. The automatically generated message mentions //place//, its contents, and the desired type. An implementation may choose to generate a somewhat differently worded error message if it recognizes that //place// is of a particular form, such as one of the arguments to the function that called **[[CL:Macros:check-type]]**. //[[CL:Glossary:string]]// is allowed because some applications of **[[CL:Macros:check-type]]** may require a more specific description of what is wanted than can be generated automatically from //typespec//.

====Examples====

<blockquote> ([[CL:Macros:defparameter]] aardvarks '(sam harry fred)) → (SAM HARRY FRED) (check-type aardvarks (array * (3)))
▷ Error: The value of AARDVARKS, (SAM HARRY FRED),
▷ is not a 3-long array.
▷ To continue, type :CONTINUE followed by an option number:
▷ 1: Specify a value to use instead.
▷ 2: Return to Lisp Toplevel.
▷ Debug> \IN{:CONTINUE 1}
▷ Use Value: \IN{#(SAM FRED HARRY)} → NIL aardvarks → #<ARRAY-T-3 13571> (map 'list #'identity aardvarks) → (SAM FRED HARRY) ([[CL:Macros:defparameter]] aardvark-count 'foo) → FOO (check-type aardvark-count (integer 0 *) "A positive integer")
▷ Error: The value of AARDVARK-COUNT, FOO, is not a positive integer.
▷ To continue, type :CONTINUE followed by an option number:
▷ 1: Specify a value to use instead.
▷ 2: Top level.
▷ Debug> \IN{:CONTINUE 2} </blockquote>

<blockquote> (defmacro define-adder (name amount) (check-type name (and symbol (not null)) "a name for an adder function") (check-type amount integer) `(defun ,name (x) (+ x ,amount)))

(macroexpand '(define-adder add3 3)) → (defun add3 (x) (+ x 3))

(macroexpand '(define-adder 7 7))
▷ Error: The value of NAME, 7, is not a name for an adder function.
▷ To continue, type :CONTINUE followed by an option number:
▷ 1: Specify a value to use instead.
▷ 2: Top level.
▷ Debug> \IN{:Continue 1}
▷ Specify a value to use instead.
▷ Type a form to be evaluated and used instead: \IN{'ADD7} → (defun add7 (x) (+ x 7))

(macroexpand '(define-adder add5 something))
▷ Error: The value of AMOUNT, SOMETHING, is not an integer.
▷ To continue, type :CONTINUE followed by an option number:
▷ 1: Specify a value to use instead.
▷ 2: Top level.
▷ Debug> \IN{:Continue 1}
▷ Type a form to be evaluated and used instead: \IN{5} → (defun add5 (x) (+ x 5))

</blockquote>

Control is transferred to a handler.

====Side Effects====

The debugger might be entered.

====Affected By====

**[[CL:Variables:*break-on-signals*]]**

The implementation.

====Exceptional Situations====

None.

====See Also====

{\secref\ConditionSystemConcepts}

====Notes====

<blockquote> (check-type //place// //typespec//) ≡ (assert (typep //place// '//typespec//) (//place//) 'type-error :datum //place// :expected-type '//typespec//) </blockquote>

