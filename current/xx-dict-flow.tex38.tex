====== Function COMPLEMENT ======

====Syntax====
  * **complement** //function// → //complement-function//

====Arguments and Values====
  * //function// - a //[[CL:Glossary:function]]//.
  * //complement-function// - a //[[CL:Glossary:function]]//.

====Description====
Returns a //[[CL:Glossary:function]]// that takes the same //[[CL:Glossary:argument|arguments]]// as //function//, and has the same side-effect behavior as //function//, but returns only a single value: a //[[CL:Glossary:generalized boolean]]// with the opposite truth value of that which would be returned as the //[[CL:Glossary:primary value]]// of //function//. That is, when the //function// would have returned //[[CL:Glossary:true]]// as its //[[CL:Glossary:primary value]]// the //complement-function// returns //[[CL:Glossary:false]]//, and when the //function// would have returned //[[CL:Glossary:false]]// as its //[[CL:Glossary:primary value]]// the //complement-function// returns //[[CL:Glossary:true]]//.

====Examples====
<blockquote>
([[CL:Functions:funcall]] (complement #'[[CL:Functions:zerop]]) 1) → //[[CL:Glossary:true]]//
([[CL:Functions:funcall]] (complement #'[[CL:Functions:characterp]]) #\\A) → //[[CL:Glossary:false]]//
([[CL:Functions:funcall]] (complement #'[[CL:Functions:member]]) 'a '(a b c)) → //[[CL:Glossary:false]]//
([[CL:Functions:funcall]] (complement #'[[CL:Functions:member]]) 'd '(a b c)) → //[[CL:Glossary:true]]//
</blockquote>

====Side Effects====
None.

====Affected By====
None.

====Exceptional Situations====
None.

====See Also====
  * **[[CL:Functions:not|Function NOT]]**

====Notes====
<blockquote>
(complement //x//) ≡ #'([[CL:Symbols:lambda]] (&rest arguments) ([[CL:Functions:not]] ([[CL:Functions:apply]] //x// arguments)))
</blockquote>

In Common Lisp, functions with names like ''//xxx//-if-not'' are related to functions with names like ''//xxx//-if'' in that

<blockquote>
(//xxx//-if-not //f// . arguments) ≡ (//xxx//-if (complement //f//) . arguments)
</blockquote>

For example,

<blockquote>
([[CL:Functions:find-if-not]] #'[[CL:Functions:zerop]] '(0 0 3))
  ≡ ([[CL:Functions:find-if]] (complement #'[[CL:Functions:zerop]]) '(0 0 3)) <r>3</r>
</blockquote>

Note that since the ''//xxx//-if-not'' //[[CL:Glossary:function|functions]]// and the **'':test-not''** arguments have been deprecated, uses of ''//xxx//-if'' //[[CL:Glossary:function|functions]]// or **'':test''** arguments with **[[CL:Functions:complement]]** are preferred.

\issue{FUNCTION-COMPOSITION:JAN89-X3J13}
