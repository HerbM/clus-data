====== Function LOG ======

====Syntax====
  * **log** //number ''&optional'' base// → //logarithm//

====Arguments and Values====
  * //number// - a non-zero //[[CL:Glossary:number]]//.
  * //base// - a //[[CL:Glossary:number]]//.
  * //logarithm// - a //[[CL:Glossary:number]]//.

====Description====
**log** returns the logarithm of //number// in base //base//. If //base// is not supplied its value is ''e'', the base of the natural logarithms.

**log** may return a //[[CL:Glossary:complex]]// when given a //[[CL:Glossary:real]]// negative //number//.

<blockquote>
(log -1.0) ≡ (complex 0.0 (float pi 0.0))
</blockquote>

If //base// is zero, **log** returns zero.

The result of ''(log 8 2)'' may be either ''3'' or ''3.0'', depending on the implementation. An implementation can use floating-point calculations even if an exact integer result is possible.

The branch cut for the logarithm function of one argument (natural logarithm) lies along the negative real axis, continuous with quadrant II. The domain excludes the origin.

The mathematical definition of a complex logarithm is as follows, whether or not minus zero is supported by the implementation:

<blockquote>
(log ''x'') ≡ ([[CL:Functions:complex]] ([[CL:Functions:log]] ([[CL:Functions:abs]] ''x'')) ([[CL:Functions:phase]] ''x''))
</blockquote>

Therefore the range of the one-argument logarithm function is that strip of the complex plane containing numbers with imaginary parts between ''-π'' (exclusive) and ''π'' (inclusive) if minus zero is not supported, or ''-π'' (inclusive) and~''π'' (inclusive) if minus zero is supported.

The two-argument logarithm function is defined as:

<blockquote>
(log //base// //number//) ≡ ([[CL:Functions:math-divide|/]] (log //number//) (log //base//))
</blockquote>

This defines the //[[CL:Glossary:principal]]// //[[CL:Glossary:values]]// precisely. The range of the two-argument logarithm function is the entire complex plane.

====Examples====
<blockquote>
(log 100 10) <r>2.0 
//or// 2 </r>
(log 100.0 10) <r>2.0 </r>
(log #c(0 1) #c(0 -1)) <r>#C(-1.0 0.0) 
//or// #C(-1 0)</r>
(log 8.0 2) <r>3.0</r>
(log #c(-16 16) #c(2 2)) <r>3 
//or// #c(3.0 0.0) ; approximately
//or// 3.0 ; approximately, unlikely</r>
</blockquote>

====Affected By====
The implementation.

====Exceptional Situations====
None.

====See Also====
  * **[[CL:Functions:exp|Function EXP]]**
  * **[[CL:Functions:expt|Function EXPT]]**
  * {\secref\FloatSubstitutability}

====Notes====
None.

\issue{REAL-NUMBER-TYPE:X3J13-MAR-89} \issue{IEEE-ATAN-BRANCH-CUT:SPLIT} \issue{IEEE-ATAN-BRANCH-CUT:SPLIT} \issue{COMPLEX-RATIONAL-RESULT:EXTEND}
