====== Type UNSIGNED-BYTE ======

====Supertypes====

**[[CL:Types:unsigned-byte]]**, **[[CL:Types:signed-byte]]**, **[[CL:Types:integer]]**, **[[CL:Types:rational]]**, **[[CL:Types:real]]**, **[[CL:Types:number]]**, **[[CL:Types:t]]**

====Description====

The atomic //[[CL:Glossary:type specifier]]// **[[CL:Types:unsigned-byte]]** denotes the same type as is denoted by the //[[CL:Glossary:type specifier]]// ''(integer 0 *)''.

====Compound Type Specifier Kind====

Abbreviating.

====Compound Type Specifier Syntax====

\Deftype{unsigned-byte}{\ttbrac{//s// | **[[CL:Types:wildcard|*]]**}}

====Compound Type Specifier Arguments====

//s// - a positive //[[CL:Glossary:integer]]//.

====Compound Type Specifier Description====

This denotes the set of non-negative //[[CL:Glossary:integers]]// that can be represented in a byte of size //s// (bits). This is equivalent to ''(mod //m//)'' for ''//m//=2^s'', or to ''(integer 0 //n//)'' for ''//n//=2^s-1''. The type **[[CL:Types:unsigned-byte]]** or the type ''(unsigned-byte *)'' is the same as the type ''(integer 0 *)'', the set of non-negative //[[CL:Glossary:integers]]//.

====Notes====

The //[[CL:Glossary:type]]// ''(unsigned-byte 1)'' is also called **[[CL:Types:bit]]**.

\issue{REAL-NUMBER-TYPE:X3J13-MAR-89}
