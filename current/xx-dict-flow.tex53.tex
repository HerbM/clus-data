====== Accessor VALUES ======

====Syntax====

\DefunWithValues values {''&rest'' object} {\starparam{object}}

(**setf** (**values** //''&rest'' place//) //new-values//)

====Arguments and Values====

//object// - an //[[CL:Glossary:object]]//.

//place// - a //[[CL:Glossary:place]]//.

//new-value// - an //[[CL:Glossary:object]]//.

====Description====

**[[CL:Functions:values]]** returns the //objects// as //[[CL:Glossary:multiple values]]//.

**[[CL:Macros:setf]]** of **[[CL:Functions:values]]** is used to store the //[[CL:Glossary:multiple values]]// //new-values// into the //places//. see section {\secref\SETFofVALUES}.

====Examples====

<blockquote> (values) → \novalues (values 1) → 1 (values 1 2) → 1, 2 (values 1 2 3) → 1, 2, 3 (values (values 1 2 3) 4 5) → 1, 4, 5 (defun polar (x y) (values (sqrt (+ (* x x) (* y y))) (atan y x))) → POLAR (multiple-value-bind (r theta) (polar 3.0 4.0) (vector r theta)) → #(5.0 0.927295) </blockquote>

Sometimes it is desirable to indicate explicitly that a function returns exactly one value. For example, the function

<blockquote> (defun foo (x y) (floor (+ x y) y)) → FOO </blockquote> returns two values because **[[CL:Functions:floor]]** returns two values. It may be that the second value makes no sense, or that for efficiency reasons it is desired not to compute the second value. **[[CL:Functions:values]]** is the standard idiom for indicating that only one value is to be returned:

<blockquote> (defun foo (x y) (values (floor (+ x y) y))) → FOO </blockquote> This works because **[[CL:Functions:values]]** returns exactly one value for each of //args//; as for any function call, if any of //args// produces more than one value, all but the first are discarded.

====Affected By====

None.

====Exceptional Situations====

None.

====See Also====

**[[CL:Functions:values-list]]**, **[[CL:Macros:multiple-value-bind]]**, **[[CL:Constant Variables:multiple-values-limit]]**, {\secref\Evaluation}

====Notes====

Since **[[CL:Functions:values]]** is a //[[CL:Glossary:function]]//, not a //[[CL:Glossary:macro]]// or //[[CL:Glossary:special form]]//, it receives as //[[CL:Glossary:arguments]]// only the //[[CL:Glossary:primary values]]// of its //[[CL:Glossary:argument]]// //[[CL:Glossary:forms]]//.

\issue{SETF-OF-VALUES:ADD} \issue{SETF-OF-VALUES:ADD} \issue{SETF-OF-VALUES:ADD}
