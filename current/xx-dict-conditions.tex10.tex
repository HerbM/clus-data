====== Macro ASSERT ======

====Syntax====

\DefmacWithValuesNewline assert {test-form \brac{\paren{\starparam{place}} \brac{datum-form \starparam{argument-form}}}} {**[[CL:Constant Variables:nil]]**}

====Arguments and Values====

//test-form// - a //[[CL:Glossary:form]]//; always evaluated.

//place// - a //[[CL:Glossary:place]]//; evaluated if an error is signaled.

//datum-form// - a //[[CL:Glossary:form]]// that evaluates to a //datum//. Evaluated each time an error is to be signaled, or not at all if no error is to be signaled.

//argument-form// - a //[[CL:Glossary:form]]// that evaluates to an //argument//. Evaluated each time an error is to be signaled, or not at all if no error is to be signaled.

//datum//, //arguments// - //[[CL:Glossary:designators]]// for a //[[CL:Glossary:condition]]// of default type **[[CL:Types:error]]**. (These //[[CL:Glossary:designators]]// are the result of evaluating //datum-form// and each of the //argument-forms//.)

====Description====

**[[CL:Functions:assert]]** assures that //test-form// evaluates to //[[CL:Glossary:true]]//. If //test-form// evaluates to //[[CL:Glossary:false]]//, **[[CL:Macros:assert]]** signals a //[[CL:Glossary:correctable]]// //[[CL:Glossary:error]]// (denoted by //datum// and //arguments//). Continuing from this error using \therestart{continue} makes it possible for the user to alter the values of the //places// before **[[CL:Macros:assert]]** evaluates //test-form// again. If the value of //test-form// is //[[CL:Glossary:non-nil]]//, **[[CL:Functions:assert]]** returns **[[CL:Constant Variables:nil]]**.

The //places// are //[[CL:Glossary:generalized references]]// to data upon which //test-form// depends, whose values can be changed by the user in attempting to correct the error. //[[CL:Glossary:Subforms]]// of each //place// are only evaluated if an error is signaled, and might be re-evaluated if the error is re-signaled (after continuing without actually fixing the problem).

The order of evaluation of the //places// is not specified; see section {\secref\GenRefSubFormEval}.\idxtext{order of evaluation}\idxtext{evaluation order}

If a //place// //[[CL:Glossary:form]]// is supplied that produces more values than there are store variables, the extra values are ignored. If the supplied //[[CL:Glossary:form]]// produces fewer values than there are store variables, the missing values are set to **[[CL:Constant Variables:nil]]**.


====Examples==== <blockquote> ([[CL:Macros:defparameter]] x (make-array '(3 5) :initial-element 3)) → #2A((3 3 3 3 3) (3 3 3 3 3) (3 3 3 3 3)) ([[CL:Macros:defparameter]] y (make-array '(3 5) :initial-element 7)) → #2A((7 7 7 7 7) (7 7 7 7 7) (7 7 7 7 7)) (defun matrix-multiply (a b) (let ((*print-array* nil)) (assert (and (= (array-rank a) (array-rank b) 2) (= (array-dimension a 1) (array-dimension b 0))) (a b) "Cannot multiply ~S by ~S." a b) (really-matrix-multiply a b))) → MATRIX-MULTIPLY (matrix-multiply x y)
▷ Correctable error in MATRIX-MULTIPLY:
▷ Cannot multiply #<ARRAY ...> by #<ARRAY ...>.
▷ Restart options:
▷ 1: You will be prompted for one or more new values.
▷ 2: Top level.
▷ Debug> \IN{:continue 1}
▷ Value for A: \IN{x}
▷ Value for B: \IN{(make-array '(5 3) :initial-element 6)} → #2A((54 54 54 54 54) (54 54 54 54 54) (54 54 54 54 54) (54 54 54 54 54) (54 54 54 54 54)) </blockquote>

<blockquote> (defun double-safely (x) (assert (numberp x) (x)) (+ x x)) (double-safely 4) → 8

(double-safely t)
▷ Correctable error in DOUBLE-SAFELY: The value of (NUMBERP X) must be non-NIL.
▷ Restart options:
▷ 1: You will be prompted for one or more new values.
▷ 2: Top level.
▷ Debug> \IN{:continue 1}
▷ Value for X: \IN{7} → 14 </blockquote>

====Affected By====

**[[CL:Variables:*break-on-signals*]]**

The set of active //[[CL:Glossary:condition handlers]]//.

====Exceptional Situations====

None.

====See Also====

**[[CL:Functions:check-type]]**, **[[CL:Functions:error]]**, {\secref\GeneralizedReference}

====Notes====

The debugger need not include the //test-form// in the error message, and the //places// should not be included in the message, but they should be made available for the user's perusal. If the user gives the "continue" command, the values of any of the references can be altered. The details of this depend on the implementation's style of user interface.

\issue{FORMAT-STRING-ARGUMENTS:SPECIFY} \issue{ASSERT-ERROR-TYPE:ERROR} \issue{PUSH-EVALUATION-ORDER:FIRST-ITEM} \issue{SETF-MULTIPLE-STORE-VARIABLES:ALLOW}
