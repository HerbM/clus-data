====== Special Operator PROGN ======

====Syntax====

\DefspecWithValues progn {\starparam{form}} {\starparam{result}}

====Arguments and Values====

//forms// - an //[[CL:Glossary:implicit progn]]//.

//results// - the //[[CL:Glossary:value|values]]// of the //[[CL:Glossary:forms]]//.

====Description====

\specref{progn} evaluates //forms//, in the order in which they are given.

The values of each //form// but the last are discarded.

If \specref{progn} appears as a //[[CL:Glossary:top level form]]//, then all //[[CL:Glossary:forms]]// within that \specref{progn} are considered by the compiler to be //[[CL:Glossary:top level forms]]//.

====Examples==== <blockquote> (progn) → NIL (progn 1 2 3) → 3 (progn (values 1 2 3)) → 1, 2, 3 ([[CL:Macros:defparameter]] a 1) → 1 (if a (progn ([[CL:Macros:defparameter]] a nil) 'here) (progn ([[CL:Macros:defparameter]] a t) 'there)) → HERE a → NIL </blockquote>

====Affected By====

None.

====Exceptional Situations====

None.

====See Also====

**[[CL:Macros:prog1]]**, **[[CL:Macros:prog2]]**, {\secref\Evaluation}

====Notes====

Many places in \clisp\ involve syntax that uses //[[CL:Glossary:implicit progns]]//. That is, part of their syntax allows many //[[CL:Glossary:forms]]// to be written that are to be evaluated sequentially, discarding the results of all //[[CL:Glossary:forms]]// but the last and returning the results of the last //[[CL:Glossary:form]]//. Such places include, but are not limited to, the following: the body of a //[[CL:Glossary:lambda expression]]//; the bodies of various control and conditional //[[CL:Glossary:forms]]// (//e.g.// **[[CL:Macros:case]]**, \specref{catch}, \specref{progn}, and **[[CL:Macros:when]]**).

