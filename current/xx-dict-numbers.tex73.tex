====== Function DECODE-FLOAT, SCALE-FLOAT, FLOAT-RADIX, FLOAT-SIGN, FLOAT-DIGITS, FLOAT-PRECISION, INTEGER-DECODE-FLOAT ======

====Syntax====
  * **decode-float** //float// → //significand, exponent, sign//
  * **scale-float** //float integer// → //scaled-float//
  * **float-radix** //float// → //float-radix//
  * **float-sign** //float-1** ''&optional'' float-2// → //signed-float//
  * **float-digits** //float// → //digits1//
  * **float-precision** //float// → //digits2//
  * **integer-decode-float** //float// → //significand, exponent, integer-sign//

====Arguments and Values====
  * //digits1// - a non-negative //[[CL:Glossary:integer]]//.
  * //digits2// - a non-negative //[[CL:Glossary:integer]]//.
  * //exponent// - an //[[CL:Glossary:integer]]//.
  * //float// - a //[[CL:Glossary:float]]//.
  * //float-1// - a //[[CL:Glossary:float]]//.
  * //float-2// - a //[[CL:Glossary:float]]//.
  * //float-radix// - an //[[CL:Glossary:integer]]//.
  * //integer// - a non-negative //[[CL:Glossary:integer]]//.
  * //integer-sign// - the //[[CL:Glossary:integer]]// ''-1'', or the //[[CL:Glossary:integer]]// ''1''.
  * //scaled-float// - a //[[CL:Glossary:float]]//.
  * //sign// - A //[[CL:Glossary:float]]// of the same //[[CL:Glossary:type]]// as //float// but numerically equal to ''1.0'' or ''-1.0''.
  * //signed-float// - a //[[CL:Glossary:float]]//.
  * //significand// - a //[[CL:Glossary:float]]//.

====Description====

**decode-float** computes three values that characterize //float//. The first value is of the same //[[CL:Glossary:type]]// as //float// and represents the significand. The second value represents the exponent to which the radix (notated in this description by ''b'') must be raised to obtain the value that, when multiplied with the first result, produces the absolute value of //float//. If //float// is zero, any //[[CL:Glossary:integer]]// value may be returned, provided that the identity shown for **scale-float** holds. The third value is of the same //[[CL:Glossary:type]]// as //float// and is 1.0 if //float// is greater than or equal to zero or -1.0 otherwise.

**decode-float** divides //float// by an integral power of ''b'' so as to bring its value between ''1/b'' (inclusive) and ''1'' (exclusive), and returns the quotient as the first value. If //float// is zero, however, the result equals the absolute value of //float// (that is, if there is a negative zero, its significand is considered to be a positive zero).

**scale-float** returns ''(* //float// (expt (float ''b'' //float//) //integer//))'', where ''b'' is the radix of the floating-point representation. //float// is not necessarily between ''1/b'' and ''1''.

**float-radix** returns the radix of //float//.

**float-sign** returns a number ''z'' such that ''z'' and //float-1// have the same sign and also such that ''z'' and //float-2// have the same absolute value. If //float-2// is not supplied, its value is ''(float 1 //float-1//)''.

If an implementation has distinct representations for negative zero and positive zero, then ''(float-sign -0.0)'' → ''-1.0''.

**float-digits** returns the number of radix ''b'' digits used in the representation of //float// (including any implicit digits, such as a "hidden bit").

**float-precision** returns the number of significant radix ''b'' digits present in //float//; if //float// is a //[[CL:Glossary:float]]// zero, then the result is an //[[CL:Glossary:integer]]// zero.

For //[[CL:Glossary:normalized]]// //[[CL:Glossary:floats]]//, the results of **float-digits** and **float-precision** are the same, but the precision is less than the number of representation digits for a //[[CL:Glossary:denormalized]]// or zero number.

**integer-decode-float** computes three values that characterize //float// - the significand scaled so as to be an //[[CL:Glossary:integer]]//, and the same last two values that are returned by **decode-float**. If //float// is zero, **[[integer-decode-float** returns zero as the first value. The second value bears the same relationship to the first value as for **decode-float**:

<blockquote>
([[CL:Macros:multiple-value-bind]] (signif expon sign) 
    (integer-decode-float f) 
  (scale-float ([[CL:Functions:float]] signif f) expon)) ≡ ([[CL:Functions:abs]] f)
</blockquote>

====Examples====
Note that since the purpose of this functionality is to expose details of the //[[CL:Glossary:implementation]]//, all of these examples are necessarily very //[[CL:Glossary:implementation-dependent]]//. Results may vary widely. Values shown here are chosen consistently from one particular //[[CL:Glossary:implementation]]//.

<blockquote>
(decode-float .5) <r>0.5
0
1.0 </r>
(decode-float 1.0) <r>0.5
1
1.0 </r>
(scale-float 1.0 1) <r>2.0 </r>
(scale-float 10.01 -2) <r>2.5025 </r>
(scale-float 23.0 0) <r>23.0 </r>
(float-radix 1.0) <r>2 </r>
(float-sign 5.0) <r>1.0 </r>
(float-sign -5.0) <r>-1.0 </r>
(float-sign 0.0) <r>1.0 </r>
(float-sign 1.0 0.0) <r>0.0 </r>
(float-sign 1.0 -10.0) <r>10.0 </r>
(float-sign -1.0 10.0) <r>-10.0 </r>
(float-digits 1.0) <r>24 </r>
(float-precision 1.0) <r>24 </r>
(float-precision least-positive-single-float) <r>1 </r>
(integer-decode-float 1.0) <r>8388608
-23
1 </r>
</blockquote>

====Side Effects====
None.

====Affected By====
The implementation's representation for //[[CL:Glossary:floats]]//.

====Exceptional Situations====
The functions **decode-float**, **float-radix**, **float-digits**, **float-precision**, and **integer-decode-float** should signal an error if their only argument is not a //[[CL:Glossary:float]]//.

The //[[CL:Glossary:function]]// **scale-float** should signal an error if its first argument is not a //[[CL:Glossary:float]]// or if its second argument is not an //[[CL:Glossary:integer]]//.

The //[[CL:Glossary:function]]// **float-sign** should signal an error if its first argument is not a //[[CL:Glossary:float]]// or if its second argument is supplied but is not a //[[CL:Glossary:float]]//.

====See Also====
None.

====Notes====
The product of the first result of **decode-float** or **integer-decode-float**, of the radix raised to the power of the second result, and of the third result is exactly equal to the value of //float//.

<blockquote>
([[CL:Macros:multiple-value-bind]] (signif expon sign) 
    (decode-float f) 
  (scale-float signif expon)) 
  ≡ ([[CL:Functions:abs]] f)

([[CL:Macros:multiple-value-bind]] (signif expon sign) 
    (decode-float f) 
  ([[CL:Functions:math-multiply|*]] (scale-float signif expon) sign)) 
  ≡ f
</blockquote>

