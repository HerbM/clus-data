====== Function ABS ======

====Syntax====

**abs** //number// → //absolute-value//

====Arguments and Values====

//number// - a //[[CL:Glossary:number]]//.

//absolute-value// - a non-negative //[[CL:Glossary:real]]//.

====Description====

**[[CL:Functions:abs]]** returns the absolute value of //number//.

If //number// is

a //[[CL:Glossary:real]]//,

the result is of the same //[[CL:Glossary:type]]// as //number//.

If //number// is a //[[CL:Glossary:complex]]//, the result is a positive

//[[CL:Glossary:real]]//

with the same magnitude as //number//. The result can be a //[[CL:Glossary:float]]// \reviewer{Barmar: Single-float.} even if //number//'s components are //[[CL:Glossary:rationals]]// and an exact rational result would have been possible.

Thus the result of ''(abs #c(3 4))'' can be either ''5'' or ''5.0'', depending on the implementation.

====Examples====

<blockquote> (abs 0) → 0 (abs 12/13) → 12/13 (abs -1.09) → 1.09 (abs #c(5.0 -5.0)) → 7.071068 (abs #c(5 5)) → 7.071068 (abs #c(3/5 4/5)) → 1 or approximately 1.0 (eql (abs -0.0) -0.0) → //[[CL:Glossary:true]]// </blockquote>

====Affected By====

None.

====Exceptional Situations====

None.

====See Also====

{\secref\FloatSubstitutability}

====Notes====

If //number// is a //[[CL:Glossary:complex]]//, the result is equivalent to the following:

''(sqrt (+ (expt (realpart //number//) 2) (expt (imagpart //number//) 2)))''

An implementation should not use this formula directly for all //[[CL:Glossary:complexes]]// but should handle very large or very small components specially to avoid intermediate overflow or underflow.

\issue{REAL-NUMBER-TYPE:X3J13-MAR-89} \issue{REAL-NUMBER-TYPE:X3J13-MAR-89} \issue{COMPLEX-RATIONAL-RESULT:EXTEND}
