====== Special Operator THROW ======

====Syntax====
  * **throw** //tag result-form//

====Arguments and Values====
  * //tag// - a //[[CL:Glossary:catch tag]]//; evaluated.
  * //result-form// - a //[[CL:Glossary:form]]//; evaluatedspecial.

====Description====
**throw** causes a non-local control transfer to a **[[CL:Special Operators:catch]]** whose tag is **[[CL:Functions:eq]]** to //tag//.

//tag// is evaluated first to produce an //[[CL:Glossary:object]]// called the throw tag; then //result-form// is evaluated, and its results are saved. If the //result-form// produces multiple values, then all the values are saved. The most recent outstanding **[[CL:Special Operators:catch]]** whose //tag// is **[[CL:Functions:eq]]** to the throw tag is exited; the saved results are returned as the value or values of **[[CL:Special Operators:catch]]**.

The transfer of control initiated by **throw** is performed as described in \secref\TransferOfControl.

====Examples====
<blockquote> 
([[CL:Special Operators:catch]] 'result 
  ([[CL:Special Operators:let]] ((i 0) (j 0)) 
    ([[CL:Macros:loop]] 
      ([[CL:Macros:incf]] j 3) 
      ([[CL:Macros:incf]] i) 
      ([[CL:Macros:when]] ([[CL:Functions:math-equal|=]] i 3) 
        (throw 'result ([[CL:Functions:values]] i j)))))) <r>3
        9</r>
        
([[CL:Special Operators:catch]] [[CL:Constant Variables:nil]]
  ([[CL:Special Operators:unwind-protect]]
      (throw [[CL:Constant Variables:nil]] 1) 
    (throw [[CL:Constant Variables:nil]] 2))) → 2 
</blockquote>

<blockquote>
([[CL:Special Operators:catch]] 'foo
  ([[CL:Functions:format]] [[CL:Constant Variables:t]] "The inner catch returns ~s.~%"
    ([[CL:Special Operators:catch]] 'foo 
      ([[CL:Special Operators:unwind-protect]] 
          (throw 'foo :first-throw) 
        (throw 'foo :second-throw))))
  :outer-catch)
<o>The inner catch returns :SECOND-THROW </o>
<r>:OUTER-CATCH</r>
</blockquote>

The consequences of the following are undefined because the **[[CL:Special Operators:catch]]** of ''b'' is passed over by the first **throw**, hence portable programs must assume that its //[[CL:Glossary:dynamic extent]]// is terminated. The //[[CL:Glossary:binding]]// of the //[[CL:Glossary:catch tag]]// is not yet //[[CL:Glossary:disestablished]]// and therefore it is the target of the second **throw**.

<blockquote>
([[CL:Special Operators:catch]] 'a 
  ([[CL:Special Operators:catch]] 'b 
    ([[CL:Special Operators:unwind-protect]] 
        (throw 'a 1) 
      (throw 'b 2))))
</blockquote>

====Affected By====
None.

====Exceptional Situations====
If there is no outstanding //[[CL:Glossary:catch tag]]// that matches the throw tag, no unwinding of the stack is performed, and an error of type **[[CL:Types:control-error]]** is signaled. When the error is signaled, the //[[CL:Glossary:dynamic environment]]// is that which was in force at the point of the **throw**.

====See Also====
  * **[[CL:Special Operators:block|Special Operator BLOCK]]**
  * **[[CL:Special Operators:catch|Special Operator CATCH]]**
  * **[[CL:Special Operators:return-from|Special Operator RETURN-FROM]]**
  * **[[CL:Special Operators:unwind-protect|Special Operator UNWIND-PROTECT]]**
  * {\secref\Evaluation}

====Notes====
**[[CL:Special Operators:catch]]** and **throw** are normally used when the //[[CL:Glossary:exit point]]// must have //[[CL:Glossary:dynamic scope]]// (//e.g.// the **throw** is not lexically enclosed by the **[[CL:Special Operators:catch]]**), while **[[CL:Special Operators:block]]** and **[[CL:Special Operators:return]]** are used when //[[CL:Glossary:lexical scope]]// is sufficient.

\issue{EXIT-EXTENT:MINIMAL}
