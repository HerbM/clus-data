====== Function PARSE-INTEGER ======

====Syntax====
  * **parse-integer** //string ''&key'' start end radix junk-allowed// → //integer, pos//

====Arguments and Values====
  * //string// - a //[[CL:Glossary:string]]//.
  * //start//, //end// - //[[CL:Glossary:bounding index designators]]// of //string//. The defaults for //start// and //end// are ''0'' and **[[CL:Constant Variables:nil]]**, respectively.
  * //radix// - a //[[CL:Glossary:radix]]//. The default is ''10''.
  * //junk-allowed// - a //[[CL:Glossary:generalized boolean]]//. The default is //[[CL:Glossary:false]]//.
  * //integer// - an //[[CL:Glossary:integer]]// or //[[CL:Glossary:false]]//.
  * //pos// - a //[[CL:Glossary:bounding index]]// of //string//.

====Description====
**parse-integer** parses an //[[CL:Glossary:integer]]// in the specified //radix// from the substring of //string// delimited by //start// and //end//.

**parse-integer** expects an optional sign (''+'' or ''-'') followed by a a non-empty sequence of digits to be interpreted in the specified //radix//. Optional leading and trailing //[[CL:Glossary:whitespace]]// is ignored.

**parse-integer** does not recognize the syntactic radix-specifier prefixes ''#O'', ''#B'', ''#X'', and ''#''n''R'', nor does it recognize a trailing decimal point.

If //junk-allowed// is //[[CL:Glossary:false]]//, an error of type **[[CL:Types:parse-error]]** is signaled if substring does not consist entirely of the representation of a signed //[[CL:Glossary:integer]]//, possibly surrounded on either side by //[[CL:Glossary:whitespace]]// //[[CL:Glossary:characters]]//.

The first //[[CL:Glossary:value]]// returned is either the //[[CL:Glossary:integer]]// that was parsed, or else **[[CL:Constant Variables:nil]]** if no syntactically correct //[[CL:Glossary:integer]]// was seen but //junk-allowed// was //[[CL:Glossary:true]]//.

The second //[[CL:Glossary:value]]// is either the index into the //[[CL:Glossary:string]]// of the delimiter that terminated the parse, or the upper //[[CL:Glossary:bounding index]]// of the substring if the parse terminated at the end of the substring (as is always the case if //junk-allowed// is //[[CL:Glossary:false]]//).

====Examples==== 
<blockquote> 
(parse-integer "123") <r>123, 3 </r>
(parse-integer "123" :start 1 :radix 5) <r>13, 3 </r>
(parse-integer "no-integer" :junk-allowed [[CL:Constant Variables:t]]) <r>NIL, 0 </r>
</blockquote>

====Side Effects====
None.

====Affected By====
None.

====Exceptional Situations====
If //junk-allowed// is //[[CL:Glossary:false]]//, an error is signaled if substring does not consist entirely of the representation of an //[[CL:Glossary:integer]]//, possibly surrounded on either side by //[[CL:Glossary:whitespace]]// characters.

====See Also====
None.

====Notes====
None.

\issue{SUBSEQ-OUT-OF-BOUNDS} \issue{RANGE-OF-START-AND-END-PARAMETERS:INTEGER-AND-INTEGER-NIL}
