====== Function SQRT, ISQRT ======

====Syntax====
  * **sqrt** //number// → //root//
  * **isqrt** //natural// → //natural-root//

====Arguments and Values====
  * //number//, //root// - a //[[CL:Glossary:number]]//.
  * //natural//, //natural-root// - a non-negative //[[CL:Glossary:integer]]//.

====Description====

**sqrt** and **isqrt** compute square roots.

**sqrt** returns the //[[CL:Glossary:principal]]// square root of //number//. If the //number// is not a //[[CL:Glossary:complex]]// but is negative, then the result is a //[[CL:Glossary:complex]]//.

**isqrt** returns the greatest //[[CL:Glossary:integer]]// less than or equal to the exact positive square root of //natural//.

If //number// is a positive //[[CL:Glossary:rational]]//, it is //[[CL:Glossary:implementation-dependent]]// whether //root// is a //[[CL:Glossary:rational]]// or a //[[CL:Glossary:float]]//. If //number// is a negative //[[CL:Glossary:rational]]//, it is //[[CL:Glossary:implementation-dependent]]// whether //root// is a //[[CL:Glossary:complex rational]]// or a //[[CL:Glossary:complex float]]//.

The mathematical definition of complex square root (whether or not minus zero is supported) follows:

<blockquote>
(sqrt //x//) = ([[CL:Functions:exp]] ([[CL:Functions:math-divide|/]] ([[CL:Functions:log]] //x//) 2))
</blockquote>

The branch cut for square root lies along the negative real axis, continuous with quadrant II. The range consists of the right half-plane, including the non-negative imaginary axis and excluding the negative imaginary axis.

====Examples====

<blockquote>
(sqrt 9.0) <r>3.0 </r>
(sqrt -9.0) <r>#C(0.0 3.0) </r>
(isqrt 9) <r>3 </r>
(sqrt 12) <r>3.4641016 </r>
(isqrt 12) <r>3 </r>
(isqrt 300) <r>17 </r>
(isqrt 325) <r>18 </r>
(sqrt 25) <r>5 
//or// 5.0 </r>
(isqrt 25) <r>5 </r>
(sqrt -1) <r>#C(0.0 1.0) </r>
(sqrt #c(0 2)) <r>#C(1.0 1.0)</r>
</blockquote>

====Side Effects====
None.

====Affected By====
None.

====Exceptional Situations====
The //[[CL:Glossary:function]]// **sqrt** should signal **[[CL:Types:type-error]]** if its argument is not a //[[CL:Glossary:number]]//.

The //[[CL:Glossary:function]]// **isqrt** should signal **[[CL:Types:type-error]]** if its argument is not a non-negative //[[CL:Glossary:integer]]//.

The functions **sqrt** and **isqrt** might signal **[[CL:Types:arithmetic-error]]**.

====See Also====
  * **[[CL:Functions:exp|Function EXP]]**
  * **[[CL:Functions:log|Function LOG]]**
  * {\secref\FloatSubstitutability}

====Notes====
**isqrt** may be defined as follows, although in real //[[CL:Glossary:implementation|implementations]]// this function might be potentially more efficient.

<blockquote>
([[CL:Macros:defun]] isqrt (x) 
  ([[CL:Functions:values]] ([[CL:Functions:floor]] (sqrt x))))
</blockquote> 

\issue{IEEE-ATAN-BRANCH-CUT:SPLIT}
