\section{Common Lisp UltraSpec - A Project For Modern Common Lisp
Documentation}\label{common-lisp-ultraspec---a-project-for-modern-common-lisp-documentation}

by Michal ``phoe'' Herda 2017

\subsection{TODO Abstract}\label{todo-abstract}

\subsection{TODO Introduction}\label{todo-introduction}

\subsection{Previous work}\label{previous-work}

\subsubsection{ANSI CL Standard}\label{ansi-cl-standard}

\begin{enumerate}
\item
  Published standard

  The ANSI Common Lisp standard is the specification of the Common Lisp
  language, published in 1994.

  The specification itself is a written document of over one thousand
  pages of formatted text. Such a large amount of technical data was a
  natural candidate to be turned into a digital database browsable by
  humans.
\item
  Derived work

  The most famous work derived from the ANSI CL standard is the Common
  Lisp HyperSpec (henceforth abbreviated as CLHS)
  (\url{http://www.lispworks.com/documentation/HyperSpec/Front/index.htm}).
  It is a hyperlinked Web version of the original standard, which allows
  for easy navigation of the standard. Its HTML form also allows for
  search and quick access using external search engines, such as Google
  or IRC (Internet Relay Chat) bots.

  The CLHS is released under an essentially non-free license which
  allows verbatim copying of CLHS as a whole, but prohibits any changes
  to it and creating derivative works. It is therefore not possible to
  create a unified piece of Common Lisp documentation based on the CLHS.
\end{enumerate}

\subsubsection{Lisp content aggregators}\label{lisp-content-aggregators}

The Common Lisp documentation spans far and wide beyond the Common Lisp
standard. Even during the time of Common Lisp standarization, many
extensions to the language existed, with their respective pieces of
documentation.

From this arose the obvious need of aggregating Lisp content, both with
respect to code and documentation. Below, I will outline three
contemporary services which provide Lisp users with content.

\begin{enumerate}
\item
  l1sp.org

  The l1sp.org (\url{http://l1sp.org}) service is a content aggregation
  tool created by Zach Beane. Its main purpose is to enable lookup of
  symbols inside various pieces of documentation scattered around the
  web. It currently contains links to above 20 pieces of documentation,
  including the CLHS, documentation for various CL implementations and
  many commonly used language extensions and libraries, such as the
  Metaobject Protocol, ASDF and Alexandria.

  This redirection service is very useful and allows for easy lookup,
  but such an approach depends on the presence of all the pieces of
  documentation in their respective places all around the Web. Also, the
  pieces of documentation are not linked to each other; for example, it
  is impossible to reach the Common Lisp reference from within the
  Metaobject Protocol reference (\url{http://metamodular.com/CLOS-MOP/})
  and vice versa. The individual pieces of reference also greatly vary
  in style, which means both the typesetting and graphical layout and
  the textual form in which the information is presented to the reader.
\item
  Quicklisp

  Quicklisp (\url{https://www.quicklisp.org/}), created by Zach Beane,
  provides a centralized repository of Lisp libraries through a piece of
  Lisp code, which in turn allows the programmer to automatically
  resolve dependencies, download and compile a particular library on
  their Lisp system.

  While Quicklisp is invaluable as a library repository, it does not
  provide any sort of documentation service as such is out of scope of
  the Quicklisp project. Therefore it cannot be a direct aid in creating
  a Lisp documentation project.
\item
  Quickdocs

  The Quickdocs service (\url{http://quickdocs.org/}) is a content
  aggregation tool created by Eitaro Fukamachi expressly for automated
  collection and generation of documentation for Common Lisp libraries;
  therefore, it aids the issue outlined in the paragraph above by
  expanding Quicklisp with documentation capabilities. The documentation
  itself is generated automatically from the source code of libraries
  found in the Quicklisp repositories. It consists of a system's
  Quicklisp description and a list of exported symbols along with the
  type of objects they refer to and any documentation strings they may
  contain.

  Such an automation provides a very good and aesthetically pleasing
  means of reading about a given system's protocol. The issue with such
  automatic generation is, it forces the library authors to follow a
  convention of documenting their libraries in a particular way, which
  must be recognizable by the tool parsing the Quicklisp systems -
  otherwise, the documentation will not be visible in Quickdocs.
  Additionally, Quicklisp descriptions often contain little more than
  links to external websites documenting the code, which deprives
  Quickdocs of the ability to automatically generate documentation for
  it.
\end{enumerate}

\subsection{My work}\label{my-work}

\subsubsection{Idea}\label{idea}

The idea of creating a unified, hyperlinked piece of Common Lisp
documentation which additionally spanned over multiple language
extensions and libraries had been growing in me since I began my journey
with Common Lisp back in 2015. I had been irritated by the separation of
particular bodies of Lisp knowledge and lack of connection between them.
In the beginning of 2016, I started looking for means to improve this
situation.

During my research, it became obvious to me that - no matter which
particular way would be chosen in this case - the project of creating
and maintaining a modern, unified repository of Common Lisp
documentation would require substantial work. It would be necessary to
choose the appropriate pieces of work the repository would consist of,
find most recent versions of their documentation, solve any legal issues
of creating derivative works of them, parse the existing documents and
keep the repository maintained in the face of the changing versions of
Common Lisp libraries.

\subsubsection{Requirements}\label{requirements}

The idea for building such a piece of documentation was presented at the
European Lisp Symposium 2016 during a lightning talk that I gave. I
would like to expand on a particular slide of that presentation, which
outlines the qualities I expect of a Common Lisp documentation project.

\begin{enumerate}
\item
  Editable

  It needs to be modifiable and extensible by anyone willing to expand
  it.
\item
  Complete

  It should aim for completeness and maximizing its coverage of the
  Common Lisp universe.
\item
  Downloadable

  It should be usable locally, without an Internet connection.
\item
  Mirrorable/Clonable

  It should be easy to create mirrors and copies of it on the Internet
  and on hard drives.
\item
  Versioned

  It should use version control.
\item
  Modular

  It should be splittable into separate modules with cross-module
  hyperlinks breaking as the only side effect.
\item
  Updatable

  It should be easy to update it to its newest version.
\item
  Portable

  It should be exportable as a static HTML website.
\item
  Unified

  It should be consistent in style.
\item
  Community-based

  It should belong to the Lisp community and be further developed and
  extended there.

  \begin{center}\rule{0.5\linewidth}{\linethickness}\end{center}

  The implementation of this idea is a project created by me that I have
  named the Common Lisp UltraSpec, henceforth abbreviated CLUS.

  The dpANS source makes it \textbf{editable}.

  Git (\url{https://git-scm.com/}) as version control makes it
  \textbf{downloadable}, \textbf{mirrorable/clonable},
  \textbf{versioned} and \textbf{updatable}.

  Hosting it on GitHub (\url{https://github.com}) allows it to be
  \textbf{community-based}.

  DokuWiki (\url{https://www.dokuwiki.org/}) allows it to be
  \textbf{modular} and \textbf{portable}.

  The goals are - to make it \textbf{complete} and \textbf{unified}.
\end{enumerate}

\subsubsection{Source - dpANS CL (see
below)}\label{source---dpans-cl-see-below}

The whole process was made possible by the availability of the LaTeX
source code for ``draft preview Americal National Standard'',
abbreviated as dpANS, for Common Lisp. These sources were put into
public domain by Kent M. Pitman and other members of the X3J13
committee.

While not being the actual standard itself, the dpANS is close enough to
it to be usable as a proper reference of Common Lisp while also being in
the public domain, which allows me to create derivative works of it. It
turned out to be a feasible source upon which I could begin implementing
the first part of the UltraSpec.

\subsubsection{Work done so far}\label{work-done-so-far}

At the moment of writing these words, I have translated six dictionaries
from the dpANS sources into pages in DokuWiki markup syntax, corrected
the pages and hyperlinked the code examples found inside.

Additionally, I have created a customized version of DokuWiki meant for
displaying the CLUS content. While I have not yet published the source
code of this modified DokuWiki instance, it was successfully deployed
(\url{http://phoe.tymoon.eu/clus/}) with the specification data
translated so far.

I expect to have the whole sources parsed and translated before the
European Lisp Symposium 2017.

\subsubsection{Demonstration of used methods and
tools}\label{demonstration-of-used-methods-and-tools}

The presence of feasible source for creating a unified and modernized
piece of Common Lisp documentation allowed me to download the sources
and start looking for means of parsing and processing it. The following
subchapters describe the tools I have been using and explain the reasons
for them being chosen.

\begin{enumerate}
\item
  Notepad++ (\url{https://notepad-plus-plus.org/}) - the text editor

  When it came to the main editor for doing most of the parsing work, I
  could choose between Emacs (\url{https://www.gnu.org/software/emacs/})
  and Notepad++, a pair of GPL-licensed
  (\url{https://www.gnu.org/licenses/gpl-3.0.en.html}) programmer's
  editors. (Emacs is a keyboard-oriented editor, available for all major
  operating systems; Notepad++ is a WYSIWYG, keyboard-and-mouse-oriented
  editor written for Windows that I was able to run on my Linux setup
  using the Wine (\url{https://www.winehq.org/}) toolkit.) I chose the
  latter mostly because I have been using Notepad++ for the past few
  years and also due to the entry threshold associated with Emacs; I am
  still learning this editor despite having used it for more than a year
  now, and I have been using it mostly as a Lisp programming
  environment.
\item
  DokuWiki - the engine for displaying HTML

  DokuWiki is a GPL-licensed wiki software written in PHP
  (\url{http://php.net/}). In my experience, it was able to fulfill all
  the requirements I had for a displaying engine: it does not need
  database access and instead relies on flat files, which allows me for
  easy versioning the data with Git; it has a simple markup syntax that
  I consider sane; it is extensible and hackable, which so far proves
  very useful; I have had some previous experience in using and
  configuring; and last but not least, it simply works and allows me to
  deliver the contents in a readable and aesthetically pleasing way,
  which is the most important reason.
\item
  Regular expressions, Unix coreutils - the tool for parsing the sources

  The most important choice that I have had to make in the beginning
  was, how to parse the source files of the dpANS. The source code is a
  large body of LaTeX code, created by multiple people over a large span
  of time. It contains highly customized TeX macros, used irregularly
  among the source code.

  The initial research led me towards TeX parsers written in various
  languages, such as Parsec (\url{https://wiki.haskell.org/Parsec})
  written in Haskell (\url{https://wiki.haskell.org/}). My initial
  attempts of feeding the dpANS sources to the parsers I found were
  failures though; the individual bodies of code were too complex and my
  knowledge about these parsers was too little for me to succeed. I
  realized that, in order to properly parse the TeX source code of the
  draft, I would need to create a substantially large set of parsing
  rules; even afterwards, I would need to spend a lot of time doing
  manual polishing and fixing of the corner cases, such as TeX macros
  used only in a few places within the source files or actual mistakes
  within formatting, such as utilizing function markup for macros and
  vice versa.

  Because of this, I decided to abandon the approach of parsing the
  standard with a parser capable of processing TeX directly and instead
  go for a simpler choice: utilizing a set of regular expressions to
  parse a subset of utilized TeX macros and formatting. It would mean
  later polishing the preprocessed data by hand, though I would like to
  note that this last step would be necessary anyway regardless of the
  technique used.

  My editor of choice, Notepad++, contained a powerful enough RegEx
  engine that was capable of guiding me through the process. Various
  bulk edits were also made through the assorted unix utilities: grep,
  sed, awk, rename.
\item
  Git - versioning system, GitHub - project hosting

  The data for the whole project is kept in a Git repository, stored at
  GitHub (\url{https://github.com/phoe/clus-data}) and publicly
  available. Because DokuWiki keeps all data as flat text files, I can
  easily modify and deploy new versions of data to upstream websites.
\end{enumerate}

\subsubsection{Problems encountered}\label{problems-encountered}

Most of the problems I have encountered are connected with the dpANS
sources being a big and complicated piece of documentation and usage of
regular expressions to parse the TeX sources.

As I have mentioned before, the source code had been created over a
lengthy period of time with multiple people contributing to it. Because
of that, many parts of the specification are formatted differently: they
utilize different TeX macros, specific to the people creating the source
and the part of the language that was worked upon. Despite the
irregularities, I was able to employ the regular expressions and
capabilities of my editor to fix most of the cases globally and fix the
corner cases manually.

A significant part of the required work was hyperlinking. Although I was
able to parse the code for TeX glossary entries, I also needed to take
the English grammar into account, such as plural and past forms of
glossary entries.

I have had some minor problems with DokuWiki's rendering and markup
capabilities, though none of them have been significant enough to be
mentioned in detail here.

\subsection{Conclusions and future
work}\label{conclusions-and-future-work}

\subsubsection{Benefits/Disadvantages}\label{benefitsdisadvantages}

The benefits of my approach come as logical continuations of the slogans
used in section \textbf{Requirements} TODO FIX REFERENCE.

The most obvious one, which is also the goal of the project, is the
construction of a contemporary source of Common Lisp documentation and a
singular resource capable of containing most of the knowledge a Common
Lisp programmer might need.

Another upside is modernization of the specification by fixing its
issues and bugs, expanding its examples sections, clarifying any
inconsistencies and questions that have emerged since the creation of
the standard and giving it a more aesthetically pleasing look.

A beneficial side effect of my approach is generation of a version of
the Common Lisp specification in a markup format. Such a format can then
be easily parsed by automated tools to produce a document of any
required typesetting qualities.

\begin{center}\rule{0.5\linewidth}{\linethickness}\end{center}

The disadvantages of my current approach occur on different layers.

First of all, it is easy to keep a single static website on the Web for
years without any changes, but CLUS is far from static because of its
design. The body of code that CLUS will turn into, as the time
progresses, will require maintenance in order to stay clear and
readable; it will require reviewers to check the input from anyone
wanting to contribute to the CLUS repositories.

Second, although it does apply specifically to the dpANS sources,
parsing and hyperlinking the chapters of the specification takes
significant time. Additionally, because of the variety of forms other
bodies of Lisp documentation have, it will be non-trivial to import them
into CLUS - it will require separate effort to have them parsed and
prepared for inclusion.

Third, the legal status and licensing issues of the various pieces of
documentation will require separate thought. Creating a compilation work
of all these elements will be essentially creating a derivative of them
all and legal caution will need to be taken in case of documents with
unknown or confusing legal status. It might be required to negotiate the
terms of inclusion of particular pieces of work into CLUS with the
respective holders of rights to them.

\subsubsection{Thoughts}\label{thoughts}

Among all the literature available for studying Common Lisp, I would
like to mention the the dpANS source files as a valuable read from a
non-technical point of view.

The standard was created before the era of ubiquitous versioning
systems. Because of this, the draft source contains many comments, some
of them timestamped. They show the technical problems and decisions the
langauge specifiers faced and solved in the process of creating a formal
standard for a programming language. They alsooutline the features which
were deprecated and removed - or, on the contrary, created and added
along the way, some of which I personally find quite enlightening. What
I want to emphasize here, though, is that they show X3J13 as a group of
human beings working on a common goal. The comments there show various
aspects of their work: from communicating messages between particular
people, through decision-making and commented-out pieces of
specification itself, to the in-jokes and humor of the people.

In my opinion, studying the original sources for all three draft
previews (all of which are available online) might be valuable for any
person who wants to research specification development or software
development in general from a more humane point of view as well as Lisp
programmers who are interested in extending their background and the
process through which Common Lisp came to life.

\begin{center}\rule{0.5\linewidth}{\linethickness}\end{center}

Another thought that I would like to mention here is the fact that, in
the beginning, I had imagined my work as simple translation of the
sources from their TeX format into wiki markup in order to let the
DokuWiki engine format them into HTML. Reality has verified these ideas
- I quickly realized that the standard itself has its share of
inconsistencies, bugs and other issues. It is of course expected for
such a huge body of documentation to have issues and these issues do not
undermine the value of the specification as a whole, but I have
unexpectedly found myself to be able to fix them as I progress through
the sources.

Suddenly, from a simple translator, I had become an editor of the Common
Lisp standard itself. What I am creating right now is not the draft
sources being translated into DokuWiki markup - it is an edited version
which contains many improvements and fixes to many issues that were
impossible to fix in the previous CL specifications based on the work of
X3J13.

It is a very responsible role that has emerged - but also one that I
consider very satisfying.

\subsubsection{Plans}\label{plans}

It is impossible to speak of future plans without mentioning the Lisp
community here.

The Common Lisp UltraSpec was meant from the start to be a
community-based project, meaning that it belongs to the Lisp community
and is meant to be utilized and expanded within it. I hope that other
people will aid me in my process by suggesting changes, submitting
patches, possibly integrating the documentation for respective Common
Lisp libraries into the code and maintaining them later on.

Once the specification is completely integrated, I intend on extending
its scope to include common facilities and extensions included and/or
used in most contemporary Common Lisp implementations, such as the
Metaobject Protocol, ASDF, Quicklisp and the compatibility libraries
which provide cross-platform functionalities not included in the
standard such as parallelism or networking.

I want to create quality standards for the respective types of pages and
enforce them in order to keep the quality of the documentation high and
its style consistent across pages and modules.

\subsection{TODO Bibliography}\label{todo-bibliography}
